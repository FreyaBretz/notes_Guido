\section{Interior penalty methods}
\label{sec:interior-penalty}

\begin{intro}
  In this section we extend the weakening of continuity, which we
  explored for boundary values in Section~\ref{sec:nitsches-method}
  using Nitsche's method to interior interfaces between mesh
  cells. While the methods obtained may look much more complicated,
  the mathematical analysis is completely analogue to that
  section. Thus, we can be fairly brief.
\end{intro}

\begin{definition}
  Let $\T_h$ be a mesh of $\Omega \subset \R^d$ consisting of mesh
  cells $T_i$. For every boundary facet $F\subset \partial T_i$, we
  assume\footnote{This assumption can indeed be relaxed} that either
  $F \subset \partial \Omega$ or $F$ is a boundary facet of another
  cell $T_j$. In the second case, we indicate this relation by
  labeling this facet $F_{ij}$. The set of all facets $F_{ij}$ is the
  set of interior faces $\F_h^i$. The set of facets on the boundary is
  $\F_h^\partial$.
\end{definition}

\begin{definition}
  The discontinuous finite element space on $\T_h$ is constructed by
  concatenation of all shape function spaces $P_T$ for $T\in \T_h$
  without additional continuity requirements:
  \begin{gather}
    V_h = \bigl\{v\in L^2(\Omega) \big|
    v_{|T} \in P_T \;\forall T\in \T_h\bigr\}
  \end{gather}
\end{definition}

%%% Local Variables:
%%% mode: latex
%%% TeX-master: "main"
%%% End:




