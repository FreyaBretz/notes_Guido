\svnid{$Id$}

\begin{intro}
  There are two fundamentally different definitions of Sobolev spaces,
  which are usually referred to as the spaces $H^k$ and $W^{k,p}$. The
  first group is obtained by completing a space of continuously
  differentiable functions with respect to a given norm. The second
  definition relies on the introduction of distributional derivatives
  and then restricts the set of functions with such derivatives to
  those bounded with respect to a given norm. Thus, we can say that
  $H^k$ approximates the desired space from the inside, while
  $W^{k,p}$ bounds it from the outside. An important result of modern
  analysis was the conclusion that both classes are actually the same.

  Details on Sobolev spaces, including most of the material below can
  be found in~\cite{AdamsFournier03}. A very condensed introduction is
  also in\cite[Chapter 7]{GilbargTrudinger98}.
\end{intro}

%%%%%%%%%%%%%%%%%%%%%%%%%%%%%%%%%%%%%%%%%%%%%%%%%%%%%%%%%%%%%%%%%%%%%%
%%%%%%%%%%%%%%%%%%%%%%%%%%%%%%%%%%%%%%%%%%%%%%%%%%%%%%%%%%%%%%%%%%%%%%
\section{The Sobolev spaces $H^{1}(\Omega)$}
\label{sec:h1}
%%%%%%%%%%%%%%%%%%%%%%%%%%%%%%%%%%%%%%%%%%%%%%%%%%%%%%%%%%%%%%%%%%%%%%
%%%%%%%%%%%%%%%%%%%%%%%%%%%%%%%%%%%%%%%%%%%%%%%%%%%%%%%%%%%%%%%%%%%%%%

\begin{intro}
  These spaces are defined by first defining a norm for them, then
  completing for instance the space $\co^\infty$ with respect to this
  norm. This will lead to some difficulties with the involved symbols,
  which we will resolve in Section~\ref{sec:weak-derivatives}. We note
  that the problem of definiteness of the norm is the same as in the
  definition of $L^2(\Omega)$, which is, why we again have to take
  equivalence classes.
\end{intro}

\begin{definition}
  For functions $f,g\in \co^\infty(\Omega)$, we define the inner
  products, the
  $H^{1}$-seminorm $|.|_1$ and the $H^{1}$-norm $\norm{.}_1$ as
  \begin{xalignat}2
    \scal(f,g)_1 &= \int_\Omega \nabla f\cdot \nabla g\dx,
    &
    |f|_1 &= \norm{\nabla f}_0,
    \\
    \Scal(f,g) &= \scal(f,g)_0 + \scal(f,g)_1,
        & \norm{f}_1 &= \norm{f}_0 + |f|_1.
      \end{xalignat}
  Here, $\scal(.,.)_0$ and $\norm{.}_0$ refer to the inner product and
  norm in $L^2(\Omega)$, respectively.
\end{definition}

\begin{definition}
  \index{h1@$H^1(\Omega)$}
  First we compute the completion of
  $\co^\infty(\Omega)$ with respect to the norm $\norm{f}_1$, that is,
  the set of limits of all Cauchy sequences with respect to the
  $H^1$-norm consisting of elements in $\co^\infty(\Omega)$ with
  uniformly bounded $H^1$-norm. The $H^1$-norm of the limit function
  is defined as the limit of the norms of the sequence.
  
  The \define{Sobolev space} $H^1(\Omega)$ is the set of equivalence classes in
  this completion, where we say $f\simeq g$ if $\norm{f-g}_1=0$.
\end{definition}

\begin{note}
  The space $\co^\infty(\Omega)$ in this definition could have been
  replaced by $\co^1(\Omega)$ with no different effect.
\end{note}

\begin{example}
  The completion process in the definition above indeed yields
  functions which were not in $\co^\infty$. For instance, it is
  possible to construct a smooth sequence of functions converging to
  the function $f(x) = |x|$ with respect to the $H^1$-norm on $[-1,1]$.
\end{example}

\begin{definition}
  The space $\co^\infty_0(\Omega)$ is the space fo functions with
  ininitely many continuous derivatives and compact support in
  $\Omega$.
\end{definition}

\begin{definition}
  \index{h10@$H^1_0(\Omega)$}
  The Sobolev space $H^1_0(\Omega)$ is obtained by completing the
  space $\co^\infty_0(\Omega)$ with respect to the $H^1$-norm and
  taking equivalence classes such that $f\simeq g$ if $\norm{f-g}_1=0$.
\end{definition}

\begin{note}
  It remains to be shown that the spaces $H^1(\Omega)$ and
  $H^1_0(\Omega)$ are actually different.
\end{note}
%%%%%%%%%%%%%%%%%%%%%%%%%%%%%%%%%%%%%%%%%%%%%%%%%%%%%%%%%%%%%%%%%%%%%%
%%%%%%%%%%%%%%%%%%%%%%%%%%%%%%%%%%%%%%%%%%%%%%%%%%%%%%%%%%%%%%%%%%%%%%
\section{Weak derivatives and the Sobolev spaces $W^{1,2}$}
\label{sec:weak-derivatives}
%%%%%%%%%%%%%%%%%%%%%%%%%%%%%%%%%%%%%%%%%%%%%%%%%%%%%%%%%%%%%%%%%%%%%%
%%%%%%%%%%%%%%%%%%%%%%%%%%%%%%%%%%%%%%%%%%%%%%%%%%%%%%%%%%%%%%%%%%%%%%

\begin{intro}
  When we introduced the space $H^1(\Omega)$, we started with a
  subspace of continuous functions and extended this space by adding
  the limits of Cauchy sequences. The result is a complete space with
  respect to the norm $\norm{.}_1$. On the other hand, it is not clear
  whether this is the biggest function space for which this norm is
  finite. Therefore, in this section, we take the opposite approach:
  we define a derivative in a very broad sense and then reduce to
  those derivatives bounded with respect to the norm $\norm{.}_1$.
\end{intro}

\begin{definition}
  If for a given function $u$ there exists a function $w$ such that
  \begin{gather}
    \label{eq:hk:1}
     \int_\Omega w \phi \dx
     =
     -\int_\Omega u \partial_i \phi \dx,
     \qquad\forall \phi\in \co^\infty_0(\Omega),
  \end{gather}
  then we define $\partial_i u := w$ as the \define{distributional
    derivative} (partial) of $u$ with respect to $x_i$. Similarly through
  integration by parts, we define distributional directional
  derivatives, distributional gradients etc.
\end{definition}

\begin{note}
  The formula~\eqref{eq:hk:1} is the usual integration by
  parts. Therefore, whenever $u\in\co^1$ in a neighborhood of $x$, the
  distributional derivative and the usual derivative coincide.
\end{note}

\begin{example}
  Let $\Omega=\R$ and $u(x) = |x|$. Intuitively,
  it is clear that the distributional derivative, if it exists, must
  be the \define{Heaviside function}
  \begin{gather}
    \label{eq:hk:2}
    w(x) =
    \begin{cases}
      -1 & x<0 \\ 1 & x>0.
    \end{cases}
  \end{gather}
  The proof that this is actually the distributional derivative is
  left to the reader.
\end{example}

\begin{example}
  For the derivative of the \putindex{Heaviside function}
  in~\eqref{eq:hk:2}, we first observe that it must be zero whenever
  $x\neq 0$, since the function is continuously differentiable
  there. Now, we take a test function $\phi\in\co^\infty$ with support
  in the interval $(-\epsilon,\epsilon)$ for some positive
  $\epsilon$. Let $w'(x)$ be the derivative of $w$. Then, by
  integration by parts
  \begin{gather*}
    \int_{-\epsilon}^\epsilon w(x) \phi'(x)\dx
    = -\int_{-\epsilon}^0 w(x)' \phi(x)\dx
    -\int_0^\epsilon w(x)' \phi(x)\dx
    + 2 \phi(0) = 2\phi(0),
  \end{gather*}
  since $w'(x) = 0$ under both integrals. Thus, $w'(x)$ is an object
  which is zero everywhere except at zero, but its integral against a
  test function $\phi$ is nonzero. This contradicts our notion, that
  integrable functions can be changed on a set of measure zero without
  changing the integral. Indeed, $w'$ is not a function in the usual
  sense, and we write $w'(x) = 2 \delta(x)$, where $\delta(x)$ is the
  \define{Dirac $\delta$-distribution}, which is defined by the two
  conditions
  \begin{gather*}
    \begin{alignedat}{2}
      \delta(x) &= 0, & \forall x & \neq 0
      \\
      \int_\R \delta(x) \phi(x)\dx &= \phi(0), \quad & \forall \phi
      &\in \co^0(\R).
    \end{alignedat}
  \end{gather*}
  We stress that $\delta$ is not an integrable function, or a function
  at all.
\end{example}


\begin{example}
   Take for instance $\Omega
  = [0,1]$. It is known that \putindex{Lipschitz-continuous}
  functions on $\R$ are absolutely continuous and thus continuously
  differentiable almost everywhere, with their derivatives bounded by
  the Lipschitz constant, say $L$. Thus, the function itself is
  bounded on $[0,1]$, say by $M$. Therefore, such a function $f$ is
  weakly differentiable and its norm is bounded by
  \begin{gather*}
    \norm{f}_1 \le L+M.
  \end{gather*}
\end{example}

\begin{definition}
  For a function $u\in L^2(\Omega)$, we call a distributional
  derivative a \define{weak derivative}, if the derivative is in
  $L^2(\Omega)$ as well. For such a \define{weakly differentiable} function,
  the seminorm $|.|_1$ and thus the norm $\norm{.}_1$ is defined, if
  the gradient is understood in the distributional sense.
  
  The space of weakly differentiable functions defined in this manner
  is the \putindex{Sobolev space} $W^{1,2}(\Omega)$, where the
  superscript one stands for the order of derivatives and the two is
  the exponent in the norm.
\end{definition}

\begin{remark}
  In this section and the previous section, we have seen two different
  definition of Sobolev spaces. One of the important theorems of
  modern analysis due to Meyers and Serrin~\cite{MeyersSerrin64}
  states that these two definitions actually specify the same object.
\end{remark}

%%%%%%%%%%%%%%%%%%%%%%%%%%%%%%%%%%%%%%%%%%%%%%%%%%%%%%%%%%%%%%%%%%%%%%
%%%%%%%%%%%%%%%%%%%%%%%%%%%%%%%%%%%%%%%%%%%%%%%%%%%%%%%%%%%%%%%%%%%%%%
\section{Higher order derivatives and different exponents}
\label{sec:higher-derivatives}
%%%%%%%%%%%%%%%%%%%%%%%%%%%%%%%%%%%%%%%%%%%%%%%%%%%%%%%%%%%%%%%%%%%%%%
%%%%%%%%%%%%%%%%%%%%%%%%%%%%%%%%%%%%%%%%%%%%%%%%%%%%%%%%%%%%%%%%%%%%%%

\begin{intro}
  Here we generalize the definitions of Sobolev spaces in
  Sections~\ref{sec:h1} and~\ref{sec:weak-derivatives}, respectively,
  to higher order derivatives. Most of this section will consist of
  introducing ugly notation, which is, why we postponed
  this. Mathematically, like for continuous derivatives, the same
  concepts as above are applied to obtain the new spaces.
\end{intro}

\begin{definition}
  A $d$-dimensional \define{multi-index} is a tuple
  $\alpha=(\alpha_1,\alpha_2,\dots,\alpha_d)$ with values $\alpha_i$
  being nonnegative integers. We define partial derivatives of a
  function $u\in C^\infty$ with respect to a multi-index as
  \begin{gather*}
    \partial_{\alpha}u(x) = \frac{\partial^\alpha}{\partial x^\alpha}
    u(x)
    = \frac{\partial^{\alpha_1}}{\partial x_1^{\alpha_1}}
    \frac{\partial^{\alpha_2}}{\partial x_2^{\alpha_2}}
    \dots
    \frac{\partial^{\alpha_d}}{\partial x_d^{\alpha_d}}
    u(x_1,x_2,\dots,x_d).
  \end{gather*}
  The order of such a derivative is
  \begin{gather*}
    |\alpha| = \sum_{i=1}^d \alpha_i.
  \end{gather*}
\end{definition}

\begin{definition}
  Let $u$ be integrable on $\Omega\subseteq \R^d$. If a function $w$
  exists such that
  \begin{gather}
    \int_\Omega w \phi\dx = (-1)^{|\alpha|} \int_\Omega
    u, \partial_\alpha \phi \dx,
    \qquad\forall \phi\in \co^\infty_0(\Omega),
  \end{gather}
  then we call $\partial_\alpha u = w$ a \putindex{distributional derivative} of
  order $|\alpha|$ of $u$. If this derivative is in $L^2(\Omega)$, we
  call it a \putindex{weak derivative}.
\end{definition}

\begin{definition}
  The space $W^{k,2}(\Omega)$ is the space of functions $u\in
  L^2(\Omega)$ such that all distributional derivatives of order
  $|\alpha| \le k$ are in $L^2(\Omega)$. The $W^{k,2}$-seminorm and
  -norm are defined by
  \begin{gather*}
    |f|_k = \sqrt{\sum_{|\alpha| = k} \norm{\partial_\alpha u}^2},
    \qquad
    \norm{f}_k = \sqrt{\sum_{|\alpha| \le k} \norm{\partial_\alpha u}^2}.
  \end{gather*}
\end{definition}

\begin{remark}
  If we give up the notion of an inner product, all definitions above
  extend to the case where we replace $L^2$ by a space based on a norm
  with different exponent $1 \le p \le \infty$. This leads to the
  spaces $W^{k,p}(\Omega)$.
\end{remark}
%%%%%%%%%%%%%%%%%%%%%%%%%%%%%%%%%%%%%%%%%%%%%%%%%%%%%%%%%%%%%%%%%%%%%%
%%%%%%%%%%%%%%%%%%%%%%%%%%%%%%%%%%%%%%%%%%%%%%%%%%%%%%%%%%%%%%%%%%%%%%
\section{Properties of Sobolev spaces}
\label{sec:hk:properties}
%%%%%%%%%%%%%%%%%%%%%%%%%%%%%%%%%%%%%%%%%%%%%%%%%%%%%%%%%%%%%%%%%%%%%%
%%%%%%%%%%%%%%%%%%%%%%%%%%%%%%%%%%%%%%%%%%%%%%%%%%%%%%%%%%%%%%%%%%%%%%

\begin{intro}
  For continuously differentiable functions, the inclusion $\co^{k+1}
  \subset \co^k$ is obvious. The same inclusion holds for $W^{k+1,p}$
  and $W^{k,p}$ by definition. Continuous functions are obviously
  continuous on smooth submanifolds, but Sobolev functions are a
  priori not even defined there. Nevertheless, we will see below, that
  there are operators which allow us to define traces of Sobolev
  functions on lower dimensional submanifolds, and even embeddings
  into spaces of continuous functions.
\end{intro}

\begin{definition}
  Let $\Omega \subset \R^d$. For the \putindex{Sobolev space}
  $W^{k,p}(\Omega)$, we define the \define{Sobolev number}
  \begin{gather}
    \label{eq:hk:3}
    \sigma = k - \frac dp.
  \end{gather}
\end{definition}

\begin{todo}
  For the following theorems, verification is required when the cases $p=1$ and
  $p=\infty$ are included, when equality of the Sobolev numbers is
  sufficient and when not, and which assumptions on the domains are reasonable.
\end{todo}

\begin{theorem}
  Let $\Omega \subset \R^d$ be a bounded Lipschitz domain. Given two
  Sobolev spaces $W^{k_1, p_1}(\Omega)$ and $W^{k_2, p_2}(\Omega)$
  with $1 \le p_1, p_2 < \infty$. If $\sigma_1 \ge \sigma_2$, then
  there exists a continuous embedding $W^{k_1,
    p_1}(\Omega)\hookrightarrow W^{k_2, p_2}(\Omega)$ such that
\begin{gather*}
  \norm{u}_{W^{k_2, p_2}(\Omega)} \le C \norm{u}_{W^{k_1, p_1}(\Omega)}.
\end{gather*}
\end{theorem}

\begin{theorem}
  Let $\Omega_1 \subset \R^{d_1}$ be a bounded Lipschitz domain and
  $\Omega_2\subset \overline\Omega_1$ a smooth submanifold of
  dimension $d_2$. Then, if $\sigma_1 \ge \sigma_2$, there exists a
  continuous trace operator from $W^{k_1,
    p_1}(\Omega_1)\hookrightarrow W^{k_2, p_2}(\Omega_2)$, such that
  \begin{gather*}
    \norm{u}_{W^{k_2, p_2}(\Omega_2)} \le C \norm{u}_{W^{k_1, p_1}(\Omega_1)}.
  \end{gather*}
\end{theorem}

% \begin{definition}
%   A function is called \define{Hölder continuous} with exponent $\alpha$, if there is a
%   constant $C$, such that
%   \begin{gather*}
%     \forall x,y\in \Omega: \quad |f(x)-f(y)| \le C 
%   \end{gather*}
% \end{definition}

% \begin{theorem}
%   Let $\Omega_1 \subset \R^{d_1}$ be a bounded Lipschitz domain. 
% \end{theorem}

%%% Local Variables: 
%%% mode: latex
%%% TeX-master: "main"
%%% End: 
