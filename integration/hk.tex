\svnid{$Id$}

\begin{intro}
  There are two fundamentally different definitions of Sobolev spaces,
  which are usually referred to as the spaces $H^k$ and $W^{k,p}$. The
  first group is obtained by completing a space of continuously
  differentiable functions with respect to a given norm. The second
  definition relies on the introduction of distributional derivatives
  and then restricts the set of functions with such derivatives to
  those bounded with respect to a given norm. Thus, we can say that
  $H^k$ approximates the desired space from the inside, while
  $W^{k,p}$ bounds it from the outside. An important result of modern
  analysis was the conclusion that both classes are actually the same.

  Details on Sobolev spaces, including most of the material below can
  be found in~\cite{AdamsFournier03}.
\end{intro}

%%%%%%%%%%%%%%%%%%%%%%%%%%%%%%%%%%%%%%%%%%%%%%%%%%%%%%%%%%%%%%%%%%%%%%
%%%%%%%%%%%%%%%%%%%%%%%%%%%%%%%%%%%%%%%%%%%%%%%%%%%%%%%%%%%%%%%%%%%%%%
\section{The Sobolev spaces $H^1(\Omega)$}
\label{sec:h1}
%%%%%%%%%%%%%%%%%%%%%%%%%%%%%%%%%%%%%%%%%%%%%%%%%%%%%%%%%%%%%%%%%%%%%%
%%%%%%%%%%%%%%%%%%%%%%%%%%%%%%%%%%%%%%%%%%%%%%%%%%%%%%%%%%%%%%%%%%%%%%

\begin{intro}
  These spaces are defined by first defining a norm for them, then
  completing for instance the space $\co^\infty$ with respect to this
  norm. This will lead to some difficulties with the involved symbols,
  which we will resolve in Section~\ref{sec:weak-derivatives}. We note
  that the problem of definiteness of the norm is the same as in the
  definition of $L^2(\Omega)$, which is, why we again have to take
  equivalence classes.
\end{intro}

\begin{definition}
  For a function $f\in \co^\infty(\Omega)$, we define the
  $H^1$-seminorm $|.|_1$ and the $H^1$-norm $\norm{.}_1$ as
  \begin{align}
    |f|_1 &= \norm{\nabla f}_0,
    \norm{f}_1 &= \norm{f}_0 + |f|_1.
  \end{align}
  Here, $\norm{.}_0$ refers to the $L^2$-norm on $\Omega$. Since
  the $L^2$-norm ignores values on sets of measure zero, we will adopt
  the notion, that the $H^1$-seminorm is well-defined for a function
  $f$ which is differentiable almost everywhere.
\end{definition}

\begin{definition}
  First we compute the completion of
  $\co^\infty(\Omega)$ with respect to the norm $\norm{f}_1$, that is,
  the set of limits of all Cauchy sequences with respect to the
  $H^1$-norm consisting of elements in $\co^\infty(\Omega)$ with
  uniformly bounded $H^1$-norm.
  
  The Sobolev space $H^1(\Omega)$ is the set of equivalence classes in
  this completion, where we say $f\simeq g$ if $\norm{f-g}_1=0$.
\end{definition}

\begin{note}
  The space $\co^\infty(\Omega)$ in this definition could have been
  replaced by $\co^1(\Omega)$ with no different effect.
\end{note}

\begin{example}
  The completion process in the definition above indeed yields
  functions which were not in $\co^\infty$. For instance, it is
  possible to construct a smooth sequence of functions converging to
  the function $f(x) = |x|$ with respect to the $H^1$-norm on $[-1,1]$.
\end{example}

%%%%%%%%%%%%%%%%%%%%%%%%%%%%%%%%%%%%%%%%%%%%%%%%%%%%%%%%%%%%%%%%%%%%%%
%%%%%%%%%%%%%%%%%%%%%%%%%%%%%%%%%%%%%%%%%%%%%%%%%%%%%%%%%%%%%%%%%%%%%%
\section{Weak derivatives}
\label{sec:weak-derivatives}
%%%%%%%%%%%%%%%%%%%%%%%%%%%%%%%%%%%%%%%%%%%%%%%%%%%%%%%%%%%%%%%%%%%%%%
%%%%%%%%%%%%%%%%%%%%%%%%%%%%%%%%%%%%%%%%%%%%%%%%%%%%%%%%%%%%%%%%%%%%%%

\begin{example}
   Take for instance $\Omega
  = [0,1]$. It is known that \putindex{Lipschitz-continuous}
  functions on $\R$ are absolutely continuous and thus continuously
  differentiable almost everywhere, with their derivatives bounded by
  the Lipschitz constant, say $L$. Thus, the function itself is
  bounded on $[0,1]$, say by $M$. Therefore, such a function $f$ is
  weakly differentiable and its norm is bounded by
  \begin{gather*}
    \norm{f}_1 \le L+M.
  \end{gather*}
\end{example}

%%% Local Variables: 
%%% mode: latex
%%% TeX-master: "main"
%%% End: 
