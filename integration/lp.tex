\svnid{$Id$}

%%%%%%%%%%%%%%%%%%%%%%%%%%%%%%%%%%%%%%%%%%%%%%%%%%%%%%%%%%%%%%%%%%%%%%
%%%%%%%%%%%%%%%%%%%%%%%%%%%%%%%%%%%%%%%%%%%%%%%%%%%%%%%%%%%%%%%%%%%%%%
\section{The Lebesgue function spaces $L^p$}
\label{sec:lp}
%%%%%%%%%%%%%%%%%%%%%%%%%%%%%%%%%%%%%%%%%%%%%%%%%%%%%%%%%%%%%%%%%%%%%%
%%%%%%%%%%%%%%%%%%%%%%%%%%%%%%%%%%%%%%%%%%%%%%%%%%%%%%%%%%%%%%%%%%%%%%

\subsection{Integral inequalities}

\begin{lemma}[Young's inequality]
  \defindex{Young's inequality}
  For two arbitrary numbers $a$ and $b$, and a positive number
  $\gamma$ holds
  \begin{gather}
    \label{eq:lp:1}
    2\left|ab\right| \le \gamma a^2 + \frac1\gamma b^2.
  \end{gather}
\end{lemma}
\begin{proof}
  From the binomial formulas we have
  \begin{gather*}
    0 \le
    \begin{cases}
      (a+b)^2 &= a^2+b^2+2ab,
      \\
      (a+b)^2 &= a^2+b^2-2ab,
    \end{cases}
  \end{gather*}
  and thus by bringing $2ab$ to the left,
  \begin{gather}
    \label{eq:lp:2}
    \pm 2ab \le a^2+b^2.
  \end{gather}
  For $\gamma \neq 1$, the inequality is obtained by replacing $a$
  with $\gamma a$ and $b$ with $b/\gamma$ in~\eqref{eq:lp:2} and
  observing that this does not change the left hand side.
\end{proof}

\begin{lemma}[Cauchy-Bunyakovsky-Schwarz inequality\footnote{Also known as
    Cauchy-Schwarz or Schwarz's inequality}]
  \defindex{Cauchy-BunyakovskySchwarz inequality}
  If functions $f$ and $g$ and their squares are integrable on a
  subset $\Omega$ of $\R^d$, so is their pointwise product $fg$, and
  there holds
  \begin{gather}
    \label{eq:lp:3}
    \int_\Omega f(x)g(x)\dx \le \sqrt{\int_\Omega f^2(x)\dx} \sqrt{\int_\Omega g^2(x)\dx}.
  \end{gather}
\end{lemma}

\begin{proof}
  First, by Young's inequality we obtain for arbitrary
  $\gamma$ that
  \begin{gather*}
    2 \int_\Omega f(x)g(x)\dx
    \le \gamma \int_\Omega f^2(x)\dx
    + \frac1\gamma \int_\Omega g^2(x)\dx.
  \end{gather*}
  Choosing $\gamma$ such that both terms on the right are equal,
  namely
  \begin{gather*}
    \gamma = \left. \sqrt{\int_\Omega g^2(x)\dx} \right/ \sqrt{\int_\Omega f^2(x)\dx},
  \end{gather*}
  the inequality is obtained.
\end{proof}

\begin{note}
  The Cauchy-Bunyakovsky-Schwarz inequality is an immediate extension
  of the Cauchy inequality for discrete sums
  \begin{gather}
    \sum_k a_k b_k \le \sqrt{\sum_ka_k^2} \sqrt{\sum_kb_k^2}
  \end{gather}
\end{note}

\begin{lemma}[Hölder's inequality]
  Assume that the functions $|f|^p$ and $|g|^q$ are integrable with
  $1<p,q<\infty$ and
  \begin{gather}
    \label{eq:lp:5}
    \frac1p+\frac1q=1,
  \end{gather}
  then the function $fg$ is integrable on $\Omega$ and
  \begin{gather}
    \label{eq:lp:6}
    \int_\Omega |fg| \dx \le \sqrt[p] {\int_\Omega |f|^p \dx} \sqrt[q] {\int_\Omega |g|^q \dx}.
  \end{gather}
\end{lemma}

\begin{lemma}[Minkowski's inequality]
  Let the functions $|f|^p$ and $|g|^p$ be integrable on $\Omega$ for
  $1\le p < \infty$. Then, the function $|f+g|^p$ is integrable on
  $\Omega$ and
  \begin{gather}
    \label{eq:lp:7}
    \sqrt[p] {\int_\Omega |f+g|^p \dx}
    \le \sqrt[p] {\int_\Omega |f|^p \dx}
    +\sqrt[p] {\int_\Omega |g|^p \dx}
  \end{gather}
\end{lemma}

%%%%%%%%%%%%%%%%%%%%%%%%%%%%%%%%%%%%%%%%%%%%%%%%%%%%%%%%%%%%%%%%%%%%%%
\subsection{The real Hilbert spaces $L^2(\Omega)$}
%%%%%%%%%%%%%%%%%%%%%%%%%%%%%%%%%%%%%%%%%%%%%%%%%%%%%%%%%%%%%%%%%%%%%%

\begin{intro}
  For a bounded or unbounded subset $\Omega\subseteq \R^d$, the set of
  functions with squares, which are integrable with finite integrals
  forms a vector space. This is assured by the Minkowski's
  inequality~\eqref{eq:lp:7}. Our goal in this section is equipping this
  space with an inner product and a norm. Once a norm has been
  defined, we will show that the space obtained is complete, thus a
  \putindex{Hilbert space}.
  We start this section by the attempt to introduce an inner
  product.
\end{intro}

\begin{lemma}
\label{lemma:lp:bilinear}
  Let $\Omega \subseteq \R^d$ be bounded or unbounded.
  The form
  \begin{gather}
    \label{eq:lp:4}
    \scal(f,g) = \int_\Omega f(x)g(x)\dx
  \end{gather}
  is defined and bounded for functions $f$ and $g$ with integrable
  squares. It is bilinear, positive semidefinite, and symmetric.
\end{lemma}

\begin{proof}
  According to the Cauchy-Bunyakovsky-Schwarz
  inequality~\eqref{eq:lp:3}, boundedness of $\scal(f,g)$ follows from
  boundedness of the integrals of $f^2$ and $g^2$.
  Linearity in $f$ and $g$ are an immediate consequence of the
  linearity of the integral in
  equation~\eqref{eq:integral:3}. Symmetry follows from the fact that
  we can change the order of $f$ and $g$ in the product under the
  integral. Finally, we have
  \begin{gather*}
    0 \le \int_\Omega f^2(x)\dx = \scal(f,f).
  \end{gather*}
\end{proof}

\begin{intro}
  We have obtained $\scal(f,f) \ge 0$, but the definiteness of an
  inner product requires that $\scal(f,f) = 0$ implies $f=0$. On the
  other hand, $\scal(f,f) = 0$ holds for any function, which is zero
  almost everywhere in $\Omega$, but can have arbistrary values on a
  set of measure zero. Thus, we have no hope to define a definite
  inner product by integration on a function space. The solution to
  this dilemma is a modification of the function space by a little
  trick. Essentially, we turn around the definitions and define the
  relation $f=0$ through the condition $\scal(f,f) = 0$. We will now
  do this in a mathematically sound way.
\end{intro}

\begin{lemma}
  Let $F$ be a set of functions on $\R^d$. We introduce the relation
  $f\simeq g$ if $f$ and $g$ differ at most on a set of measure zero,
  or
  \begin{gather}
    f\simeq g
    \quad\Leftrightarrow f(x)-g(x)=0 \text{a.e.}
  \end{gather}
 This relation is an \putindex{equivalence relation}.
\end{lemma}

\begin{proof}
  We have to show reflexivity, symmetry, and transitivity of
  ``$\simeq$''. Obviously, since $f(x)=f(x)$ for all $x$, we have
  $f\simeq f$. Similarly obvious from the definition is that $f\simeq
  g$ implies $g\simeq f$. Finally, let $f\simeq g$ and $g\simeq
  h$. Then,
  \begin{gather*}
    f(x)-h(x) = \bigl(f(x)-g(x)\bigr) + \bigl(g(x)-h(x)\bigr).
  \end{gather*}
  Accordingly, the set on which $f$ and $h$ differ is at most the union of
  two set of measure zero, thus $f\simeq h$.
\end{proof}

\begin{lemma}
  The set of functions
  \begin{gather*}
    Z = \bigl\{ f \big| f\equiv 0 \bigr\},
  \end{gather*}
  where ``0'' is the function which is zero everywhere, is a vector
  space. Thus, for any vector space $V$ of functions, the quotient set
  $V/Z$ is a vector space.
\end{lemma}

\begin{proof}
  The first part of the proof is obvious. Thus, for any element of $V$
  vector addition and scalar multiplication can be split into their
  components in $Z$ and the remainder, which makes them well defined
  on the quotient set.
\end{proof}

\begin{definition}
  \defindex{L2@$L^2$}

  Let $V$ be the space of functions with bounded square integrals on a
  set $\Omega\subseteq \R^d$. Let $Z$ be the equivalence class of the
  zero function according to the preceding lemmas. We define the
  vector space $L^2(\Omega)$ as the quotient space $V/Z$. The space
  $L^2(\Omega)$ is equipped with the inner product $\scal(.,.)$
  according to equation~\eqref{eq:lp:4} and the norm
  \begin{gather}
    \norm{f} = \norm{f}_2 := \sqrt{\scal(f,f)}.
  \end{gather}
\end{definition}

\begin{note}
  We say for two functions $f$ and $g$ in $L^2(\Omega)$ that ``$f=g$''
  if the functions coincide almost everywhere.
  Thus, the missing definiteness of the inner product in
  Lemma~\ref{lemma:lp:bilinear} is obviously cured by considering
  equivalence classes. Thus, the norm is actually a norm, and
  $L^2(\Omega)$ for the moment is a pre-Hilbert space, that is, an
  inner product space, which is not necessarily complete.
\end{note}

\begin{remark}
  Consistent with other authors, we usually refer to an element of
  $L^2(\Omega)$ as a \define{square integrable function} or
  $L^2$-function. Nevertheless, it is important to keep in mind that
  these elements are not functions at all. In particular, they do not
  assign a value $f(x)$ to a given point $x\in \R^d$, since $\{x\}$ is
  a set of measure zero, and thus we would be allowed to change this
  value any time. This observation is important whenever we deal with
  function spaces $L^2(\Omega)$ and similar objects. Later we will
  have to answer the question what kind of evaluation of an
  $L^2$-function is actually permitted.
\end{remark}

\begin{definition}
  We say that a sequence $f_n$ in $L^2(\Omega)$ converges to an
  element $f\in L^2(\Omega)$ if $\norm{f_n-f}\to 0$ as $n\to\infty$. A
  sequence $f_n$ in $L^2(\Omega)$ is called a \define{Cauchy
    sequence}, if the \define{Cauchy criterion}
  \begin{gather}
    \label{eq:lp:cauchy}
    \forall \epsilon>0
    \quad \exists n_\epsilon
    \quad\forall m,n>n_\epsilon :
    \norm{f_m-f_n} < \epsilon.
  \end{gather}
  For the latter, we also say $\norm{f_m-f_n}\to 0$ as
  $n,m\to\infty$. 
\end{definition}

\begin{theorem}[Fischer-Riesz]
  Let $\Omega\subseteq \R^d$. Then,
  $L^2(\Omega)$ is a Hilbert space, that is, it is complete, that is,
  every Cauchy sequence $f_n$ in $L^2(\Omega)$ converges to an
  element $f\in L^2(\Omega)$.
\end{theorem}

\begin{proof}
  First, we note that the Cauchy criterion is necessary for
  convergence, since by the triangle inequality, we have for
  $m,n\to\infty$:
  \begin{gather*}
    \norm{f_m-f_n} \le \norm{f_m-f} + \norm{f-f_n} \to 0.
  \end{gather*}
  Now we assume that the Cauchy criterion holds. Then, there exists
  a sequence $n_1,n_2,\dots$ such that $\norm{f_{n_{k+1}}- f_{n_k}}
  <2^{-k}$. Now we have to distinguish between bounded domains
  $\Omega$ and unbounded domains. First, for bounded domains, the
  Cauchy-Bunyakovsky-Schwarz inequality implies that
  \begin{gather*}
    \int_\Omega 1 \left|f_{n_{k+1}}(x)- f_{n_k}(x)\right|\dx \le \sqrt{m(\Omega)}
    \norm{f_{n_{k+1}}- f_{n_k}} \le \sqrt{m(\Omega)}2^{-k},
  \end{gather*}
  hence the series
  \begin{gather*}
    \sum_{k=1}^\infty \int_\Omega |f_{n_{k+1}}(x)- f_{n_k}(x)|\dx
  \end{gather*}
  converges to a finite value. The sequence of functions defined by
  \begin{gather*}
    s_m(x) = \sum_{k=1}^m |f_{n_{k+1}}(x)- f_{n_k}(x)|,
  \end{gather*}
  is monotonically increasing. Furthermore, we have seen above that
  their integrals are uniformly bounded. Thus, by the Beppo-Levi
  theorem, $s_m(x)$ converges almost everywhere to an integrable
  function. This on the other hand implies that for almost every $x\in
  \Omega$ the sequence $f_{n_k}(x)$ is a Cauchy sequence and thus
  converges to a limit value $f(x)$.
  
  If the domain $\Omega$ is unbounded and its measure is not finite,
  the above argument can be applied to any finite subdomain. Covering
  $\Omega$ with a countable sequence of such subdomains, we can
  likewise conclude convergence to a limit function $f$ for
  almost every $x\in\Omega$.

  Now we prove that $f\in L^2(\Omega)$. We observe that
  \begin{gather*}
    \norm{f_{n_k}} \le \norm{f_{n_1}} + \norm{f_{n_k}-f_{n_1}} \le \norm{f_{n_1}}+\frac12.
  \end{gather*}
  Thus, $\norm{f_{n_k}}^2$ is uniformly bounded and as $|f_{n_k}(x)|^2
  \to |f(x)|^2$ almost everywhere, Fatou's lemma asserts that
  $\norm{f}$ is finite and accordingly $f\in L^2(\Omega)$.

  It remains to show that $\norm{f_n-f}\to 0$.
  \begin{todo}
    Show that norms converge
  \end{todo}
\end{proof}

%%%%%%%%%%%%%%%%%%%%%%%%%%%%%%%%%%%%%%%%%%%%%%%%%%%%%%%%%%%%%%%%%%%%%%
\subsection{The real Banach spaces $L^p(\Omega)$}
%%%%%%%%%%%%%%%%%%%%%%%%%%%%%%%%%%%%%%%%%%%%%%%%%%%%%%%%%%%%%%%%%%%%%%

\begin{intro}
  We saw that the Cauchy-Bunyakovsky-Schwarz inequality has a
  generalization for exponents different from 2 in Hölder's
  inequality, further that Minkowski's inequality holds for arbitrary
  $p$ with $1\le p<\infty$. This suggests to define a norm through
  \begin{gather}
    \label{eq:lp:8}
    \norm{f}_p = \sqrt[p] {\int_\Omega |f|^p \dx},
  \end{gather}
  and it is indeed an easy exercise to prove the norm properties
  analogue to the previous subsection.
\end{intro}

\begin{definition}
  A function $f:\Omega \to \R$ is called \define{essentially bounded}
  from above, if there is a number $a$ such that the set
  \begin{gather*}
    (f>a) := \bigl\{x\in\Omega \big| f(x)>a \bigr\},
  \end{gather*}
  has measure zero. The number $a$ is called an \define{essential
    upper bound}. For an essentially bounded function, we define the
  \define{essential supremum} $\esssup f$ as the infimum of all
  essential upper bounds:
  \begin{gather*}
    \esssup f = \inf_{a\in \R} (f>a) \text{ has measure zero.}
  \end{gather*}
\end{definition}

\begin{definition}
  For $1\le p < \infty$, the space $L^p(\Omega)$ is the space of all
  functions such that $|f|^p$ is integrable with its norm defined
  in~\eqref{eq:lp:8}. The space $L^\infty(\Omega)$ is the space of all
  essentially bounded functions on $\Omega$, and its norm is
  \begin{gather*}
    \norm{f}_\infty = \esssup_{x\in\Omega} f
  \end{gather*}
\end{definition}

\begin{note}
  For $p\neq2$, the $L^p$-norm is not defined by an inner product,
  thus $L^p(\Omega)$ cannot be a Hilbert space.
\end{note}

\begin{theorem}
  The spaces $L^p(\Omega)$ are complete, thus, they are Banach spaces.
\end{theorem}

%%% Local Variables: 
%%% mode: latex
%%% TeX-master: "main"
%%% End: 
