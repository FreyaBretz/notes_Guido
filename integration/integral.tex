\svnid{$Id$}

%%%%%%%%%%%%%%%%%%%%%%%%%%%%%%%%%%%%%%%%%%%%%%%%%%%%%%%%%%%%%%%%%%%%%%
%%%%%%%%%%%%%%%%%%%%%%%%%%%%%%%%%%%%%%%%%%%%%%%%%%%%%%%%%%%%%%%%%%%%%%
\section{The Lebesgue Integral}
\label{sec:lebesgue-integral}
%%%%%%%%%%%%%%%%%%%%%%%%%%%%%%%%%%%%%%%%%%%%%%%%%%%%%%%%%%%%%%%%%%%%%%
%%%%%%%%%%%%%%%%%%%%%%%%%%%%%%%%%%%%%%%%%%%%%%%%%%%%%%%%%%%%%%%%%%%%%%
\begin{intro}
  This section introduces the Lebesgue integral mostly following the
  presentation in the monograph by Riesz and
  Sz.-Nagy~\cite{RieszNagy}. We deviate from this work only in two
  respects: first, we restrict the presentation to the results
  pertaining to the definition and properties of $L^p$-spaces. Second,
  we elaborate more on higher-dimensional integrals and modify the
  development of the one-dimensional case in order to be closer to
  higher dimensions.
\end{intro}


%%% Local Variables: 
%%% mode: latex
%%% TeX-master: "main"
%%% End: 
