\svnid{$Id$}

%%%%%%%%%%%%%%%%%%%%%%%%%%%%%%%%%%%%%%%%%%%%%%%%%%%%%%%%%%%%%%%%%%%%%%
%%%%%%%%%%%%%%%%%%%%%%%%%%%%%%%%%%%%%%%%%%%%%%%%%%%%%%%%%%%%%%%%%%%%%%
\section{The Lebesgue Integral}
\label{sec:lebesgue-integral}
%%%%%%%%%%%%%%%%%%%%%%%%%%%%%%%%%%%%%%%%%%%%%%%%%%%%%%%%%%%%%%%%%%%%%%
%%%%%%%%%%%%%%%%%%%%%%%%%%%%%%%%%%%%%%%%%%%%%%%%%%%%%%%%%%%%%%%%%%%%%%
\begin{intro}
  This section introduces the Lebesgue integral mostly following the
  presentation in the monograph by Riesz and
  Sz.-Nagy~\cite{RieszNagy}. We deviate from this work only in two
  respects: first, we restrict the presentation to the results
  pertaining to the definition and properties of $L^p$-spaces. Second,
  we elaborate more on higher-dimensional integrals and modify the
  development of the one-dimensional case in order to be closer to
  higher dimensions.
\end{intro}

%%%%%%%%%%%%%%%%%%%%%%%%%%%%%%%%%%%%%%%%%%%%%%%%%%%%%%%%%%%%%%%%%%%%%%
\subsection{Step functions and their integrals}
%%%%%%%%%%%%%%%%%%%%%%%%%%%%%%%%%%%%%%%%%%%%%%%%%%%%%%%%%%%%%%%%%%%%%%

\begin{definition}
  \defindex{lattice}
  We introduce \textbf{lattices} $\Q_n$ of $\R^d$ consisting of half
  open cubes
  \begin{gather*}
    Q^{(n)}_{i_1,\dots,i_d} =
    \left]\tfrac{i_1}{2^n},\tfrac{i_1+1}{2^n}\right]
    \times
    \left]\tfrac{i_2}{2^n},\tfrac{i_2+1}{2^n}\right]
    \times\dots\times
    \left]\tfrac{i_d}{2^n},\tfrac{i_d+1}{2^n}\right],
    \qquad i_k\in\mathbb Z.
  \end{gather*}
  The lattice is said to be of width $2^{-n}$.
\end{definition}

\begin{note}
  Independent on the dimension $d$, the number of cubes in $\Q_n$ is
  countable. Thus, we can replace the multiple indices indicating the
  position of a cube in the lattice by a single enumeration index
  $k$. Furthermore, it is easy to see that the number of cubes in
  $\Q_n$ contained in a bounded set is finite, albeit depending on $n$.
\end{note}

\begin{definition}
  A function $f$ is called a \define{step function} on the lattice
  $\Q_n$, if $f$ is constant on each cube $Q_k$. We refer to the value
  of $f$ on $Q_k$ as $f(Q_k)$ or short $f_k$.
  value 
\end{definition}

\begin{definition}
  \index{integral!of a step function}
  The \textbf{integral} of a step function $f$ on $\Q_n$ is
  defined in the obvious way as
  \begin{gather*}
    \int_{\R^d} f(x) \,dx = \sum_{Q_k} f(q_k). 
  \end{gather*}
  Here, we explicitly allow the values $\infty$ and $-\infty$ for the integral.
\end{definition}

\begin{definition}
  The \define{support} of a function is the set
  \begin{gather}
    \supp f = \overline{
      \bigl\{x\in \R^d\big| f(x) \neq 0 \bigr\}}.
  \end{gather}
\end{definition}
\begin{note}
  If the support of $f$ is bounded, its integral is finite.
\end{note}

%%%%%%%%%%%%%%%%%%%%%%%%%%%%%%%%%%%%%%%%%%%%%%%%%%%%%%%%%%%%%%%%%%%%%%
\subsection{Sets of measure zero}
%%%%%%%%%%%%%%%%%%%%%%%%%%%%%%%%%%%%%%%%%%%%%%%%%%%%%%%%%%%%%%%%%%%%%%

%%%%%%%%%%%%%%%%%%%%%%%%%%%%%%%%%%%%%%%%%%%%%%%%%%%%%%%%%%%%%%%%%%%%%%
\subsection{Definition of the integral}
%%%%%%%%%%%%%%%%%%%%%%%%%%%%%%%%%%%%%%%%%%%%%%%%%%%%%%%%%%%%%%%%%%%%%%

%%%%%%%%%%%%%%%%%%%%%%%%%%%%%%%%%%%%%%%%%%%%%%%%%%%%%%%%%%%%%%%%%%%%%%
\subsection{Integration of limits}
%%%%%%%%%%%%%%%%%%%%%%%%%%%%%%%%%%%%%%%%%%%%%%%%%%%%%%%%%%%%%%%%%%%%%%

%%% Local Variables: 
%%% mode: latex
%%% TeX-master: "main"
%%% End: 
