\begin{intro}
  We can embed finite element methods for the Darcy problem, also for
  the Maxwell problem, into a common framework based on the de Rham
  complex. If we wanted to do this in its full mathematical beauty, we
  would have to spend some time introducing the concept and notation
  of differential forms. As an alternative, we can use the concrete
  vector spaces $\Hdiv(\domain)$ and $\Hcurl(\domain)$. The drawback
  is, that we have to prove deveral particular cases, where the
  abstract theory only knows one common case. Nevertheless, it is
  worthwhile to begin this way, such that the reader has an easier
  task reading the full theory
  in~\cite{ArnoldFalkWinther06acta,ArnoldFalkWinther10}. As a byproduct,
  we will prove in generality some of the properties of polynomial
  spaces in Chapter~\ref{cha:darcy}.
\end{intro}

\section{Maxwell equations}

\begin{intro}
  Electromagnetic fields are governed by four laws of nature put together
  by James Clerk Maxwell to a single system. The laws are
  \begin{enumerate}
  \item Gauss' law for the electric field: the electric flux
    through a closed surface equals $1/\epsilon$ times the electric
    charge enclosed by the surface:
    \begin{gather*}
      \int_{\d V} E\cdot \n \ds = \int_V \frac\rho\epsilon \dx.
    \end{gather*}
  \item There are no magnetic monopoles, therefore the magnetic flux
    through any closed surface vanishes:
    \begin{gather*}
      \int_{\d V} B\cdot \n \ds = 0.
    \end{gather*}
  \item Faraday's law of induction: the voltage induced in a closed
    loop is proportional to the rate of change of the magnetic field
    through the surface encloded by the loop:
    \begin{gather*}
      \int_{\d A} E\cdot\ds = -\frac{d}{dt}\int_A B\cdot \n \ds.
    \end{gather*}
  \item Ampère's law: the magnetic field induced in a closed loop is
    proportional to the electric current plus the change of electric
    field through that loop:
    \begin{gather*}
      \int_{\d A} B\cdot\ds
      = \mu \int_A J\cdot \n \ds
      + \mu\epsilon\frac{d}{dt}\int_A E\cdot \n \ds.
    \end{gather*}
  \end{enumerate}
  
  Using the Gauss theorem for the first two and the Stokes theorem for
  the remaining two laws, we obtain the \define{Maxwell equations} of
  electromagnetics
  \begin{xalignat}2
    \div E &= \frac\rho\epsilon
    & \curl E &= -\d_t B,\\
    \div B &= 0
    & \curl B &= \mu J + \mu\epsilon E.
  \end{xalignat}
  They are an hyperbolic system of equations and typically have wave
  solutions. Many simplifications have been developed to suit
  particular purposes.
\end{intro}

\begin{intro}
  An important simplification of the Maxwell equations is obtained by
  assuming an isolating material, that is, the electric current $J$
  vanishes. Additionally, we may assume that there are no electric
  charges, such that $\div E=0$. Then, taking the curl of the equation
  for $\curl E$ and inserting the formula for $\curl B$, we obtain
  \begin{gather}
    \mu\epsilon \d_t^2 E + \curl\curl E = 0
    \qquad \div E=0.
  \end{gather}
  We can even go further and study the stationary limit
  \begin{gather}
    \curl\curl E=0 \qquad \div E=0.
  \end{gather}
\end{intro}

\begin{Notation}{curl}
  With $\curl u$ we describe the curl of a vector field $u$, which in
  three dimensions is defined as
  \begin{gather}
    \curl u =
    \begin{pmatrix}
      ???
    \end{pmatrix}
  \end{gather}
\end{Notation}

\begin{remark}
  A polular error in the literature consists of the following
  argument: since $\div E = 0$, there also holds $\nabla \div E =
  0$. Therefore, we can use the formula
  \begin{gather*}
    \Delta u = \nabla\div u - \curl\curl u,
  \end{gather*}
  and avoid the $\div-\curl$-problem alltogether.
\end{remark}

\section{The de Rham complex of Hilbert spaces}

\begin{intro}
  We now know three differential operators, $\nabla$, $\curl$, and
  $\div$ with the interesting property
  \begin{gather}
    \curl\nabla \phi = 0
    \qquad \div\curl E=0.
  \end{gather}
  As a consequence, for $\phi\in H^1(\domain)$ we not only have
  $\nabla \phi\in L^2(\domain;\R^3)$, we also have
  $\curl\nabla\phi=0\in L^2(\domain;\R^3)$. This gives rise to the sequence
  \begin{gather}
    \R
    \overset{\subset}{\longrightarrow} H^1(\domain)
    \overset{\nabla}{\longrightarrow} \Hcurl(\domain)
    \overset{\curl}{\longrightarrow} \Hdiv(\domain)
    \overset{\div}{\longrightarrow} L^2(\domain)
    \longrightarrow 0,
  \end{gather}
  such that the range of an operator is always in the kernel of the
  operator to its right.
\end{intro}

\begin{Notation}{hlambda}
  The notation of exterior calculus of differential forms allows us to
  write this sequence elegantly as
  \begin{gather}\minCDarrowwidth20pt
    \begin{CD}
      \R
      @>{d}>> H\Lambda^0(\domain)
      @>{d}>> H\Lambda^1(\domain)
      @>{d}>> H\Lambda^2(\domain)
      @>{d}>> H\Lambda^3(\domain)
      @>>> 0
      \\
      @.
      @V{\cong}VV
      @V{\cong}VV
      @V{\cong}VV
      @V{\cong}VV
      \\
      \R
      @>{\subset}>> H^1(\domain)
      @>{\nabla}>> \Hcurl(\domain)
      @>{\curl}>> \Hdiv(\domain)
      @>{\div}>> L^2(\domain)
      @>>> 0,
    \end{CD}
  \end{gather}
  such that $d\colon H\Lambda^k(\domain) \to H\Lambda^{k+1}(\domain)$ and
  \begin{gather}
    d^2 = d\circ d = 0.
  \end{gather}
\end{Notation}

\begin{remark}
  The spaces $H\Lambda^k(\domain)$ are Hilbert spaces with values in
  the spaces of alternating $k$-forms on $\R^d$. From linear algebra,
  we know that all alternating $k$-forms are zero if $k$ exceeds the
  dimension of the vector space.  Therefore, the sequence above is
  only valid in three dimensions, and it must be shorter by one member
  in two dimensions. Changing our view back to differential operators,
  we realize that there are two relevant sequences in two
  dimensions. In the following diagram, the sequence on top can be
  used to formulate Maxwell problems in $\Hcurl$ in two dimensions,
  while the sequence on the bottom relates to the mixed form of the
  Laplacian.

  We introduce the sequences in two dimensions and afterwards will
  focus our arguments on the more general case of three dimensions
  again. Specialization to two dimensions are straight forward.
\end{remark}

\begin{Notation}{hlambda-2d}
  In two dimensions, we consider the de Rham sequences
  \begin{gather}\minCDarrowwidth20pt
    \begin{CD}
      \R
      @>{\subset}>> H^1(\domain)
      @>{\nabla}>> \Hcurl(\domain)
      @>{\curl}>> L^2(\domain)
      @>>> 0
      \\
      @.
      @A{\cong}AA
      @A{\cong}AA
      @A{\cong}AA
      \\
      \R
      @>{d}>> H\Lambda^0(\domain)
      @>{d}>> H\Lambda^1(\domain)
      @>{d}>> H\Lambda^2(\domain)
      @>>> 0
      \\
      @.
      @V{\cong}VV
      @V{\cong}VV
      @V{\cong}VV
      \\
      \R
      @>{\subset}>> H^1(\domain)
      @>{\curl}>> \Hdiv(\domain)
      @>{\div}>> L^2(\domain)
      @>>> 0,
    \end{CD}
  \end{gather}
\end{Notation}

%%% Local Variables: 
%%% mode: latex
%%% TeX-master: "main"
%%% End: 
