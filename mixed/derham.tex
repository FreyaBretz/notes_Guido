\begin{intro}
  We can embed finite element methods for the Darcy problem, also for
  the Maxwell problem, into a common framework based on the de Rham
  complex. If we wanted to do this in its full mathematical beauty, we
  would have to spend some time introducing the concept and notation
  of differential forms. As an alternative, we can use the concrete
  vector spaces $\Hdiv(\domain)$ and $\Hcurl(\domain)$. The drawback
  is, that we have to prove deveral particular cases, where the
  abstract theory only knows one common case. Nevertheless, it is
  worthwhile to begin this way, such that the reader has an easier
  task reading the full theory
  in~\cite{ArnoldFalkWinther06acta,ArnoldFalkWinther10}. As a byproduct,
  we will prove in generality some of the properties of polynomial
  spaces in Chapter~\ref{cha:darcy}.
\end{intro}

\begin{intro}
  We now know three differential operators, $\nabla$, $\curl$, and
  $\div$ with the interesting property
  \begin{gather}
    \curl\nabla \phi = 0
    \qquad \div\curl E=0.
  \end{gather}
  As a consequence, for $\phi\in H^1(\domain)$ we not only have
  $\nabla \phi\in L^2(\domain;\R^3)$, we also have
  $\curl\nabla\phi=0\in L^2(\domain;\R^3)$. This gives rise to the sequence
  \begin{gather}
    \R
    \overset{\subset}{\longrightarrow} H^1(\domain)
    \overset{\nabla}{\longrightarrow} \Hcurl(\domain)
    \overset{\curl}{\longrightarrow} \Hdiv(\domain)
    \overset{\div}{\longrightarrow} L^2(\domain)
    \longrightarrow 0,
  \end{gather}
  such that the range of an operator is always in the kernel of the
  operator to its right.
\end{intro}

\begin{Notation}{hlambda}
  The notation of exterior calculus of differential forms allows us to
  write this sequence elegantly as
  \begin{gather}\minCDarrowwidth20pt
    \begin{CD}
      \R
      @>{d}>> H\Lambda^0(\domain)
      @>{d}>> H\Lambda^1(\domain)
      @>{d}>> H\Lambda^2(\domain)
      @>{d}>> H\Lambda^3(\domain)
      @>>> 0
      \\
      @.
      @V{\cong}VV
      @V{\cong}VV
      @V{\cong}VV
      @V{\cong}VV
      \\
      \R
      @>{\subset}>> H^1(\domain)
      @>{\nabla}>> \Hcurl(\domain)
      @>{\curl}>> \Hdiv(\domain)
      @>{\div}>> L^2(\domain)
      @>>> 0,
    \end{CD}
  \end{gather}
  such that $d=d_k\colon H\Lambda^k(\domain) \to H\Lambda^{k+1}(\domain)$ and
  \begin{gather}
    d^2 = d\circ d = 0.
  \end{gather}
\end{Notation}

\begin{remark}
  The spaces $H\Lambda^k(\domain)$ are Hilbert spaces with values in
  the spaces of alternating $k$-forms on $\R^d$. From linear algebra,
  we know that all alternating $k$-forms are zero if $k$ exceeds the
  dimension of the vector space.  Therefore, the sequence above is
  only valid in three dimensions, and it must be shorter by one member
  in two dimensions. Changing our view back to differential operators,
  we realize that there are two relevant sequences in two
  dimensions. In the following diagram, the sequence on top can be
  used to formulate Maxwell problems in $\Hcurl$ in two dimensions,
  while the sequence on the bottom relates to the mixed form of the
  Laplacian.

  We introduce the sequences in two dimensions and afterwards will
  focus our arguments on the more general case of three dimensions
  again. Specialization to two dimensions are straight forward.
\end{remark}

\begin{Notation}{hlambda-2d}
  In two dimensions, we consider the de Rham sequences
  \begin{gather}\minCDarrowwidth20pt
    \begin{CD}
      \R
      @>{\subset}>> H^1(\domain)
      @>{\nabla}>> \Hcurl(\domain)
      @>{\curl}>> L^2(\domain)
      @>>> 0
      \\
      @.
      @A{\cong}AA
      @A{\cong}AA
      @A{\cong}AA
      \\
      \R
      @>{d}>> H\Lambda^0(\domain)
      @>{d}>> H\Lambda^1(\domain)
      @>{d}>> H\Lambda^2(\domain)
      @>>> 0
      \\
      @.
      @V{\cong}VV
      @V{\cong}VV
      @V{\cong}VV
      \\
      \R
      @>{\subset}>> H^1(\domain)
      @>{\curl}>> \Hdiv(\domain)
      @>{\div}>> L^2(\domain)
      @>>> 0,
    \end{CD}
  \end{gather}
\end{Notation}

\begin{Notation}{hlambda-norm}
  The spaces $H\Lambda^k(\domain)$ are Hilbert spaces with the inner product
  \begin{gather}
    \scal(u,v)_{H\Lambda^k} = \scal(u,v)_{L^2} + \scal(d u, d v)_{L^2}.
  \end{gather}
\end{Notation}

\begin{Theorem}{de-rham}
  Assume the domain $\domain$ is Lipschitz.  If $\domain$ is simply
  conntected, the sequences in equations~\eqref{} and~\eqref{} are
  exact, that is, there holds
  \begin{gather}
    \ker {d_{k+1}} = \range{d_k}.
  \end{gather}
  If it is not simply connected, the codimension of $\range{d_k}$ in
  $\ker{d_{k+1}}$ is finite. In particular, in both cases,
  $\range{d_k}$ is closed in $H\Lambda^{k+1}(\domain)$.
\end{Theorem}

\begin{Lemma}{hlambda-0}
  The bounded Hilbert cochain complex
  \begin{gather}\minCDarrowwidth20pt
    \begin{CD}
      \R
      @>{d}>> H\Lambda^0_0(\domain)
      @>{d}>> H\Lambda^1_0(\domain)
      @>{d}>> H\Lambda^2_0(\domain)
      @>{d}>> H\Lambda^3_0(\domain)
      @>>> 0
      \\
      @.
      @V{\cong}VV
      @V{\cong}VV
      @V{\cong}VV
      @V{\cong}VV
      \\
      \R
      @>{\subset}>> H^1_0(\domain)
      @>{\nabla}>> \Hcurl_0(\domain)
      @>{\curl}>> \Hdiv_0(\domain)
      @>{\div}>> L^2_0(\domain)
      @>>> 0,
    \end{CD}
  \end{gather}
  has the same properties as stated for the Hilbert complex without
  boundary conditions.
\end{Lemma}

\subsection{A polynomial complex}

\begin{intro}
  We have already seen that adding $x\P_k$ to the space $\P_k^d$, we
  obtain a surjective divergence operator from the Raviart-Thomas
  element to the pressure space $\P_k$. In this section, we see that
  there is a general principle behind this concept and it can be
  extended to the curl and gradient operators.
\end{intro}



\subsection{The complex of tensor products}

%%% Local Variables: 
%%% mode: latex
%%% TeX-master: "main"
%%% End: 
