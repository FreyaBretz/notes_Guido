\begin{intro}
  In the previous chapter, we studied discretizations with
  $\div V_h = Q_h$ with two advantages. First, due to
  \slideref{Corollary}{galerkin-mixed-u-kerb} the velocity error is
  independent of the pressure. Second, the divergence converges faster
  than the gradient. A natural question arising is whether we can do
  something similar for the Stokes problem. There, the equation
  \begin{gather}
    \form(\div v_h, q_h) = 0 \qquad\forall q_h\in Q_h,
  \end{gather}
  would immediately imply $\div v_h=0$, that is, the discrete solution
  is exactly divergence free.
  
  The answer to this question is a current research topic. So far,
  beginning with the element by Scott and Vogelius, several methods
  have been proposed for special mesh geometry or macro meshes. The
  difficulty is balancing the condition $\div V_h = Q_h$ with the
  $H^1$-conformity of the velocity space. All the spaces in the
  previous chapter were only $\Hdiv$-conforming with discontinuous
  tangential components.
  
  A fairly simple solution to this question though can be obtained by
  using discontinuous Galerkin methods. These were introduced to
  obtain formulations \emph{consistent} with $H^1$ while not
  \emph{conforming}. Thus, we can apply them directly to
  Raviart-Thomas and Brezzi-Douglas-Marini elements to obtain a
  consistent method with divergence free solutions.

  We begin this chapter by a quick review of the interior penalty
  method before diving into divergence conforming methods.
\end{intro}

\section{The interior penalty method}

% \begin{intro}
%   In this section we extend the weakening of continuity, which we
%   explored for boundary values in Section~\ref{sec:nitsches-method}
%   using Nitsche's method to interior interfaces between mesh
%   cells. While the methods obtained may look much more complicated,
%   the mathematical analysis is completely analogue to that
%   section. Thus, we can be fairly brief.
% \end{intro}

\begin{intro}
  We review the basic definitions necessary to describe discontinuous
  Galerkin (DG) methods. In particular, we need the sets of faces
  $\F_h$ of a mesh, discontinuous piecewise polynomial spaces and
  broken integrals.
\end{intro}

\begin{Definition}{dg-faces}
  Let $\T_h$ be a mesh of $\Omega \subset \R^d$ consisting of mesh
  cells $T_i$. For every boundary facet $F\subset \partial T_i$, we
  assume\footnote{This assumption can indeed be relaxed} that either
  $F \subset \partial \Omega$ or $F$ is a boundary facet of another
  cell $T_j$. In the second case, we indicate this relation by
  labeling this facet $F_{ij}$. The set of all facets $F_{ij}$ is the
  set of interior faces $\F_h^i$. The set of facets on the boundary is
  $\F_h^\partial$.
\end{Definition}

\begin{Definition}{dg-spaces}
  The discontinuous finite element space on $\T_h$ is constructed by
  concatenation of all shape function spaces $P_T$ for $T\in \T_h$
  without additional continuity requirements:
  \begin{gather}
    V_h = \bigl\{v\in L^2(\Omega) \big|
    v_{|T} \in P_T \;\forall T\in \T_h\bigr\}.
  \end{gather}
\end{Definition}

\begin{Definition}{broken-integrals}
  For any set of cells $\mesh_h$ or faces $\faces_h$, we define the bilinear
  forms
  \begin{align}
    \form(u,v)_{\mesh_h} &= \sum_{\cell\in\mesh_h} \form(u,v)_\cell, \\
    \forme(u,v)_{\faces_h} &= \sum_{\face\in\faces_h} \forme(u,v)_\face. \\
  \end{align}
\end{Definition}

\begin{intro}
  We start out with the equation
  \begin{gather*}
    -\Delta u = f.
  \end{gather*}
  Integrating by parts on each mesh cell yields
  \begin{gather*}
    \form(-\Delta u,v )_\cell
    = \form(\nabla u, \nabla v)_\cell - \forme(\d_n u, v)_{\d\cell} = \form(f,v)_T.
  \end{gather*}
  We realize that the choice of discontinuous finite element spaces
  introduces a consistency term on the interfaces between cells and on
  the boundary.

  On interior faces, there is the issue that $u$ and
  $\d_n u$ actually have two values on the interface, one from the
  left cell and one from the right. Therefore, we have to consolidate
  these two values into one. To this end, we introduce the concept of
  a numerical flux, which constructs a single value out of these
  two. Thus, we introduce on the interface $\face$ between two cells
  $\cell^+$ and $\cell^-$
  \begin{gather*}
    \mathcal F(\nabla u) = \frac{\nabla u^+ + \nabla u^-}{2} = :
    \mvl{\nabla u}.
  \end{gather*}
  
  Using $\forme(\d_n u,v) = \forme(\nabla u,v\n)$ we change our point
  of view and instead of integrating over the boundary $\d\cell$, we
  integrate over a face $\face$ between two cells $\cell^+$ and
  $\cell^-$. Adding up integrals from both sides, we obtain the term
  \begin{gather*}
    -\forme(\mvl{\nabla u},v^+\n^+ +v^-\n^-)_{\face}
    = -2\forme(\mvl{\nabla u},\mvl{v\n})_{\face}.
  \end{gather*}
  On boundary faces, we simply get
  \begin{gather*}
    \forme(\d_\n u,v)_{\face}.
  \end{gather*}
  
  Adding over all cells and faces, we obtain the equation
  \begin{gather*}
    \form(\nabla u,\nabla v)_{\T_h}
    -2\forme(\mvl{\nabla u},\mvl{v\n})_{\F_h^i}
    -\forme(\d_\n u,v)_{\F_h^\d} = \form(f,v)_{\domain}.
  \end{gather*}

  Following the idea of Nitsche, we symmetrize this term
  to obtain
  \begin{multline*}
    \form(\nabla u,\nabla v)_{\T_h}
    -2\forme(\mvl{\nabla u},\mvl{v\n})_{\F_h^i}
    -2\forme(\mvl{u\n},\mvl{\nabla v})_{\F_h^i}
    \\
    -\forme(\d_\n u,v)_{\F_h^\d}
    -\forme(u,\d_\n v)_{\F_h^\d}
    = \form(f,v)_{\domain}
    - \forme(u^o,\d_n v)_{\F_h^\d}.
  \end{multline*}
  Here the second term on the right was introduced for consistency.
  Finally, it turns out that this method is not stable and nees
  stabilization by a jumpt term. This will be done in
  \blockref{Definition}{ip}. Before, we introduce the notation for
  averaging and jump operators.
\end{intro}

\begin{Notation}{dg-operators}
  Let $\face$ be a face between the cells $\cell^+$ and $\cell^-$. Let
  $\n^+$ and $\n^-=-\n^+$ be the outer normal vectors of the cells at a
  point $x\in \face$. For a function $u\in V_h$, the traces $u^+$ and
  $u_-$ of $u$ on $\face$ taken from the cell $\cell^+$
  and $\cell^-$ are defined as:
  \begin{align*}
    u^+(x) &= \lim_{\epsilon\searrow 0} u(x-\epsilon\n^+), \\
    u^-(x) &= \lim_{\epsilon\searrow 0} u(x-\epsilon\n^-).
  \end{align*}
  We define the \define{averaging operator} $\mvl{.}$ and the
  \define{jump operator} $\jmp{.}$ as
  \begin{gather}
    \label{eq:ip:1}
    \mvl{u} = \frac{u^++u^-}{2},
    \qquad
    \jmp{u} = u^+-u^-.
  \end{gather}
  Not that the sign of the jump of $u$ depends on the choice of the
  cells $\cell^+$ and $\cell^-$. It will only be used in quadratic
  terms.
\end{Notation}

\begin{remark}
  The jump can be denoted as the mean value of the product of a
  function and the normal vector,
  \begin{gather}
    \jmp{u} = 2\mvl{u\n}\cdot\n^+ = -2\mvl{u\n}\cdot\n^-.
  \end{gather}
\end{remark}

\begin{Definition}{ip}
  The \define{interior penalty method}\footnote{Also known as
    symmetric interior penalty (SIPG) or IP-DG.} uses the bilinear
  form
  \begin{multline}
    \label{eq:ip:2}
    a_h(u,v) = \form(\nabla u,\nabla v)_{\mesh_h}
    + \forme(\ipp_h\jmp{u},\jmp{v})_{\faces_h^-}
    + \forme(\ipp_h u,v)_{\faces_h^\d}
    \\
    -2\forme(\mvl{\nabla u},\mvl{v\n})_{\faces_h^i}
    -2\forme(\mvl{u\n},\mvl{\nabla v})_{\faces_h^i}
    \\
    - \forme(\d_n u,v)_{\faces_h^\d}
    - \forme(u,\d_n v)_{\faces_h^\d},
  \end{multline}
  and the linear form
  \begin{gather}
    \label{eq:ip:3}
    f_h(v) = \form(f,v)_{\domain} - \forme(u^D,\d_n v)_{\faces_h^\d}
    + \forme(\ipp_h u,v)_{\faces_h^\d},
  \end{gather}
  where $f$ is the right hand side of the equation and $u^D$ the
  Dirichlet boundary value.
\end{Definition}

\begin{Definition}{ip-norm}
  On the space $V_h$ we define the norm $\norm{.}_{1,h}$ by
  \begin{gather}
    \label{eq:ip:4}
    \norm{v}_{1,h}^2 = \sum_{\cell\in\mesh_h} \norm{\nabla v}_\cell^2
    + \sum_{\face\in\faces_h^i} \norm{\sigma_h\jmp{v}}_\face^2
    + \sum_{\face\in\faces_h^\d} \norm{\sigma_hv}_\face^2.
  \end{gather}
\end{Definition}

\begin{Problem}{ip-norm}
  Prove that the norm defined in (\ref{eq:ip:4}) is indeed a norm on $V_h$.
\end{Problem}

\begin{Lemma}{ip-stability}
  Let $\T_h$ be shape-regular and chosen on each face $\face$ as
  $\sigma_h = \sigma_0/h_\face$, where $h_T$ is the minimal diameter
  of a cell adjacent to $\face$. Then, there is a $\sigma_0>0$ such
  that there exists a constant $\ellipa>0$, such that independent of
  $h$ there holds
  \begin{gather}
    \label{eq:ip:5}
    a_h(u_h,u_h) \ge \ellipa \norm{u_h}_{1,h}^2 \quad \forall u_h\in V_h.
  \end{gather}
\end{Lemma}

\begin{Problem}{ip-stability}
  Prove \blockref{Lemma}{ip-stability}.
\end{Problem}

\begin{Lemma}{ip-consistence}
  Let $f\in L^2(\domain)$ and let the boundary conditions admit that
  for the solution to
  \begin{xalignat*}2
    -\Delta u &= f &\text{in }&\domain, \\
    u &= u^D &\text{on }&\d\domain,
  \end{xalignat*}
  there holds $u\in H^{1+\epsilon}(\domain)$ for a positive
  $\epsilon$. Then, the interior penalty method is consistent, that
  is,
  \begin{gather}
    a_h(u,v_h) = f_h(v_h)\quad\forall v_h\in V_h.
  \end{gather}
\end{Lemma}

\begin{proof}
  From $f\in L^2(\domain)$ we deduce that
  $\nabla u\in \Hdiv(\domain)$. Thus, with the extra regularity, the
  traces of $\d_n u$ on faces are well-defined abd coincide from both
  sides. The remainder is integration by parts.
\end{proof}

\begin{Theorem}{ip-convergence}
  For $k\ge 1$ let $\P_k\subset P_\cell$ and $u\in H^{s+1}(\domain)$ with
  $1/2 \le s \le k$. Then, the interior penalty method admits the
  error estimate
  \begin{gather}
    \norm{u-u_h}_{1,h} \le c h^s \snorm{u}_{s+1}.
  \end{gather}
  If furthermore hte boundary condition admits \putindex{elliptic
    regularity},
there holds
  \begin{gather}
    \norm{u-u_h}_{0} \le c h^{s+1} \snorm{u}_{s+1}.
  \end{gather}
\end{Theorem}

%%% Local Variables:
%%% mode: latex
%%% TeX-master: "main"
%%% End:





\subsection{Bounded formulation in $H^1$}
\begin{intro}
  The interior penalty method introduced so far is $V_h$-elliptic and
  consistent, but it is not bounded on $H^1(\domain)$. This was a
  reason, why we could not use standard techniques for the proof of
  the convergence result and after applying consistency had to
  estimate each term separately.

  In this section, we will introduce a reformulation of the interior
  penalty method, which is equivalent to the original method on $V_h$,
  but is also bounded in $H^1(\domain)$. As an unpleasant side effect,
  it turns out that this method is inconsistent, and we have to
  estimate the consistency error.
  
  The main technique applied here is the use of lifting operators,
  such that the traces of derivatives on faces can be replaced by
  volume terms. Note that the lifting operators, while very useful for
  the analysis of the method, are not actually used in the
  implementation of the interior penalty method.
\end{intro}

\begin{Definition}{dg-lifting}
  Define the auxiliary space 
  \begin{gather}
    \label{eq:ip-lifting:1}
    \Sigma_h = \bigl\{ \tau\in L^2(\domain;\R^d) \big\vert
    \,\forall \cell\in \mesh_h: \tau_{|\cell} \in \Sigma_T \bigr\},
  \end{gather}
  where $\Sigma_T$ is a (possibly mapped) polynomial space chosen such that
  $\nabla V_T \subset \Sigma_T$. Then, we define the \define{lifting
    operator}
  \begin{gather}
    \label{eq:ip-lifting:2}
    \lifting\colon V+V_h \to \Sigma_h
  \end{gather}
  by
  \begin{gather}
    \label{eq:ip-lifting:3}
    \form(\lifting v,\tau)_{\mesh_h}
    = 2\forme(\mvl{\tau},\mvl{v\n})_{\faces_h^i}
    + \forme(\tau\cdot\n,v)_{\faces_h^\d}.
  \end{gather}
\end{Definition}

\begin{Lemma}{ip-lifting-bounded}
  The lifting operator is a bounded operator from $L^2(\faces_h)$ to
  $\Sigma_h$, such that
  \begin{gather}
    \label{eq:ip-lifting:4}
    \norm{\lifting v}_{L^2(\domain)}
    \le c \norm*{\tfrac1{\sqrt{h}}\jmp{v}}_{\faces_h^i}
    + \norm*{\tfrac1{\sqrt{h}} v}_{\faces_h^\d}.
  \end{gather}
  In particular, it is bounded on $H^1(\domain)$.
\end{Lemma}

\begin{proof}
  It is clear, that the operator is bounded on $L^2(\faces_h)$, since
  its definition involves face integrals weighted with polynomial
  functions. The dependence on the mesh size is due to the standard
  scaling argument.
\end{proof}

\begin{Definition}{ip-lifting}
  The \define{interior penalty method} with lifting operators uses the
  bilinear form
  \begin{multline}
    \label{eq:ip-lifting:5}
    a_h(u,v) = \form(\nabla u,\nabla v)_{\mesh_h}
    - \form(\lifting u, \nabla v)_{\mesh_h}
    - \form(\nabla u, \lifting v)_{\mesh_h}
    \\
    + \forme(\ipp_h\jmp{u},\jmp{v})_{\faces_h^-}
    + \forme(\ipp_h u,v)_{\faces_h^\d}
    .
  \end{multline}
  and the linear form~\eqref{eq:ip:3} of the original interior penalty
  method. Its residual operator is
  \begin{gather}
    \label{eq:ip-lifting:7}
    \Res(u,v) = a_h(u,v) - \form(f,v).
  \end{gather}
\end{Definition}

\begin{Lemma}{ip-equivalence}
  The interior penalty method in flux form (\blockref{Definition}{ip})
  and in lifting form (\blockref{Definition}{ip}) coincide on the
  discrete space $V_h$ if $\Sigma_h$ is chosen such that $\nabla V_h
  \subset \Sigma_h$.
\end{Lemma}

\begin{proof}
  Since $\nabla V_h \subset \Sigma_h$, $\nabla u_h$ and $\nabla v_h$
  are valid test functions in the definition~\eqref{eq:ip-lifting:3}
  of the lifting operator, and  the equality
  \begin{gather*}
    \form(\lifting{u_h},\nabla v_h)_{\mesh_h}
    = 2\forme(\mvl{u_h\n},\mvl{\nabla v_h})_{\faces_h^i}
    + \forme(u_h,\d_n v_h)_{\faces_h^\d}.
  \end{gather*}
\end{proof}

\begin{Definition}{ip-residual}
  Let $V\subset H^1(\domain)$ and let $u,u^*\in V$ solve the primal
  and dual problems
  \begin{gather}
    a(u,v) = f(v),
    \qquad
    a(v,u^*) = \psi(v),
    \qquad
    \forall v\in V,
  \end{gather}
  with a bounded, $V$-elliptic bilinear form $a(.,.)$. For a discrete
  bilinear form $a_h(.,.)$ defined on $V+V_h$, we define the primal
  and dual \define{residual operator}s
  \begin{gather}
    \label{eq:ip-lifting:8}
    \begin{split}
      \Res(u,v) &= a_h(u,v) - f(v), \\
      \Res^*(u^*,v) &= a_h(v,u^*) - \psi(v).
    \end{split}
  \end{gather}
\end{Definition}

% This IS Strang's second lemma!
\begin{Lemma}{ip-lifting-strang}
  Let $a_h(.,.)$ be a bounded bilinear form on $V+V_h$ and elliptic on
  $V_h$ with norm $\norm{.}_{V_h}$ and constant $\ellipa$. Then, the error
  $u-u_h$ admits the estimate
  \begin{gather}
    \label{eq:ip-lifting:9}
    \norm{u-u_h}_{V_h} \le \frac1{\ellipa}
    \norm{\Res(u,.)}_{V_h^*}
    + \left(1+\frac{\norm{a_h}}{\ellipa}\right)
    \inf_{w_h\in V_h}\norm{u-w_h}
  \end{gather}
\end{Lemma}

\begin{proof}
  First, by the definition of the residual, we have the error equation
  \begin{gather}
    a_h(u-u_h, v_h) = \Res(u,v_h),\qquad\forall v_h\in V_h.
  \end{gather}
  Inserting $w_h-w_h$ for an arbitrary element $w_h\in V_h$, we obtain
  \begin{gather*}
    a_h(w_h-u_h, v_h) = \Res(u,v_h) - a_h(u-w_h, v_h),\qquad\forall v_h\in V_h.
  \end{gather*}
  Using $v_h = w_h-u_h$ and ellipticity, we obtain
  \begin{align*}
    \ellipa \norm{w_h-u_h}_{V_h}^2
    &\le a_h(w_h-u_h,w_h-u_h)\\
    & = \Res(u,w_h-u_h) - a_h(u-w_h, w_h-u_h)\\
    & \le \bigl(\norm{\Res(u,.)}_{V_h^*} + \norm{a_h} \norm{u-w_h}_{V_h}\bigr)
      \norm{w_h-u_h}_{V_h}.
  \end{align*}
  Hence, by triangle inequality
  \begin{gather*}
    \norm{u-u_h}_{V_h} \le \frac1{\ellipa} \norm{\Res(u,.)}_{V_h^*}
    + \left(1+\frac{\norm{a_h}}{\ellipa}\right)
    \inf_{w_h\in V_h}\norm{u-w_h}_{V_h}
  \end{gather*}
\end{proof}

% From Girault/Kanschat/Riviere

\begin{Lemma}{ip-lifting-residual-1}
  Let $u\in V$ be the solution to the Poisson equation
  with right hand side $f\in L^2(\domain)$. Assume $u\in H^s(\domain)$
  with $s>3/2$. Then, we have for $v\in V+V_h$:
  \begin{gather}
    \label{eq:hdivdg:11}
    \form(f,v) = \form(\nabla u,\nabla v)_{\mesh_h}
    - 2\forme(\nabla u,\mvl{v \n})_{\faces_h^i}
    -\forme(\d_n u,v)_{\faces_h^\d}.
  \end{gather}
\end{Lemma}

\begin{proof}
  We set out from the strong form of the Poisson equation and
  integrate by parts.
  \begin{gather*}
    \form(f,v) = \form(-\Delta u, v)
    = \form(\nabla u,\nabla v)_{\mesh_h}
    - \sum_{\cell\in\mesh_h} \forme(\d_n u,v)_{\d\cell}
    .
  \end{gather*}
  Under the regularity assumptions of the lemma, all of these
  integrals make sense at least as duality pairings. In particular,
  $\d_n u\in L^2(\d\cell)$, and thus we can split $\d\cell$ into
  individual faces. Therefore,
  \begin{gather*}
    \sum_{\cell\in\mesh_h} \forme(\d_n u,v)_{\d\cell}
    = 2\forme(\nabla u,\mvl{v\otimes n})_{\faces_h^i}
    + \forme(\d_n u,v)_{\faces_h^\d}.
  \end{gather*}
  The proof concludes by collecting the results.
\end{proof}

\begin{Lemma}{ip-lifting-residual-2}
  Let $k\ge 1$ and let $V_h$ such that $\P_{k-1} \subset \Sigma_T$. Then, if
  $u\in H^{k+1}(\domain)$ and $v\in V+V_h$, there holds
  \begin{gather}
    \label{eq:ip-lifting:10}
    \begin{split}
    \abs{\Res(u,v)}
    &\le c h^{k} \snorm{u}_{k+1}
    \bigl(
    \norm{\sqrt{\ipp_h}\jmp{v}}_{\faces_h^i}
    +
    \norm{\sqrt{\ipp_h}v}_{\faces_h^\d}\bigr)
    \\
    &\le c h^{k} \snorm{u}_{k+1} \norm{v}_{1,h}.
    \end{split}
  \end{gather}
  % Furthermore, for $v_h\in V_h$, there holds
  % \begin{gather}
  %   \label{eq:ip-lifting:11}
  %   \Res(u,v_h) = 0.
  % \end{gather}
\end{Lemma}

\begin{proof}
  First, we observe that by the regularity assumption, $\jmp{u} = 0$
  and thus, $\lifting u=0$. Hence,
  \begin{gather*}
    a_h(u,v) = \form(\nabla u, \nabla v)_{\mesh_h}
    - \form(\nabla u,\lifting v)_{\mesh_h}.
  \end{gather*}
  By \blockref{Lemma}{ip-lifting-residual-1} and regularity of $u$,
  \begin{align*}
    \Res(u,v)
    &= 2\forme(\nabla u,\mvl{v\n})_{\faces_h^i}
      + \forme(\d_n u,v)_{\faces_h^\d}
      - \form(\nabla u,\lifting v)_{\mesh_h}
    \\
    &= 2\forme(\mvl{\nabla u},\mvl{v\n})_{\faces_h^i}
      + \forme(\d_n u,v)_{\faces_h^\d}
      - \form(\nabla u,\lifting v)_{\mesh_h}
    \\
    &= 2\forme(\mvl{\nabla u},\mvl{v\n})_{\faces_h^i}
      + \forme(\d_n u,v)_{\faces_h^\d}
      - \form(\Pi_{\Sigma_h} \nabla u,\lifting v)_{\mesh_h},
  \end{align*}
  where $\Pi_{\Sigma_h}$ is the $L^2$-projection. Now, we can apply
  the definition of the lifting term to obtain
  \begin{multline*}
    \Res(u,v) = 2\forme(\tfrac1{\ipp_h}\mvl{\nabla u -
      \Pi_{\Sigma_h}\nabla u}, \ipp_h\mvl{v\n})_{\faces_h^i}
    \\
    + \forme(\tfrac1{\ipp_h} (\nabla u -\Pi_{\Sigma_h} \nabla
    u)\cdot\n, \ipp_h v)_{\faces_h^\d}.
  \end{multline*}
  Application of standard approximation and trace estimates yields the
  result observing that $\ipp_h = \ipp_0/h$.
\end{proof}

\begin{Theorem}{ip-lifting-h1}
  Let $k\ge 1$ and $V_h$ such that $\P_k \subset V_\cell$. Let
  $u\in H^{k+1}(\domain)$ be the solution to the continuous Poisson
  problem. Let $a_h(.,.)$ be the interior penalty method with lifting
  operators such that $\nabla V_h\subset\Sigma_h$. Then, there holds
  \begin{gather}
    \norm{u-u_h}_{1,h} \le c h^{k} \snorm{u}_{k+1}.
  \end{gather}
\end{Theorem}

\begin{proof}
  Application of \blockref{Lemma}{ip-lifting-strang},
  \blockref{Lemma}{ip-lifting-residual-2}, and standard interpolation
  results.
\end{proof}

\begin{Theorem}{ip-lifting-l2}
  Let the assumptions of \blockref{Theorem}{ip-lifting-h1} hold and in
  addition assume that the problem
  \begin{gather*}
    a(v,u^*) = \psi(v),\qquad\forall v\in V,
  \end{gather*}
  admits the \putindex{elliptic regularity} estimate
  \begin{gather}
    \label{eq:ip-lifting:12}
    \norm{u^*}_{H^2(\domain)} \le c \norm{\psi}_{L^2(\domain)}.
  \end{gather}
  Then, there holds
  \begin{gather}
    \label{eq:ip-lifting:13}
    \norm{u-u_h}_{L^2(\domain)} \le c h^{k+1} \snorm{u}_{H^{k+1}(\domain)}.
  \end{gather}
\end{Theorem}

\begin{proof}
  The proof uses the duality argument by Aubin and Nitsche, which sets
  out solving the auxiliary problem
  \begin{gather*}
    a(v,u^*) = \form(u-u_h,v),\qquad\forall v\in V.
  \end{gather*}
  Using the definition of the dual residual, we obtain the equation
  \begin{gather*}
    \form(u-u_h, v) = a_h(v,u^*) - \Res^*(u^*,v),\qquad\forall v\in V+V_h.
  \end{gather*}
  Testing with $v=u-u_h$ yields
  \begin{gather*}
    \norm{u-u_h}^2 = a_h(u-u_h,u^*) - \Res^*(u^*,u-u^*).
  \end{gather*}
  Additionally, we us the error equation
  \begin{gather*}
    a_h(u-u_h, v_h) = \Res(u,v_h),
  \end{gather*}
  tested with $v_h = I_h u^*$, to obtain
  \begin{multline*}
    \norm{u-u_h}^2 = a_h(u-u_h, u^*-I_h u^*) - \Res^*(u^*,u-u_h)
    + \Res(u,I_h u^*).
  \end{multline*}
  Using the regularity of $u^*$, the first term on the right
  admits the estimate
  \begin{gather*}
    \abs{a_h(u-u_h, u^*-I_h u^*)}
    \le \norm{u-u_h}_{1,h}\norm{u^*-I_hu^*}_{1,h}
    \le c h \norm{u-u_h}_{1,h}.
  \end{gather*}
  For the second term, we use \blockref{Lemma}{ip-lifting-residual-2}
  to obtain
  \begin{gather*}
    \abs{\Res^*(u^*,u-u_h)} \le c h \snorm{u^*}_2 \norm{u-u_h}_{1,h}.
  \end{gather*}
  Finally, using $\jmp{u^*} = 0$, the same lemma yields
  \begin{align*}
    \abs{\Res(u, I_h u^*)}
    &\le c h \snorm{u}_2
      \bigl(\norm{\sqrt{\ipp_h}\jmp{I_h u^*}}_{\faces_h^i}
      + \norm{\sqrt{\ipp_h}I_h u^*}_{\faces_h^\d}\bigr)
    \\
    & = c h \snorm{u}_2
      \bigl(\norm{\sqrt{\ipp_h}\jmp{u^*-I_h u^*}}_{\faces_h^i}
      + \norm{\sqrt{\ipp_h}(u^*-I_h u^*)}_{\faces_h^\d}\bigr)
    \\
    & \le c h \snorm{u}_2 h^k \snorm{u^*}_{k+1}
  \end{align*}
  Using the energy estimate in \blockref{Theorem}{ip-lifting-h1} we
  can conclude the prove.
\end{proof}

%%% Local Variables:
%%% mode: latex
%%% TeX-master: "main"
%%% End:


\section{Divergence conforming IP}
\begin{remark}
  The extension of the interior penalty method to vector-valued
  problems is obvious. Furthermore, since the method generates an
  elliptic bilinear form on the discontinuous space $V_h$, this
  ellipticity is inherited by any subspace of
  $V_h\cap\Hdiv(\domain)$. Thus, we can write down the weak
  formulation of a divergence conforming DG method for the Stokes
  equations. In the following definition, we assume slip or no-slip
  boundary conditions, that is, $v\cdot\n=0$ on the whole boundary.
\end{remark}

\begin{Definition}{hdiv-ip}
  A divergence conforming DG method for the Stokes equations consists
  of a discrete velocity space $V_h\subset \Hdiv_0(\domain)$ and a
  pressure space $Q_h\subset L^2_0(\domain)$ such that
  \begin{gather}
    \label{eq:hdivdg:1}
    \div V_h = Q_h.
  \end{gather}
  Using the interior penalty bilinear form $a_h(.,.)$, we search for
  solutions $(u_h,p_h)\in V_h\times Q_h$ such that for all $(v,q)\in
  V_h\times Q_h$ there holds
  \begin{gather}
    \label{eq:hdivdg:2}
    a_h(u_h,v) +\form(\div v,p_h)+\form(\div u_h,q) = f(v).
  \end{gather}
\end{Definition}

\begin{remark}
  Due to the fact that $V_h\not\subset V$, we have introduced the  norm
  $\norm{.}_{1,h}$ on $V_h$. In particular, the norm $\norm{.}_1$ is
  not defined for all elements of $V_h$. Therefore, we need a
  modification of Fortin's lemma (\slideref{Lemma}{fortin}), where the
  norm on the left hand side of the stability
  estimate~\eqref{eq:galerkin:16} uses the discrete norm, namely,
  \begin{gather}
    \norm{\Pi_{V_h}v}_{V_h} \le c \norm{v}_V,
  \end{gather}
\end{remark}

\begin{Lemma}{dg-fortin}
  Let $\{\mesh_h\}$ be a shape-regular sequence of meshes. Then, the
  \putindex{canonical interpolation} operators of the
  Brezzi-Douglas-Marini and Raviart-Thomas elements admit the bound
  \begin{gather}
    \label{eq:hdivdg:4}
    \norm{I_h v}_{1,h} \le c \snorm{v}_1
  \end{gather}
\end{Lemma}

\begin{proof}
  First, we note that all degrees of freedom are defined as cell or
  face integrals with smooth weight functions. Thus, they are bounded
  on $H^1$. Thus, since the local polynomial spaces are finite
  dimensional, there holds on the reference cell $\refcell$ and its
  faces $\refface$:
  \begin{align}
    \norm{I_{\refcell} v}_{1;\refcell} &\le c \snorm{v}_{1;\refcell},
    \\
    \norm{I_{\refcell} v}_{0;\refface} &\le c \snorm{v}_{1;\refcell}.   
  \end{align}
  On shape regular meshes, we have the scaling property
  \begin{align}
    \snorm{f}_{m;\cell} &\simeq h_\cell^{\frac{d}{2}-m},
    \\
    \snorm{f}_{m;\face} &\simeq h_\face^{\frac{d-1}{2}-m},
  \end{align}
  such that for a face $\face$ of cell $\cell$
  \begin{align}
    \norm{I_{\cell} v}_{1;\cell} & \le c \snorm{v}_{1;\cell},
    \\
    \norm{I_{\cell} v}_{0;\refface} &\le c h^{\frac12} \snorm{v}_{1;\cell}.
  \end{align}
  We conclude
  \begin{gather}
    \norm{I_h v}_{1,h}^2 \le \sum_{\cell\in\mesh_h}
    \biggl[\norm{I_{\cell} v}_{1;\cell}^2
    + 4 \sum_{\face\subset\d\cell}
    \norm*{\tfrac{\ipp_0}{h_F} I_{\cell} v}_{0;\face}^2
    \biggr] \le c \snorm{v}_{1}^2.
  \end{gather}
\end{proof}

\begin{Corollary}{hdivdg-infsup}
  Assume that the inf-sup condition~\eqref{eq:stokes:1} in
  \slideref{Theorem}{stokes-infsup} holds.
  Then, the method in \slideref{Definition}{hdiv-ip}
  admits the inf-sup condition
  \begin{gather}
    \label{eq:hdivdg:3}
    \inf_{q_h\in Q_h} \sup_{v_h\in V_h}
    \frac{\form(\div v_h,q_h)}{\norm{v_h}_{1,h}\norm{q_h}_0} \ge \beta,
  \end{gather}
  with a constant $\beta >0$ independent of $h$.
\end{Corollary}

\begin{proof}
  First, we make use of the fact that $q_h\in Q_h \subset Q$ to deduce
  from \slideref{Theorem}{stokes-infsup} the there is a function $w\in
  V$ with $\div v=q_h$ and $\norm{v}_1 \le \norm{q_h}_0$. To this
  function, we apply the \putindex{Fortin operator} to define $v_h =
  I_h v$. By the preceding lemma, we have
  \begin{gather}
    \norm{v_h}_{1,h} \le c \norm{v}_1 \le \norm{q_h}_0,
  \end{gather}
  which proves the inf-sup condition.
\end{proof}

\begin{Theorem}{hdivdg-convergence}
  Assume that $(u_h,p_h)\in V_h\times Q_h$ is the solution to the
  divergence conforming DG method in \slideref{Definition}{hdiv-ip} and
  that the continuous Stokes problem is well-posed as in
  \slideref{Theorem}{stokes-infsup}. Then, for the Raviart-Thomas
  pairs $RT_k/\P_k$ and $RT_{[k]}/\Q_k$ with $k\ge 1$ and
  $u$ sufficiently smooth there holds
  \begin{align}
    \label{eq:hdivdg:5}
    \norm{u-u_h}_{1,h} &\le h^k \snorm{u}_{k+1}, \\
    \label{eq:hdivdg:7}
    \norm{p-p_h}_{0} &\le h^k \bigl(\snorm{u}_{k+1} + \snorm{p}_{k}\bigr).
  \end{align}
  Furthermore,
  \begin{gather}
    \label{eq:hdivdg:6}
    \div u_h = 0.
  \end{gather}
\end{Theorem}

\begin{proof}
  The proof follows the lines of the abstract theory of
  \slideref{Theorem}{galerkin-mixed-u-kerbh} and
  \slideref{Theorem}{galerkin-mixed-p}. But since the setting with
  $V_h\not\subset V$ exceeds the assumptions of the abstract theory,
  we adapt the proofs instead of using the results.

  Due to consistency of the method, we have
  \begin{gather}
    a_h(u-u_h, v_h) + \form(\div v_h, p-p_h)
    + \form(\div u-\div u_h, q_h) = 0.
  \end{gather}
  Testing with $v_h=0$ and using $\div V_h = Q_h$ immediately yields
  $\div u_h = \div u = 0$, or
  \begin{gather}
    \ker{B_h} \subset \ker B.
  \end{gather}
  In order to use the ellipticity of $a_h(.,.)$, we insert
  arbitrary functions $w_h \in \ker{B_h}$ and $r_h\in Q_h$. Choosing
  $q_h = 0$ yields the error equation
  \begin{gather}
    \label{eq:hdivdg:8}
    a_h(u_h-w_h, v_h) + \form(\div v_h, p_h-r_h)
    = a_h(u-w_h, v_h) + \form(\div v_h, p-r_h).
  \end{gather}
  Testing with $v_h=u_h-w_h$ and employing $\div v_h=0$, we obtain
  \begin{gather}
    \ellipa \norm{u_h-w_h}_{1,h}^2
    \le a_h(u_h-w_h,u_h-w_h)
    = a_h(u-w_h,u_h-w_h).
  \end{gather}
  Now, we use the canonical interpolation $w_h = I_h u$ to obtain
  \begin{gather}
    \ellipa \norm{u_h-w_h}_{1,h}^2
    \le \frac\ellipa2 \norm{u_h-w_h}_{1,h}^2
    + \frac{c}{2\ellipa} h^{2k} \snorm{u}_{k+1}^2.
  \end{gather}

  Finally, we use the inf-sup condition to find a test function
  $v_h\in V_h$ such that $\div v_h = p_h-r_h$ and $\beta \norm{v_h}_{1,h}
  \le \norm{p_h-r_h}$. Then, the error equation~\eqref{eq:hdivdg:8}
  yields
  \begin{multline}
    \norm{p_h-r_h}
    = \frac{\form(\div v_h,p_h-r_h)}{\norm{p_h-r_h}}
    \\
    = \frac{a_h(u-u_h, v_h) + \form(\div v_h, p-r_h)}
    {\norm{p_h-r_h}}
    \le \tfrac{\norm{a_h}}\beta \norm{u-u_h}_{1,h}
    + \norm{p-r_h}_0
    .
  \end{multline}
  Using the previously proven error estimate for $u_h$ and the
  $L^2$-projection $r_h = \Pi_h p$ yields the result.
\end{proof}

\section{Error estimates by duality}

\begin{intro}
  So far, we have only considered estimates in the so called
  \putindex{energy norm}, that is, a norm such that $a_h(.,.)$ is
  bounded and elliptic\footnote{We use the term energy norm loosely
    here. Strictly speaking, the energy norm would be
    $\norm{v}_A = \sqrt{a_h(v,v)}$.}.

  In the context of elliptic equations, we have seen the duality
  argument of Aubin and Nitsche, which allows us to obtain optimal
  estimates in weaker norms, for instance in $L^2$.

  A particular difficulty here is the fact, that we have to test the
  dual solution with the error \emph{and} exploit some kind of
  Galerkin orthogonality. Thus, we cannot use consistency as before
  and will introduce residual operators later. The analysis here is a
  simplified version of the corresponding results
  in~\cite{GiraultKanschatRiviere14}.
\end{intro}

\begin{Definition}{dual-stokes}
  The \putindex{dual problem} to the Stokes problem in weak for
  consists of finding $(u^*,p^*)\in V_h\times Q_h$ such that for all
  $v\in V$ and $q\in Q$ there holds
  \begin{gather}
    \label{eq:hdivdg:9}
    \form(\nabla v,\nabla u^*) + \form(\div u^*,q) + \form(\div v,p^*)
    = \form(\psi,v).
  \end{gather}
\end{Definition}

\begin{Assumption}{stokes-regularity}
  The dual Stokes problem admits the elliptic regularity estimate
  \begin{gather}
    \label{eq:hdivdg:10}
    \norm{u^*}_{2} \le c \norm{f}_0.
  \end{gather}
\end{Assumption}

\begin{remark}
  Like for scalar elliptic equations, the elliptic regularity
  assumption holds for domains with smooth boundary or with piecewise
  smooth boundary where every corner is convex.
\end{remark}

\begin{Definition}{hdivdg-residual-operators}
  For the solutions $(u,p)\in V\times Q$ and $(u^*,p^*)\in V\times Q$
  of the primal and dual Stokes problem, respectively, we define the
  residual operators
  \begin{align}
    \operatorname{Res}(u,p;v) &= a_h(u,v)+\form(\div v,p) - \form(f,v),
    \\
    \operatorname{Res}^*(v;u^*,p^*)
                              &= a_h(v,u^*)+\form(\div v,p^*) -
                                \form(\psi,v),
  \end{align}
  for $v\in V+V_h$.
\end{Definition}

% From Girault/Kanschat/Riviere

\begin{Lemma}{hdivdg-residual-1}
  Let $(u,p)\in V\times Q$ be the solution to the Stokes problem with
  right hand side $f\in L^2(\domain;\R^d)$. Assume $u\in
  H^s(\domain;\R^d)$ and $p\in H^{s-1}(\domain)$ with $s>3/2$. Then,
  we have for $v\in V+V_h$:
  \begin{gather}
    \label{eq:hdivdg:11}
    \form(f,v) = \form(\nabla u,\nabla v)_{\mesh_h}
    -\forme(\nabla u,\mvl{v\otimes n})_{\faces_h^i}
    -\forme(\d_n u,v)_{\faces_h^\d}
    + \form(\div v,p).
  \end{gather}
\end{Lemma}

\begin{proof}
  We set out from the strong form of the Stokes equations and
  integrate by parts.
  \begin{align}
    \form(f,v) &= \form(-\Delta u + \nabla p, v)
    \\
    &= \form(\nabla u,\nabla v)_{\mesh_h}
      - \sum_{\cell\in\mesh_h} \forme(\d_n u,v)_{\d\cell}
      - \form(\div v,p)
      .
  \end{align}
  Under the regularity assumptions of the lemma, all of these
  integrals make sense at least as duality pairings. In particular,
  $\d_n u\in L^2(\d\cell)$, and thus we can split $\d\cell$ into
  individual faces. Therefore,
  \begin{gather}
    \sum_{\cell\in\mesh_h} \forme(\d_n u,v)_{\d\cell}
    = \forme(\nabla u,\mvl{v\otimes n})_{\faces_h^i}
    +\forme(\d_n u,v)_{\faces_h^\d}.
  \end{gather}
  The proof concludes by collecting the results.
\end{proof}

\begin{Corollary}{hdivdg-residual-2}
  The residual operators can be expressed as
  \begin{gather}
    \label{eq:hdivdg:12}
    \begin{split}
    \operatorname{Res}(u,p;v)
    &= a_h(u,v)
      - \form(\nabla u,\nabla v)_{\mesh_h}
      \\
      &\qquad
      + \forme(\nabla u,\mvl{v\otimes n})_{\faces_h^i}
      +\forme(\d_n u,v)_{\faces_h^\d}.
      \\
    \operatorname{Res}^*(u^*,p^*;v)
    &= a_h(v,u^*)
      - \form(\nabla u,\nabla v)_{\mesh_h}
      \\
      &\qquad
      + \forme(\nabla u,\mvl{v\otimes n})_{\faces_h^i}
      +\forme(\d_n u,v)_{\faces_h^\d}.      
    \end{split}
  \end{gather}
  In particular, the residual operators do not depend on the pressure
  solutions.
\end{Corollary}

\begin{Theorem}{hdivdg-l2}
  Let the assumptions of \slideref{Theorem}{hdivdg-convergence}
  and \slideref{Assumption}{stokes-regularity} hold. Then,
  \begin{gather}
    \norm{u-u_h}_0 \le c h^{k+1} \snorm{u}_{k+1}.
  \end{gather}
\end{Theorem}

\begin{Problem}{hdivdg-l2}
  Adapt the proof of \slideref{Theorem}{ip-lifting-l2} to prove
  \slideref{Theorem}{hdivdg-l2}.
\begin{solution}
   We again consider the auxiliary problem
  \begin{gather}
    a(v,u^*) +(\nabla\cdot v, p^*)= \form(u-u_h,v),\qquad\forall v\in V.
  \end{gather}
  Using the definition of the dual residual, we obtain the equation
  \begin{gather}
    \form(u-u_h, v) = a_h(v,u^*) +(\nabla\cdot v, p^*)- \Res^*(u^*,p^*;v),\qquad\forall v\in V+V_h.
  \end{gather}
  Testing with $v=u-u_h$ yields
  \begin{gather}
    \norm{u-u_h}^2 = a_h(u-u_h,u^*) +(\nabla\cdot (u-u_h), p^*)- \Res^*(u^*, p^*;u-u^*).
  \end{gather}
  Additionally, we us the error equation
  \begin{gather}
    a_h(u-u_h, v_h)+(\nabla\cdot v_h, p-p_h) = \Res(u,p;v_h),
  \end{gather}
  tested with $v_h = I_h u^*$, to obtain
  \begin{align}
    \norm{u-u_h}^2 &= a_h(u-u_h,u^*-I_h u^*) +(\nabla\cdot (u-u_h), p^*-q_h)\\
    &\quad- \Res^*(u^*, p^*;u-u^*)
    -\underbrace{(\nabla\cdot I_h u^*, p-p_h)}_{0} + \Res(u,p;I_h u^*).    
  \end{align}
  Using the regularity of $u^*$, the first term on the right
  admits the estimate
  \begin{gather}
    \abs{a_h(u-u_h, u^*-I_h u^*)}
    \le \norm{u-u_h}_{1,h}\norm{u^*-I_hu^*}_{1,h}
    \le c h \norm{u-u_h}_{1,h}.
  \end{gather}
  as before.
  
  For the second term consider $q_h=\Pi_{Q_h} p^*$
  \begin{align}
   (\nabla\cdot (u-u_h), p^*-\Pi_{Q_h} p^*)\leq \norm{\nabla \cdot  (u-u_h)} \norm{p^*-\Pi_{Q_h} p^*}=0
  \end{align}
  
  For the third term we use \slideref{Lemma}{ip-lifting-residual-2}
  to obtain
  \begin{gather}
    \abs{\Res^*(u^*,u-u_h)} \le c h \snorm{u^*}_2 \norm{u-u_h}_{1,h}.
  \end{gather}
  Finally, using $\jmp{u^*} = 0$, the same lemma yields
  \begin{align}
    \abs{\Res(u, I_h u^*)}
    &\le c h \snorm{u}_2
      \bigl(\norm{\sqrt{\ipp_h}\jmp{I_h u^*}}_{\faces_h^i}
      + \norm{\sqrt{\ipp_h}I_h u^*}_{\faces_h^\d}\bigr)
    \\
    & = c h \snorm{u}_2
      \bigl(\norm{\sqrt{\ipp_h}\jmp{u^*-I_h u^*}}_{\faces_h^i}
      + \norm{\sqrt{\ipp_h}(u^*-I_h u^*)}_{\faces_h^\d}\bigr)
    \\
    & \le c h \snorm{u}_2 h^k \snorm{u^*}_{k+1}
  \end{align}
  This is exactly the same proof we used before.
  Using the energy estimate in \slideref{Theorem}{ip-lifting-h1} we
  can conclude the prove.
\end{solution}

\end{Problem}

%%% Local Variables: 
%%% mode: latex
%%% TeX-master: "main"
%%% End: 
