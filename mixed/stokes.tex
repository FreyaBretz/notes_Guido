
\section{Well-posedness of the continuous problem}

\begin{intro}
  We begin our investigation into the Stokes problem by investigating
  the well-posedness of the continuous problem. This is particularly
  simple, since we have
  \begin{gather*}
    a(u,v) = \form(\strain u,\strain v)
  \end{gather*}
  for the original Stokes problem in \blockref{Definition}{stokes-eq1}
  and
  \begin{gather*}
    a(u,v) = \form(\nabla u,\nabla v)
  \end{gather*}
  for the simplified \putindex{Stokes equations} in
  \blockref{Definition}{stokes-eq2}. From the standard theory for the
  Laplacian, we know that the second one is $V$-elliptic on
  $V=H^1_0(\domain;\R^d)$. For the first one, we conclude this by using a
  \putindex{Korn inequality}. Therefore, we can already conclude a
  first result:
\end{intro}

\begin{Lemma}{stokes-a-elliptic}
  Let $V=H^1_0(\domain, \R^d)$ and $V_h\subset V$ a finite dimensional
  subspace. Then, $a(.,.)$ is elliptic on $\ker B$ and on $\ker{B_h}$
  independent of the choices of $Q$ and $Q_h$.
\end{Lemma}

\begin{remark}
  We focus here on no-slip boundary condition on the whole boundary as
  the exemplary case. Other boundary conditions are possible, but as
  soon as the Dirichlet boundary for one velocity component becomes
  too small, the ellipticity of $a(.,.)$ on $V$ must be established by
  new arguments known for instance for Robin boundary conditions. In
  the extreme case of natural boundary conditions all around, $V$ is
  the subspace of $H^1(\domain, \R^d)$ obtained by dividing by the
  space of all translations for the simplified form and by the space
  of all rigid body movements.

  Note that we have established already in
  \blockref{Lemma}{divergence-compatibility} that the condition
  $V=H^1_0(\domain, \R^d)$ implies the reduction of the pressure to
  the space $Q = L^2_0(\domain)$ from
  \blockref{Notation}{pressure-constant}.
\end{remark}

\begin{intro}
  The previous lemma guarantees well-posedness of the
  \putindex{reduced problem} in all possible cases. Therefore, the
  remainder of this section is only concerned with the inf-sup
  condition for the divergence operator. We
  follow~\cite{GiraultRaviart86} in this presentation.
\end{intro}

\begin{Lemma}{stokes-helmholtz}
  Let $V=H^1_0(\domain,\R^d)$. Then, the divergence operator
  $\div\colon V \to L^2(\domain)$ is continuous and the subspace
  \begin{gather*}
    V^0 = \ker \div
    = \bigl\{v\in V \big|
    \div v = 0 \text{ a.e.} \bigr\}
  \end{gather*}
  is closed in $V$ and $V$ admits the orthogonal decomposition
  \begin{gather*}
    V = V^0\oplus V^\perp.
  \end{gather*}
\end{Lemma}

\begin{proof}
  We have that
  \begin{gather*}
    \norm{\div v}_{L^2(\domain)}^2
    = \int_\domain \left(\sum \d_iv_i\right)^2\dx
    \le d \int_\domain \sum \abs{\d_iv_i}^2\dx
    \le d \norm{v}_{H^1(\domain;\R^d)}^2.
  \end{gather*}
  Thus, the divergence operator is a continuous mapping from $V$ to
  $L^2(\domain)$. The definition of $V^0$ is equivalent to the
  definition of zero in $L^2(\domain)$. Finally, since the kernel is
  the pre-image of a closed set under a continuous map, it is
  closed. The existence of the decomposition follows from
  \blockref{Theorem}{orthogonal-complement}.
\end{proof}

\begin{Lemma}{stokes-grad}
  If $f\in V^* = H^{-1}(\domain;\R^d)$ satisfies
  \begin{gather*}
    f(v) = 0 \quad\forall v\in V^0,
  \end{gather*}
  then, there exists $p\in L^2(\domain)$ such that
  \begin{gather*}
    f = \nabla p.
  \end{gather*}
  If $\domain$ is connected, then $p$ is unique up to an additive
  constant.
\end{Lemma}

\begin{proof}
  First, we identify $L^2(\domain)$ with its dual. Then, by
  \begin{gather*}
    \scal(-\nabla p, v)_{V^*\times V}
    = \scal(p, \div v)_{L^2(\domain)},
    \qquad\forall v\in V,
  \end{gather*}
  we see that $-\nabla\colon L^2(\domain)\to V^*$ is the dual to the
  divergence operator. Using the Cauchy-sequence argument, we see that
  $\range{\div}$ is closed in $L^2(\domain)$ and the closed range
  theorem applies. Thus, $\range{-\nabla}$ is closed in $V^*$ and
  \begin{gather*}
    \range{\nabla} = \polar{(V^0)} \cong V^\perp
  \end{gather*}
  is the polar set of the kernel $V^0$. This implies the statement
  that there is a $p$ for every $f$. Uniqueness follows by the fact
  that the only differentiable functions on a connected domain with
  $\nabla p=0$ are the constant functions, and by density of such
  functions in $L^2(\domain)$.
\end{proof}

\begin{Corollary}{stokes-iso}
  Let $\domain$ be connected. Then,
  \begin{enumerate}
  \item $\nabla\colon L^2_0(\domain) \to V^0$ is an isomorphism
  \item $\div\colon V^\perp \to L^2_0(\domain)$ is an isomorphism
  \end{enumerate}
\end{Corollary}

\begin{Theorem}{stokes-infsup}
  Let $\domain\subset \R^d$ be a Lipschitz-domain,
  $V=H^1_0(\domain,\R^d)$ and $Q=L^2_0(\domain)$. Then, there is a
  constant $\beta>0$ depending only on the geometry of $\domain$ such
  that
  \begin{gather}
    \label{eq:stokes:1}
    \inf_{q\in Q}\sup_{v\in V}\frac{\form(\div
      v,q)}{\norm{v}_V\norm{q}_Q} \ge \beta.
  \end{gather}
  Furthermore, the problem finding $(u,p)\in V\times Q$ such that
  \begin{gather}
    \label{eq:stokes:3}
    a(u,v)+\form(\div v,p)+\form(\div u,q) = f(v)+g(q)
    \quad\forall v\in V, q\in Q,
  \end{gather}
  has a unique solution for any right hand side $f\in V^*$ and $g\in
  \range{\div}$.
\end{Theorem}

\section{Stable discretizations}

\begin{intro}
  We begin by application of the generic theory of the previous
  chapter to the Stokes problem in order to obtain a generic error
  estimate based on the concrete choice of norms and a single
  assumption. Guided by this theorem, we spend the remaining part of
  this section exploring different options for the discrete spaces.
\end{intro}

\begin{Theorem}{stokes-convergence}
  Let $V=H^1_0(\domain;\R^d)$ and $Q=L^2_0(\domain)$. Let furthermore
  $V_h\subset V$ and $Q_h\subset Q$ be discrete subspaces such that
  there exists $\beta>0$ independent of $h$ such that
  \begin{gather}
    \label{eq:stokes:2}
    \inf_{q_h\in Q_h}\sup_{v_h\in V_h}\frac{\form(\div
      v_h,q_h)}{\norm{v_h}_V\norm{q_h}_Q} \ge \beta.
  \end{gather}
  Then, the Galerkin approximation of~\eqref{eq:stokes:3} admits a
  unique solution $(u_h, p_h)\in V_h\times Q_h$ with the
  quasi-bestapproximation property
  \begin{gather}
    \label{eq:stokes:4}
    \begin{split}
      \norm{u-u_h}_1
      &\le c_1 \inf_{v_h\in V_h}\norm{u-v_h}_1
      + c_2 \inf_{q_h\in Q_h}\norm{p-q_h}_0
      \\
      \norm{p-p_h}_1
      &\le c_3 \inf_{v_h\in V_h}\norm{u-v_h}_1
      + c_4 \inf_{q_h\in Q_h}\norm{p-q_h}_0.
    \end{split}
  \end{gather}
\end{Theorem}

\begin{Corollary}{stokes-convergence2}
  Under the assumptions of \blockref{Theorem}{stokes-convergence},
  let there be in addition interpolation operators $I_{V_h}$ and
  $I_{Q_h}$ such that
  \begin{gather}
    \label{eq:stokes:5}
    \begin{split}
      \norm{u-I_{V_h} u}_1 &\le c h^k \snorm{u}_{k+1} \\
      \norm{p-I_{Q_h} p}_0 &\le c h^k \snorm{p}_{k}.
    \end{split}
  \end{gather}
  Then, there is a constant $c$ independent of the approximation
  spaces such that
  \begin{gather}
    \label{eq:stokes:6}
    \begin{split}
      \norm{u-u_h}_1 &\le c h^k \bigl(\snorm{u}_{k+1} +
      \snorm{p}_{k}\bigr)
      \\
      \norm{p-p_h}_1 &\le c h^k \bigl(\snorm{u}_{k+1} +
      \snorm{p}_{k}\bigr).
    \end{split}
  \end{gather}
\end{Corollary}
\begin{intro}
  We continue showing that the most natural discretizations
  in two dimensions are not inf-sup stable. This holds for the
  discretization using continuous linear or bilinear elements for both
  velocity components and the pressure as well as for continuous
  linear or bilinear velocity functions combined with piecewise
  constant pressure functions.
\end{intro}

\begin{example}
  We begin with a one-dimensional example. Piecewise linear velocity
  and piecewise linear pressure. Both continuous. Then, $\div v_h$ is
  piecewise constant. Consequently, a pressure function which has zero
  mean value on each cell is in the kernel of $B_h$.
  \begin{figure}[tp]
    \centering
    \includegraphics[width=.6\textwidth]{./fig/p1-p1-1d.tikz}
    \caption[Example for the $P_1-P_1$ pair in one
    dimension]{Piecewise linear pressure (\tikz\draw[color=cyan] (0,0)
      -- (1em,0);) and divergence (\tikz\draw[color=red] (0,0)
      -- (1em,0);) of
      piecewise linear velocity.}
    \label{fig:stokes:p1p1-1d}
  \end{figure}
\end{example}

\begin{example}
  Take a patch of four quadrilaterals or triangles meeting in a common
  vertex. Let $\domain$ be the union of these grid cells. Choose
  linear and bilinear shape functions for $V_h$, respectively. Then, $\dim
  V_h = 2$, since we have one interior vertex with one basis function for
  each velocity component. Choose piecewise constant pressure
  functions. Dividing out the global constant, we conclude that $\dim
  Q_h = 3$. Thus, the statement
  \begin{gather*}
    \forall q_h\in Q_h \;\exists v_h\in V_h:
    \quad \norm{v_h}_1 = \norm{q_h}_0
    \;\wedge\; b(v_h, q_h) \ge \beta \norm{q_h}^2
  \end{gather*}
  cannot hold true. Therefore, the inf-sup condition does not hold. In
  fact, $\ker{B_h} = \{0\}$.
  \begin{figure}[tp]
    \begin{center}
    \hfill
    \includegraphics[width=.3\textwidth]{./fig/patch1.tikz}
    \hfill
    \includegraphics[width=.3\textwidth]{./fig/patch2.tikz}
    \hfill\mbox{}
    \end{center}
    \caption[Very coarse meshes with Dirichlet boundary.]{Very coarse meshes with Dirichlet boundary. Degrees of freedom for pressure (\tikz\draw[shape pressure] (0,0) circle (1ex);) and for both velocity components(\tikz\draw[shape veloxy] (0,0) circle (1ex);).}
    \label{fig:stokes:example1}
  \end{figure}

  Thus, we conclude that for this combination of shape function
  spaces, there is a mesh such that they are not suited for the
  approximation of the Stokes problem. But, this may be a problem of a
  mesh with too few cells. In fact, asymptotically, a triangular mesh
  contains twice as many vertices as cells, a quadrilateral mesh as
  many. Therefore, $\dim V_h > \dim Q_h$ as soon as the mesh is
  sufficiently fine. Will this be sufficient?
\end{example}

\begin{Problem}{p1-p0-unstable}
The domain $\Omega=[0,1]^2$ is decomposed into $N \times N$ congruent squares where each
of them is again divided into two triangles. The decomposition $\mathcal{T}_h$
is given by these triangles.

We again choose piecewise linear ansatz functions for the velocity for $V_h$
(vanishing on $\partial \Omega$) and piecewise constant ansatz functions
for $Q_h$.

Is there a $N$ and an orientation of the triangles such that $V_h\times Q_h$ is
inf-sup stable?
\begin{solution}
  The number of degrees of freedom for the velocity is $2(N-1)^2$ and for the
  pressure $2N^2-1$. Hence, there is always a pressure ansatz functions such that
  \begin{align*}
    (\nabla \cdot \boldsymbol{v}_h, q_h)=0 \quad \forall \boldsymbol{v}_h\in V_h.
  \end{align*}
\end{solution}
\end{Problem}

\begin{Problem}{checker-board}
  Let $\domain = (0,1)^2$ be the unit square and let the mesh consist
  of Cartesian squares of side length $1/n$. Choose $V_h \subset V$
  based on bilinear shape functions. Show that the piecewise constant
  pressure function $p_c=\pm 1$ in a checkerboard fashion is in the
  kernel of $B_h^T$, that is
  \begin{gather*}
    b(v_h, p_c) = 0 \quad\forall v_h\in V_h.
  \end{gather*}
\begin{solution}
This time there are $2N^2$ for the velocity and $2N^2-1$ ansatz functions for
the pressure. Therefore, we have to look a big deeper.

Denote by $p_{i+\frac12, j+\frac12}$ the pressure constant on cell $K_{i,j}$
($0\leq i,j \leq N-1$). The values of the velocity at the four vertices are
denoted by $u^1_{i,j}$, $u^1_{i,j+1}$, $u^1_{i+1,j}$, $u^1_{i+1,j+1}$ for the
first component and similarly for the second component.

Using this notation we can write the divergence constraint in terms of nodal and cell values:
\begin{align*}
 &(\nabla \cdot \boldsymbol{v}_h, q_h)\\
   &= \sum_{i,j} q_{i+\frac12,j+\frac12} \int_{K_{i,j}} \nabla \cdot \boldsymbol{v}_h \,\mathrm{d}x\\
   &= \sum_{i,j} q_{i+\frac12,j+\frac12} \int_{\partial K_{i,j}} \boldsymbol{n} \cdot \boldsymbol{v}_h \,\mathrm{d}x\\
   &= \sum_{i,j} q_{i+\frac12,j+\frac12} \frac{h}{2}(u^1_{i+1,j}-u^1_{i,j+1}+u^1_{i+1,j}-u^1_{i+1,j+1})\\
      &\quad +\sum_{i,j} q_{i,j} \frac{h}{2}(u^2_{i,j+1}-u^2_{i+1,j}+u^2_{i,j+1}-u^2_{i+1,j+1})\\
   &= \frac{h}{2}\sum_{i,j} u^1_{i,j}(q_{i-\frac12,j+\frac12}+q_{i-\frac12,j-\frac12}-q_{i+\frac12,j+\frac12}-q_{i+\frac12,j-\frac12})\\
      &\quad +\frac{h}{2}\sum_{i,j} u^2_{i,j}(q_{i+\frac12,j-\frac12}+q_{i-\frac12,j-\frac12}-q_{i+\frac12,j+\frac12}-q_{i-\frac12,j+\frac12})
\end{align*}
Thus,
\begin{align*}
(\nabla \cdot \boldsymbol{v}_h, q_h)=0 \quad \forall \boldsymbol{v}_h\in V_h
\end{align*}
implies
\begin{align*}
q_{i-\frac12,j+\frac12}+q_{i-\frac12,j-\frac12}-q_{i+\frac12,j+\frac12}-q_{i+\frac12,j-\frac12} &=0 \\
q_{i+\frac12,j-\frac12}+q_{i-\frac12,j-\frac12}-q_{i+\frac12,j+\frac12}-q_{i-\frac12,j+\frac12} &=0.
\end{align*}
for all ($0\leq i,j \leq N-1$).
These two constraints can be rephrased as
\begin{align*}
q_{i-\frac12,j+\frac12} &= q_{i+\frac12,j-\frac12} \\
q_{i-\frac12,j-\frac12} &= q_{i+\frac12,j+\frac12}
\end{align*}
for all ($0\leq i,j \leq N-1$). Since the mean value of the pressure must be zero,
the set of pressure functions that fulfill
\begin{align*}
(\nabla \cdot \boldsymbol{v}_h, q_h)=0 \quad \forall \boldsymbol{v}_h\in V_h
\end{align*}
can be described by
\begin{align*}
 p_{i,j} = c \quad \text{if } (i+j) \operatorname{mod} 2 = 0\\
 p_{i,j} = -c \quad \text{if } (i+j) \operatorname{mod} 2 = 1
\end{align*}
where $c\not=0$.
\end{solution}
\end{Problem}

\subsection{The MINI element}

\begin{Definition}{barycentric-coordinates}
  A simplex $T\in \R^d$ with vertices $x_0,\dots,x_d$ is described by
  a set of $d+1$ \define{barycentric coordinates}
  $\lambda_0,\dots,\lambda_d$ such that
  \begin{xalignat}2
    0\le\lambda_i(x) &\le 1& i&=0,\dots,d;\quad x\in T\\
    \lambda_i(x_j) &= \delta_{ij}& i,j&=0,\dots,d\\
    \sum \lambda_i(x) &= 1.
  \end{xalignat}
\end{Definition}

\begin{remark}
  The functions $\lambda_i(x)$ are the shape functions of the linear
  $P_1$ element on $T$. They allow us to define basis functions on the
  cell $T$ without use of a reference element $\widehat T$.
  
  Note that $\lambda_i\equiv 0$ on the face opposite to the
  vertex $x_i$.
\end{remark}

\begin{example}
  We can use barycentric coordinates to define shape functions on
  simplicial meshes easily, as in
  Table~\ref{tab:barycentric-shapes}.
  \begin{table}[tp]
    \centering
    \begin{tabular}{|c|l|}
      \hline Degrees of freedom
      & Shape functions \\\hline
      \adjustbox{valign=center,margin=3pt}{\includegraphics[width=2cm]{./fig/p1-p.tikz}}
      &
        {\begin{minipage}[b]{6cm}
          \begin{gather*}
            \phi_i = \lambda_i,
            \quad i=0,1,2
          \end{gather*}
        \end{minipage}}
      \\\hline
      \adjustbox{valign=center,margin=3pt}{\includegraphics[width=2cm]{./fig/p2-p.tikz}}
      &
        {\begin{minipage}[b]{6cm}
          \begin{xalignat*}2
            \phi_{ii} &= 2\lambda_i^2 - \lambda_i,
            &i&=0,1,2\\
            \phi_{ij} &= 4\lambda_i\lambda_j
            &j&\neq i
          \end{xalignat*}
        \end{minipage}}
        \\\hline
      \adjustbox{valign=center,margin=3pt}{\includegraphics[width=2cm]{./fig/p3-p.tikz}}
      &
        {\begin{minipage}[b]{6cm}
          \begin{xalignat*}2
          \phi_{iii} &= \tfrac12 \lambda_i(3\lambda_i-1)(3\lambda_i-2)
          &i&=0,1,2\\
          \phi_{ij} &= \tfrac92\lambda_i\lambda_j(3\lambda_j-1)
          &j&\neq i\\
          \phi_0 &= 27\lambda_0\lambda_1\lambda_2
        \end{xalignat*}
        \end{minipage}}
        \\\hline
    \end{tabular}
    \caption{Degrees of freedom and shape functions of simplicial elements
      in terms of barycentric coordinates}
    \label{tab:barycentric-shapes}
  \end{table}  
\end{example}

\begin{Notation}{piecewise-polynomial-spaces}
  We denote by
  \begin{gather}
    \label{eq:stokes:8}
    H^k_h(\mathcal P) =
    \bigl\{ v\in H^k(\Omega) \big\vert
    v_{|\cell} \in \mathcal P \;\forall \cell\in\mesh_h\bigr\}
  \end{gather}
  the finite element space which is based on the shape function space
  $\mathcal P$, the mesh $\mesh_h$ and is a subspace of
  $H^k(\domain)$. Examples are the continuous spaces of piecewise
  polynomials or tensor product polynomials of degree $k$
  \begin{gather*}
    H^1_h(\P_k) \qquad H^1_h(\Q_k),
  \end{gather*}
  and the discontinuous spaces
  \begin{gather*}
    H^0_h(\P_k) \qquad H^0_h(\Q_k).
  \end{gather*}
\end{Notation}

\begin{Definition}{h1-bubble-space}
  An $H^1$-\define{bubble function} on a mesh cell $\cell$ is a
  function $b\in H^1_0(\cell)$. A \define{bubble space} $b_\cell$ on
  $\cell$ is a discrete vector space of such bubble functions.  We
  denote the space of bubble functions on the mesh $\mesh_h$ by
  \begin{gather*}
    B_h(b_\cell) = \bigl\{ v\in H^1(\domain) \big\vert
    v_{|_\cell} \in b_\cell \;\forall \cell\in\mesh_h
    \bigr\}.
  \end{gather*}
  If there is no confusion about the local bubble space $b_T$, we also
  write just $B_h$.
\end{Definition}

\begin{example}
  A bubble function on a triangle $\cell$ is easily defined by
  \begin{gather}
    \label{eq:stokes:7}
    b_3 = \lambda_0\lambda_1\lambda_2.
  \end{gather}
\end{example}

\begin{Definition}{mini-element-p}
  The \define{MINI element} consists of the spaces
  \begin{gather}
    V_h = \bigl(H^1_h(\P_1) \oplus B_h(b_3) \cap V\bigr)^2,
    \qquad
    Q_h = H^1_h(\P_1) \cap Q.
  \end{gather}
  Its degrees of freedom are:
  \begin{center}
    \includegraphics[width=.2\textwidth]{./fig/p-mini-v.tikz}
    \hspace{1cm}
    \includegraphics[width=.2\textwidth]{./fig/p1-p.tikz}
  \end{center}
\end{Definition}

\begin{intro}
  We will show now that the MINI element is indeed inf-sup stable. To
  this end, we construct the \putindex{Fortin projection} according to
  \blockref{Lemma}{fortin}. Since the construction of such a
  projection operator turns out a bit complicated, we first introduce
  a construction principle, which will help us in our further
  analysis. The idea of this principle is separating the interpolation
  into $V_h$ from the preservation of the divergence.
\end{intro}

\begin{Lemma}{fortin-construction-1}
  Let there be operators $\Pi_1,\Pi_2\colon V \to V_h$ such that
  \begin{xalignat}2
    \label{eq:stokes:10}
    \norm{\Pi_1 v}_V &\le c1 \norm{v}_V
    &\forall v&\in V,\\
    \label{eq:stokes:11}
    \norm{\Pi_2(\identity-\Pi_1)v}_V &\le c_2 \norm{v}_V
    &\forall v&\in V,\\
    \label{eq:stokes:12}
    b(v-\Pi_2v,q_h) &= 0
    &\forall v&\in V, \;q_h\in Q_h,
  \end{xalignat}
  with constants $c_1$ and $c_2$ independent of the discretization
  parameter $h$. Then, the operator
  \begin{gather}
    \label{eq:stokes:9}
    \Pi_h = \Pi_1 + \Pi_2 - \Pi_2\Pi_1
  \end{gather}
  is a \putindex{Fortin projection}, that is, it is bounded on $V$ and
  \begin{gather*}
    b(v-\Pi_h v, q_h) =0 \qquad\forall q_h\in Q_h.
  \end{gather*}
\end{Lemma}

\begin{proof}
  Boundedness of $\Pi_h$ is obvious, such that we only focus on
  preservation of the kernel ob $B$:
  \begin{multline*}
    b(v-\Pi_h v,q_h) = b(v-\Pi_1 v - \Pi_2 v + \Pi_2\Pi_1 v, q_h)
    \\
    = b(v-\Pi_2 v, q_h) - b(\Pi_1 v - \Pi_2\Pi_1 v,q_h) = 0-0 = 0.
  \end{multline*}
\end{proof}

\begin{Assumption}{h1-stable-interpolation}
  There exists an $H^1$-stable interpolation operator $I_h:V\to V_h$
  such that for each cell $\cell \in \mesh_h$ there holds for $m=0,1$
  \begin{gather}
    \label{eq:stokes:13}
    \snorm{v-I_h v}_{m,\cell} \le c \sum_{\cell'\cap\cell
      \neq\emptyset} h_{\cell'}^{1-m}\snorm{v}_{1,\cell'},
  \end{gather}
  with a constant $c$ independent of the mesh parameter $h$.
\end{Assumption}

\begin{remark}
  The interpolation operators of Clément, Scott and Zhang, Schöberl or
  Ern and Guermond fullfil these assumptions.
\end{remark}

\begin{Definition}{locally-quasi-uniform}
  A family of meshes $\{\mesh_h\}$ is called \define{locally
    quasi-uniform}, if there is a constant $c$ such that
  \begin{gather}
    \label{eq:stokes:14}
    \forall h
    \;
    \forall \cell,\cell'\in \T_h
    \quad
    \cell\cap\cell'\neq \emptyset
    \Rightarrow
    h_\cell \le c h_{\cell'}.
  \end{gather}
\end{Definition}

\begin{Assumption}{locally-quasi-uniform}
  We assume of all families of meshes that they are shape regular and
  locally quasi-uniform, such that with
  \blockref{Assumption}{h1-stable-interpolation} there holds
  \begin{gather}
    \label{eq:stokes:15}
    \snorm{v-I_h v}_{m,\cell} \le c h_\cell \snorm{v}_{1,\domain_\cell},
  \end{gather}
  where $\Omega_\cell$ is the union of all cells with nonempty
  intersection with $\cell$.
\end{Assumption}

\begin{Theorem}{mini-stability}
  Under \blockref{Assumption}{locally-quasi-uniform},
  the MINI element is inf-sup stable.
\end{Theorem}

\begin{proof}
  We construct a \putindex{Fortin projection} by choosing
  $\Pi_1 = I_h$, where $I_h:V\to \bigl(H^1_h(\P_1)\bigr)^2$ is an
  $H^1$-stable interpolation operator into the standard linear finite
  element space. Now, we construct $\Pi_2: V \to \bigl(B_h\bigr)^2$
  such that for all $q_h\in Q_h$
  \begin{gather*}
    \int_\domain \div(\Pi_2 v-v) q_h \dx
    = \int_\domain (v-\Pi_2v)\cdot\nabla q_h\dx
    = 0.
  \end{gather*}
  Indeed, $\Pi_2 v$ can be chosen on each cell. Since $\nabla q_h$ is
  constant on a cell $\cell$, we choose
  \begin{gather*}
    \int_\cell \Pi_2 v_i \dx
    = \alpha_{\cell,i} \int_\cell b_3\dx
    = \int_\cell v_i\dx,
  \end{gather*}
  where $i=1,2$ enumerates the velocity components. This is possible,
  since the mean value of $b_3$ is strictly positive. Assuming shape
  regularity, we can use the inverse estimate for $b_3$ to obtain
  \begin{gather*}
    \norm{\Pi_2 v}_{1,\cell}
    \le c h_\cell^{-1} \norm{\Pi_2 v}_{0,\cell}
    \le c h_\cell^{-1} \norm{v}_{0,\cell}.
  \end{gather*}
  Finally, we use the estimates for $I_h$ to obtain
  \begin{gather*}
    \norm{\Pi_2 (\identity-\Pi_1) v}_{1,\cell}
    \le c h_\cell^{-1} \norm{v-I_h v}_{0,\cell}
    \le c \snorm{v}_{1,\domain_\cell}.
  \end{gather*}
  Since the number of intersecting of cells of shape regular meshes is
  bounded, the finale term is bounded by $\norm v_{1,\domain}$.
\end{proof}

\begin{Problem}{quadrilateral-mini}
  Show that the MINI element can be generalized to quadrilateral
  meshes. Design a suitable bubble function $b_Q$ of minimal tensor
  degree such that
  \begin{gather*}
    V_h = \bigl(H^1_h(\Q_1) \oplus B_h(b_Q) \cap V\bigr)^2,
    \qquad
    Q_h = H^1_h(\P_1) \cap Q,
  \end{gather*}
  and the degrees of freedom are
  \begin{center}
    \includegraphics[width=.2\textwidth]{./fig/q-mini-v.tikz}
    \hspace{1cm}
    \includegraphics[width=.2\textwidth]{./fig/q1-p.tikz}
  \end{center}

  Discuss extensions to tetrahedra and hexahedra in three dimensions.
\end{Problem}

\subsection{The $P_2-P_0$ element}

\section{Nearly incompressible elasticity}

%%% Local Variables:
%%% mode: latex
%%% TeX-master: "main"
%%% End:
