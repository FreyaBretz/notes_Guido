
\section{Well-posedness of the continuous problem}

\begin{intro}
  We begin our investigation into the Stokes problem by investigating
  the well-posedness of the continuous problem. This is particularly
  simple, since we have
  \begin{gather*}
    a(u,v) = \form(\strain u,\strain v)
  \end{gather*}
  for the original Stokes problem in \blockref{Definition}{stokes-eq1}
  and
  \begin{gather*}
    a(u,v) = \form(\nabla u,\nabla v)
  \end{gather*}
  for the simplified \putindex{Stokes equations} in
  \blockref{Definition}{stokes-eq2}. From the standard theory for the
  Laplacian, we know that the second one is $V$-elliptic on
  $V=H^1_0(\domain;\R^d)$. For the first one, we conclude this by using a
  \putindex{Korn inequality}. Therefore, we can already conclude a
  first result:
\end{intro}

\begin{Lemma}{stokes-a-elliptic}
  Let $V=H^1_0(\domain, \R^d)$ and $V_h\subset V$ a finite dimensional
  subspace. Then, $a(.,.)$ is elliptic on $\ker B$ and on $\ker{B_h}$
  independent of the choices of $Q$ and $Q_h$.
\end{Lemma}

\begin{remark}
  We focus here on no-slip boundary condition on the whole boundary as
  the exemplary case. Other boundary conditions are possible, but as
  soon as the Dirichlet boundary for one velocity component becomes
  too small, the ellipticity of $a(.,.)$ on $V$ must be established by
  new arguments known for instance for Robin boundary conditions. In
  the extreme case of natural boundary conditions all around, $V$ is
  the subspace of $H^1(\domain, \R^d)$ obtained by dividing by the
  space of all translations for the simplified form and by the space
  of all rigid body movements.

  Note that we have established already in
  \blockref{Lemma}{divergence-compatibility} that the condition
  $V=H^1_0(\domain, \R^d)$ implies the reduction of the pressure to
  the space $Q = L^2_0(\domain)$ from
  \blockref{Notation}{pressure-constant}.
\end{remark}

\begin{intro}
  The previous lemma guarantees well-posedness of the
  \putindex{reduced problem} in all possible cases. Therefore, the
  remainder of this section is only concerned with the inf-sup
  condition for the divergence operator. We
  follow~\cite{GiraultRaviart86} in this presentation.
\end{intro}

\begin{Lemma}{stokes-helmholtz}
  Let $V=H^1_0(\domain,\R^d)$. Then, the divergence operator
  $\div\colon V \to L^2(\domain)$ is continuous and the subspace
  \begin{gather*}
    V^0 = \ker \div
    = \bigl\{v\in V \big|
    \div v = 0 \text{ a.e.} \bigr\}
  \end{gather*}
  is closed in $V$ and $V$ admits the orthogonal decomposition
  \begin{gather*}
    V = V^0\oplus V^\perp.
  \end{gather*}
\end{Lemma}

\begin{proof}
  We have that
  \begin{gather*}
    \norm{\div v}_{L^2(\domain)}^2
    = \int_\domain \left(\sum \d_iv_i\right)^2\dx
    \le d \int_\domain \sum \abs{\d_iv_i}^2\dx
    \le d \norm{v}_{H^1(\domain;\R^d)}^2.
  \end{gather*}
  Thus, the divergence operator is a continuous mapping from $V$ to
  $L^2(\domain)$. The definition of $V^0$ is equivalent to the
  definition of zero in $L^2(\domain)$. Finally, since the kernel is
  the pre-image of a closed set under a continuous map, it is
  closed. The existence of the decomposition follows from
  \blockref{Theorem}{orthogonal-complement}.
\end{proof}

\begin{Lemma}{stokes-grad}
  If $f\in V^* = H^{-1}(\domain;\R^d)$ satisfies
  \begin{gather*}
    f(v) = 0 \quad\forall v\in V^0,
  \end{gather*}
  then, there exists $p\in L^2(\domain)$ such that
  \begin{gather*}
    f = \nabla p.
  \end{gather*}
  If $\domain$ is connected, then $p$ is unique up to an additive
  constant.
\end{Lemma}

\begin{proof}
  First, we identify $L^2(\domain)$ with its dual. Then, by
  \begin{gather*}
    \scal(-\nabla p, v)_{V^*\times V}
    = \scal(p, \div v)_{L^2(\domain)},
    \qquad\forall v\in V,
  \end{gather*}
  we see that $-\nabla\colon L^2(\domain)\to V^*$ is the dual to the
  divergence operator. Using the Cauchy-sequence argument, we see that
  $\range{\div}$ is closed in $L^2(\domain)$ and the closed range
  theorem applies. Thus, $\range{-\nabla}$ is closed in $V^*$ and
  \begin{gather*}
    \range{\nabla} = \polar{(V^0)} \cong V^\perp
  \end{gather*}
  is the polar set of the kernel $V^0$. This implies the statement
  that there is a $p$ for every $f$. Uniqueness follows by the fact
  that the only differentiable functions on a connected domain with
  $\nabla p=0$ are the constant functions, and by density of such
  functions in $L^2(\domain)$.
\end{proof}

\begin{Corollary}{stokes-iso}
  Let $\domain$ be connected. Then,
  \begin{enumerate}
  \item $\nabla\colon L^2_0(\domain) \to V^0$ is an isomorphism
  \item $\div\colon V^\perp \to L^2_0(\domain)$ is an isomorphism
  \end{enumerate}
\end{Corollary}

\begin{Theorem}{stokes-infsup}
  Let $\domain\subset \R^d$ be a Lipschitz-domain,
  $V=H^1_0(\domain,\R^d)$ and $Q=L^2_0(\domain)$. Then, there is a
  constant $\beta>0$ depending only on the geometry of $\domain$ such
  that
  \begin{gather}
    \label{eq:stokes:1}
    \inf_{q\in Q}\sup_{v\in V}\frac{\form(\div
      v,q)}{\norm{v}_V\norm{q}_Q} \ge \beta.
  \end{gather}
  Furthermore, the problem finding $(u,p)\in V\times Q$ such that
  \begin{gather}
    \label{eq:stokes:3}
    a(u,v)+\form(\div v,p)+\form(\div u,q) = f(v)+g(q)
    \quad\forall v\in V, q\in Q,
  \end{gather}
  has a unique solution for any right hand side $f\in V^*$ and $g\in
  \range{\div}$.
\end{Theorem}

\section{Stable discretizations}

\begin{intro}
  We begin by application of the generic theory of the previous
  chapter to the Stokes problem in order to obtain a generic error
  estimate based on the concrete choice of norms and a single
  assumption. Guided by this theorem, we spend the remaining part of
  this section exploring different options for the discrete spaces.
\end{intro}

\begin{Theorem}{stokes-convergence}
  Let $V=H^1_0(\domain;\R^d)$ and $Q=L^2_0(\domain)$. Let furthermore
  $V_h\subset V$ and $Q_h\subset Q$ be discrete subspaces such that
  there exists $\beta>0$ independent of $h$ such that
  \begin{gather}
    \label{eq:stokes:2}
    \inf_{q_h\in Q_h}\sup_{v_h\in V_h}\frac{\form(\div
      v_h,q_h)}{\norm{v_h}_V\norm{q_h}_Q} \ge \beta.
  \end{gather}
  Then, the Galerkin approximation of~\eqref{eq:stokes:3} admits a
  unique solution $(u_h, p_h)\in V_h\times Q_h$ with the
  quasi-bestapproximation property
  \begin{gather}
    \label{eq:stokes:4}
    \begin{split}
      \norm{u-u_h}_1
      &\le c_1 \inf_{v_h\in V_h}\norm{u-v_h}_1
      + c_2 \inf_{q_h\in Q_h}\norm{p-q_h}_0
      \\
      \norm{p-p_h}_1
      &\le c_3 \inf_{v_h\in V_h}\norm{u-v_h}_1
      + c_4 \inf_{q_h\in Q_h}\norm{p-q_h}_0.
    \end{split}
  \end{gather}
\end{Theorem}

\begin{Corollary}{stokes-convergence2}
  Under the assumptions of \blockref{Theorem}{stokes-convergence},
  let there be in addition interpolation operators $I_{V_h}$ and
  $I_{Q_h}$ such that
  \begin{gather}
    \label{eq:stokes:5}
    \begin{split}
      \norm{u-I_{V_h} u}_1 &\le c h^k \snorm{u}_{k+1} \\
      \norm{p-I_{Q_h} p}_0 &\le c h^k \snorm{p}_{k}.
    \end{split}
  \end{gather}
  Then, there is a constant $c$ independent of the approximation
  spaces such that
  \begin{gather}
    \label{eq:stokes:6}
    \begin{split}
      \norm{u-u_h}_1 &\le c h^k \bigl(\snorm{u}_{k+1} +
      \snorm{p}_{k}\bigr)
      \\
      \norm{p-p_h}_1 &\le c h^k \bigl(\snorm{u}_{k+1} +
      \snorm{p}_{k}\bigr).
    \end{split}
  \end{gather}
\end{Corollary}
\begin{intro}
  We continue showing that the most natural discretizations
  in two dimensions are not inf-sup stable. This holds for the
  discretization using continuous linear or bilinear elements for both
  velocity components and the pressure as well as for continuous
  linear or bilinear velocity functions combined with piecewise
  constant pressure functions.
\end{intro}

\begin{example}
  We begin with a one-dimensional example. Piecewise linear velocity
  and piecewise linear pressure. Both continuous. Then, $\div v_h$ is
  piecewise constant. Consequently, a pressure function which has zero
  mean value on each cell is in the kernel of $B_h$.
  \begin{figure}[tp]
    \centering
    \includegraphics[width=.6\textwidth]{./fig/p1-p1-1d.tikz}
    \caption[Example for the $P_1-P_1$ pair in one
    dimension]{Piecewise linear pressure (\tikz\draw[color=cyan] (0,0)
      -- (1em,0);) and divergence (\tikz\draw[color=red] (0,0)
      -- (1em,0);) of
      piecewise linear velocity.}
    \label{fig:stokes:p1p1-1d}
  \end{figure}
\end{example}

\begin{example}
  Take a patch of four quadrilaterals or triangles meeting in a common
  vertex. Let $\domain$ be the union of these grid cells. Choose
  linear and bilinear shape functions for $V_h$, respectively. Then, $\dim
  V_h = 2$, since we have one interior vertex with one basis function for
  each velocity component. Choose piecewise constant pressure
  functions. Dividing out the global constant, we conclude that $\dim
  Q_h = 3$. Thus, the statement
  \begin{gather*}
    \forall q_h\in Q_h \;\exists v_h\in V_h:
    \quad \norm{v_h}_1 = \norm{q_h}_0
    \;\wedge\; b(v_h, q_h) \ge \beta \norm{q_h}^2
  \end{gather*}
  cannot hold true. Therefore, the inf-sup condition does not hold. In
  fact, $\ker{B_h} = \{0\}$.
  \begin{figure}[tp]
    \centering
    \includegraphics[width=.4\textwidth]{./fig/patch1.tikz}
    \hfill
    \includegraphics[width=.4\textwidth]{./fig/patch2.tikz}
    \caption[Very coarse meshes with Dirichlet boundary.]{Very coarse meshes with Dirichlet boundary. Degrees of freedom for pressure (\tikz\draw[shape pressure] (0,0) circle (1ex);) and for both velocity components(\tikz\draw[shape veloxy] (0,0) circle (1ex);).}
    \label{fig:stokes:example1}
  \end{figure}

  Thus, we conclude that for this combination of shape function
  spaces, there is a mesh such that they are not suited for the
  approximation of the Stokes problem. But, this may be a problem of a
  mesh with too few cells. In fact, asymptotically, a triangular mesh
  contains twice as many vertices as cells, a quadrilateral mesh as
  many. Therefore, $\dim V_h > \dim Q_h$ as soon as the mesh is
  sufficiently fine. Will this be sufficient?
\end{example}

\begin{Problem}{checker-board}
  Let $\domain = (0,1)^2$ be the unit square and let the mesh consist
  of Cartesian squares of side length $1/n$. Choose $V_h \subset V$
  based on bilinear shape functions. Show that the piecewise constant
  pressure function $p_c=\pm 1$ in a checkerboard fashion is in the
  kernel of $B_h^T$, that is
  \begin{gather*}
    b(v_h, p_c) = 0 \quad\forall v_h\in V_h.
  \end{gather*}
\end{Problem}

\subsection{The Mini element}

\begin{Definition}{h1-bubble-space}
  An $H^1$-\define{bubble function} on a mesh cell $\cell$ is a
  function $b\in H^1_0(\cell)$. A \define{bubble space} $b_\cell$ on
  $\cell$ is a discrete vector space of such bubble functions.  We
  denote the space of bubble functions on the mesh $\mesh_h$ by
  \begin{gather*}
    B_h(b_\cell) = \bigl\{ v\in H^1(\mesh_h) \big\vert
    v_{|_\cell} \in b_\cell \;\forall \cell\in\mesh_h
    \bigr\}.
  \end{gather*}
\end{Definition}

\begin{Definition}{barycentric-coordinates}
  A simplex $T\in \R^d$ with vertices $x_0,\dots,x_d$ is described by
  a set of $d+1$ \define{barycentric coordinates}
  $\lambda_0,\dots,\lambda_d$ such that
  \begin{xalignat}2
    0\le\lambda_i(x) &\le 1& i&=0,\dots,d;\quad x\in T\\
    \lambda_i(x_j) &= \delta_{ij}& i,j&=0,\dots,d\\
    \sum \lambda_i(x) &= 1.
  \end{xalignat}
\end{Definition}

\begin{remark}
  The functions $\lambda_i(x)$ are the shape functions of the linear
  $P_1$ element on $T$. They allow us to define basis functions on the
  cell $T$ without use of a reference element $\widehat T$.
  
  Note that $\lambda_i\equiv 0$ on the face opposite to the
  vertex $x_i$.
\end{remark}
\section{Nearly incompressible elasticity}

%%% Local Variables:
%%% mode: latex
%%% TeX-master: "main"
%%% End:
