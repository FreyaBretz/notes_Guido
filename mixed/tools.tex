\begin{Lemma*}{poincare-friedrichs}{Poincaré-Friedrichs inequalities}
  There is a constant $c_P$ only depending on the domain $\domain$
  such that the Poincaré inequality
  \begin{gather}
    \norm{v-\overline v}_{L^2(\domain)} \le c_P \norm{\nabla v}_{L^2(\domain)}
    \qquad \forall v\in H^1(\domain)
  \end{gather}
  holds. Furthermore, there is a constant also denoted as $c_P$ such that Friedrichs' inequality
  \begin{gather}
    \norm{v}_{L^2(\domain)} \le c_P \norm{\nabla v}_{L^2(\domain)}
    \qquad \forall v\in H^1_0(\domain)
  \end{gather}
  holds.
\end{Lemma*}

\begin{Assumption}{mapping-decomposition}
  For more general mappings $\Phi\colon \refcell\to \cell$, we
  make the assumption, that they can be decomposed into three factors,
  \begin{gather}
    \Phi = \Phi_O \circ \Phi_S \circ \Phi_W,
  \end{gather}
  where $\Phi_O$ is a combination of translation and rotation,
  $\Phi_S$ is a scaling with a characteristic length $h_T$, and
  $\Phi_W$ is a warping function not changing the characteristic length.
\end{Assumption}

\begin{Lemma*}{scaling-1}{Scaling lemma}
  Let the typical length of a cell $\cell$ be $h_\cell$. Assume there
  are constants $0 < M_\cell, m_\cell, d_\cell, D_\cell$, such that
  \begin{gather}
    \begin{split}
      \norm{\nabla\Phi_W(\refvx)} \le M_\cell,
      \\
      \norm{\nabla\Phi_W^{-1}(\refvx)} \le m_\cell^{-1} ,
      \\
      d^2_\cell \le \det \nabla\Phi_W(\refx)) \le D^2_\cell.      
    \end{split}
  \end{gather}
  for all $\refvx\in\refcell$. Then, for $k=0,1$ and a constant $c$
  \begin{gather}
    \begin{split}
      \snorm{\refu}_{k,\refcell}
      &\le c \frac{M_\cell}{d_\cell}  h_\cell^{k-\nicefrac d2}
      \snorm{u}_{k,\cell},\\
      \snorm{u}_{k,\cell}
      &\le c \frac{D_\cell}{m_\cell} h_\cell^{\nicefrac d2-k}
      \snorm{\refu}_{k,\refcell}.
    \end{split}
  \end{gather}
  This extends to higher derivatives under assumptions on higher
  derivatives of $\Phi_\cell$.
\end{Lemma*}
