\begin{Problem}{schur-complement}
Let $A\in\mathbb{R}^{n\times n}$, $B\in\mathbb{R}^{k\times n}$, $k\leq n$.
Moreover, assume that $B$ has full rank and that $A$ is symmetric and
positive definite.\\
Consider the problem
\begin{align}
\begin{aligned}
\begin{pmatrix} A & B^* \\ B & 0 \end{pmatrix}
\begin{pmatrix} x \\ y \end{pmatrix}
= \begin{pmatrix} F \\ G \end{pmatrix}
\end{aligned}
\tag{*}
\label{problem9-stokes}
\end{align}
\begin{enumerate}
\item Prove that then $S := BA^{-1}B^*$ is symmetric and positive definite, too.
How can this matrix be used to solve (\ref{problem9-stokes})?
\item Show that
\begin{align}
 P := I - B^*(BB^*)^{-1}B.
\end{align}
is a projector on the kernel of $B$ with $\norm{P}_2=1$.
\item Show for the case $G = 0$ that $x$ is a solution of
\begin{align}
PAPx = PF
\end{align}
if $(x,y)$ is a solution of (\ref{problem9-stokes}).
\end{enumerate}
\begin{solution}
\begin{enumerate}
 \item Let's first consider the symmetry:
  \begin{align}
    S^*=(BA^{-1}B^*)^*=B(A^*)^{-1}B^*=BA^{-1}B^*=S
  \end{align}
  We further observe
  \begin{align}
   \norm{v}_S^2 &= v^*Sv = v^*BA^{-1}B^*v = (B^*v)^*A^{-1}B^*v \leq 0 \\
   \norm{v}_S^2 &\Longleftrightarrow B^*v = 0 \Longleftrightarrow v=0
  \end{align}
  since $A$ and $B$ have full rank.

  The first equation implies
  \begin{align}
   x=A^{-1} (F-B^* y)
  \end{align}
  and eliminating $x$ in the second equation yields
  \begin{align}
   BA^{-1} (F-B^* y)=G \Longleftrightarrow BA^{-1} B^* y=BA^{-1} F-G.
   \end{align}

  \item \begin{align}P^2&=(I - B^*(BB^*)^{-1}B)(I - B^*(BB^*)^{-1}B)\\
            &=I - 2B^*(BB^*)^{-1}B+ B^*(BB^*)^{-1}BB^*(BB^*)^{-1}B\\
            &= I-B^*(BB^*)^{-1}B=P
        \end{align}
        Therefore, $\norm{P}_2=1$ if $\operatorname{ker} B\not=\{0\}$.
        Note that $k<n$ implies this.

        Further, $Px=x$ for all $x\in \operatorname{ker} B$ and $BPy=0$ for all $y$.
  \item Let $x$ be a solution of (\ref{problem9-stokes}) for $G=0$.
  \begin{align}
PAPx &= (I-B^*(BB^*)^{-1}B) A (I-B^*(BB^*)^{-1}B) x\\
&= (I-B^*(BB^*)^{-1}B) Ax\\
&= (I-B^*(BB^*)^{-1}B) (F-B^* y)\\
&= (I-B^*(BB^*)^{-1}B) F = PF
        \end{align}
\end{enumerate}
\end{solution}
\end{Problem}


%%% Local Variables:
%%% mode: latex
%%% TeX-master: "main"
%%% End:
