\begin{Definition}{spanned-simplex}
  Let $x_0,\dots,x_k$ for $k\le \sdim$ be a set of $k+1$ points in
  $\R^\sdim$. Then, we call the set of convex combinations of these points
  the $k$-simplex $f$ spanned by $\{x_0,\dots,x_k\}$.
\end{Definition}

\begin{Definition}{subsimplices}
  Let $\cell$ be the simplex in $\R^\sdim$ spanned by the points
  $x_0,\dots,x_\sdim$. Then, every ascending subset
  $\sigma \subset \{0,\dots,\sdim\}$ of length $k+1$ defines a
  $k$-dimensional \define{subsimplex} of $\cell$ denoted as $f_\sigma$.

  The set of all subsimplices of $\cell$, including $\cell$ itself is
  called es
  \index{Delta@$\Delta(\cell)$} $\Delta(\cell)$. The set of all $k$-dimensional subsimplices
  is \index{Delta k@$\Delta_k(\cell)$} $\Delta_k(\cell)$.
\end{Definition}

\begin{example}
  A $\sdim$-dimensional simplex $\cell$ has $\binom{\sdim+1}{k+1}$
  subsimplices of dimension $k$.

  The one-dimensional simplex $[x_0,x_1]$ has two subsimplices of
  dimension zero, namely the two points $x_0$ and $x_1$.

  The triangle spanned by the points $x_0, x_1, x_3$ has three
  one-dimensional subsimplices (edges) and three zero-dimensional
  subsimplices (vertices).

  The tetrahedron spanned by the points $x_0,\dots,x_3$ has
  \begin{itemize}
  \item 4 triangular faces,
  \item 6 edges,
  \item 4 vertices.
  \end{itemize}
\end{example}

\begin{Definition}{barycentric-coordinates}
  A simplex $T\in \R^\sdim$ with vertices $x_0,\dots,x_\sdim$ is described by
  a set of $\sdim+1$ \define{barycentric coordinates}
  $\lambda_0,\dots,\lambda_d$ such that
  \begin{xalignat}2
    0\le\lambda_i(x) &\le 1& i&=0,\dots,d;\quad x\in T\\
    \lambda_i(x_j) &= \delta_{ij}& i,j&=0,\dots,d\\
    \sum \lambda_i(x) &= 1.
  \end{xalignat}
\end{Definition}

\begin{remark}
  The functions $\lambda_i(x)$ are the shape functions of the linear
  $P_1$ element on $T$. They allow us to define basis functions on the
  cell $T$ without use of a reference element $\widehat T$.

  Note that $\lambda_i\equiv 0$ on the face opposite to the
  vertex $x_i$.
\end{remark}

\begin{example}
  We can use barycentric coordinates to define shape functions on
  simplicial meshes easily, as in
  Table~\ref{tab:barycentric-shapes}.
  \begin{table}[tp]
    \centering
    \begin{tabular}{|c|l|}
      \hline Degrees of freedom
      & Shape functions \\\hline
      \adjustbox{valign=center,margin=3pt}{\includegraphics[width=2cm]{./fig/p1-p.tikz}}
      &
        {\begin{minipage}[b]{6cm}
          \begin{gather}
            \phi_i = \lambda_i,
            \quad i=0,1,2
          \end{gather}
        \end{minipage}}
      \\\hline
      \adjustbox{valign=center,margin=3pt}{\includegraphics[width=2cm]{./fig/p2-p.tikz}}
      &
        {\begin{minipage}[b]{6cm}
          \begin{xalignat*}2
            \phi_{ii} &= 2\lambda_i^2 - \lambda_i,
            &i&=0,1,2\\
            \phi_{ij} &= 4\lambda_i\lambda_j
            &j&\neq i
          \end{xalignat*}
        \end{minipage}}
        \\\hline
      \adjustbox{valign=center,margin=3pt}{\includegraphics[width=2cm]{./fig/p3-p.tikz}}
      &
        {\begin{minipage}[b]{6cm}
          \begin{xalignat*}2
          \phi_{iii} &= \tfrac12 \lambda_i(3\lambda_i-1)(3\lambda_i-2)
          &i&=0,1,2\\
          \phi_{ij} &= \tfrac92\lambda_i\lambda_j(3\lambda_j-1)
          &j&\neq i\\
          \phi_0 &= 27\lambda_0\lambda_1\lambda_2
        \end{xalignat*}
        \end{minipage}}
        \\\hline
    \end{tabular}
    \caption{Degrees of freedom and shape functions of simplicial elements
      in terms of barycentric coordinates}
    \label{tab:barycentric-shapes}
  \end{table}
\end{example}

\begin{Lemma}{barycentric-subsimplex}
  Let $\cell$ be a simplex in $\R^\sdim$ with barycentric coordinates
  $\lambda_0,\dots,\lambda_\sdim$. Then, the subset
  $\lambda_{\sigma_1},\dots,\lambda_{\sigma_k}$ form the barycentric
  coordinates of the subsimplex $f_\sigma\in\Delta_k(\cell)$.
\end{Lemma}

\begin{Notation}{dual-barycentric}
  For a simplex $\cell\subset \R^\sdim$, the vectors $t_i = x_i-x_0$
  for $i=1,\dots,\sdim$ form a basis of $\R^n$. The dual basis in
  $\Alt^1\R^\sdim$ is denoted by $\dlambda_1,\dots,\dlambda_\sdim$.

  For a subsimplex $f_\sigma$ of dimension $k$, the vectors
  $\{t_{\sigma_1},\dots,t_{\sigma_k}\}$ span the tangent space
  $Tf_\sigma$ and the restrictions of $\{\dlambda_{\sigma_1},\dots,\dlambda_{\sigma_k}\}$ to this tangent space form a basis for $\Alt^1Tf_\sigma$.
\end{Notation}

\begin{Notation}{alt-k-barycentric}
  The algebraic $k$-forms
  \begin{gather}
    \dlambda_\sigma = \dlambda_{\sigma_1}\wedge\dots\wedge\dlambda_{\sigma_k},
  \end{gather}
  for $\sigma\in\Sigma(k,n)$ form a basis for $\Alt^k(\R^n)$. Thus,
  each differential form on a simplex $\cell$ can be written as
  \begin{gather}
    \omega = \sum_{\sigma\in\Sigma(k,n)} a_\sigma \dlambda_\sigma,
  \end{gather}
  with coefficient functions $a_\sigma\colon \cell\to \R$.
\end{Notation}



%%% Local Variables:
%%% mode: latex
%%% TeX-master: "main"
%%% End:
