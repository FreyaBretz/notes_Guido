\begin{intro}
  If we discretize the mixed formulation of almost incompressible
  elasticity with any of the stable Stokes pairs of the preceding
  sections, we can apply
  \slideref{Corollary}{stabilized-mixed-convergence} to obtain optimal
  error estimates. Nevertheless, our problem with almost
  incompressible elasticity was not the approximation of the pressure
  (which was introduced artificially anyway), but locking. So, how
  does the choice of an inf-sup stable pair avoid locking?
  
  Locking, in the terminology developed in the previous chapter, can
  be described as the fact that the kernel of the discrete divergence
  operator is too small, in the example presented even
  \begin{gather}
    \ker B_h = \{0\}.
  \end{gather}
  Note though, that there might be more subtle locking effects, where
  the approximation is reduced but not destroyed.

  In view of \slideref{Theorem}{galerkin-mixed-u-kerbh}, locking means
  that $\vV_h^g$ is too small or even the zero space, and therefore the
  quasi best-approximation result of this theorem is useless, since
  \begin{gather}
    \inf_{w_h\in \vV_h^g} \norm{\vu-w_h}_{1} \not\to 0
    \quad\text{as } h \to 0.
  \end{gather}
  
  The additional assumption of the inf-sup condition
  in\slideref{Theorem}{galerkin-mixed-p} on the other hand guarantees
  that an approximation of the kernel is possible, and thus, locking
  becomes impossible.
\end{intro}

\begin{Lemma}{reduced integration}
  Let $\vV_h\times Q_h$ be a stable pair for the Stokes problem
  admitting a Korn inequality. Let furthermore $\Pi_Q$ be the
  $L^2$-projection onto $Q_h$. Then, the solution $\vu_h\in \vV_h$ to the
  weak formulation
  \begin{gather}
    \label{eq:stokes:33}
    2\mu \form(\strain{\vu_h},\strain \vv) + \lambda (\Pi_Q \div
    \vu_h,\Pi_q\div \vv) = (f,v)
    \quad\forall \vv\in \vV_h,
  \end{gather}
  admits the quasi-optimality estimate
  \begin{gather}
    \norm{\vu-\vu_h}_1 \le c \sup_{\vv_h\in \vV_h}\norm{\vu-\vv_h}_1,
  \end{gather}
  with a constant $c$ independent of the quotient $\lambda/\mu$.
\end{Lemma}

\begin{proof}
  We introduce the auxiliary variable $p_h\in Q_h$ by the condition
  \begin{gather}
    \int_\domain \div \vu_h q_h \dx = \int_\domain p_h q_h\dx
    \quad\forall q_h\in Q_h.
  \end{gather}
  By definition of the $L^2$-projection, we have
  \begin{gather}
    \form(\div \vu_h, q_h) = \form(\Pi_Q\div \vu_h,q_h) = \form(p_h,q_h).
  \end{gather}
  Since $\Pi_Q\div \vu_h$ and $p_h$ are in the same space, this implies
  that $p_h = \Pi_Q\div \vu_h$ pointwise. In addition, we observe that
  \begin{gather}
    \form (p_h, \Pi_Q\div \vv) = \form (p_h, \div \vv).
  \end{gather}
  Hence, the formulation~\eqref{eq:stokes:33} is equivalent to
  \begin{gather}
    \arraycolsep2pt
    \begin{matrix}
      2\mu\form(\strain{\vu_h},\strain \vv) &+& \form(p_h,\div \vv)
      &=& \form(f,v) &\qquad&\forall \vv\in \vV_h\\
      \form(\div \vu_h, q) &-&\tfrac1\lambda\form(p_h,q)&=&0&&\forall q\in Q_h.
    \end{matrix}
  \end{gather}
  This is the Stokes problem augmented by a positive definite bilinear
  form $c(.,.)$, such that
  \slideref{Theorem}{mixed-stabilized-well-posed} and
  \slideref{Corollary}{stabilized-mixed-convergence} apply.
\end{proof}

\begin{remark}
  The technique in the previous lemma is often called \define{reduced
    integration}. This refers to the fact, that we can replace the
  explicit projection $\Pi_Q$ by using a quadrature formula which is
  exact on $Q_h$, but zero for all higher order polynomials occuring
  in $\div \vu_h$.
\end{remark}

%%% Local Variables:
%%% mode: latex
%%% TeX-master: "main"
%%% End:
