
\begin{intro}
  The key to the mixed analysis which is also underlying our
  quasi-best\-ap\-proxi\-mation result was a splitting of the solution
  process into the reduced problem for $u$ and then applying the
  inf-sup condition for $b(.,.)$ in order to estimate $p$. This way,
  we will be able to obtain estimates for the Stokes problem, but we
  have tacitly abandoned weakly compressible elasticity. Indeed, the
  mixed form of the Lamé-Navier equations is not a constrained
  minimization problem. In this section, we will fill the gap and
  derive estimates for the solution of this problem which are robust
  in $\lambda$.

  In the Lamé-Navier equations, we had
  \begin{gather}
    c(p,q) = -\tfrac1\lambda \scal(q,p)_{L^2(\domain)},
  \end{gather}
  which suggests assuming symmetric and $Q$-elliptic. But, we want
  estimates independent of $\lambda$! Therefore, we should only
  require semi-definite, which on the other hand turns out a bit too
  weak.
\end{intro}

\begin{Assumption}{mixed-elliptic-stabilized}
  In addition to \slideref{Assumption}{mixed-elliptic}, let $c(.,.)$
  be positive semi-definite and elliptic on $\ker{B^\transpose}$,
  \begin{gather}
    c(q,q) \ge u\quad\forall q\in Q,
    \qquad
    c(q,q) \ge \ellipc \norm{q}_Q^2 \quad\forall q\in \ker{B^\transpose}.
  \end{gather}
\end{Assumption}

\begin{remark}
  Again, this assumption is not necessary for the analysis, but it
  yields a convenient and useful theorem which goes far beyond weakly
  compressible elasticity and covers stabilized methods for spaces
  where the inf-sup condition for $b(.,.)$ does not hold for the whole
  space $Q$.
\end{remark}

\begin{Theorem}{mixed-stabilized-well-posed}
  Let \slideref{Assumption}{mixed-elliptic-stabilized} hold and
  let $a(.,.)$ and $c(.,)$ be symmetric. In addition, let there be
  $\infsupc>0$ such that
  \begin{gather}
    \begin{split}
      \inf_{q\in \ortho{\ker{B^\transpose}}}  \sup_{v\in V}
      \frac{b(v,q)}{\norm{v}_V\norm{q}_Q} &\ge \infsupc\\
      \inf_{v\in \ortho{\ker{B}}}  \sup_{q\in Q}
      \frac{b(v,q)}{\norm{v}_V\norm{q}_Q} &\ge \infsupc
    \end{split}
  \end{gather}
  Then, the problem finding $(u,p)\in V\times Q$ such that
  \begin{multline}
    a(u,v) + b(v,p) + b(u,q) - c(p,q) = f(v)+g(q)
    \\
    \forall v\in V, q\in Q
  \end{multline}
  has a unique solution for all $f\in V^*$ and $g\in Q^*$ and there is
  a constant $C$ such that
  \begin{gather}
    \norm{u}_V+\norm{p}_Q
    \le C \bigl(\norm{f}_{V^*} + \norm{g}_{Q^*}\bigr).
  \end{gather}
\end{Theorem}

\begin{proof}
  First, note that the ellipticity assumptions as well as the inf-sup
  conditions are symmetric in $V$ and $Q$. Indeed, replacing the test
  functions and the form $b(.,.)$ by their negatives, we can transform
  the problem into one where $V$ and $Q$ have exchanged their
  roles. Thus, it is sufficient to show well-posedness for $f=0$. The
  same result then holds for $g=0$ and it holds for both nonzero by
  linearity.
  
  We note that by the inf-sup conditions both $\range B$ and
  $\range{B^\transpose}$ are closed. Thus, we can decompose $u=u^0+u^\perp$
  with $u^0\in\ker B$ and $u^\perp$ in its orthogonal
  complement. Assuming $f=0$ and testing with $q=0$ we obtain the
  equation
  \begin{gather}
    \label{eq:infsup:14}
    a(u,v) = -b(v,p) = 0.
  \end{gather}
  In particular, testing with $v=u^0$ yields
  \begin{gather}
    a(u,u^0) = -b(u^0,p) = 0.
  \end{gather}
  Hence,
  \begin{gather}
    \ellipa \norm{u^0}^2_V \le a(u^0,u^0)
    = -a(u^\perp,u^0)
    \le \bounda \norm{u^\perp}\norm{u^0},
  \end{gather}
  which implies
  \begin{gather}
    \label{eq:infsup:15}
    \norm{u^0}_V \le \frac{\bounda}{\ellipa} \norm{u^\perp}_V.
  \end{gather}
  Testing with $v=u$ and $q=-p$ yields
  \begin{gather}
    a(u,u)+c(p,p) \le g(p) = g^0(p^0) + g^\perp(p^\perp),
  \end{gather}
  where $p^0\in \ker{B^\transpose}$, $g^\perp\in \polar{\ker{B^\transpose}}$ and the
  other two are defined by orthogonality in $Q$ and $Q^*$,
  respectively. Let first $g^0=0$. Then, by~\eqref{eq:infsup:14} and
  the inf-sup condition for $p$, there is $v\in V$ with $\norm{v}_V = 1$
  such that
  \begin{gather}
    \infsupc \norm{p^\perp} \le
    \abs{b(v,p^\perp)} = \abs{b(v,p)} = \abs{a(u,v)}
    \le \sqrt{a(u,u)}\sqrt{a(v,v)},
  \end{gather}
  by the Bunyakovsky-Cauchy-Schwarz inequality for symmetric bilinear
  forms.  Therefore, squaring and using the definition of the operator
  norm of $g^\perp$ yields
  \begin{gather}
    \norm{p^\perp}_Q \le \frac{\bounda}{\infsupc^2}\norm{g^\perp}_{Q^*}.
  \end{gather}
  Furthermore, we have
  \begin{gather}
    c(p,p^0) = b(u,p^0) - g^\perp(p^0) = 0.
  \end{gather}
  Hence,
  \begin{gather}
    \ellipc \norm{p^0}_Q^2 \le c(p^0,p^0)
    = c(p^\perp,p^0) \le \norm c \norm{p^0}_Q \norm{p^\perp}_Q,
  \end{gather}
  concluding
  \begin{gather}
    \norm{p^0}_Q
    \le \frac{\bounda \norm c}{\ellipc\infsupc^2}
    \norm{g^\perp}_{Q^*}.
  \end{gather}

  We continue our proof for $g^\perp=0$ and $g^0 \neq 0$. Testing with
  $q=p^0$, we obtain
  \begin{align}
    c(p^0,p^0)
    &= c(p,p^0) - c(p^\perp,p^0) \\
    &= b(u,p^0) - g^0(p^0) - c(p^\perp,p^0) \\
    &\le \norm{g^0}_{Q^*} \norm{p^0}_Q
      + \norm c \norm{p^\perp}_Q \norm{p^0}_Q,
  \end{align}
  yielding
  \begin{gather}
    \norm{p^0}_Q \le \frac1{\ellipc}
    \left(\norm{g^0}_{Q^*} + \norm c \norm{p^\perp}_Q\right).
  \end{gather}
  $p^\perp$ is estimated as before by
  \begin{align}
    \norm{p^\perp}_Q^2
    &\le \frac{\bounda}{\infsupc^2}\norm{g^0}_{Q^*} \norm{p^0}_Q \\
    &\le \frac{\bounda}{\ellipc\infsupc^2} \norm{g^0}_{Q^*}^2
      + \frac{\bounda\norm c}{\ellipc\infsupc^2}
      \norm{g^0}_{Q^*}\norm{p^\perp}_Q\\
    &\le \frac{\bounda}{\ellipc\infsupc^2} \norm{g^0}_{Q^*}^2
      + \frac12 \norm{p^\perp}_Q^2
      + \frac{\bounda^2 \norm c^2}{2{\ellipc}^2\infsupc^4} \norm{g^0}_{Q^*}^2.
  \end{align}
  We conclude that $p^\perp$ and $p^0$ are bounded by $g^0$. Summing
  up, we obtain for $f=0$ and $g\in Q^*$ the estimate
  \begin{gather}
    \norm{p}_Q \le c \norm{g}_{Q^*}.
  \end{gather}

  We estimate $u^0$ by $u^\perp$ using~\eqref{eq:infsup:15} and
  $u^\perp$ by the inf-sup condition, choosing $q\in Q$ with
  $\norm{q}_Q = 1$ such that
  \begin{gather}
    \infsupc\norm{u^\perp} = b(u,q) = c(p,q) + g(q)
    \le \norm{c}\norm{p}_Q +  \norm{g}_{Q^*}.
  \end{gather}
  Thus, we have estimated all components of the solution by the norm
  of $g$, assuming $f=0$. Now we conclude the proof by reverting the
  roles of $u$ and $p$, respectively $f$ and $g$.
\end{proof}

\begin{intro}
  The extension of \slideref{Theorem}{galerkin-mixed-p} to the
  saddle-point problem of \slideref{Definition}{saddle-point-abstract}
  with bilinear form $c(.,.)$ is even more cumbersome than this
  theorem. Nevertheless, the use of residual operators as a technique
  to structure the proof of convergence is instructive and may come
  handy at some point.
\end{intro}

\begin{Definition}{mixed-residual}
  For the saddle-point problem
  \begin{gather}
    a(u,v) + b(v,p) + b(u,q) - c(p,q),
  \end{gather}
  and functions $w_h\in V_h$ and $r_h\in Q_h$ we introduce the the
  residual operators $R_f \in V_h^*$ and $R_g\in Q_h^*$ as
  \begin{gather}
    \begin{split}
      R_f(v_h) &= a(u-w_h, v_h) + b(v_h, p-r_h) \\
      R_g(q_h) &= b(u-w_h, q_h) - c(p-r_h, q_h).
    \end{split}
  \end{gather}
\end{Definition}

\begin{Corollary}{mixed-residual-bounded}
  Under \slideref{Assumption}{mixed-elliptic-stabilized}, we have
  \begin{gather}
    \begin{split}
      \abs{R_f(v_h)}
      &\le \bigl(\bounda \norm{u-w_h}_V + \boundb \norm{p-r_h}\bigr)
      \norm{v_h}_V
      \\
      \abs{R_g(q_h)}
      &\le \bigl(\boundb \norm{u-w_h}_V + \norm c \norm{p-r_h}\bigr)
      \norm{q_h}_Q.
    \end{split}
  \end{gather}
\end{Corollary}

\begin{Lemma}{stabilized-mixed-approximation}
  Let the assumptions of
  \slideref{Theorem}{mixed-stabilized-well-posed} hold. Then, there
  are constants $c_1$ to $c_4$ independent of the solutions $u$, $p$,
  $u_h$, and $p_h$ and the discretization parameter $h$, such that for
  any $v_h\in V_h$ and $q_h\in Q_h$
  \begin{gather}
    \label{eq:infsup:16}
    \begin{split}
      \norm{u_h-v_h} &\le c_1 \norm{R_f}_{V_h^*} + c_2
      \norm{R_g}_{Q_h^*}
      \\
      \norm{p_h-q_h} &\le c_3 \norm{R_f}_{V_h^*} + c_4
      \norm{R_g}_{Q_h^*}.
    \end{split}
  \end{gather}
\end{Lemma}

\begin{proof}
  The proof is lengthy and follows the lines of the proof of
  well-posedness for
  \slideref{Theorem}{mixed-stabilized-well-posed}. It is obtained by
  considering the components $u_h^0-v_h^0$ and $u_h^\perp-v_h^\perp$
  as well as $p_h^0-q_h^0$ and $p_h^\perp-q_h^\perp$ separately.
\end{proof}

\begin{intro}
  In spite of the bad treatment the proof of the previous lemma
  received in these notes, it contains the main parts of the
  convergence proof, and whenever a saddle-point problem including
  $c(.,.)$ is solved, it has to be reproduced. It is just the fact
  that the proof is overwhelmingly technical that led to the decision
  to leave this experience to the first time the reader actually needs
  this result.
\end{intro}

\begin{Corollary}{stabilized-mixed-convergence}
    Let the assumptions of
  \slideref{Theorem}{mixed-stabilized-well-posed} hold. Then, there
  are constants $c_1$ to $c_4$ independent of the solutions $u$, $p$,
  $u_h$, and $p_h$ and the discretization parameter $h$, such that
  \begin{gather}
    \label{eq:infsup:17}
    \begin{split}
      \norm{u-u_h}
      &\le c_1 \inf_{v_h\in V_h}\norm{u-v_h}_V
      + c_2 \inf_{q_h\in Q_h} \norm{p-q_h}_Q
      \\
      \norm{p-p_h}
      &\le c_3 \inf_{v_h\in V_h}\norm{u-v_h}_V
      + c_4 \inf_{q_h\in Q_h} \norm{p-q_h}_Q.
    \end{split}
  \end{gather}
\end{Corollary}

\begin{proof}
  The proof begins with the standard approach with triangle inequality
  \begin{align}
    \norm{u-u_h} &\le \norm{u-v_h} + \norm{v_h-u_h} \\
    \norm{p-p_h} &\le \norm{p-q_h} + \norm{q_h-p_h}.
  \end{align}
  Then, we employ \slideref{Lemma}{stabilized-mixed-approximation} on
  the terms on the right and use the estimate of
  \slideref{Corollary}{mixed-residual-bounded}.
\end{proof}

