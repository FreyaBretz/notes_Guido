\begin{Notation}{combinations}
  The set of \putindex{combination}s of $k$ elements from the set
  $\{1,\dots,n\}$ is denoted by $\Sigma(k,n)$. An element
  $\sigma\in\Sigma(k,n)$ is represented by a rising sequence
  \begin{gather}
    \sigma = (\sigma_1,\dots,\sigma_k),
    \qquad
    1\le \sigma_1 < \dots < \sigma_k \le n.
  \end{gather}
  By $\overline\sigma \in\Sigma(n-k,n)$ we denote the complement of
  $\sigma$ in the set $\{1,\dots,n\}$.

  An alternative notation for $\sigma$ is its \define{characteristic vector}
  \begin{gather}
    \chi_\sigma\in \R^n,
    \qquad
    \chi_i =
    \begin{cases}
      1 & i\in\sigma\\
      0& i\not\in\sigma.
    \end{cases}
  \end{gather}
  
  We also introduce the set $\Sigma_0(k,n)$ denoting the combinations
  $(\sigma_0,\dots,\sigma_k)$ of $k+1$ values from the set
  $\{0,\dots,n\}$.
\end{Notation}
