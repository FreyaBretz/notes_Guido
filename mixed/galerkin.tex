\begin{intro}
  The Galerkin approximation of mixed problems starts out the same way
  as for elliptic problems, namely, choose discrete subspaces
  $V_h\subset V$ and $Q_h \subset Q$. There is a fundamental
  difference though: the inf-sup condition is not inherited
  automatically on the subspaces like $V$-ellipticity. It actually becomes
  an additional requirement on the choice of $V_h$ and
  $Q_h$. We will thus work our way in several steps towards the final
  result.
\end{intro}

\begin{Definition}{mixed-galerkin}
  Let $V_h\subset V$ and $Q_h\subset Q$. Then, the \define{Galerkin
    approximation} of the mixed problem in
  \slideref{Definition}{mixed-weak} is: find
  $(u_h, p_h)\in V_h\times Q_h$ such that
  \begin{gather}
    \label{eq:galerkin:3}
        \arraycolsep.1em
    \begin{matrix}
      a(u_h,v_h) &+& b(v_h,p_h) &=& f(v_h) &\quad&\forall v_h\in V_h, \\
      b(u_h,q_h) && &=& g(q_h) &&\forall q_h\in Q_h.
    \end{matrix}
  \end{gather}
\end{Definition}

\begin{Definition}{kerbh}
  Let $V_h\subset V$ and $Q_h\subset Q$. Then, we define the subspace
  \begin{gather}
    \label{eq:galerkin:1}
    V_h^0 = \ker{B_h} = \bigl\{v_h\in V_h \big|
    b(v_h, q_h) = 0 \quad\forall q_h\in Q_h
    \bigr\}.
  \end{gather}
  Furthermore, we define the affine space
  \begin{gather}
    \label{eq:galerkin:2}
    V_h^g = \bigl\{v_h\in V_h \big|
    b(v_h, q_h) = g(q) \quad\forall q_h\in Q_h
    \bigr\}.
  \end{gather}
  The discrete reduced problem is: find $u_h\in V_h^g$ such that
  \begin{gather}
    \label{eq:galerkin:4}
    a(u_h, v_h) = f(v_h), \qquad\forall v_h \in \ker{B_h}.
  \end{gather}
\end{Definition}


\begin{Theorem}{galerkin-mixed-u-kerbh}
  Let $V_h^g$ be nonempty and let $a(.,.)$ and $b(.,.)$ be bounded
  with $\bounda$ and $\boundb$, respectively, and let there be constant and $\gamma_h$
  such that
  \begin{gather}
    \label{eq:galerkin:5}
    a(v_h, v_h) \ge \gamma_h \norm{v_h}^2_V,
    \qquad\forall v_h\in \ker{B_h}.
  \end{gather}
  Let furthermore the continuous mixed problem be well-posed with
  solution $(u,p)\in V\times Q$.
  Then, the discrete reduced problem~\eqref{eq:galerkin:4} has a
  unique solution $u_h\in V_h$ and there holds
  \begin{gather}
    \label{eq:galerkin:8}
    \norm{u-u_h}_V \le \left(1+\frac{\bounda}{\gamma_h}\right)
    \inf_{w_h\in V_h^g}\norm{u-w_h}_V
    + \frac{\boundb}{\gamma_h}
    \inf_{q_h\in Q_h}\norm{p-q_h}_Q
  \end{gather}
\end{Theorem}

\begin{proof}
  Let $u_h^g \in V_h^g$ arbitrary. By the ellipticity assumption,
  there is a unique function $u_h^0\in \ker{B_h}$ such that
  \begin{gather}
    a(u_h^0,v_h) = f(v_h) - a(u_h^g,v_h),
    \qquad\forall v_h\in \ker{B_h}.
  \end{gather}
  Hence, $u_h = u_h^g + u_h^0$ is the unique solution
  to~\eqref{eq:galerkin:4}. Choose now $w_h\in V_h^g$
  arbitrarily. Then, $v_h = u_h-w_h\in \ker{B_h}$ and using
  \begin{gather}
    f(v_h) = a(u, v_h) - b(v_h, p),
  \end{gather}
  we obtain
  \begin{gather}
    \begin{split}
      \label{eq:galerkin:9}
      a(v_h, v_h)
      &= f(v_h) - a(u_h-v_h, v_h) \\
      &= a(u-w_h, v_h) - b(v_h, p) \\
      &= a(u-w_h, v_h) - b(v_h, p-q_h)
    \end{split}
  \end{gather}
  for any $q_h\in Q_h$, yielding
  \begin{gather}
    \gamma_h\norm{v_h}^2_V
    \le \abs{a(v_h, v_h)}
    \le \bounda \norm{u-w_h}_V \norm{v_h}_V
      + \boundb \norm{p-q_h}_Q \norm{v_h}_V.
  \end{gather}
  We conclude by the standard triangle inequality argument
  \begin{multline}
    \norm{u-u_h}_V \le \norm{u-w_h}_V + \norm{u_h-w_h}_V
    \\
    \le \norm{u-w_h}_V + \frac{\bounda}{\gamma_h} \norm{u-w_h}_V
    + \frac{\boundb}{\gamma_h} \norm{p-q_h}_Q.
  \end{multline}
  The estimate follows since $w_h\in V_h^g$ and $q_h\in Q_h$ were
  chosen arbitrarily.
\end{proof}

\begin{Corollary}{galerkin-mixed-u-kerb}
  If in addition to the assumptions of
  \slideref{Theorem}{galerkin-mixed-u-kerbh} there holds
  \begin{gather}
    \label{eq:galerkin:6}
    \ker{B_h}\subset \ker B,
  \end{gather}
  then
  \begin{gather}
    \label{eq:galerkin:7}
    \norm{u-u_h}_V \le \left(1+\frac{\bounda}{\gamma_h}\right)
    \inf_{w_h\in V_h^g}\norm{u-w_h}_V
  \end{gather}
\end{Corollary}

\begin{proof}
  Consider that in equation~\eqref{eq:galerkin:9} there holds
  $v_h\in\ker B$.
\end{proof}

\begin{remark}
  Note that we used ellipticity of $a(.,.)$ on the subspace
  $\ker{B_h}$ for the discrete problem and on $\ker B$ for the
  continuous problem. If $\ker{B_h}\not\subset\ker{B}$, then
  ellipticity on $\ker B$ like in \slideref{Theorem}{mixed-elliptic}
  is not sufficient for well-posedness of the discrete problem.
  
  In practice, we will encounter two situations: either ellipticity
  holds on the whole space $V$ (Stokes equations) or
  $\ker{B_h}\subset\ker{B}$ (the other examples in these notes).
\end{remark}

\begin{Theorem}{galerkin-mixed-p}
  Assume in addition the assumptions of
  \slideref{Theorem}{galerkin-mixed-u-kerbh} that there are constants
  $\infsupc_h$, possibly depending on the parameter $h$, such that
  \begin{gather}
    \label{eq:galerkin:10}
    \inf_{q_h\in Q_h} \sup_{v_h\in V_h}
    \frac{b(v_h, q_h)}{\norm{v_h}_V\norm{q_h}_Q}
    \ge \infsupc_h.
  \end{gather}
  Then, there is a unique solution $p_h\in Q_h$ such that $(u_h, p_h)$
  is the unique solution to the discrete mixed
  problem~\eqref{eq:galerkin:3}. There are a constants $c_h^{(i)}$
  only depending on $\bounda$, $\boundb$, $\gamma_h$ and $\infsupc_h$
  such that
  \begin{align}
    \label{eq:galerkin:11}
    \norm{u-u_h}_V
    &\le c_h^{(1)} \inf_{v_h\in V_h}\norm{u-v_h}_V
    + c_h^{(2)} \inf_{q_h\in Q_h}\norm{p-q_h}_Q \\
    \norm{p-p_h}_Q
    &\le c_h^{(3)} \inf_{v_h\in V_h}\norm{u-v_h}_V
    + c_h^{(4)} \inf_{q_h\in Q_h}\norm{p-q_h}_Q.
  \end{align}
\end{Theorem}

\begin{proof}
  Applying \slideref{Theorem}{infsup-mixed2} to the discrete
  problem~\eqref{eq:galerkin:3}, we conclude that there is a unique
  solution $(u_h,p_h)\in V_h\times Q_h$. Let us begin estimating the
  error by establishing the bound
  \begin{gather}
    \label{eq:galerkin:12}
    \inf_{w_h\in V_h^g} \norm{u-w_h}_V
    \le \left(1+\frac{\boundb}{\infsupc_h}\right)
    \inf_{v_h\in V_h} \norm{u-v_h}_V.
  \end{gather}
  By the third condition in
  \slideref{Theorem}{infsup-well-equivalence}, there is a unique
  $z_h\in \ortho{\ker{B_h}}$ such that
  \begin{gather}
    B_h z_h = B_h(u-v_h),\qquad \forall v_h\in V_h,
  \end{gather}
  and
  \begin{gather}
    \norm{z_h}_V \le \frac1{\infsupc_h}\norm{B_h(u-v_h)}_{Q_h^*}
    \le\frac{\boundb}{\infsupc_h} \norm{u-v_h}_V.
  \end{gather}
  Let $w_h = z_h+v_h$. Then, $w_h\in V_h^g$ since
  \begin{gather}
    b(w_h, q_h) = b(u-v_h, q_h) = b(u, q_h) = g(q_h),
    \qquad\forall q_h\in Q_h.
  \end{gather}
  Furthermore,
  \begin{gather}
    \norm{u-w_h}_V \le \norm{u-v_h}_V + \norm{z_h}_V
    \le \left(1+\frac{\boundb}{\infsupc_h}\right)
    \norm{u-v_h}_V.
  \end{gather}
  Since $v_h \in V_h$ was chosen arbitrarily, we have
  proven~\eqref{eq:galerkin:12} and thus by
  \slideref{Theorem}{galerkin-mixed-u-kerbh} the estimate for
  $\norm{u-u_h}_V$ with
  \begin{gather}
    c_h^{(1)} = \left(1+\frac{\boundb}{\infsupc_h}\right)
    \left(1+\frac{\bounda}{\gamma_h}\right),
    \qquad
    c_h^{(2)} = \left(1+\frac{\boundb}{\infsupc_h}\right)
    \frac{\boundb}{\gamma_h}.
  \end{gather}
  It remains to prove the estimate for $\norm{p-p_h}_Q$. Using Galerkin
  orthogonality for the test function $v_h$, we obtain
  \begin{gather}
    \label{eq:galerkin:13}
    a(u-u_h, v_h) + b(v_h, p-p_h) = 0.
  \end{gather}
  Hence, for any $q_h\in Q_h$ there is by the inf-sup condition
  $v_h\in V_h$ with $\norm{v_h}_V=1$ such that
  \begin{align}
    \infsupc_h \norm{p_h-q_h}_Q
    &\le b(v_h, p_h-q_h)\\
    & = a(u-u_h, v_h) + b(v_h, p-q_h)\\
    &\le \bounda \norm{u-u_h}_V + \boundb \norm{p-q_h}_Q.
  \end{align}
  Again, the estimate for $\norm{p-p_h}_Q$ follows by triangle
  inequality.
\end{proof}

\begin{remark}
  Indeed, if~\eqref{eq:galerkin:10} holds, we do not have to require
  that $V_h^g$ is nonempty anymore, since $B_h\colon V_h\to Q_h^*$ is
  surjective.
\end{remark}

\begin{remark}
  We purposely proved the preceding theorems with $\gamma_h$ and
  $\infsupc_h$ depending on the parameter $h$, typically the mesh
  size. This is the minimal condition for well-posedness of the
  discrete problems. Nevertheless, this well-posedness is not uniform
  in $h$, which causes loss of approximation, as the following problem
  shows. Therefore, we will only be satisfied with uniform inf-sup
  constants in applications.
\end{remark}

\begin{Problem}{infsup-uniform}
  Let the following interpolation estimates hold:
  \begin{gather}
    \inf_{v_h\in V_h}\norm{u-v_h}_V = \mathcal O(h^k),
    \qquad
    \inf_{q_h\in Q_h}\norm{p-q_h}_Q = \mathcal O(h^k).
  \end{gather}
  Then, the estimates in \slideref{Theorem}{galerkin-mixed-p}
  are asymptotically optimal if and only if there are
  constants $\tilde \gamma>0$ and $\tilde \infsupc>0$ independent of
  $h$ such that
  \begin{gather}
    \label{eq:galerkin:14}
    \gamma_h\ge \tilde\gamma,\quad\infsupc_h\ge\tilde\infsupc,
  \end{gather}
  independent of $h$.
\begin{solution}
  The estimates in \slideref{Theorem}{galerkin-mixed-p} state
  \begin{align}
    \norm{u-u_h}_V &\le c_h^{(1)} \inf_{v_h\in V_h^g}\norm{u-v_h}_V
	      + c_h^{(2)} \inf_{q_h\in Q_h}\norm{p-q_h}_Q \\
    \norm{p-p_h}_Q &\le c_h^{(3)} \inf_{v_h\in V_h}\norm{u-v_h}_V
	      + c_h^{(4)} \inf_{q_h\in Q_h}\norm{p-q_h}_Q.
  \end{align}
  such that the additional assumptions imply
  \begin{align}
    \norm{u-u_h}_V &\le c_h^{(1)} C h^k + c_h^{(2)} C h^k \\
    \norm{p-p_h}_Q &\le c_h^{(3)} C h^k + c_h^{(4)} C h^k.
  \end{align}
    where the first constants are given by
  \begin{align}
    c_h^{(1)} = \left(1+\frac{\boundb}{\infsupc_h}\right)
    \left(1+\frac{\bounda}{\gamma_h}\right),
    \qquad
    c_h^{(2)} = \left(1+\frac{\boundb}{\infsupc_h}\right)
    \frac{\boundb}{\gamma_h}.
  \end{align}
  For the last two estimates, the relevant estimates read
  \begin{align}
    \infsupc_h \norm{p_h-q_h}_Q
    &\le b(v_h,p_h-q_h)\\
    & = a(u-u_h, v_h) + b(v_h, p-q_h)\\
    &\le \bounda \norm{u-u_h}_V + \boundb \norm{p-q_h}_Q\\
    \Rightarrow \norm{p-p_h}_Q
    &\le \norm{p-q_h} + \norm{p_h-q_h}_Q \\
    &\le \frac{\bounda}{\infsupc_h} \norm{u-u_h}_V + \left(1+\frac{\boundb}{\infsupc_h}\right) \norm{p-q_h}_Q.
  \end{align}
  Therefore, the remaining constants are
  \begin{align}
    c_h^{(3)} = \frac{\bounda}{\infsupc_h}c_h^{(1)}, \qquad
    c_h^{(4)} = \frac{\bounda}{\infsupc_h}c_h^{(2)}+\left(1+\frac{\boundb}{\infsupc}\right).
  \end{align}
\end{solution}
\end{Problem}

\begin{intro}
  As we can see from the form
  \begin{gather}
    \forall q_h\in Q_h \;
    \exists v_h\in V_h\colon
    \quad B_h v_h = q_h
    \quad\wedge\quad
    \norm{v_h}_V \le \norm{q_h}_Q,
  \end{gather}
  the uniform, discrete inf-sup condition introduces a compatibility
  condition between the spaces $V_h$ and $Q_h$. An immediate necessary
  condition is
  \begin{gather}
    \dim V_h \ge \dim Q_h.
  \end{gather}
  We often say that the space $V_h$ is ``rich enough'' to control
  functions in $Q_h$. Obviously, counting dimensions is not
  sufficient, since we could have added basis functions in
  $\ker{B_h}$. Even the condition
  \begin{gather}
    \dim\ortho{\ker{B_h}} = \dim Q_h
  \end{gather}
  is necessary and sufficient only for the existence of an inf-sup
  constant $\infsupc_h$ depending on $h$. Therefore, we need a stronger
  argument in order to ensure compatibility of the discrete
  spaces. Such an argument is the following lemma by Fortin. The
  projection operator $\Pi_{V_h}$ introduced there is usually referred
  to as \define{Fortin projection}.
\end{intro}

\begin{Lemma}{fortin}
  Let the inf-sup condition for the bilinear form $b(.,.)$
  hold on $V\times Q$ with a constant
  $\infsupc>0$. Then, it holds on $V_h\times Q_h$ uniformly with a
  constant $\tilde\infsupc>0$ if and only if there exists a linear
  operator $\Pi_{V_h}\colon V\to V_h$ satisfying for any $v\in V$
  \begin{align}
    \label{eq:galerkin:15}
    b(v-\Pi_{V_h}v,q_h) &= 0,\qquad\forall q_h\in Q_h,
    \\
    \label{eq:galerkin:16}
    \norm{\Pi_{V_h}v}_V \le c \norm{v}_V,
  \end{align}
  with $c$ independent of $h$. There holds $\tilde\infsupc = \infsupc/c$.
\end{Lemma}

\begin{proof}
  Assume first that $\Pi_{V_h}$ exists. then, there holds for any
  $q_h\in Q_h$
  \begin{gather}
    \sup_{v_h\in V_h} \frac{b(v_h, q_h)}{\norm{v_h}_V}
    \ge
    \sup_{v\in V} \frac{b(\Pi_{V_h}v, q_h)}{\norm{\Pi_{V_h} v}_V}
    =
    \sup_{v\in V} \frac{b(v, q_h)}{\norm{\Pi_{V_h} v}_V}
    \ge \frac{\infsupc}{c} \norm{q_h}_Q.
  \end{gather}
  Conversely, we assume the existence of a uniform, discrete inf-sup
  constant $\tilde\infsupc>0$. Then, for any $v\in V$ let
  $g(.) = b(v,.) \in Q_h^*$. By
  \slideref{Theorem}{infsup-well-equivalence}, there is a unique
  element $\Pi_{V_h} v \in \ortho{\ker{B_h}}$ such that
  \begin{gather}
    b(\Pi_{V_h}v,q_h) = b(v, q_h),\qquad\forall q_h\in Q_h
  \end{gather}
  and
  \begin{gather}
    \norm{\Pi_{V_h}v}_V
    \le \frac1{\tilde\infsupc}\norm{B_h v}_{Q_H^*}
    \le \frac{\boundb}{\tilde\infsupc} \norm{v}_V.
  \end{gather}
  Thus, $\Pi_{V_h}$ is bounded and~\eqref{eq:galerkin:15} holds with
  $c=\boundb/\tilde\infsupc$.
\end{proof}


%%% Local Variables:
%%% mode: latex
%%% TeX-master: "main"
%%% End:

