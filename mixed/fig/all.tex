\documentclass{article}
\usepackage{tikz,tikzscale}
\usetikzlibrary{calc}
\usetikzlibrary{math}
\usetikzlibrary{arrows.meta}
\usetikzlibrary{shapes.geometric}
\tikzset{velox/.style={color=black,draw,fill=red,thick,%
    shape=diamond,aspect=.4,
    inner sep=1.3pt,transform shape}}
\tikzset{veloy/.style={color=black,draw,fill=red,thick,%
    shape=diamond,aspect=2.5,
    inner sep=1.3pt,transform shape}}
\tikzset{veloxy/.style={color=black,draw,fill=red,thick,%
    shape=star,star points=4,star point ratio=2.2,
    inner sep=1.3pt,transform shape}}
\tikzset{pressure/.style={color=black,draw,fill=cyan,thick,%
    shape=circle,inner sep=2pt,transform shape}}
\tikzset{velo/.style={transform shape,double=red,arrows={-Stealth[open,fill=red]}}}

%% Macros for drawing degrees of freedom for different shapes/elements.
%% Arguments are always:
%%   #1: Starting point
%%   #2: End point
%%   #3: polynomial degree
%%   #4: node settings

\tikzset{pics/edgenormal/.style args={#1/#2/#3/#4}{%
    code={%
      \draw #1 -- #2
      node foreach \x [evaluate=\x as \xval] in {1,...,#3} [#4,sloped,pos=\xval/(#3+1)] {};
      }
}}


%% Macros for drawing degrees of freedom for different shapes/elements.
%% Arguments are always:
%%   #1: polynomial degree
%%   #2: node settings

\tikzset{pics/tripile/.style args={#1/#2}{%
    code={%
      \coordinate (top) at (0,#1);
      \foreach \i in{0,...,#1}
      \foreach \j in{0,...,\i}
      {
        \tikzmath{
          \y = .3*(2/3*#1-\i)*cos(30);
          \x = .3*(\i/2-\j);
        }
        \node[#2] at (\x,\y) {};
      }
    }
}}

\tikzset{pics/tensor/.style args={#1/#2/#3}{%
    code={%
      \coordinate (top) at (0,#1);
      \foreach \i in{0,...,#1}
      \foreach \j in{0,...,#2}
      {
        \tikzmath{
          \y = 2*(\i+1)/(#1+2);
          \x = 2*(\j+1)/(#2+2);
        }
        \node[#3] at (\x,\y) {};
      }
    }
}}

\tikzset{pics/pfem/.style args={#1/#2}{%
    code={%
      \tikzmath{ \ytop=2*cos(30); }
      \coordinate (top) at (0,\ytop);

      \foreach \i in{0,...,#1}
      \foreach \j in{0,...,\i}
      {
        \tikzmath{
          \y = \ytop-\ytop*\i/#1;
          \x = 2*(\i/2-\j)/#1+1;
        }
        \node[#2] at (\x,\y) {};
      }
    }
}}

\tikzset{pics/qfem/.style args={#1/#2}{%
    code={%
      \foreach \i in{0,...,#1}
      \foreach \j in{0,...,#1}
      {
        \tikzmath{
          \y = 2-2*\i/#1;
          \x = 2-2*\j/#1;
        }
        \node[#2] at (\x,\y) {};
      }
    }
}}

%%% Local Variables:
%%% mode: latex
%%% TeX-master: "all"
%%% End:


\newcommand{\showtikz}[2]{\begin{minipage}{#1\textwidth}
    \begin{center}
      \includegraphics[width=\textwidth]{./#2.tikz}
      \\
      #2
    \end{center}
  \end{minipage}
}

\begin{document}

\includegraphics[width=.24\textwidth]{./triangle.tikz}
\includegraphics[width=.24\textwidth]{./rectangle.tikz}
\includegraphics[width=.24\textwidth]{./p-mini-v.tikz}
\includegraphics[width=.24\textwidth]{./q-mini-v.tikz}

\showtikz{.24}{p0-p}
\showtikz{.24}{p1-p}
\showtikz{.24}{p2-p}
\showtikz{.24}{p3-p}

\showtikz{.24}{p2-v}
\showtikz{.24}{p3-v}

\showtikz{.24}{q0-p}
\showtikz{.24}{q1-p}
\showtikz{.24}{q2-p}

\showtikz{.24}{q2-v}
\showtikz{.24}{q-p1-p}
\showtikz{.24}{q3-v}
\showtikz{.24}{q-p2-p}

\showtikz{.24}{bdm1-tri}
\showtikz{.24}{bdm2-tri}
\showtikz{.24}{dgp1-p-tri}
\showtikz{.24}{dgp2-p-tri}

\showtikz{.24}{rt0-tri}
\showtikz{.24}{rt1-tri}
\showtikz{.24}{rt2-tri}

\showtikz{.24}{rt0-quad}
\showtikz{.24}{rt1-quad}
\showtikz{.24}{rt2-quad}

\includegraphics[width=.4\textwidth]{./patch1.tikz}
\includegraphics[width=.4\textwidth]{./patch2.tikz}

\includegraphics[width=.8\textwidth]{./p1-p1-1d.tikz}

\includegraphics[width=.2\textwidth]{./q-p1-p.tikz}
\includegraphics[width=.3\textwidth]{./q-p1-p.tikz}
\includegraphics[width=.4\textwidth]{./q-p1-p.tikz}

\includegraphics[width=.2\textwidth]{./q-p2-p.tikz}
\includegraphics[width=.3\textwidth]{./q-p2-p.tikz}
\includegraphics[width=.4\textwidth]{./q-p2-p.tikz}

\fbox{\includegraphics[width=.2\textwidth]{./rt5-tri.tikz}}
\fbox{\includegraphics[width=.3\textwidth]{./rt5-tri.tikz}}
\fbox{\includegraphics[width=.4\textwidth]{./rt5-tri.tikz}}

\end{document}

%%% Local Variables:
%%% mode: latex
%%% TeX-master: t
%%% End:
