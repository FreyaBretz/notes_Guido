
\section{Finite-dimensional problems}
\begin{intro}
  So far, our power horse for well-posedness was the Lax-Milgram
  lemma, which applied under the conditions
  \begin{xalignat}2
    a(u,v) &\le M \norm{u}\norm{v} & \forall u,v&\in V\\
    \label{eq:infsup:elliptic}
    a(u,u) &\ge \gamma \norm{u}^2 & \forall u&\in V.
  \end{xalignat}
  The second condition can also be rewritten in terms of the
  \putindex{Rayleigh quotient} as
  \begin{gather*}
    0 < \gamma = \inf_{u\in V}\frac{a(u,u)}{\norm{u}^2}.
  \end{gather*}
  Restricting this to finite a dimensional space, the notation usually
  changes from
  \begin{gather}
    a(u,v) = f(v)
    \qquad\text{to}\qquad
    v^TA u = v^Tf,
  \end{gather}
  where $A\in \R^{n\times n}$ is the matrix associated with the
  bilinear form. The bound for the Rayleigh quotient means nothing but
  that the real parts of all eigenvalues of $A$ are bounded from below
  by $\gamma$. Thus, a matrix $A$ for which we can apply the
  Lax-Milgram lemma is positive definite. And the statement of the
  lemma in finite dimension is, that a positive definite matrix is
  invertible. We know from linear algebra that this is true, but we
  also know that the condition is all but necessary.
\end{intro}

\begin{problem}
  Find an invertible, symmetric matrix $A\in \R^{2\times 2}$ and a
  vector $v\in \R^2$ such that $v^TAv=0$ and thus the Lax-Milgram
  lemma is inconclusive.
\end{problem}

The question of well-posedness in finite dimensions can be answered by:

\begin{Theorem}{la-invertible}
  A matrix $A\in\R^{n\times n}$ is invertible if and only if one of
  the following equivalent coditions holds:
  \begin{enumerate}
  \item all its (possibly complex) eigenvalues are nonzero,
  \item all its singular values are nonzero,
  \item for each $v\in\R^n$ holds $Av\neq 0$.
  \end{enumerate}
\end{Theorem}

\begin{intro}
  Why did we replace this clear theorem by the weaker Lax-Milgram
  lemma, when we studied elliptic partial differential equations?  For
  the first condition, it should be noted that spectral properties of
  operators between spaces of infinite dimension are much harder to
  obtain. Further, we do not need information on the whole spectrum,
  but only on the eigenvalue closest to zero. Therefore, we used a
  simple estimate in order to avoid discussing the spectrum at
  all. But, there is an important difference between
  Theorem~\ref{Theorem:la-invertible} and the
  estimate~\eqref{eq:infsup:elliptic}: the assumption of the theorem
  is qualitativ, $\lambda \neq 0$, while the assumption of Lax-Milgram
  is quantitative,
  \begin{gather*}
    \Re\lambda \ge \gamma> 0.
  \end{gather*}
  Why this change?
\end{intro}

\begin{problem}
  On the space $\ell_2(\R)$ define the operator $A$ by its eigenvalue
  decomposition
  \begin{align*}
    A: \ell_2(\R) &\to \ell_2(\R)\\
    e_k & \mapsto \tfrac1k e_k.
  \end{align*}
  Show that this operator does not have a bounded inverse, albeit all
  its eigenvalues are positive.
\end{problem}

\begin{intro}
  We focus on the second and third conditions, respectively, in
  Theorem~\ref{Theorem:la-invertible}.
  But, the problem above tells us that we
  will run into trouble, if we do not quantify this. Therefore, we
  start our attempt by requiring:
  \begin{gather*}
    \norm{Au}^2 \ge \gamma \norm{u}^2 \forall u\in V.
  \end{gather*}
  But while this is a condition we can easily write down for matrices
  and operators, it does not work that well for bilinear forms. For
  those, we have the condition which was introduced in this form by
  Nečas. It implies that the linear form on $V$, defined by $a(u,.)$
  is not the zero-form.

  It turns out, if we go this far, we can even consider bilinear forms
  $a(.,.)$ on two different spaces $V$ and $W$ with an associated
  operator $A: V\to W^*$.
\end{intro}

\begin{Definition}{infsup1}
  A bilinear form $a(.,.)$ on $V\times W$ is said to admit the
  \define{inf-sup condition} or is called \define{inf-sup stable}, if
  there holds
  \begin{gather}
    \label{eq:infsup:1}
    \inf_{u\in V} \sup_{w\in W} \frac{a(u,w)}{\norm{u}_V\norm{w}_W}
    \ge \gamma > 0.
  \end{gather}
\end{Definition}

\begin{Lemma}{infsup2}
  The following statements are equivalent to the inf-sup
  condition~\eqref{eq:infsup:1}:
  \begin{gather}
    \label{eq:infsup:2}
    \forall u\in V \;\exists w\in W \;:\; a(u,w) \ge \gamma \norm{u}_V\norm{w}_W
  \end{gather}
  \begin{gather}
    \label{eq:infsup:3}
    \forall u\in V
    \;\exists w\in W \;:\;
    \left\{
    \arraycolsep0.3ex
    \begin{array}{rcl}
      \norm{w}_W &\le&\norm{u}_V\\
      a(u,w) &\ge& \gamma \norm{u}_V^2
    \end{array}
    \right.
  \end{gather}
\end{Lemma}

\begin{proof}
  The proof is left as a problem.
\end{proof}

\begin{Theorem}{infsup-well-posedness}
  Let $a(.,.)$ on $V\times W$ be a bounded bilinear form such that
  \begin{gather}
    a(u,w) \le M \norm{u}_V \norm{w}_W.
  \end{gather}
  Then, the problem finding $u\in V$ such that
  \begin{gather}
    a(u,w) = f(w) \qquad\forall w\in W,
  \end{gather}
  has a unique solution and
  \begin{gather}
    \norm{u}_V \le \frac1\gamma \norm{f}_{W^*}
  \end{gather}
\end{Theorem}
%%% Local Variables: 
%%% mode: latex
%%% TeX-master: "main"
%%% End: 
