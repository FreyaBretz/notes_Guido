
\section{Finite-dimensional problems}
\begin{intro}
  So far, our power horse for well-posedness was the Lax-Milgram
  lemma, which can be applied under the conditions
  \begin{xalignat}2
    a(u,v) &\le M \norm{u}\norm{v} & \forall u,v&\in V\\
    \label{eq:infsup:elliptic}
    a(u,u) &\ge \ellipa \norm{u}^2 & \forall u&\in V.
  \end{xalignat}
  The second condition can also be rewritten in terms of the
  \putindex{Rayleigh quotient} as
  \begin{gather*}
    0 < \ellipa = \inf_{u\in V}\frac{a(u,u)}{\norm{u}^2}.
  \end{gather*}
  Restricting this to a finite dimensional space, the notation usually
  changes from
  \begin{gather}
    a(u,v) = f(v)
    \qquad\text{to}\qquad
    v^TA u = v^T f,
  \end{gather}
  where $A\in \R^{n\times n}$ is the matrix associated with the
  bilinear form. The bound for the Rayleigh quotient means nothing but
  that the real parts of all eigenvalues of $A$ are bounded from below
  by $\ellipa$. Thus, a matrix $A$ for which we can apply the
  Lax-Milgram lemma is positive definite. And the statement of the
  lemma in finite dimension is, that a positive definite matrix is
  invertible. We know from linear algebra that this is true, but we
  also know that the condition is all but necessary.
\end{intro}

\begin{intro}
  Why did we replace this clear theorem by the weaker Lax-Milgram
  lemma, when we studied elliptic partial differential equations?  For
  the first condition, it should be noted that spectral properties of
  operators between spaces of infinite dimension are much harder to
  obtain. Further, we do not need information on the whole spectrum,
  but only on the eigenvalue closest to zero. Therefore, we used a
  simple estimate in order to avoid discussing the spectrum at
  all. But, there is an important difference between
  Theorem~\ref{Theorem:la-invertible} and the
  estimate~\eqref{eq:infsup:elliptic}: the assumption of the theorem
  is qualitative, $\lambda \neq 0$, while the assumption of Lax-Milgram
  is quantitative,
  \begin{gather*}
    \Re\lambda \ge \ellipa> 0.
  \end{gather*}
  The following problem shows why such a change is necessary.
\end{intro}

\begin{Problem}{unbounded-inverse}
  On the space $\ell_2(\R)$ define the operator $A$ by its eigenvalue
  decomposition
  \begin{align*}
    A: \ell_2(\R) &\to \ell_2(\R)\\
    e_k & \mapsto \tfrac1k e_k.
  \end{align*}
  Here, $\{e_k\}$ is the orthogonal basis of unit vectors of the form
  \begin{gather*}
    \arraycolsep0.1em
    \begin{array}{cccccccc}
      e_k =(0&,\ldots,&0&,&1&,&0&,\ldots)^T.\\
      &&&&\uparrow\\
      &&&&k
    \end{array}
  \end{gather*}
  \begin{enumerate}
  \item Show that this operator does not have a bounded inverse, albeit
    its eigenvalues are positive.
  \item Show that the range of $A$ is not closed in $\ell_2(\R)$
  \end{enumerate}
\begin{solution}
  \begin{enumerate}
  \item For each $e_k$, the inverse is $A^{-1} e_k = k e_k$. In particular, $A$ is injective.
    On the other hand, it holds
    \begin{gather*}
      \lim_{k\to\infty}\frac{\norm{A^{-1}e_k}}{\norm{e_k}}=\lim_{k\to\infty}k=\infty
    \end{gather*}
      and the inverse cannot be bounded.
  \item We have to construct a convergent sequence in the range of $A$
    such that the pre-image of the sequence does not converge.
    \begin{enumerate}
    \item Choose
      \begin{gather*}
        v_n = \sum_{k=1}^n \frac1k e_k.
      \end{gather*}
      \item $v_n$ is a Cauchy-sequence, since
        \begin{gather*}
          \norm{v_m-v_n}^2 = \norm*{\sum_{k=m}^n \frac1k e_k}^2
          = \sum_{k=m}^n \frac1{k^2}\norm*{e_k}^2
          \le \frac1{m^2} \sum_{k=1}^\infty \frac1{k^2}
          = \frac{\pi^2}{6} \frac1{m^2}.
        \end{gather*}
      \item We conclude that $v=\lim_{n\to\infty}v_n$ exists in the
        closure of the range of $A$.
      \item There holds
        \begin{gather*}
          v_n = A \sum_{k=1}^n e_k =: A u_n.
        \end{gather*}
      \item Due to the injectivity of $A$ for $v$ to be in the range of $A$,
            $u_n$ has to converge in $\ell_2(\R)$.
      \item The sequence $u_n$ is not a Cauchy sequence, since
        \begin{gather*}
          \norm{v_m-v_n}^2 = \norm*{\sum_{k=m}^n e_k}^2
          = \sum_{k=m}^n \norm*{e_k}^2
          = n-m.
        \end{gather*}
    \end{enumerate}
  \end{enumerate}
\end{solution}
\end{Problem}

\begin{Problem}{lax-milgram-not-applicable}
  Find an invertible, symmetric matrix $A\in \R^{2\times 2}$ and a
  vector $v\in \R^2$ such that $v^T A v=0$ and thus the Lax-Milgram
  lemma is inconclusive.
\begin{solution}
  \begin{gather*}
    A =
    \begin{pmatrix}
      1 & 0 \\ 0 & -1
    \end{pmatrix}
  \end{gather*}
\end{solution}
\end{Problem}

The question of well-posedness in finite dimensions can be answered by:

\begin{Theorem}{la-invertible}
  A matrix $A\in\R^{n\times n}$ is invertible if and only if one of
  the following equivalent conditions holds:
  \begin{enumerate}
  \item all its (possibly complex) eigenvalues are nonzero,
  \item all its singular values are nonzero,
  \item for each $v\in\R^n$ holds $Av\neq 0$.
  \end{enumerate}
\end{Theorem}

\begin{intro}
  We focus on the second and third conditions, respectively, in
  Theorem~\ref{Theorem:la-invertible}.
  But, the problem above tells us that we
  will run into trouble, if we do not quantify this. Therefore, we
  start our attempt by requiring:
  \begin{gather*}
    \norm{Au}^2 \ge \ellipa \norm{u}^2 \qquad\forall u\in V.
  \end{gather*}
  But while this is a condition we can easily write down for matrices
  and operators, it does not work that well for bilinear forms. Thus,
  we first look at the singular value decomposition.
\end{intro}

\begin{Theorem}{svd}
  Let $A\in\R^{m\times n}$ be a real matrix. Then, there exist two
  orthogonal matrices $U\in\R^{m\times m}$ and $V\in \R^{n\times n}$
  as well as a real, nonnegative diagonal matrix $\hat\Sigma$, such
  that
  \begin{gather}
    \label{eq:svd:1}
    A = U \Sigma V^\transpose,
    \qquad\text{and }
    \Sigma =
    \begin{cases}
      \begin{bmatrix}
        \hat\Sigma &0
      \end{bmatrix}
      &\text{for } m<n\\
      \hat\Sigma&\text{for } m=n\\
      \begin{bmatrix}
        \hat\Sigma \\0
      \end{bmatrix}
      &\text{for } m>n
    \end{cases}
  \end{gather}
  This is the \define{singular value decomposition} (\define{SVD}) of
  $A$, the diagonal entries of $\hat \Sigma$ are the
  \define{singular values} of $A$ and the column vectors of $U$ and
  $V$ are the left and right \define{singular vectors} of $A$,
  respectively. The same theorem holds for complex matrices with
  unitary $U$ and $V$.
\end{Theorem}

\begin{proof}
  We prove constructively for the real case by induction. For $m=1$ or
  $n=1$ the theorem is obvious. Let now $m,n>1$ and assume the theorem
  has been proven for $A\in \R^{(m-1)\times (n-1)}$. Let $\sigma_1^2$
  be the largest eigenvalue of $A^\transpose A$, which due to the symmetry of
  $A^\transpose A$ is real and nonnegative. Actually, if it is zero, then
  $A^\transpose A=0$ and thus $A=0$ and $\Sigma=0$. Now assume $\sigma_1^2 >
  0$. Choose $x_1$ as an eigenvector to the eigenvalue $\sigma_1^2$ of
  $A^\transpose A$ and
  \begin{gather}
    y_1 = \frac1{\sigma_1} A x_1.
  \end{gather}
  We can complete both $x_1$ and $y_1$ to an orthonormal basis $X$ and
  $Y$, respectively. Then, there holds for $e_1 \in \R^n$ and
  $\bar e_1\in \R^m$:
  \begin{align}
    Y^\transpose AX e_1 &= Y^\transpose A x_1 = \sigma_1 Y^\transpose y_1 = \sigma_1 \bar e_1\\
    (Y^\transpose AX)^\transpose \bar e_1 = X^\transpose A^\transpose Y\bar e_1,
              &= X^\transpose A^\transpose y_1 =  \tfrac1{\sigma_1} X^\transpose A^\transpose A x_1
                = \sigma_1 X^\transpose x_1 = \sigma_1 e_1.
  \end{align}
  Thus,
  \begin{gather}
    Y^\transpose AX =
    \begin{bmatrix}
      \sigma_1 & 0 \\ 0 & \tilde A
    \end{bmatrix}
  \end{gather}
  with $\tilde A\in \R^{(m-1)\times (n-1)}$. By induction, there holds
  $\tilde A = \tilde U \tilde \Sigma \tilde V^\transpose$ with orthogonal
  matrices $U$ and $V$ and $\tilde\Sigma$ of the same form as $\Sigma$
  and both dimensions reduced by 1. Now let
  \begin{gather}
    U = Y
    \begin{bmatrix}
      1 & 0\\ 0& \tilde U
    \end{bmatrix},
    \qquad
    V = X
    \begin{bmatrix}
      1 & 0 \\ 0& \tilde V
    \end{bmatrix},
    \qquad
    \Sigma =
    \begin{bmatrix}
      \sigma_1 & 0 \\ 0 & \tilde\Sigma
    \end{bmatrix}
  \end{gather}
  $U$ and $V$ as the product of orthogonal matrices are orthogonal,
  $\sigma$ has the claimed structure and there holds $A=U\Sigma V^\transpose$.
\end{proof}

\begin{Corollary}{svd-order}
  By the construction of the proof, the singular values are ordered by
  decreasing magnitude,
  \begin{gather}
    \label{eq:svd:2}
    \sigma_1 \ge \sigma_2 \ge \dots \ge \sigma_r > 0.
  \end{gather}
  The number $r$ of nonzero singular values is the dimension of the
  range of the matrix $A$.
\end{Corollary}

%%% Local Variables:
%%% mode: latex
%%% TeX-master: "main"
%%% End:


\begin{Definition}{ker-range-rn}
  \index{ker}
  Let $A: V\to W$ be a linear operator. Then, we define
  the \define{kernel} and the \define{range} of $A$ as
  \begin{align*}
    \ker A &= \bigl\{ v\in V \big| Av = 0\bigr\} \\
    \range A &= \bigl\{ w\in W \big| \;\exists\,v\in V: Av=w\bigr\}.
  \end{align*}
\end{Definition}

\begin{Definition}{orthogonal1}
  Let $V\subset\R^n$ be a subspace. We define the \define{orthogonal
    complement} of $V$ as
  \begin{gather}
    \label{eq:infsup:4}
    \ortho{V} = \bigl\{w\in \R^n \big| \;\forall\,v\in V \scal(w,v) = 0 \bigr\}.
  \end{gather}
\end{Definition}

\begin{Lemma}{ker-coker-rn}
  Let $A\in \R^{m\times n}$ and $A^T$ its transpose. Then, there holds
  \begin{gather}
    \label{eq:infsup:5}
    \begin{split}
      \ker A &= \ortho{\range{A^T}}\\
      \range A &= \ortho{\ker{A^T}}\\
      \ker{A^T} &= \ortho{\range A}\\
      \range{A^T} &= \ortho{\ker{A}}
    \end{split}
  \end{gather}
\end{Lemma}

\begin{proof}
  % First, we note that
  % \begin{gather*}
  %   \R^n = \ker A \oplus \ortho{\ker A},
  %   \qquad
  %   \R^m = \ker{A^T} \oplus \ortho{\ker{A^T}}.
  % \end{gather*}
  Let $A=U\Sigma V^T$ be the singular value decomposition of $A$ and
  $r$ be the number of nonzero singular values. Then, the first $r$
  vectors of $U$ span the range of $A$ and the last $n-r$ vectors of
  $V$ span its kernel. Furthermore,
  \begin{gather}
    A^T = \bigl(U\Sigma V^T\bigr)^T = V \Sigma U^T.
  \end{gather}
  Therefore, the first $r$ vectors of V span the range of $A^T$ and
  the last $n-r$ vectors of $U$ span its kernel.
\end{proof}

\begin{Corollary}{ker-coker-iso}
  Let $A\in \R^{m\times n}$ and $A^T$ its transpose. Then, the
  restrictions $A\colon \ortho{\ker A} \to \range A$ and $A^T\colon
  \ortho{\ker{A^T}} \to \range{A^T}$ are isomorphisms.
\end{Corollary}

\begin{proof}
  We note that $\dim \range A = \dim \range{A^T}$. Thus, by
  \blockref{Lemma}{ker-coker-rn} the dimensions of domain and range of
  each of the restricted operators are equal, say $\dim \range A =
  r$. The singular value decomposition of the operators is
  \begin{gather*}
    A = U\Sigma V^T \qquad A^T = V\Sigma U^T,
  \end{gather*}
  where all matrices are in $\R^{r\times r}$ and
  \begin{gather*}
    \Sigma = \diag(\sigma_1,\dots,\sigma_r),
  \end{gather*}
  and all singular values are positive. Thus, $A$ and $A^T$ are invertible.
\end{proof}

\begin{Corollary}{svd-infsup}
  Let $r=\dim \ortho{\ker A}$. Then, for the smallest nonzero singular
  value there holds
  \begin{gather}
    \label{eq:infsup:6}
    \sigma_r
    = \inf_{v\in \ortho{\ker A}} \sup_{w\in \R^m} \frac{w^T A v}{\abs{v}\abs{w}}
    = \inf_{w\in \ortho{\ker{A^T}}} \sup_{v\in \R^n} \frac{w^T A v}{\abs{v}\abs{w}}.
  \end{gather}
\end{Corollary}

\begin{proof}
  Since the Cauchy-Schwarz inequality turns into an equation if and
  only if the two vectors are coaligned, there holds for any $v\in \R^n$:
  \begin{gather*}
    \sup_{w\in\R^m}\frac{w^T A v}{\abs{w}} = \frac{v^T A^T A v}{\abs{Av}}.
  \end{gather*}
  Therefore,
  \begin{gather*}
    \inf_{v\in \ortho{\ker A}} \sup_{w\in \R^m}
    \frac{w^T A v}{\abs{v}\abs{w}}
    = \inf_{v\in \ortho{\ker A}} \frac{\abs{Av}}{\abs{v}}.
  \end{gather*}
  Now, let $v = \sum \alpha_i v_i$ where $v_i$ are the columns of $V$
  in the SVD of $A$. Then,
  \begin{gather*}
    \abs{Av}^2 = \abs*{A\sum_{i=1}^r \alpha_i v_i}^2
    = \abs*{\sum_{i=1}^r \sigma_i \alpha_i u_i}^2
    = \sum_{i=1}^r \sigma_i^2 \alpha_i^2.
  \end{gather*}
  The quotient
  \begin{gather*}
    \frac{\sum_{i=1}^r \sigma_i^2 \alpha_i^2}{\sum_{i=1}^r \alpha_i^2}
  \end{gather*}
  clearly has its minimum if $\alpha_1 = \dots=\alpha_{r-1} = 0$.
\end{proof}

\begin{Definition}{infsup1}
  A bilinear form $a(\cdot,\cdot)$ on $V\times W$ is said to admit the
  \define{inf-sup condition} or is called \define{inf-sup stable}, if
  there holds
  \begin{gather}
    \label{eq:infsup:1}
    \inf_{u\in V} \sup_{w\in W} \frac{a(u,w)}{\norm{u}_V\norm{w}_W}
    \ge \ellipa > 0.
  \end{gather}
\end{Definition}

\begin{remark}
  In this finite dimensional exposition, is clear that $V$ and $W$
  must have the same dimension, and thus $V=W=\R^n$. This will be
  different, when we consider infinite dimensional spaces and indeed
  consider different spaces $V$ and $W$.
\end{remark}

\begin{Lemma}{infsup2}
  The following statements are equivalent to the inf-sup
  condition~\eqref{eq:infsup:1}:
  \begin{gather}
    \label{eq:infsup:2}
    \forall u\in V \;\exists w\in W \;:\; a(u,w) \ge \ellipa \norm{u}_V\norm{w}_W
  \end{gather}
  \begin{gather}
    \label{eq:infsup:3}
    \forall u\in V
    \;\exists w\in W \;:\;
    \left\{
    \arraycolsep0.3ex
    \begin{array}{rcl}
      \norm{w}_W &\le&\norm{u}_V\\
      a(u,w) &\ge& \ellipa \norm{u}_V^2
    \end{array}
    \right.
  \end{gather}
  \begin{gather}
    \label{eq:infsup:3a}
    \forall u\in V
    \;\exists w\in W \;:\;
    \left\{
    \arraycolsep0.3ex
    \begin{array}{rcl}
      \ellipa \norm{w}_W &\le&\norm{u}_V\\
      Aw &=& u
    \end{array}
    \right.
  \end{gather}
\end{Lemma}

\begin{Problem}{inf-sup-equivalence}
  Prove Lemma 2.1.16.
\begin{solution}
  We have to prove the following statements are equivalent to the inf-sup condition:
  \begin{align}
    \forall u\in V \;\exists w\in W \;:\; a(u,w) \ge \ellipa \norm{u}_V\norm{w}_W     \tag{1}
  \end{align}
  \begin{align}
   \begin{aligned}
    \forall u\in V \;\exists w\in W \;:\;
    \begin{cases}
      \norm{w}_W \le\norm{u}_V\\
      a(u,w) \ge \ellipa \norm{u}_V^2
    \end{cases}
   \end{aligned}
   \tag{2}
  \end{align}
  \begin{align}
   \begin{aligned}
    \forall u\in V \;\exists w\in W \;:\;
    \begin{cases}
      \ellipa \norm{w}_W \le \norm{u}_V\\
      Aw = u
      \end{cases}
  \end{aligned}
      \tag{3}
  \end{align}
  The inf-sup condition reads
  \begin{align}
   \inf_{u\in V} \sup_{w\in W} \frac{a(u,w)}{\norm{u}_V\norm{w}_W} \ge \ellipa \tag{IS}
  \end{align}

  $(IS)\Rightarrow(3)$\\
  The inf-sup condition is equivalent to $A: ker(A)^\perp\to V^*$ being an isomorphism.
  By the Riesz representation theorem there exists for a given $u\in V$ a $w\in W$ such that
  $Aw=Ju$ where $J$ is the Riesz map. Hence, it holds
  \begin{align*}
a(u,w)=\langle A w, u\rangle = \langle J u, u\rangle = \norm{u}_V^2.
  \end{align*}
 Due to $w\in ker(A)^\perp$, $A^{-1}$ is bounded and
 \begin{align*}
 \norm{w}_W=\norm{A^{-1}u}_V\leq \frac{1}{\ellipa}\norm{u}_V.
  \end{align*}

  $(3)\Rightarrow(2)$ \\
  Define $\tilde{w}=\ellipa w$. Then,
  \begin{align*}
a(u,\tilde{w})=\ellipa a(u,w) = \ellipa \norm{u}_V^2
  \end{align*}
  and
  \begin{align*}
\norm{\tilde{w}}_W=\ellipa \norm{w}_W\leq\norm{u}_V
  \end{align*}

  $(2)\Rightarrow(1)$ \\
  \begin{align*}
  a(u,w) \ge \ellipa \norm{u}_V^2 \ge \ellipa \norm{u}_V \norm{w}_W
  \end{align*}

  $(1)\Rightarrow(IS)$ \\
  \begin{align*}
  &\forall u\in V \exists w\in W: a(u,w)\ge \ellipa \norm{u}_V\norm{w}_W\\
  &\Rightarrow \forall u\in V: \sup_{w\in W} \frac{a(u,w)}{\norm{w}_W}\ge \ellipa \norm{u}_V\\
  &\Rightarrow \inf_{u\in V}\sup_{w\in W} \frac{a(u,w)}{\norm{w}_W\norm{u}_V}\ge \ellipa
  \end{align*}
\end{solution}
\end{Problem}


%%%%%%%%%%%%%%%%%%%%%%%%%%%%%%%%%%%%%%%%%%%%%%%%%%%%%%%%%%%%%%%%%%%%%%
%%%%%%%%%%%%%%%%%%%%%%%%%%%%%%%%%%%%%%%%%%%%%%%%%%%%%%%%%%%%%%%%%%%%%%
\section{Infinite dimensional Hilbert spaces}
%%%%%%%%%%%%%%%%%%%%%%%%%%%%%%%%%%%%%%%%%%%%%%%%%%%%%%%%%%%%%%%%%%%%%%
%%%%%%%%%%%%%%%%%%%%%%%%%%%%%%%%%%%%%%%%%%%%%%%%%%%%%%%%%%%%%%%%%%%%%%

\begin{intro}
  In the previous section, we derived quantitative conditions to
  ensure the invertibility of a matrix $A$ or its restriction to its
  cokernel $\ortho{\ker A}$. The arguments there have a natural
  extension to infinite dimensional Hilbert spaces, which we will
  derive in this section. We already saw in
  \blockref{Problem}{unbounded-inverse} that we may run into trouble
  if the range of $A$ is not closed. On the other hand, it turns out
  that most notions of linear algebra related to orthogonality can be
  maintained in Hilbert spaces if closed subspaces are considered.
  We begin by citing the most important results.
\end{intro}

\begin{Definition}{polar-orthogonal}
  Let $W\subset V$ be a subspace of a Hilbert space $V$. We define its
  \define{polar space} $\polar{W}\subset V^*$ and its
  \define{orthogonal complement} $\ortho{W}\subset V$ by
  \begin{gather}
    \label{eq:infsup:7}
    \begin{split}
    \polar{W} &= \bigl\{f\in V^* \big| \scal(f,w)_{V^*\times V} = 0
    \quad\forall\,w\in W\bigr\},
    \\
    \ortho{W} &= \bigl\{v\in V \big| \scal(v,w)_{V} = 0
    \quad\forall\,w\in W\bigr\}.
    \end{split}
  \end{gather}
  For a subspace $U\subset V^*$, we define its polar space
  \begin{gather}
    \polar{U} = \bigl\{v\in V \big| \scal(u,v)_{V^*\times V} = 0
    \quad\forall\,u\in U\bigr\}
  \end{gather}
\end{Definition}

\begin{Lemma}{orthogonal-closed}
  The polar space $\polar{W}$ and the orthogonal complement $\ortho{W}$ of a
  subspace $W\subset V$ are both closed. So is the polar space $\polar{U}$
  of a subspace $U\subset V^*$.
\end{Lemma}

\begin{proof}
  Consider the mapping
  \begin{align*}
    \Phi_w\colon V^* &\to \R,\\
    v&\mapsto \scal(v,w)_{V^*\times V}.
  \end{align*}
  For any $w$, the kernel of $\phi$ is closed as
  the pre-image of a closed set. $\polar{W}$ is closed since it is the
  intersection of these kernels for all $w\in W$.

  The inner product is continuous on $V\times V$. Therefore, the
  mapping
  \begin{align*}
    \phi_w\colon V &\to \R,\\
    v&\mapsto \scal(v,w),
  \end{align*}
  is continuous. The argument continues as above. Similar for $\polar{U}$.
\end{proof}

\begin{Theorem}{orthogonal-complement}
  Let $W$ be a subspace of a Hilbert space $V$ and $\ortho{W}$ its
  orthogonal complement. Then, $\ortho{W} = \ortho{\overline{W}}$. Further,
  $V = W \oplus \ortho{W}$ if and only if $W$ is closed.
\end{Theorem}

\begin{proof}
  Clearly, $\ortho{\overline{W}} \subset \ortho{W}$ since
  $W\subset\overline{W}$. Let now $u\in \ortho{W}$. Then, $\phi =
  \scal(u,\cdot)$ is a continuous linear functional on $V$. Therefore,
  if a sequence $w_n \subset W$ converges to $w\in \overline{W}$, we
  have
  \begin{gather*}
    \scal(u,w) = \lim_{n\to\infty} \scal(u,w_n) = 0.
  \end{gather*}
  Hence, $u\in \ortho{\overline{W}}$ and $\ortho{W} = \ortho{\overline{W}}$.

  Now, the ``only if'' follows by the fact, that if $W$ is not
  closed, there is an element $w\in \overline{W}$ but not in $W$ such that
  $\scal(w,u)=0$ for all $u\in \ortho{W}$. Thus, $w\not\in \ortho{W}$ and
  consequently $w\not\in \ortho{W} \oplus W$.

  Let now $W$ be closed. We show that there is a unique decomposition
  \begin{gather}
    \label{eq:infsup:8}
    v = w + u,\qquad w\in W, \;u\in \ortho{W},
  \end{gather}
  which is equivalent to $V = W \oplus \ortho{W}$. Uniqueness follows,
  since
  \begin{gather*}
    v = w_1+u_1 = w_2+u_2
  \end{gather*}
  implies that for any $y\in V$
  \begin{gather*}
    0 = \scal(w_1-w_2+u_1-u_2,y) = \scal(w_1-w_2,y) + \scal(u_1-u_2,y).
  \end{gather*}
  Choosing $y=u_1-u_2$ and $w_1-w_2$ in turns, we see that one of the
  inner products vanishes for orthogonality and the other implies that
  the difference is zero.

  If $v\in W$, we choose $w=v$ and $u=0$. For $v\not\in W$, we prove
  existence by considering that due to the closedness of $W$ there holds
  \begin{gather*}
    d=\inf_{w\in W} \norm{v-w} >0.
  \end{gather*}
  Let $w_n$ be a minimizing sequence. Using the parallelogram identity
  \begin{gather*}
    \norm{a+b}^2+\norm{a-b}^2 = 2\norm{a}^2+2\norm{b}^2,
  \end{gather*}
  we prove that $\{w_n\}$ is a Cauchy sequence by
  \begin{align*}
    \norm{w_m-w_n}^2 &= \norm{(v-w_n)-(v-w_m)}^2\\
    &= 2\norm{v-w_n}^2+2\norm{v-w_m}^2-\norm{2v-w_m-w_n}^2\\
    &= 2\norm{v-w_n}^2+2\norm{v-w_m}^2-4\norm*{v-\frac{w_m+w_n}2}^2\\
    &\le 2\norm{v-w_n}^2+2\norm{v-w_m}^2-4d^2,
  \end{align*}
  since $(w_m+w_n)/2\in W$ and $d$ is the infimum. Now we use the
  minimizing property to obtain
  \begin{gather*}
    \lim_{m,n\to\infty}\norm{w_m-w_n}^2 = 2d^2-2d^2 -4d^2=0.
  \end{gather*}
  By completeness of $V$, $w=\lim w_n$ exists and by the closedness of
  $W$, we have $w\in W$. Let $u=v-w$. By continuity of the norm, we
  have $\norm{u}=d$. It remains to show that $u\in \ortho{W}$. To this
  end, we introduce the variation $w+\epsilon \tilde w$ with $\tilde
  w\in W$ to obtain
  \begin{align*}
    d^2 &\le \norm{v-w-\epsilon \tilde w}^2\\
    &= \norm{u}^2-2\epsilon\scal(u,\tilde w)+\epsilon^2 \norm{\tilde w},
  \end{align*}
  implying for any $\epsilon>0$
  \begin{gather*}
    0\le-2\epsilon\scal(u,\tilde w)+\epsilon^2 \norm{\tilde w},
  \end{gather*}
  which requires $\scal(u,\tilde w) = 0$.
\end{proof}

\begin{Definition}{orthogonal-projection}
  Let $V$ be a Hilbert space and $W\subset V$ be a closed
  subspace. For a vector $v\in V$, let $v=w+u$ be the unique
  decomposition with $w\in W$ and $u\in \ortho{W}$. Then we call $w$ and
  $u$ the \define{orthogonal projection}s of $v$ into $W$ and $\ortho{W}$,
  respectively. We write
  \begin{gather*}
    \Pi_W = w, \qquad \Pi_{\ortho{W}} = u.
  \end{gather*}
\end{Definition}

\begin{Lemma}{polar-orthogonal-hilbert}
  Let $V$ be a Hilbert space and $W\subset V$ be a closed
  subspace. Then, the polar space $\polar{W}\subset V^*$ and the orthogonal
  space $\ortho{W}$ can be isometrically identified by \putindex{Riesz
    representation}.
\end{Lemma}

\begin{proof}
  For every $f$ in the
  dual of $\ortho{W}$, define $g\in V^*$ by
  \begin{gather*}
    \scal(g,v)_{V^*\times V} =
    \scal(f,\Pi_{\ortho{V}}v)_{(\ortho{V})^*\times \ortho{V}}.
  \end{gather*}
  Clearly, $g(v)=0$ for $v\in W$, therefore $g\in \polar{W}$.
\end{proof}

% \begin{Corollary}

% \end{Corollary}


\begin{Theorem}{closed-range}
  Let $V,W$ be Hilbert spaces and $A\colon V\to W$ a continuous linear
  operator. Then, the following statements are equivalent:
  \begin{gather}
    \label{eq:infsup:9}
    \begin{split}
      \range A &\text{ is closed in } W,\\
      \range{A^T} &\text{ is closed in } V^*,\\
      \range A &= \polar{\ker{A^T}},\\
      \range{A^T} &= \polar{\ker A}.
    \end{split}
  \end{gather}
\end{Theorem}

\begin{remark}
  This is the famous \emph{\putindex{closed range theorem}} by Banach.
  It actually holds under weaker assumptions, for instance $V,W$ only
  Banach spaces. The proof can be found for instance
  in~\cite[p.~205--209]{Yosida80}.
\end{remark}

\begin{Theorem}{open-mapping}
  Let $A\colon V\to W$ be continuous and surjective. Then, the image
  $A(U)\subset W$ of any open set $U\subset V$ is open.
\end{Theorem}

\begin{remark}
  This is the \emph{\putindex{open mapping theorem}} by Banach. The
  proof can be found for instance in~\cite[p.75--76]{Yosida80}.
\end{remark}

\begin{Lemma}{closed-infsup}
  Let $A\colon V\to W$ be continuous. Then, $\range A$ is closed in
  $W$ if and only if there exists $\ellipa>0$ such that
  \begin{gather}
    \label{eq:infsup:10}
    \forall w\in \range A\;
    \exists v\in V\quad
    Av = w
    \;\wedge\;
    \ellipa \norm{v}_V \le \norm{w}_W.
  \end{gather}
\end{Lemma}

\begin{proof}
  We first show that the inf-sup condition~\eqref{eq:infsup:10}
  implies $\range A$ closed. To this end, let $\{w_n\}$ be a Cauchy
  sequence in $\range A$ converging to a point $w\in W$. Thus, there
  is a sequence $\{v_n\}$ in $V$ such that $Av_n = w_n$ and
  $\ellipa \norm{v_n} \le \norm{w_n}$. Hence,
  \begin{gather*}
    \norm{v_m-v_n}_V \le \frac1\ellipa \norm{w_m-w_n}_W,
  \end{gather*}
  and $\{v_n\}$ is a Cauchy sequence in $V$. Therefore, $v_n\to v\in
  V$ and due to continuity of $A$ we obtain $Av=w$ and thus $w\in
  \range A$.

  Conversely, let $\range A$ be closed in $W$. Thus, it is a Banach
  space and the \putindex{open mapping theorem} applies to $A\colon
  V\to\range A$. We map the open unit ball $B_1(0)\subset V$ and
  obtain that $A(B_1(0))$ is open in $\range A$, implying that there
  is an open ball $B_\delta(0) \subset A(B_1(0))$. This is sufficient
  to construct $v$:

  Let $w\in\range A$. Then,
  \begin{gather*}
    \tilde w\frac\delta2 \frac{w}{\norm{w}} \in B_\delta(0) \subset A(B_1(0)).
  \end{gather*}
  Hence, there is $v\in V$ with $\norm{v}<1$ such that $Av=\tilde w$,
  which proves the lemma.
\end{proof}

\begin{Theorem}{infsup-well-equivalence}
  Let $a(\cdot,\cdot)$ on $V\times W$ be a bounded bilinear form such that
  \begin{gather}
    a(v,w) \le M \norm{v}_V \norm{w}_W,
  \end{gather}
  and $A\colon V\to W^*$ its associated operator.
  Then, the following statements are equivalent:
  \begin{enumerate}
  \item There exists $\ellipa>0$ such that
    \begin{gather}
      \label{eq:infsup:11}
      \inf_{w\in W}\sup_{v\in V}
      \frac{a(v,w)}{\norm{v}_V\norm{w}_W}
      \ge \ellipa.
    \end{gather}
  \item The operator $A^T\colon W\to \polar{\ker A}$ is an isomorphism and
    \begin{gather}
      \label{eq:infsup:12}
      \norm{A^Tw}_{V^*} \ge \ellipa \norm{w}_{W} \qquad\forall w\in W.
    \end{gather}
  \item The operator $A\colon \ortho{\ker A}\to W^*$ is an isomorphism
    and
    \begin{gather}
      \label{eq:infsup:13}
      \norm{Av}_{W^*} \ge \ellipa\norm{v}_V\qquad \forall v\in \ortho{\ker A}.
    \end{gather}
  \end{enumerate}
\end{Theorem}

\begin{proof}
  First, we show the equivalence of the first two statements. Let us
  rephrase the inf-sup condition to
  \begin{gather*}
    \norm{A^T w}_{V^*}
    = \sup_{v\in V}\frac{\scal(A^T w,v)}{\norm{v}_V}
    = \sup_{v\in V}\frac{a(v,w)}{\norm{v}_V}
    \ge \ellipa\norm{w} \qquad
    \forall w\in W.
  \end{gather*}
  Thus, equations~\eqref{eq:infsup:11} and~\eqref{eq:infsup:12} are
  equivalent and we have already proven that the second statement
  implies the first. It remains to show the $A^T$ is an isomorphism
  from $W$ onto $\polar{\ker A}$. Equation~\eqref{eq:infsup:12} implies that
  $A^T\colon W \to \range{A^T}$ is an isomorphism and its inverse is
  bounded by $1/\ellipa$ (multiply both sides by $A^{-1}$). Using
  \blockref{Lemma}{closed-infsup}, we obtain that $\range{A^T}$ is
  closed in $V^*$ and the \putindex{closed range theorem} settles the
  issue.

  In order to prove equivalence of the second and third statement, we
  use \blockref{Lemma}{polar-orthogonal-hilbert} to isometrically
  identify $(\ortho{\ker A})^*$ with $\polar{\ker A}$. Thus, $A$ is an
  isomorphism from $\ortho{\ker A}$ onto $W$ if and only if $A^T$ is an
  isomorphism from $W$ onto $(\ortho{\ker A})^* = \polar{\ker A}$. and
  \begin{gather*}
    \norm{A}_{W^*\to \ortho{\ker A}} = \norm{A^T}_{\polar{\ker A}\to W}.
  \end{gather*}
\end{proof}

\begin{Corollary}{infsup-well-posedness1}
  Let $a(\cdot,\cdot)$ on $V\times W$ be a bounded bilinear form such that
  \begin{gather}
    a(v,w) \le M \norm{v}_V \norm{w}_W.
  \end{gather}
  Let the inf-sup-condition
  \begin{gather*}
    \inf_{w\in W}\sup_{v\in V}
    \frac{a(v,w)}{\norm{v}_V\norm{w}_W}
    \ge \ellipa > 0
  \end{gather*}
  hold.  Then, the problem finding $w\in W$ such that
  \begin{gather*}
    a(v,w) = f(v) \qquad\forall v\in V,
  \end{gather*}
  has a unique solution for $f\in \polar{\ker A}$ and
  \begin{gather}
    \norm{w}_W \le \frac1\ellipa \norm{f}_{V^*}.
  \end{gather}
\end{Corollary}

\begin{remark}
  \blockref{Theorem}{infsup-well-equivalence} exhibits an asymmetry
  between the left and right argument. In particular, we obtain a
  unique solution only for the adjoint operator $A^T$, which is
  exactly what we need, when we compute say a pressure from the
  divergence of a velocity field. In general, we consider the
  restriction of $f$ to the polar set of the kernal in the above
  well-posedness result detrimental and would prefer a result that
  holds for all $f\in V^*$. This on the other hand requires
  $\ker A=\{0\}$, or $\overline{\range{A^T}} = W^*$. Then, on the
  other hand, we see that $\range{A^T}$ is closed since $\range{A}$ is
  closed and the closed range theorem holds. Therefore, we obtain the
  following theorem for the case that we require a unique solution for
  all right hand sides.
\end{remark}

\begin{Theorem}{infsup-well-posedness2}
  Let $a(\cdot,\cdot)$ on $V\times W$ be a bounded bilinear form such that
  \begin{gather}
    a(v,w) \le M \norm{v}_V \norm{w}_W.
  \end{gather}
  Let for some $\ellipa>0$ the inf-sup-conditions
  \begin{align*}
    \inf_{w\in W}\sup_{v\in V}
    \frac{a(v,w)}{\norm{v}_V\norm{w}_W}
    &\ge \ellipa,\\
    \inf_{v\in V}\sup_{w\in W}
    \frac{a(v,w)}{\norm{v}_V\norm{w}_W}
    &\ge \ellipa
  \end{align*}
  hold.  Then, the problem finding $v\in V$ such that
  \begin{gather*}
    a(v,w) = f(w) \qquad\forall w\in W,
  \end{gather*}
  has a unique solution for $f\in W^*$ and
  \begin{gather}
    \norm{v}_V \le \frac1\ellipa \norm{f}_{W^*}.
  \end{gather}
\end{Theorem}

\begin{remark}
  If we compare \blockref{Theorem}{infsup-well-posedness2} with
  \blockref{Corollary}{infsup-well-posedness1}, we see that the only
  difference lies in the fact that the second inf-sup condition
  ensures surjectivity of $A$ by injectivity of $A^T$. In some cases
  it may be difficult to prove both inf-sup conditions. Then, it is
  sufficient to prove one inf-sup condition, say the first, and then
  only
  \begin{gather*}
     \inf_{v\in V}\sup_{w\in W}
    \frac{a(v,w)}{\norm{v}_V\norm{w}_W} > 0,
  \end{gather*}
  thus, injectivity of $A^T$. Although we verify less than the
  assumptions of \blockref{Theorem}{infsup-well-posedness2}, the
  closed range theorem saves us from the additional work. We further
  note that this notion is symmetric between $A$ and $A^T$, that is,
  it is sufficient to prove inf-sup for either operator and
  injectivity for the other.
\end{remark}

%%%%%%%%%%%%%%%%%%%%%%%%%%%%%%%%%%%%%%%%%%%%%%%%%%%%%%%%%%%%%%%%%%%%%%
%%%%%%%%%%%%%%%%%%%%%%%%%%%%%%%%%%%%%%%%%%%%%%%%%%%%%%%%%%%%%%%%%%%%%%
\section{The inf-sup condition for mixed problems}
%%%%%%%%%%%%%%%%%%%%%%%%%%%%%%%%%%%%%%%%%%%%%%%%%%%%%%%%%%%%%%%%%%%%%%
%%%%%%%%%%%%%%%%%%%%%%%%%%%%%%%%%%%%%%%%%%%%%%%%%%%%%%%%%%%%%%%%%%%%%%

\begin{intro}
  In the previous section, we have developed a framework for
  well-posedness of problems which are not $V$-elliptic. In principle,
  this theory can be applied to the bilinear form
  $\mathcal A((u,p),(v,q))$ as a whole. On the other hand, we can
  formally split the solution of a constrained minimization problem
  into the reduced problem and then computing the Lagrange multiplier,
  which more clearly exhibits the relation of the two spaces $V$ and
  $Q$ involved in the mixed formulation. Here are the resulting
  theorems.
\end{intro}

\begin{Theorem}{infsup-mixed1}
  Let $V$ and $Q$ be Hilbert spaces and let the mixed bilinear form
  \begin{gather*}
        \mathcal A\mixedform(u,p,v,q)
      = a(u,v) + b(v,p) + b(u,q)
  \end{gather*}
  be defined and bounded for any $u,v\in V$ and $p,q\in Q$.
  Then, the problem
  \begin{gather*}
    \mathcal A\mixedform(u,p,v,q) = \scal(f,v)+\scal(g,q)
    \quad\forall v\in V, q\in Q,
  \end{gather*}
  has a unique solution for any $f\in V^*$ and any $g\in Q^*$ if and
  only if there exists $\ellipa>0$ such that
  \begin{gather*}
    \forall
    \begin{bmatrix}
      u\in V\\p\in Q
    \end{bmatrix}
    \;
    \exists
    \begin{bmatrix}
      v\in V\\q\in Q
    \end{bmatrix}
    \colon
    \quad
    \mathcal A\mixedform(u,p,v,q) \ge \ellipa
    \norm{(u,p)}_{V\times Q}\norm{(v,q)}_{V\times Q},
  \end{gather*}
  and vice versa.
\end{Theorem}

\begin{proof}
  Straight application of \blockref{Theorem}{infsup-well-posedness2}.
\end{proof}

\begin{Theorem}{infsup-mixed2}
  Let $V$ and $Q$ be Hilbert spaces and let
  \begin{gather}
    \begin{split}
      \ker B &= \bigl\{v\in V \big| b(v,q) = 0 \;\forall q\in Q\bigr\}.
    \end{split}
  \end{gather}
  Then, the problem finding $(u,p)\in V\times Q$ such that
  \begin{gather}
    a(u,v) + b(v,p) + b(u,q) = f(v) \quad\forall v\in V, q\in Q,
  \end{gather}
  is well-posed if and only if the problem finding $u\in \ker B$ such that
    \begin{gather}
      a(u,v) = f(v) \quad\forall v\in \ker B
    \end{gather}
    is well-posed for any $f\in V^*$ and there is a positive constant
    $\beta$ such that
    \begin{gather}
      \inf_{q\in Q}\sup_{v\in V} \frac{b(v,q)}{\norm{v}_V\norm{q}_Q} \ge \beta.
    \end{gather}
\end{Theorem}

\begin{proof}
  By requiring well-posedness of the reduced problem, $u\in V$ is
  well-determined and bounded by the data $f\in V^*$ without knowledge
  of the Lagrange multiplier. Hence, with $u\in V$ given and
  $b(u,q) = 0$, the problem of determining the Lagrange multiplier $p$
  reduces to
  \begin{gather}
    b(v,p) = f(v) - a(u,v), \qquad\forall v\in V.
  \end{gather}
  Applying \blockref{Corollary}{infsup-well-posedness1} to the
  bilinear form $b(.,.)$, we deduce that this equation has a unique
  solution $p\in Q$ if and only if $f(v)-a(u,v) \in \ker B^0$, which
  is the reduced problem.
\end{proof}

\begin{remark}
  Since $V$ is a Hilbert space, the decomposition
  $V = \ker B \oplus \ker B^\perp$ is uniquely determined and there is
  a corresponding decomposition $V^* = \ker B^0+(\ker B^\perp)^0$,
  such that $f = f^0+f^\perp$ above. The way we solve the reduced
  problem first and then compute the Lagrange multiplier implies that
  the solution $u$ only depends on $f^\perp$, while the Lagrange
  multiplier $p$ only depends on $f^0$.
\end{remark}

\begin{remark}
  We have imposed well-posedness of the reduced problem only in an
  abstract way. Depending on $a(.,.)$ we can formulate two conditions:
  ellipticity on $\ker B$ or inf-sup stability on $\ker B$. Indeed,
  most problems considered in this class will have symmetric bilinear
  forms $a(.,.)$, such that ellipticity serves as our usual
  assumption.  In these cases, note that $V$-ellipticity already
  implies the well-posedness on $\ker B$.
\end{remark}

\begin{Problem}{inhomogeneous-continuity}
  Show that %\blockref{Theorem}{infsup-mixed2}
  Theorem~2.3.3 can be extended to the
  case with right hand side $f(v)+g(q)$ with $g\in Q^*$.

\begin{solution}
  We want to solve the problem
  \begin{align*}
    a(u,v) + b(v,p) + b(u,q) = f(v)+g(q) \quad\forall v\in V, q\in Q,
  \end{align*}
  where $b(v,p)$ fulfills a inf-sup condition.

  \begin{enumerate}
  \item Due to %\blockref{Theorem}{infsup-well-equivalence}
    Theorem~2.2.12, the
    operator $B: V\to Q^*$ is surjective and thus, there exists
    $u_g\in V$ such that
    \begin{gather*}
      b(u_g,q) = q(q) \quad\forall q\in Q.
    \end{gather*}
  \item Now consider the function $u_0 = u-u_g$. For $u$ to solve the
    orginal problem $u_0$ has to solve
    \begin{align*}
      a(u_0+u_g,v) + b(v,p) + b(u_0+u_g,q) = f(v)+g(q) \quad\forall v\in V, q\in Q\\
      \Leftrightarrow a(u_0,v) + b(v,p) + b(u_0,q) = f(v)-a(u_g,v) \quad\forall v\in V, q\in Q
    \end{align*}
  \item Due to $a(u_g,v) \leq \norm{a}\norm{u_g}_V\norm{v}_V$,
    the right-hand side $f(\cdot)-a(u_g,\cdot)$ is in $V^*$
    and we are in the the setting of
    %\blockref{Theorem}{infsup-mixed2}.
    Theorem~2.3.3 .
  \end{enumerate}
\end{solution}
\end{Problem}

\begin{intro}
  We summarize the result of this section in an assumption for
  well-posedness which will be the basis for further results in this
  course. We know from the discussion above that this assumption is
  only sufficient and weaker conditions may be imposed on
  $a(.,.)$. But indeed, it helps us through a lot of problems and is a
  good compromise between generality and ease of use.
\end{intro}

\begin{Assumption}{mixed-elliptic}
  Let $V$ and $Q$ be Hilbert spaces and let $a(.,.)$ and $b(.,.)$ be
  bounded bilinear forms on $V\times V$ and $V\times Q$,
  respectively. We define their norms as the smallest constants such
  that for all arguments there holds
  \begin{gather}
    a(u,v) \le \norm a \norm{u}_V \norm{v}_V,
    \qquad
    b(v,q) \le \norm b \norm{v}_V \norm{q}_Q.
  \end{gather}
  With these forms, we associate bounded operators $A$, $B$, and $B^T$
  according to \blockref{Definition}{saddle-point-operators}. With
  $b(.,.)$ we associate the spaces
  \begin{gather}
    \begin{split}
      \ker B &= \bigl\{v\in V \big| b(v,q) = 0 \;\forall q\in Q\bigr\},\\
      \ker B^T &= \bigl\{q\in Q \big| b(v,q) = 0 \;\forall v\in V\bigr\}.
    \end{split}
  \end{gather}
  Furthermore, we assume that $a(.,.)$ is positive semi-definite on
  $V$ and elliptic on $\ker B$,
  \begin{gather}
    a(u,u) \ge 0 \quad\forall u\in V,
    \qquad
    a(u,u) \ge \ellipa \norm{u}_V^2 \quad\forall u\in \ker B.
  \end{gather}
\end{Assumption}

%%%%%%%%%%%%%%%%%%%%%%%%%%%%%%%%%%%%%%%%%%%%%%%%%%%%%%%%%%%%%%%%%%%%%%
%%%%%%%%%%%%%%%%%%%%%%%%%%%%%%%%%%%%%%%%%%%%%%%%%%%%%%%%%%%%%%%%%%%%%%
\section{Galerkin approximation of mixed problems}
%%%%%%%%%%%%%%%%%%%%%%%%%%%%%%%%%%%%%%%%%%%%%%%%%%%%%%%%%%%%%%%%%%%%%%
%%%%%%%%%%%%%%%%%%%%%%%%%%%%%%%%%%%%%%%%%%%%%%%%%%%%%%%%%%%%%%%%%%%%%%

\begin{intro}
  The Galerkin approximation of mixed problems starts out the same way
  as for elliptic problems, namely, choose discrete subspaces
  $V_h\subset V$ and $Q_h \subset Q$. There is a fundamental
  difference though: the inf-sup condition is not inherited
  automatically on the subspaces like $V$-ellipticity. It actually becomes
  an additional requirement on the choice of $V_h$ and
  $Q_h$. We will thus work our way in several steps towards the final
  result.
\end{intro}

\begin{Definition}{kerbh}
  Let $V_h\subset V$ and $Q_h\subset Q$. Then, we define the subspace
  \begin{gather}
    \label{eq:galerkin:1}
    \ker{B_h} = \bigl\{v_h\in V_h \big|
    b(v_h, q_h) = 0 \quad\forall q_h\in Q_h
    \bigr\}.
  \end{gather}
  Furthermore, we define the affine space
  \begin{gather}
    \label{eq:galerkin:2}
    V_h^g = \bigl\{v_h\in V_h \big|
    b(v_h, q_h) = g(q) \quad\forall q_h\in Q_h
    \bigr\}.
  \end{gather}
\end{Definition}


\begin{Definition}{mixed-galerkin}
  We introduce the mixed discrete problem: find $(u_h, p_h)\in
  V_h\times Q_h$ such that
  \begin{gather}
    \label{eq:galerkin:3}
    a(u_h, v_h) + b(v_h, p_h) + b(u_h, q_h) = f(v_h)+g(q_h),
    \quad\forall v_h\in V_h, q_h\in Q_h,
  \end{gather}
  and the discrete reduced problem: find $u_h\in V_h^g$ such that
  \begin{gather}
    \label{eq:galerkin:4}
    a(u_h, v_h) = f(v_h), \qquad\forall v_h \in \ker{B_h}.
  \end{gather}
\end{Definition}

\begin{Theorem}{galerkin-mixed-u-kerbh}
  Let $V_h^g$ be nonempty and let $a(.,.)$ and $b(.,.)$ be bounded
  with norms $\norm a$ and $\norm b$, respectively, and let there be constant and $\gamma_h$
  such that
  \begin{gather}
    \label{eq:galerkin:5}
    a(v_h, v_h) \ge \gamma_h \norm{v_h}^2_V,
    \qquad\forall v_h\in \ker{B_h}.
  \end{gather}
  Let furthermore the continuous mixed problem be well-posed with
  solution $(u,p)\in V\times Q$.
  Then, the discrete reduced problem~\eqref{eq:galerkin:4} has a
  unique solution and there holds
  \begin{gather}
    \label{eq:galerkin:8}
    \norm{u-u_h}_V \le \left(1+\frac{\norm a}{\gamma_h}\right)
    \inf_{w_h\in V_h^g}\norm{u-w_h}_V
    + \frac{\norm b}{\gamma_h}
    \inf_{q_h\in Q_h}\norm{p-q_h}_Q
  \end{gather}
\end{Theorem}

\begin{proof}
  Let $u_h^g \in V_h^g$ arbitrary. By the ellipticity assumption,
  there is a unique function $u_h^0\in \ker{B_h}$ such that
  \begin{gather*}
    a(u_h^0,v_h) = f(v_h) - a(u_h^g,v_h),
    \qquad\forall v_h\in \ker{B_h}.
  \end{gather*}
  Hence, $u_h = u_h^g + u_h^0$ is the unique solution
  to~\eqref{eq:galerkin:4}. Choose now $w_h\in V_h^g$
  arbitrarily. Then, $v_h = u_h-w_h\in \ker{B_h}$ and using
  \begin{gather*}
    f(v_h) = a(u, v_h) - b(v_h, p),
  \end{gather*}
  we obtain
  \begin{gather}
    \begin{split}
      \label{eq:galerkin:9}
      a(v_h, v_h)
      &= f(v_h) - a(u_h-v_h, v_h) \\
      &= a(u-w_h, v_h) - b(v_h, p) \\
      &= a(u-w_h, v_h) - b(v_h, p-q_h)
    \end{split}
  \end{gather}
  for any $q_h\in Q_h$, yielding
  \begin{gather*}
    \gamma_h\norm{v_h}^2_V
    \le \abs{a(v_h, v_h)}
    \le \norm a \norm{u-w_h}_V \norm{v_h}_V
      + \norm b \norm{p-q_h}_Q \norm{v_h}_V.
  \end{gather*}
  We conclude by the standard triangle inequality argument
  \begin{multline*}
    \norm{u-u_h}_V \le \norm{u-w_h}_V + \norm{u_h-w_h}_V
    \\
    \le \norm{u-w_h}_V + \frac{\norm a}{\gamma_h} \norm{u-w_h}_V
    + \frac{\norm b}{\gamma_h} \norm{p-q_h}_Q.
  \end{multline*}
  The estimate follows since $w_h\in V_h^g$ and $q_h\in Q_h$ were
  chosen arbitrarily.
\end{proof}

\begin{remark}
  Note that we only used ellipticity of $a(.,.)$ on the subspace
  $\ker{B_h}$ for the discrete problem and on $\ker B$ for the
  continuous problem. Since the union of two vector spaces is not a
  vector space, this is a strange condition. In practice, we will
  encounter two situations: either ellipticity holds on the whole
  space $V$ or $\ker{B_h}\subset\ker{B}$, where the latter is again a
  vector space.
\end{remark}

\begin{Corollary}{galerkin-mixed-u-kerb}
  If in addition to the assumptions of
  \blockref{Theorem}{galerkin-mixed-u-kerbh} there holds
  \begin{gather}
    \label{eq:galerkin:6}
    \ker{B_h}\subset \ker B,
  \end{gather}
  then
  \begin{gather}
    \label{eq:galerkin:7}
    \norm{u-u_h}_V \le \left(1+\frac{\norm a}{\gamma_h}\right)
    \inf_{w_h\in V_h^g}\norm{u-w_h}_V
  \end{gather}
\end{Corollary}

\begin{proof}
  Consider that in equation~\eqref{eq:galerkin:9} there holds
  $v_h\in\ker B$.
\end{proof}

\begin{Theorem}{galerkin-mixed-existence-p}
  Assume in addition the assumptions of
  \blockref{Theorem}{galerkin-mixed-u-kerbh} that there are constants
  $\beta_h$, possibly depending on the parameter $h$, such that
  \begin{gather}
    \label{eq:galerkin:10}
    \inf_{q_h\in Q_h} \sup_{v_h\in V_h}
    \frac{b(v_h, q_h)}{\norm{v_h}_V\norm{q_h}_Q}
    \ge \beta_h.
  \end{gather}
  Then, there is a unique solution $p_h\in Q_h$ such that $(u_h, p_h)$
  is the unique solution to the discrete mixed
  problem~\eqref{eq:galerkin:3}. There are a constants $c_h^{(i)}$
  only depending on $\norm a$, $\norm b$, $\gamma_h$ and $\beta_h$
  such that
  \begin{align}
    \label{eq:galerkin:11}
    \norm{u-u_h}_V
    &\le c_h^{(1)} \inf_{v_h\in V_h^g}\norm{u-v_h}_V
    + c_h^{(2)} \inf_{q_h\in Q_h}\norm{p-q_h}_Q \\
    \norm{p-p_h}_Q
    &\le c_h^{(3)} \inf_{v_h\in V_h}\norm{u-v_h}_V
    + c_h^{(4)} \inf_{q_h\in Q_h}\norm{p-q_h}_Q.
  \end{align}
\end{Theorem}

\begin{proof}
  Applying \blockref{Theorem}{infsup-mixed2} to the discrete
  problem~\eqref{eq:galerkin:3}, we conclude that there is a unique
  solution $(u_h,p_h)\in V_h\times Q_h$. Let us begin estimating the
  error by establishing the bound
  \begin{gather}
    \label{eq:galerkin:12}
    \inf_{w_h\in V_h^g} \norm{u-w_h}_V
    \le \left(1+\frac{\norm b}{\beta_h}\right)
    \inf_{v_h\in V_h} \norm{u-v_h}_V.
  \end{gather}
  By the third condition in
  \blockref{Theorem}{infsup-well-equivalence}, there is a unique
  $z_h\in \ortho{\ker{B_h}}$ such that
  \begin{gather*}
    B_h z_h = B_h(u-v_h),\qquad \forall v_h\in V_h,
  \end{gather*}
  and
  \begin{gather*}
    \norm{z_h}_V \le \frac1{\beta_h}\norm{B_h(u-v_h)}_{Q_h^*}
    \le\frac{\norm b}{\beta_h} \norm{u-v_h}_V.
  \end{gather*}
  Let $w_h = z_h+v_h$. Then, $w_h\in V_h^g$ since
  \begin{gather*}
    b(w_h, q_h) = b(u-v_h, q_h) = b(u, q_h) = g(q_h),
    \qquad\forall q_h\in Q_h.
  \end{gather*}
  Furthermore,
  \begin{gather*}
    \norm{u-w_h}_V \le \norm{u-v_h}_V + \norm{z_h}_V
    \le \left(1+\frac{\norm b}{\beta_h}\right)
    \norm{u-v_h}_V.
  \end{gather*}
  Since $v_h \in V_h$ was chosen arbitrarily, we have
  proven~\eqref{eq:galerkin:12} and thus by
  \blockref{Theorem}{galerkin-mixed-u-kerbh} the estimate for
  $\norm{u-u_h}_V$ with
  \begin{gather}
    c_h^{(1)} = \left(1+\frac{\norm b}{\beta_h}\right)
    \left(1+\frac{\norm a}{\gamma_h}\right),
    \qquad
    c_h^{(2)} = \left(1+\frac{\norm b}{\beta_h}\right)
    \frac{\norm b}{\gamma_h}.
  \end{gather}
  It remains to prove the estimate for $\norm{p-p_h}_Q$. Using Galerkin
  orthogonality for the test function $v_h$, we obtain
  \begin{gather}
    \label{eq:galerkin:13}
    a(u-u_h, v_h) + b(v_h, p-p_h) = 0.
  \end{gather}
  Hence, for any $q_h\in Q_h$ there is by the inf-sup condition
  $v_h\in V_h$ with $\norm{v_h}_V=1$ such that
  \begin{align*}
    \beta_h \norm{p_h-q_h}_Q
    &\le b(v_h, p_h-q_h)\\
    & = a(u-u_h, v_h) + b(v_h, p-q_h)\\
    &\le \norm a \norm{u-u_h}_V + \norm b \norm{p-q_h}_Q.
  \end{align*}
  Again, the estimate for $\norm{p-p_h}_Q$ follows by triangle
  inequality.
\end{proof}

\begin{remark}
  Indeed, if~\eqref{eq:galerkin:10} holds, we do not have to require
  that $V_h^g$ is nonempty anymore, since $B\colon V\to Q^*$ is
  surjective.
\end{remark}

\begin{remark}
  We purposely proved the preceding theorems with $\gamma_h$ and
  $\beta_h$ depending on the parameter $h$, typically the mesh
  size. This is the minimal condition for well-posedness of the
  discrete problems. Nevertheless, this well-posedness is not uniform
  in $h$, which causes loss of approximation, as the following problem
  shows. Therefore, we will only be satisfied with uniform inf-sup
  constants in applications.
\end{remark}

\begin{Problem}{infsup-uniform}
  Let the following interpolation estimates hold:
  \begin{gather*}
    \inf_{v_h\in V_h}\norm{u-v_h}_V = \mathcal O(h^k),
    \qquad
    \inf_{q_h\in Q_h}\norm{p-q_h}_V = \mathcal O(h^k).
  \end{gather*}
  Then, the estimates in %\blockref{Theorem}{galerkin-mixed-existence-p}
  Theorem~2.4.7 are asymptotically optimal if and only if there are
  constants $\tilde \gamma>0$ and $\tilde \beta>0$ independent of
  $h$ such that
  \begin{gather}
    \label{eq:galerkin:14}
    \gamma_h\ge \tilde\gamma,\quad\beta_h\ge\tilde\beta,
  \end{gather}
  independent of $h$.
\begin{solution}
  The estimates in Theorem~2.4.7 state
  \begin{align*}
    \norm{u-u_h}_V &\le c_h^{(1)} \inf_{v_h\in V_h^g}\norm{u-v_h}_V
	      + c_h^{(2)} \inf_{q_h\in Q_h}\norm{p-q_h}_Q \\
    \norm{p-p_h}_Q &\le c_h^{(3)} \inf_{v_h\in V_h}\norm{u-v_h}_V
	      + c_h^{(4)} \inf_{q_h\in Q_h}\norm{p-q_h}_Q.
  \end{align*}
  such that the additional assumptions imply
  \begin{align*}
    \norm{u-u_h}_V &\le c_h^{(1)} C h^k + c_h^{(2)} C h^k \\
    \norm{p-p_h}_Q &\le c_h^{(3)} C h^k + c_h^{(4)} C h^k.
  \end{align*}
    where the first constants are given by
  \begin{align*}
    c_h^{(1)} = \left(1+\frac{\norm b}{\beta_h}\right)
    \left(1+\frac{\norm a}{\gamma_h}\right),
    \qquad
    c_h^{(2)} = \left(1+\frac{\norm b}{\beta_h}\right)
    \frac{\norm b}{\gamma_h}.
  \end{align*}
  For the last two estimates, the relevant estimates read
  \begin{align*}
    \beta_h \norm{p_h-q_h}_Q
    &\le b(v_h,p_h-q_h)\\
    & = a(u-u_h, v_h) + b(v_h, p-q_h)\\
    &\le \norm a \norm{u-u_h}_V + \norm b \norm{p-q_h}_Q\\
    \Rightarrow \norm{p-p_h}_Q
    &\le \norm{p-q_h} + \norm{p_h-q_h}_Q \\
    &\le \frac{\norm a}{\beta_h} \norm{u-u_h}_V + \left(1+\frac{\norm b}{\beta_h}\right) \norm{p-q_h}_Q.
  \end{align*}
  Therefore, the remaining constants are
  \begin{align*}
    c_h^{(3)} = \frac{\norm a}{\beta_h}c_h^{(1)}, \qquad
    c_h^{(4)} = \frac{\norm a}{\beta_h}c_h^{(2)}+\left(1+\frac{\norm b}{\beta}\right).
  \end{align*}
\end{solution}
\end{Problem}

\begin{intro}
  As we can see from the form
  \begin{gather*}
    \forall q_h\in Q_h \;
    \exists v_h\in V_h\colon
    \quad B_h v_h = q_h
    \quad\wedge\quad
    \norm{v_h}_V \le \norm{q_h}_Q,
  \end{gather*}
  the uniform, discrete inf-sup condition introduces a compatibility
  condition between the spaces $V_h$ and $Q_h$. An immediate necessary
  condition is
  \begin{gather}
    \dim V_h \ge \dim Q_h.
  \end{gather}
  We often say that the space $V_h$ is ``rich enough'' to control
  functions in $Q_h$. Obviously, counting dimensions is not
  sufficient, since we could have added basis functions in
  $\ker{B_h}$. Even the condition
  \begin{gather*}
    \dim\ker{B_h}^T = \dim Q_h
  \end{gather*}
  is necessary and sufficient only for the existence of an inf-sup
  constant $\beta_h$ depending on $h$. Therefore, we need a stronger
  argument in order to ensure compatibility of the discrete
  spaces. Such an argument is the following lemma by Fortin. The
  projection operator $\Pi_{V_h}$ introduced there is usually referred
  to as \define{Fortin projection}.
\end{intro}

\begin{Lemma}{fortin}
  Let the inf-sup condition for the bilinear form $b(.,.)$
  hold on $V\times Q$ with a constant
  $\beta>0$. Then, it holds on $V_h\times Q_h$ uniformly with a
  constant $\tilde\beta>0$ if and only if there exists a linear
  operator $\Pi_{V_h}\colon V\to V_h$ satisfying for any $v\in V$
  \begin{align}
    \label{eq:galerkin:15}
    b(v-\Pi_{V_h}v,q_h) &= 0,\qquad\forall q_h\in Q_h, \\
    \norm{\Pi_{V_h}v}_V \le c \norm{v}_V,
  \end{align}
  with $c$ independent of $h$. There holds $\tilde\beta = \beta/c$.
\end{Lemma}

\begin{proof}
  Assume first that $\Pi_{V_h}$ exists. then, there holds for any
  $q_h\in Q_h$
  \begin{gather*}
    \sup_{v_h\in V_h} \frac{b(v_h, q_h)}{\norm{v_h}_V}
    \ge
    \sup_{v\in V} \frac{b(\Pi_{V_h}v_h, q_h)}{\norm{\Pi_{V_h} v_h}_V}
    =
    \sup_{v\in V} \frac{b(v, q_h)}{\norm{\Pi_{V_h} v_h}_V}
    \ge \frac{\beta}{c} \norm{q_h}_Q.
  \end{gather*}
  Conversely, we assume the existence of a uniform, discrete inf-sup
  constant $\tilde\beta>0$. Then, for any $v\in V$ let
  $g(.) = b(v,.) \in Q_h^*$. By
  \blockref{Theorem}{infsup-well-equivalence}, there is a unique
  element $\Pi_{V_h} v \in \ortho{\ker{B_h}}$ such that
  \begin{gather*}
    b(\Pi_{V_h}v,q_h) = b(v, q_h),\qquad\forall q_h\in Q_h
  \end{gather*}
  and
  \begin{gather*}
    \norm{\Pi_{V_h}v}_V
    \le \frac1{\tilde\beta}\norm{B_h v}_{Q_H^*}
    \le \frac{\norm b}{\tilde\beta} \norm{v}_V.
  \end{gather*}
  Thus, $\Pi_{V_h}$ is bounded and~\eqref{eq:galerkin:15} holds with
  $c=\norm b/\tilde\beta$.
\end{proof}



%%% Local Variables:
%%% mode: latex
%%% TeX-master: "main"
%%% End:



%%%%%%%%%%%%%%%%%%%%%%%%%%%%%%%%%%%%%%%%%%%%%%%%%%%%%%%%%%%%%%%%%%%%%%
%%%%%%%%%%%%%%%%%%%%%%%%%%%%%%%%%%%%%%%%%%%%%%%%%%%%%%%%%%%%%%%%%%%%%%
\section{Bringing back $c(p,q)$}
%%%%%%%%%%%%%%%%%%%%%%%%%%%%%%%%%%%%%%%%%%%%%%%%%%%%%%%%%%%%%%%%%%%%%%
%%%%%%%%%%%%%%%%%%%%%%%%%%%%%%%%%%%%%%%%%%%%%%%%%%%%%%%%%%%%%%%%%%%%%%

\begin{intro}
  The key to the mixed analysis which is also underlying our
  quasi-bestapproximation result was a splitting of the solution
  process into the reduced problem for $u$ and then applying the
  inf-sup condition for $b(.,.)$ in order to estimate $p$. This way,
  we will be able to obtain estimates for the Stokes problem, but we
  have tacitly abandoned weakly compressible elasticity. Indeed, the
  mixed form of the Lamé-Navier equations is not a constrained
  minimization problem. In this section, we will fill the gap and
  derive estimates for the solution of this problem which are robust
  in $\lambda$.

  In the Lamé-Navier equations, we had
  \begin{gather*}
    c(p,q) = -\tfrac1\lambda \scal(q,p)_{L^2(\domain)},
  \end{gather*}
  which suggests assuming symmetric and $Q$-elliptic. But, we want
  estimates independent of $\lambda$! Therefore, we should only
  require semi-definite, which on the other hand turns out a bit too
  weak.
\end{intro}

\begin{Assumption}{mixed-elliptic-stabilized}
  In addition to \blockref{Assumption}{mixed-elliptic}, let $c(.,.)$
  be positive semi-definite and elliptic on $\ker{B^T}$,
  \begin{gather}
    c(q,q) \ge u\quad\forall q\in Q,
    \qquad
    c(q,q) \ge \ellipc \norm{q}_Q^2 \quad\forall q\in \ker{B^T}.
  \end{gather}
\end{Assumption}

\begin{remark}
  Again, this assumption is not necessary for the analysis, but it
  yields a convenient and useful theorem which goes far beyond weakly
  compressible elasticity and covers stabilized methods for spaces
  where the inf-sup condition for $b(.,.)$ does not hold for the whole
  space $Q$.
\end{remark}

\begin{Theorem}{mixed-stabilized-well-posed}
  Let \blockref{Assumption}{mixed-elliptic-stabilized} hold and in
  addition let there be $\beta>0$ such that
  \begin{gather}
    \begin{split}
      \inf_{q\in \ortho{\ker{B^T}}}  \sup_{v\in V}
      \frac{b(v,q)}{\norm{v}_V\norm{q}_Q} &\ge \beta\\
      \inf_{v\in \ortho{\ker{B}}}  \sup_{q\in Q}
      \frac{b(v,q)}{\norm{v}_V\norm{q}_Q} &\ge \beta
    \end{split}
  \end{gather}
  Then, the problem finding $(u,p)\in V\times Q$ such that
  \begin{multline}
    a(u,v) + b(v,p) + b(u,q) - c(p,q) = f(v)+g(q)
    \\
    \forall v\in V, q\in Q
  \end{multline}
  has a unique solution for all $f\in V^*$ and $g\in Q^*$ and there is
  a constant $C$ such that
  \begin{gather}
    \norm{u}_V+\norm{p}_Q \le C \bigl(\norm{f}_{V^*} + \norm{g}_{Q^*}.
  \end{gather}
\end{Theorem}

\begin{proof}
  First, note that the ellipticity assumptions as well as the inf-sup
  conditions are symmetric in $V$ and $Q$. Indeed, replacing the test
  functions by their negatives, we can transform the problem into one
  where $V$ and $Q$ have exchanged their roles. Thus, it is sufficient
  to show well-posedness for $g=0$. The same result then holds for
  $f=0$ and it holds for both nonzero by linearity.
  
  We note that by the inf-sup conditions both $\range B$ and
  $\range{B^T}$ are closed. Thus, assuming $g=0$ we can decompose
  $f=f^0+f^\perp$, where $f^\perp\in \polar{\ker B}$ and
  $f^0 \in\polar{{\ortho{\ker B}}}$ and $u=u^0+u^\perp$ with
  $u^0\in\ker B$ and $u^\perp$ in its orthogonal complement.
\end{proof}

%%% Local Variables: 
%%% mode: latex
%%% TeX-master: "main"
%%% End:
