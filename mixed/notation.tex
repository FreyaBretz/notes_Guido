\subsection{Combinatorics}

\begin{Notation}{combinations}
  The set of \putindex{combination}s of $k$ elements from the set
  $\{1,\dots,n\}$ is denoted by $\Sigma(k,n)$. An element
  $\sigma\in\Sigma(k,n)$ is represented by a rising sequence
  \begin{gather}
    \sigma = (\sigma_1,\dots,\sigma_k),
    \qquad
    1\le \sigma_1 < \dots < \sigma_k \le n.
  \end{gather}
  By $\overline\sigma \in\Sigma(n-k,n)$ we denote the complement of
  $\sigma$ in the set $\{1,\dots,n\}$.
  
  We also introduce the set $\Sigma_0(k,n)$ denoting the combinations
  $(\sigma_0,\dots,\sigma_k)$ of $k+1$ values from the set
  $\{0,\dots,n\}$.
\end{Notation}


\subsection{Simplices and barycentric coordinates}

\begin{Definition}{barycentric-coordinates}
  A simplex $T\in \R^d$ with vertices $x_0,\dots,x_d$ is described by
  a set of $d+1$ \define{barycentric coordinates}
  $\lambda_0,\dots,\lambda_d$ such that
  \begin{xalignat}2
    0\le\lambda_i(x) &\le 1& i&=0,\dots,d;\quad x\in T\\
    \lambda_i(x_j) &= \delta_{ij}& i,j&=0,\dots,d\\
    \sum \lambda_i(x) &= 1.
  \end{xalignat}
\end{Definition}

\begin{remark}
  The functions $\lambda_i(x)$ are the shape functions of the linear
  $P_1$ element on $T$. They allow us to define basis functions on the
  cell $T$ without use of a reference element $\widehat T$.

  Note that $\lambda_i\equiv 0$ on the face opposite to the
  vertex $x_i$.
\end{remark}

\begin{example}
  We can use barycentric coordinates to define shape functions on
  simplicial meshes easily, as in
  Table~\ref{tab:barycentric-shapes}.
  \begin{table}[tp]
    \centering
    \begin{tabular}{|c|l|}
      \hline Degrees of freedom
      & Shape functions \\\hline
      \adjustbox{valign=center,margin=3pt}{\includegraphics[width=2cm]{./fig/p1-p.tikz}}
      &
        {\begin{minipage}[b]{6cm}
          \begin{gather}
            \phi_i = \lambda_i,
            \quad i=0,1,2
          \end{gather}
        \end{minipage}}
      \\\hline
      \adjustbox{valign=center,margin=3pt}{\includegraphics[width=2cm]{./fig/p2-p.tikz}}
      &
        {\begin{minipage}[b]{6cm}
          \begin{xalignat*}2
            \phi_{ii} &= 2\lambda_i^2 - \lambda_i,
            &i&=0,1,2\\
            \phi_{ij} &= 4\lambda_i\lambda_j
            &j&\neq i
          \end{xalignat*}
        \end{minipage}}
        \\\hline
      \adjustbox{valign=center,margin=3pt}{\includegraphics[width=2cm]{./fig/p3-p.tikz}}
      &
        {\begin{minipage}[b]{6cm}
          \begin{xalignat*}2
          \phi_{iii} &= \tfrac12 \lambda_i(3\lambda_i-1)(3\lambda_i-2)
          &i&=0,1,2\\
          \phi_{ij} &= \tfrac92\lambda_i\lambda_j(3\lambda_j-1)
          &j&\neq i\\
          \phi_0 &= 27\lambda_0\lambda_1\lambda_2
        \end{xalignat*}
        \end{minipage}}
        \\\hline
    \end{tabular}
    \caption{Degrees of freedom and shape functions of simplicial elements
      in terms of barycentric coordinates}
    \label{tab:barycentric-shapes}
  \end{table}
\end{example}

%%% Local Variables:
%%% mode: latex
%%% TeX-master: "main"
%%% End:


\subsection{Domains and function spaces}

\begin{Notation}{domain}
  By $\domain$ we denote a \define{domain} in $\R^d$, that is, an
  open, connected subset. The boundary of $\domain$ is denoted by
  $\d\domain$.

  Unless stated otherwise, we will always assume that the boundary of
  the domain consists of finitely many smooth surfaces meeting in
  ridges. Actually, we will often assume polygonal domains, such that
  the finite element mesh covers it exactly.
\end{Notation}

\begin{Notation}{vector-diff-operators}
  Differential operators for vector fields $u:\R^d\to\R^d$
  are defined as follows:
  \begin{xalignat}2
    \nabla \vu &=
    \begin{pmatrix}
      \d_1 u_1 & \cdots & \d_d u_1\\
      \vdots && \vdots \\
      \d_1 u_d & \cdots & \d_d u_d
    \end{pmatrix}
    &&\text{(gradient)}
    \\
    \div \vu &= \sum_{i=1}^d \d_i u_i
    &&\text{(divergence)}\\
    \strain u &= \frac{\nabla u + (\nabla u)^\transpose}{2}
    &&\text{(symmetric gradient)}
  \end{xalignat}

  For a tensor field $\sigma: \R^d\to \R^{d\times d}$, the divergence
  is a vector defined row-wise as
  \begin{gather}
    \div\sigma = \left(\sum_{j=1}^d \d_j \sigma_{ij}\right)_{i=1,\dots,d}
  \end{gather}
\end{Notation}

\begin{Notation}{lebesgue-spaces}
  For a domain $\domain\in\R^d$, we denote by $L^2(\domain)$\index{L2@$L^2(\domain)$} the \define{Lebesgue-space}
  of square integrable ``functions'' on $\domain$ with its norm\index{norm!L2@$\norm{\cdot}_{L^2}=\norm{\cdot}_0$}
  \begin{gather}
    \norm{u} = \norm{u}_0 = \norm{u}_{L^2(\domain)}.
  \end{gather}
  We denote the inner product of $L^2(\domain)$ by
  \begin{gather}
    \scal(u,v)_{L^2(\domain} = \form(u,v) = \int_\domain u v \dvx.
  \end{gather}
\end{Notation}

\begin{Notation}{sobolev-spaces}
  By $H^k(\domain)$\index{Hk@$H^k(\domain)$} we denote the
  \define{Sobolev space} of square integrable functions on $\domain$
  with square integrable distributional derivatives up to order
  $k$. Its norm is\index{norm!Hk@$\norm{\cdot}_{H^k}=\norm{\cdot}_{k}$}
  \begin{gather}
    \norm{u}_k = \norm{u}_{H^k(\domain)} = \sum_{\abs{\alpha} \le k}
    \norm{\d^\alpha u}_{L^2(\domain)}.
  \end{gather}
  We also use the $H^k$-seminorm\index{seminorm!Hk@$\abs{\cdot}_{H^k}$}
  \begin{gather}
    \abs{u}_k = \norm{u}_{H^k(\domain)} = \sum_{\abs{\alpha} \le k}
    \norm{\d^\alpha u}_{L^2(\domain)}.
  \end{gather}
\end{Notation}

\begin{Notation}{sobolev-spaces-2}
  By $H^1_0(\domain)$\index{H10@$H^1_0(\domain)$} we denote the
  completion of $C^\infty_0(\domain)$ with respect to the norm
  $\norm{\cdot}_{H^1(\domain)}$. Similarly,
  $H^1_{\Gamma}(\domain)$\index{H1gamma@$H^1_\Gamma(\domain)$} for any
  $\Gamma\subset\d\domain$ is the completion of all functions in
  $C^\infty(\domain)$ vanishing on $\gamma$.

  By $H^{-1}(\domain)$\index{H1m@$H^{-1}(\domain)$}, we denote the
  normed dual of $H^1_0(\domain)$, that is the space of bounded linear
  functionals on this space.
\end{Notation}

\begin{Notation}{vector-valued}
  We denote vector valued quantities by boldface letters like $\vx$,
  $\vu$, $\vv$. Lebesgue spaces of vector valued functions
  are denoted either by $L^2(\domain;\R^d)$ or by $\vL(\domain)$. The inner product of these is
  \begin{gather}
    \form(u,v) = \int_\domain \vu \cdot \vv \dx.
  \end{gather}
  Vector valued Sobolev spaces $\vH^k(\domain)$ are defined accordingly.
\end{Notation}

\begin{Notation}{transpose}
  Given two Hilbert spaces $V$ and $W$ and a mapping $A\colon V\to W^*$,
  we define the \define{adjoint} of $A$ by
  \begin{gather}
    \begin{split}
      A^\transpose\colon W &\to V^* \\
      \scal(Av,w)_{W^*\times W} &= \scal(v,A^\transpose w)_{V\times V^*}
      \qquad \forall v\in V, w\in W.
    \end{split}
  \end{gather}
  The adjoint with respect to the Euclidean inner product is called
  the \define{transpose} of $A$, denoted by $A^\transpose$ as
  well. This notation is used consistently for real and complex
  Hilbert spaces.
\end{Notation}


%%% Local Variables: 
%%% mode: latex
%%% TeX-master: "main"
%%% End: 
