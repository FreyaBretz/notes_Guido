\begin{Notation}{domain}
  By $\domain$ we denote a \define{domain} in $\R^d$, that is, an
  open, connected subset. The boundary of $\domain$ is denoted by
  $\d\domain$.

  Unless stated otherwise, we will always assume that the boundary of
  the domain consists of finitely many smooth surfaces meeting in
  ridges. Actually, we will often assume polygonal domains, such that
  the finite element mesh covers it exactly.
\end{Notation}

\begin{Notation}{vector-diff-operators}
  Differential operators for vector fields $u:\R^d\to\R^d$
  are defined as follows:
  \begin{xalignat}2
    \nabla \vu &=
    \begin{pmatrix}
      \d_1 u_1 & \cdots & \d_d u_1\\
      \vdots && \vdots \\
      \d_1 u_d & \cdots & \d_d u_d
    \end{pmatrix}
    &&\text{(gradient)}
    \\
    \div \vu &= \sum_{i=1}^d \d_i u_i
    &&\text{(divergence)}\\
    \strain u &= \frac{\nabla u + (\nabla u)^T}{2}
    &&\text{(symmetric gradient)}
  \end{xalignat}

  For a tensor field $\sigma: \R^d\to \R^{d\times d}$, the divergence
  is a vector defined row-wise as
  \begin{gather}
    \div\sigma = \left(\sum_{j=1}^d \d_j \sigma_{ij}\right)_{i=1,\dots,d}
  \end{gather}
\end{Notation}

\begin{Notation}{lebesgue-spaces}
  For a domain $\domain\in\R^d$, we denote by $L^2(\domain)$\index{L2@$L^2(\domain)$} the \define{Lebesgue-space}
  of square integrable ``functions'' on $\domain$ with its norm\index{norm!L2@$\norm{\cdot}_{L^2}=\norm{\cdot}_0$}
  \begin{gather}
    \norm{u} = \norm{u}_0 = \norm{u}_{L^2(\domain)}.
  \end{gather}
  We denote the inner product of $L^2(\domain)$ by
  \begin{gather}
    \scal(u,v)_{L^2(\domain} = \form(u,v) = \int_\domain u v \dvx.
  \end{gather}
\end{Notation}

\begin{Notation}{sobolev-spaces}
  By $H^k(\domain)$\index{Hk@$H^k(\domain)$} we denote the
  \define{Sobolev space} of square integrable functions on $\domain$
  with square integrable distributional derivatives up to order
  $k$. Its norm is\index{norm!Hk@$\norm{\cdot}_{H^k}=\norm{\cdot}_{k}$}
  \begin{gather}
    \norm{u}_k = \norm{u}_{H^k(\domain)} = \sum_{\abs{\alpha} \le k}
    \norm{\d^\alpha u}_{L^2(\domain)}.
  \end{gather}
  We also use the $H^k$-seminorm\index{seminorm!Hk@$\abs{\cdot}_{H^k}$}
  \begin{gather}
    \abs{u}_k = \norm{u}_{H^k(\domain)} = \sum_{\abs{\alpha} \le k}
    \norm{\d^\alpha u}_{L^2(\domain)}.
  \end{gather}
\end{Notation}

\begin{Notation}{sobolev-spaces-2}
  By $H^1_0(\domain)$\index{H10@$H^1_0(\domain)$} we denote the
  completion of $C^\infty_0(\domain)$ with respect to the norm
  $\norm{\cdot}_{H^1(\domain)}$. Similarly,
  $H^1_{\Gamma}(\domain)$\index{H1gamma@$H^1_\Gamma(\domain)$} for any
  $\Gamma\subset\d\domain$ is the completion of all functions in
  $C^\infty(\domain)$ vanishing on $\gamma$.

  By $H^{-1}(\domain)$\index{H1m@$H^{-1}(\domain)$}, we denote the
  normed dual of $H^1_0(\domain)$, that is the space of bounded linear
  functionals on this space.
\end{Notation}

\begin{Notation}{vector-valued}
  We denote vector valued quantities by boldface letters like $\vx$,
  $\vu$, $\vv$. Lebesgue spaces of vector valued functions
  are denoted either by $L^2(\domain;\R^d)$ or by $\vL(\domain)$. The inner product of these is
  \begin{gather}
    \form(u,v) = \int_\domain \vu \cdot \vv \dx.
  \end{gather}
  Vector valued Sobolev spaces $\vH^k(\domain)$ are defined accordingly.
\end{Notation}


%%% Local Variables: 
%%% mode: latex
%%% TeX-master: "main"
%%% End: 
