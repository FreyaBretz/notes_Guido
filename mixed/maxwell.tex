\begin{Notation}{curl}
  With $\curl u$ we describe the curl of a vector field $u$, which in
  three dimensions is defined as
  \begin{gather}
    \label{eq:maxwell:2}
    \curl u =
    \begin{pmatrix}
      \d_2u_3-\d_3u_2\\\d_3u_1-\d_1u_3\\\d_1u_2-\d_2u_1
    \end{pmatrix}.
  \end{gather}
  In two dimension, we distinguish between the vector curl of a scalar
  function and the scalar curl of a vector function
  \begin{gather}
    \curl u = \d_1u_2-\d_2u_1,
    \qquad
    \curl \phi =
    \begin{pmatrix}
      \d_2 \phi \\ -\d_1\phi
    \end{pmatrix}.
  \end{gather}
\end{Notation}

\begin{remark}
  The scalar curl of a two-dimensional vector field is equal to the
  third component of the extension of this vector field by zero into
  $\R^3$, in formulas,
  \begin{gather}
    \curl
    \begin{pmatrix}
      u_1\\u_2
    \end{pmatrix}
    =
    \curl
    \begin{pmatrix}
      u_1\\u_2\\0
    \end{pmatrix}_3.
  \end{gather}
  Similarly, the vector curl of a scalar function $\phi$ in two dimensions
  consistes of the first two components of the curl of a three
  dimensional function in the last component of the vector,
  \begin{gather}
    \curl \phi = \curl
    \begin{pmatrix}
      0\\0\\\phi
    \end{pmatrix}_{1,2}.
  \end{gather}
\end{remark}

\begin{Lemma}{curl-green}
  For vector fields $u,v\in C^1(\overline{\domain})$, there holds
  \begin{gather}
    \label{eq:maxwell:3}
    \int_{\domain} \curl u\cdot v\dx = \int_{\domain} u\cdot\curl v\dx
    + \int_{\d\domain} (n\times u) \cdot v \ds.
  \end{gather}
\end{Lemma}

\begin{intro}
  Electromagnetic fields are governed by four laws of nature put together
  by James Clerk Maxwell to a single system. The laws are
  \begin{enumerate}
  \item Gauss' law for the electric field: the electric flux
    through a closed surface equals $1/\epsilon$ times the electric
    charge enclosed by the surface:
    \begin{gather}
      \int_{\d V} E\cdot \n \ds = \int_V \frac\rho\epsilon \dx.
    \end{gather}
  \item There are no magnetic monopoles, therefore the magnetic flux
    through any closed surface vanishes:
    \begin{gather}
      \int_{\d V} B\cdot \n \ds = 0.
    \end{gather}
  \item Faraday's law of induction: the voltage induced in a closed
    loop is proportional to the rate of change of the magnetic field
    through the surface encloded by the loop:
    \begin{gather}
      \label{eq:mixed:maxwell:faraday}
      \int_{\d A} E\cdot\dl = -\frac{d}{dt}\int_A B\cdot \n \ds.
    \end{gather}
  \item Ampère's law: the magnetic field induced in a closed loop is
    proportional to the electric current plus the change of electric
    field through that loop:
    \begin{gather}
      \label{eq:mixed:maxwell:ampere}
      \int_{\d A} B\cdot\dl
      = \mu \int_A J\cdot \n \ds
      + \mu\epsilon\frac{d}{dt}\int_A E\cdot \n \ds.
    \end{gather}
  \end{enumerate}
  
  Using the Gauss theorem for the first two and the Stokes theorem for
  the remaining two laws, we obtain the \define{Maxwell equations} of
  electromagnetics
  \begin{xalignat}2
    \div E &= \frac\rho\epsilon
    & \curl E &= -\d_t B,\\
    \div B &= 0
    & \curl B &= \mu J + \mu\epsilon \d_t E.
  \end{xalignat}
  They are an hyperbolic system of equations and typically have wave
  solutions. Many simplifications have been developed to suit
  particular purposes.
\end{intro}

\begin{intro}
  An important simplification of the Maxwell equations is obtained by
  assuming an isolating material, that is, the electric current $J$
  vanishes. Additionally, we may assume that there are no electric
  charges, such that $\div E=0$. Then, taking the curl of the equation
  for $\curl E$ and inserting the formula for $\curl B$, we obtain
  \begin{gather}
    \mu\epsilon \d_t^2 E + \curl\curl E = 0
    \qquad \div E=0.
  \end{gather}
  We can even go further and study the stationary limit
  \begin{gather}
    \label{eq:maxwell:1}
    \curl\curl E=0 \qquad \div E=0.
  \end{gather}
  This is the equation we are concerned with most, since its solution
  theory also provides insight into the other forms.
\end{intro}

\begin{remark}
  A popular error in the literature consists of the following
  argument: since $\div E = 0$, there also holds $\nabla \div E =
  0$. Therefore, we can use the formula
  \begin{gather}
    \Delta \vu = \nabla\div \vu - \curl\curl \vu,
  \end{gather}
  and avoid the div-curl-problem alltogether. Unfortunately, this is
  only true, if solutions of~\eqref{eq:maxwell:1} are in
  $H^1(\domain;\R^d)$, which is not true, depending on the boundary
  conditions. See the discussion in~\cite{CostabelDauge00}.
\end{remark}

\begin{Definition}{Maxwell-boundary}
  The Maxwell equation~\eqref{eq:maxwell:1} is complemented with the
  following boundary conditions:
  \begin{itemize}
  \item Perfectly conducting:
    \begin{gather}
      \n\times \vu = 0.
    \end{gather}
    \item Natural:
      \begin{gather}
        \n\times \curl \vu = 0.
      \end{gather}
    \item Impedance:
      \begin{gather}
        \n\times \curl \vu - \alpha  (\n\times \vu) \times \n = 0.
      \end{gather}
  \end{itemize}
\end{Definition}

\begin{Definition}{Hcurl}
  The space $\Hcurl(\domain)$ is the subspace of vector fields in
  $L^2(\domain;\R^d)$ with distributional curls in $L^2(\domain;\R^d)$,
  \begin{gather}
    \Hcurl(\domain) = \bigl\{ \vu \in L^2(\domain;\R^d) \; \big\vert
    \; \nabla \times \vu \in L^2(\domain;\R^d) \bigr\}.
  \end{gather}
  It is a Hilbert space with the norm defined by
  \begin{gather}
    \norm{\vu}_{\Hcurl}^2 = \norm{\vu}_{L^2}^2 + \norm{\curl\vu}_{L^2}^2.
  \end{gather}
  In two dimensions, the same definitions apply with replacement of
  the scalar curl of the vector field.
\end{Definition}

\begin{Definition}{curl-traces}
  For $u\in C^1(\overline\domain)$, we define the trace operators
  \begin{gather}
    \label{eq:maxwell:4}
    \begin{split}
      \gamma_{\tau} &= \n\times \vu_{|\d\domain}, \\
      \gamma_{T} &= \n\times \vu_{|\d\domain} \times \n.\\
    \end{split}
  \end{gather}
  The second of these is the tangential component of $u$ on the
  boundary. Furthermore, we introduce the space $\Hcurl_0$ as the
  completion of the space of differentiable functions with compact
  support under the norm of $\Hcurl$
  \begin{gather}
    \Hcurl_0 = \overline{C^\infty_{00}(\domain;\R^d)}^{\Hcurl}.
  \end{gather}
\end{Definition}


\begin{Theorem}{curl-traces}
  The trace operator $\gamma_\tau$ can be extended to a continuous,
  surjective operator
  \begin{gather}
    \gamma_\tau \colon \Hcurl(\domain) \to Y(\d\domain),
  \end{gather}
  where
  \begin{gather}
    \label{eq:maxwell:5}
    \begin{split}
      Y(\d\domain) &= \bigl\{
      \vu\in \vH^{-1/2}_\tau(\d\domain)\; \big\vert
      \;\n\cdot(\curl \vu) \in H^{-1/2}(\d\domain) \bigr\},\\
      \vH^{-1/2}_\tau(\d\domain) &= \bigl\{
      \vu\in \vH^{-1/2}(\d\domain;\R^d) \;\big\vert \;
      \vu\cdot\n=0 \text{ a.e.}\bigr\}.
    \end{split}
  \end{gather}
  Furthermore, the trace operator $\gamma_T$ can be extended to a
  continuous operator
  \begin{gather}
    \gamma_T \colon \Hcurl(\domain) \to Y(\d\domain)^*.
  \end{gather}
\end{Theorem}


\begin{intro}
  The trace theorem indicates, that $\Hcurl_0(\domain)$ is the correct
  space to solve the problem with perfectly conducting boundary
  condition on the whole boundary. It remains now to deal with the
  divergence constraint. First, we note, that the divergence operator
  is not well-defined on $\Hcurl$, and that the subspace of $\Hcurl$
  with divergence in $L^2$ is $H^1$, which must be avoided. Therefore,
  we have to resort to a dual formulation of this constraint, which
  leads to the following weak form of the perfectly conducting Maxwell
  problem.
\end{intro}

\begin{Definition}{Maxwell-mixed-0}
  The Maxwell problem for perfectly conducting boundary conditions in
  weak form reads: find $(\vu,p)\in V\times Q$, where
  $V=\Hcurl_0(\domain)$ and $Q=H^1_0(\domain)$ such that there holds
    \begin{gather}
      \label{eq:maxwell:6}
    \begin{aligned}
      \form(\curl \vu, \curl \vv) &+ \form(\vv,\nabla p) &=&\form(\vf,\vv)
      &\forall \vv&\in \vV\\
      \form(\vu,\nabla q) & &=&0
      &\forall q&\in Q.\\      
    \end{aligned}
  \end{gather}
\end{Definition}

\begin{remark}
  At this point, our task is laid out. We have to prove well-posedness
  of the Maxwell problem in mixed form, then find suitable finite
  element spaces and commuting interpolation operators.  It turns out
  that this can be done in a more general framework, called the de
  Rham complex.
\end{remark}

% \begin{Definition}{Maxwell-mixed-imp}
%   The Maxwell problem for impedance boundary conditions in
%   weak form reads: find $(u,p)\in V\times Q$, where
%   \begin{gather}
    
%   \end{gather}
%   $V=\Hcurl_0(\domain)$ and $Q=H^1_0(\domain)$ such that there holds
%     \begin{gather}
%       \label{eq:maxwell:6}
%     \begin{aligned}
%       \form(\curl u, \curl v) &+ \form(v,\nabla p) &=&\form(f,v)
%       &\forall v&\in V\\
%       \form(u,\nabla q) & &=&0
%       &\forall q&\in Q.\\      
%     \end{aligned}
%   \end{gather}
% \end{Definition}



%%% Local Variables: 
%%% mode: latex
%%% TeX-master: "main"
%%% End: 
