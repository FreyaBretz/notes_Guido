
\section{Modelling diffusion problems}

\begin{intro}
  Diffusion problems arise when a balance law, for instance for mass
  in ground water flow or for energy in temperature conduction is
  coupled with a constitutive equation relating the direction of
  movement to the gradient of the quantity of interest.
\end{intro}

\begin{intro}
  Let $\rho$ be the density of a conserved quantity. Then, for any given
  volume we have the ``mass''
  \begin{gather*}
    m = \int_V \rho\dx.
  \end{gather*}
  Changes of this mass can be due to two processes:
  \begin{enumerate}
  \item Generation of additional mass by a source $g$,
  \item Flow of mass over the boundary of $V$ at a velocity $v$.
  \end{enumerate}
  In formulas, we have
  \begin{gather*}
    \tfrac{d}{dt} m = \int_V g\dx - \oint_{\d V} J\cdot \n \ds,
  \end{gather*}
  also known as \define{Reynolds transport theorem}. Here, $J$ is the
  \define{flux}. The exact form of the flux will be modelled later.
  The formula above is somewhat unwieldy, since it combines volume and
  surface integrals. Therefore, we apply the \putindex{Gauss theorem}
  to obtain
  \begin{gather}
    \label{eq:darcy:2}
    \frac{d}{dt} \int_V \rho\dx = \int_V g\dx - \int_V \div J \dx.
  \end{gather}
  Concentrating and assuming sufficient regularity, we arrive at the
  equation
  \begin{gather}
    \label{eq:darcy:3}
    \d_t \rho + \div J = g.
  \end{gather}
  As before in these notes, we ignore the time dependence and only
  look at stationary limits. In this case, this reduces to
  \begin{gather}
    \label{eq:darcy:4}
    \div J = g.
  \end{gather}
\end{intro}

\begin{example}
  Next we consider constitutive relations between $\rho$ and $J$ such
  that we can complement equation~\eqref{eq:darcy:4} by a second
  equation and obtain a solvable system. To this end, we consider
  thermal diffusion and ground water flow.

  \begin{description}
  \item[Heat conduction:] Here, the conserved quantity is not the
    density $\rho$, but the temperature $T$. \define{Fourier's law}
    states that the flux is proportional to the gradient of the
    temperature, pointing in opposite direction:
    \begin{gather*}
      J = -k\nabla T.
    \end{gather*}
    The constant of proportionality $k$ is the heat conductivity.
  \item[Porous media flow:] The conserved quantity is the amount of
    fluid, represented by the hydraulic head or pressure
    $p$. \define{Darcy's law} says that the flux is the product of the
    hydraulic \define{permeability} of the media and the gradient of the
    pressure:
    \begin{gather*}
      J = -K\nabla p.
    \end{gather*}
    Here, the permeability $K$ is either a positive scalar function or
    a symmetric, positive definite matrix. Note that in the latter
    case, $J$ and $\nabla p$ do not point in the same direction.
    \item[General diffusion processes:] \define{Fick's law} states,
      that the flux of a diffusion process is determined by the
      gradient of the diffusing quantity $p$ by the relation
    \begin{gather*}
      J = -D\nabla p.
    \end{gather*}
    $D$ is the symmetric, positive definite \define{diffusion tensor}.
  \end{description}
\end{example}

\begin{intro}
  From the two equations for $J$, we derive the following system of
  PDE:
  \begin{gather*}
    \arraycolsep2pt
    \begin{matrix}
      K^{-1} J &-& \nabla p &=& 0
      \div J &&&= f.
    \end{matrix}
  \end{gather*}
\end{intro}


% \begin{Definition}{mass-balance}
%   We call a quantity $p$ conserved, if the \define{conservation law}
%   \begin{gather}
%     \label{eq:darcy:1}
%     \d_t p = \div p + g
%   \end{gather}
%   holds with a \define{source} function $g$. In words, the change of
%   the quantity within a control volume is equal to the balance of
%   sources and the flow over the boundary of the volume.
% \end{Definition}


%%% Local Variables:
%%% mode: latex
%%% TeX-master: "main"
%%% End:
