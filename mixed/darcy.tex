
\section{Modelling diffusion problems}

\begin{intro}
  Diffusion problems arise when a balance law, for instance for mass
  in ground water flow or for energy in temperature conduction is
  coupled with a constitutive equation relating the direction of
  movement to the gradient of the quantity of interest.
\end{intro}

\begin{intro}
  Let $\rho$ be the density of a conserved quantity. Then, for any given
  volume we have the ``mass''
  \begin{gather*}
    m = \int_V \rho\dx.
  \end{gather*}
  Changes of this mass can be due to two processes:
  \begin{enumerate}
  \item Generation of additional mass by a source $g$,
  \item Flow of mass over the boundary of $V$ at a velocity $v$.
  \end{enumerate}
  In formulas, we have
  \begin{gather*}
    \tfrac{d}{dt} m = \int_V g\dx - \oint_{\d V} J\cdot \n \ds,
  \end{gather*}
  also known as \define{Reynolds transport theorem}. Here, $J$ is the
  \define{flux}. The exact form of the flux will be modelled later.
  The formula above is somewhat unwieldy, since it combines volume and
  surface integrals. Therefore, we apply the \putindex{Gauss theorem}
  to obtain
  \begin{gather}
    \label{eq:darcy:2}
    \frac{d}{dt} \int_V \rho\dx = \int_V g\dx - \int_V \div J \dx.
  \end{gather}
  Concentrating and assuming sufficient regularity, we arrive at the
  equation
  \begin{gather}
    \label{eq:darcy:3}
    \d_t \rho + \div J = g.
  \end{gather}
  As before in these notes, we ignore the time dependence and only
  look at stationary limits. In this case, this reduces to
  \begin{gather}
    \label{eq:darcy:4}
    \div J = g.
  \end{gather}
\end{intro}

\begin{example}
  Next we consider constitutive relations between $\rho$ and $J$ such
  that we can complement equation~\eqref{eq:darcy:4} by a second
  equation and obtain a solvable system. To this end, we consider
  thermal diffusion and ground water flow.

  \begin{description}
  \item[Heat conduction:] Here, the conserved quantity is not the
    density $\rho$, but the temperature $T$. \define{Fourier's law}
    states that the flux is proportional to the gradient of the
    temperature, pointing in opposite direction:
    \begin{gather*}
      J = -k\nabla T.
    \end{gather*}
    The constant of proportionality $k$ is the heat conductivity.
  \item[Porous media flow:] The conserved quantity is the amount of
    fluid, represented by the hydraulic head or pressure
    $p$. \define{Darcy's law} says that the flux is the product of the
    hydraulic \define{permeability} of the media and the gradient of the
    pressure:
    \begin{gather*}
      J = -K\nabla p.
    \end{gather*}
    Here, the permeability $K$ is either a positive scalar function or
    a symmetric, positive definite matrix. Note that in the latter
    case, $J$ and $\nabla p$ do not point in the same direction.
    \item[General diffusion processes:] \define{Fick's law} states,
      that the flux of a diffusion process is determined by the
      gradient of the diffusing quantity $p$ by the relation
    \begin{gather*}
      J = -D\nabla p.
    \end{gather*}
    $D$ is the symmetric, positive definite \define{diffusion tensor}.
  \end{description}
\end{example}

\begin{intro}
  From the two equations for $J$, we derive the following system of
  PDE, where we replace the letter $J$ by the more familiar $u$:
  \begin{gather}
    \label{eq:darcy:5}
    \arraycolsep2pt
    \begin{matrix}
      K^{-1} u &+& \nabla p &=& 0 \\
      \div u &&&=& f.
    \end{matrix}
  \end{gather}
  This system is closed by boundary conditions. Let $\Gamma_D$ be the
  Dirichlet boundary and $\Gamma_N$ be the Neumann boundary such that
  $\Gamma_D \cap \Gamma_N = \emptyset$ and
  $\Gamma_D\cup\Gamma_N = \d\domain$. Then, we let
  \begin{gather}
    \label{eq:darcy:6}
    \begin{aligned}
      p(x) &= p^D(x) & x & \in \Gamma_D, \\
      u(x)\cdot\n &= u^N(x)\cdot n & x & \in \Gamma_N.
    \end{aligned}
  \end{gather}

  Following the concept of finding spaces such that we have an inf-sup
  condition, we are looking for a pair with minimal regularity, such
  that we have a stable and bounded inf-sup condition. We begin the
  usual way by multiplying with a test function and integrating:
  \begin{gather}
    \label{eq:darcy:7}
    \arraycolsep2pt
    \begin{matrix}
      \displaystyle\int_\domain K^{-1} u\cdot v\dx
      &+&
      \displaystyle\int_\domain \nabla p \cdot v\dx
      &=& 0 \\
      \displaystyle\int_\domain \div u q\dx
      &&&=&
      \displaystyle\int_\domain fq\dx.
    \end{matrix}
  \end{gather}
  It turns out, we have two immediate options: first, we can integrate
  the first equation by parts, having all derivatives on $u$. On the
  other hand, we can integrate by parts in the second equation,
  leaving all derivatives on $p$ on $p$ and $q$. In the second case,
  we obtain the equation
  \begin{gather*}
    -\int_\domain u\cdot\nabla q\dx + \int_{\d\domain} u\cdot\n q \ds
    = \int_\domain fq\dx.
  \end{gather*}
  Applying the boundary condition, we first follow the recipe of
  elliptic partial differential equations and implement $p=p^D$ as an
  \putindex{essential boundary condition}, that is, the test function
  space has zero trace on $\Gamma_D$. Then, we can swap in $u^N$ for
  $u$ on $\Gamma_N$, such that the boundary term ends up on the right
  hand side.
\end{intro}

\begin{Definition}{primal-mixed}
  The \define{primal mixed formulation} of the mixed diffusion
  problem~\eqref{eq:darcy:5} reads: find $(u,p)\in V\times Q$ such
  that for all $v\in V$ and $q\in Q$ holds
  \begin{gather}
    \label{eq:darcy:8}
    \arraycolsep2pt
    \begin{array}{rcccl}
      \form(K^{-1} u, v) &+& \form( \nabla p, v)
      &=& 0 \\
      -\form(u,\nabla q)
      &&&=& \form(f,q) - \forme(u^N\cdot\n, q)_{\Gamma_N}.
    \end{array}
  \end{gather}
  The spaces are
  \begin{gather}
    \label{eq:darcy:9}
    \begin{split}
      V_h &= L^2(\domain;\R^d), \\
      Q_h &= H^1_{\Gamma_D}(\domain) = \bigl\{
      q\in H^1(\domain) \big| \;q_{|\Gamma_D} = 0
      \bigr\}.
    \end{split}
  \end{gather}
\end{Definition}

\begin{remark}
  Since the first equation is tested with the test function $v$ itself
  in all terms, we can eliminate this equation and there holds
  $u= K\nabla P$ in $L^2(\domain;\R^d)$. Entering this into the second
  equation, we obtain the well-known \putindex{primal formulation}
  \begin{gather*}
    \form(K \nabla p,\nabla q) = \form(f,q)
    - \forme(u^N\cdot\n, q)_{\Gamma_N}.
  \end{gather*}
  Just keep in mind that the ``\putindex{natural boundary condition}''
  in this case is
  \begin{gather*}
    K\nabla p\cdot n = 0.
  \end{gather*}
  Hence, the primal mixed formulation does not provide any advantages
  compared to the primal formulation, and we are not going to pursue
  it further.
\end{remark}

\begin{intro}
  Now we return to the first alternative, namely integrating by parts
  in the first equation of~\eqref{eq:darcy:7}:
  \begin{gather*}
    \int_\domain K^{-1} u \cdot v \dx - \int_\domain p \div v\dx 
    + \int_{\d\domain} v\cdot \n p\ds = 0.
  \end{gather*}
  Ensuing is a formulation multiplying and integrating the divergences
  of $u$ and $v$, respectively, with functions in $Q$. In order to fit
  this into our standard framework, we have to introduce a new Sobolev
  space. In addition, since $u\cdot\n$ does not appear as a boundary
  integral, we must make this an \putindex{essential boundary
    condition}. Thus, we require that the test functions have zero
  normal trace on $\Gamma_N$ (and justify this below). Note that now
  the Dirichlet condition $p=0$ has become a ``\putindex{natural
    boundary condition}''!
\end{intro}

\begin{Definition}{hdiv}
  Let $\domain \subset \R^d$ be a domain.  We define the
  Sobolev space
  \begin{gather}
    \Hdiv(\domain) = \bigl\{
    v\in L^2(\domain;\R^d) \big\vert
    \div v\in L^2(\domain)\bigr\},
  \end{gather}
  and its inner product
  \begin{gather}
    \scal(u,v)_{\Hdiv} = \form(u,v)_0 + \form(\div u,\div v)_0.
  \end{gather}
  Furthermore, let $C^\infty_{00}(\domain)$ be the space of smooth
  functions with compact support in $\domain$. Then, we define its
  closure in $\Hdiv(\domain)$:
  \begin{gather}
    \Hdiv_0(\domain) = \overline{C^\infty_{00}(\domain)}.
  \end{gather}
  For subset $\Gamma\subset\d\domain$, the space
  $\Hdiv_\Gamma(\domain)$ is defined accordingly (compare to
  $H^1_\Gamma(\domain)$)
\end{Definition}

Using the space $\Hdiv$ and for the moment the assumption, that
$\Hdiv_0$ and $\Hdiv_\Gamma$ serve to set boundary conditions, we can
write down our second weak fromulation of the mixed diffusion problem:

\begin{Definition}{dual-mixed}
  The \define{dual mixed formulation} of the mixed diffusion
  problem~\eqref{eq:darcy:5} reads: find $(u,p) \in V\times Q$ such
  that for all $v\in V$ and $q\in Q$ holds
  \begin{gather}
    \label{eq:darcy:10}
    \arraycolsep2pt
    \begin{array}{rcccl}
      \form(K^{-1} u, v) &-& \form(p, \div v)
      &=& \forme(p^D,v\cdot \n)_{\Gamma_D} \\
      \form(\div u, q)
      &&&=& \form(f,q).
    \end{array}
  \end{gather}
  The spaces are
  \begin{gather}
    \label{eq:darcy:11}
    V_h = \Hdiv_{\Gamma_N}(\domain),
    \qquad
    Q_h = L^2(\domain).
  \end{gather}  
\end{Definition}

\subsection{Properties of $\Hdiv(\domain)$}

\begin{Theorem}{Hdiv-separable}
  Let $\domain$ be a bounded Lipschitz domain. Then, the space
  $C^\infty(\overline\domain;\R^d)$ is dense in $\Hdiv(\domain)$.
\end{Theorem}

\begin{proof}
  % We use the statement, that a subspace $W$ is dense in a space $V$ if
  % and only if all linear functionals vanishing on $W$ also vanish on
  % $V$. Therefore, let $L \in \Hdiv(\domain)^*$. By the
  % \putindex{Riesz representation theorem}, there is
  % $u\in\Hdiv(\domain)$ such that
  % \begin{gather*}
  %   L(v) = \scal(u,v)_{\Hdiv} = \sum_{i=1}^d \scal(u_i,v_i)_0
  %   + \scal(\div u,\div v)_0
  %   \quad\forall v\in \Hdiv(\domain).
  % \end{gather*}
  % Now assume $L(\phi) = 0$ for all $\phi\in
  % C^\infty(\overline\domain;\R^d)$. Let us extend $u$ and $\div u$ outside
  % of the domain $\domain$ by zero. Then, there holds
  % \begin{gather*}
  %   \int_{\R^d} u\cdot \phi \dx + \int_{\R^d}
  % \end{gather*}
  Either by a standard mollifier argument~\cite{AdamsFournier03} or
  following~\cite[Theorem 2.4]{GiraultRaviart86}
\end{proof}

\begin{remark}
  The condition of boundedness entered the assumptions since we use
  the space $C^\infty(\overline\domain)$. It could be dropped, if we
  used a more appropriate space (cf.~\cite[Theorem
  2.4]{GiraultRaviart86}).
\end{remark}

\begin{Theorem}{Hdiv-trace}
  The \putindex{trace operator}
  $\gamma_n\colon C^\infty(\overline\domain;\R^d) \to
  C^\infty(\overline{\d\domain})$
  which maps $v\mapsto v\cdot\n_{|\d\domain}$ can be extended to a
  continuous, linear mapping
  \begin{gather}
    \gamma_n\colon \Hdiv(\domain) \to H^{-1/2}(\d\domain),
  \end{gather}
  where $H^{-1/2}(\d\domain)$ is the dual of $H^{1/2}(\d\domain)$.
\end{Theorem}

\begin{proof}
  Let $q\in C^\infty(\overline\domain)$ and
  $v\in C^\infty(\overline\domain;\R^d)$. Then, there holds
  \putindex{Green's formula}
  \begin{gather*}
    \form(v,\nabla q)_\domain
    + \form(\div v,q)_\domain
    = \forme(v\cdot\n,q)_{\d\domain}.
  \end{gather*}
  Hence,
  \begin{gather*}
    \left\vert\int_{\d\domain} v\cdot \n q \ds \right\vert
    \le \norm{v}_{\Hdiv} \norm{q}_{H^1}.
  \end{gather*}
  Applying the density of $C^\infty(\overline\domain)$ in
  $H^1(\domain)$ and of $C^\infty(\overline\domain;\R^d)$ in
  $\Hdiv(\domain)$, we can let $q$ and $v$ pass to a limit, but the
  inequality holds uniformly.

  Now apply that $H^{1/2}(\d\domain)$ is the trace space of
  $H^1(\domain)$. Therefore, for any $g\in H^{1/2}(\d\domain)$, there
  is a $q\in H^1(\domain)$ such that $q_{|\d\domain} = g$ and
  $\norm{q}_{1;\domain} \le \norm{g}_{1/2;\d\domain}$. We obtain
  \begin{gather*}
    \left\vert\int_{\d\domain} v\cdot \n g \ds \right\vert
    \le \norm{v}_{\Hdiv(\domain)} \norm{g}_{H^{1/2}(\d\domain)}
    \qquad\forall v\in \Hdiv(\domain), g\in H^{1/2}(\d\domain).
  \end{gather*}
  Hence,
  \begin{gather*}
    \norm{v\cdot\n}_{H^{-1/2}(\d\domain)} \le
    \norm{v}_{\Hdiv(\domain)}
    \qquad\forall v\in \Hdiv(\domain).
  \end{gather*}
  Thus, we have proven the continuity of the extension of $\gamma_n$
  to $\Hdiv(\domain)$.
\end{proof}

\begin{remark}
  The trace theorem tells us that our interpretation of the spaces
  $\Hdiv_0(\domain)$ and $\Hdiv_\Gamma(\domain)$ as spaces with zero
  boundary condition of the normal component is justified. This notion
  will be fortified by the two theorems below. Therefore, we will
  later avoid the notational overhead of using $\gamma_n$ and will
  simply write $v\cdot\n_{|\d\domain}$.
\end{remark}

\begin{Problem}{trace-dnu}
  Show the following result. Let $p\in H^1(\domain)$ and
  $\Delta p \in L^2(\domain)$. Then, $\d_n p\in H^{-1/2}(\d\domain)$
  and
  \begin{gather*}
    \form(\nabla p,\nabla q) = -\form(\Delta p,q) + \forme(\d_n
    p,q)_{\d\domain} \quad\forall q\in H^1(\domain).
  \end{gather*}
\end{Problem}

\begin{Theorem}{Hdiv-trace-surjective}
  The trace theorem is optimal in the sense that
  $\gamma_n\colon \Hdiv(\domain) \to H^{-1/2}(\d\domain)$ is
  surjective.
\end{Theorem}

\begin{proof}
  Let $\mu \in H^{-1/2}(\d\domain)$. We have to show that there exists
  $v\in \Hdiv(\domain)$ such that
  \begin{gather*}
    v\cdot\n = \mu \quad\text{on } \d\domain
    \qquad\text{and}\qquad
    \norm{v}_{\Hdiv(\domain)} \le \norm{\mu}_{H^{-1/2}(\d\domain)}.
  \end{gather*}

  We know that the problem
  \begin{xalignat*}2
    -\Delta \phi + \phi &= 0 &\text{in }&\domain, \\
    \d_n \phi &= \mu &\text{on }&\d\domain,
  \end{xalignat*}
  has a unique solution $\phi\in H^1(\domain)$ with
  \begin{gather*}
    \norm{\phi}_{H^1(\domain)}^2 = \forme(\mu,\phi)_{\d\domain}
    \le \norm{\mu}_{H^{-1/2}(\d\domain)}\norm{\phi}_{H^1(\domain)}.
  \end{gather*}
  The first equation then implies $\Delta\phi\in L^2(\domain)$ and
  thus $v=\nabla \phi\in \Hdiv(\domain)$. Since from this equation
  there even holds $\div v=\phi$, we obtain
  \begin{gather*}
    \norm{v}_{\Hdiv(\domain)} \le \norm{\mu}_{H^{-1/2}(\d\domain)}.
  \end{gather*}
\end{proof}

\begin{Theorem}{Hdiv-trace-kernel}
  There holds
  \begin{gather}
    \ker{\gamma_n} = \Hdiv_0(\domain).
  \end{gather}
\end{Theorem}

\begin{proof}
  The inclusion $\Hdiv_0(\domain) \subset \ker{\gamma_n}$ follows
  immediately from the definition and continuity of $\gamma_n$. For
  the opposite inclusion, we have to show that the traces of functions
  in $C^\infty_{00}(\domain)$ are dense in $\ker{\gamma_n}$. We do
  this by using, that a subspace $W$ is dense in a space $V$ if and
  only if all linear functionals vanishing on $W$ also vanish on
  $V$. Choose $u\in\ker{\gamma_n}$ and use the \putindex{Riesz
    representation theorem} to associate with it
  $L\in \ker{\gamma_n}^*$ by
  \begin{gather*}
    L(v) = \scal(u,v)_{\Hdiv} \qquad\forall v\in \ker{\gamma_n}.
  \end{gather*}
  Assume now that $L(\phi) = 0$ for all $\phi\in
  C^\infty_{00}(\domain;\R^d)$. This implies by
  \begin{gather*}
    0 = L(\phi) = \form(u,\phi)_{L^2} + \form(\div u, \div \phi),
  \end{gather*}
  that $u=\nabla \div u$ in distributional sense, and by taking limits
  of $\phi$ in $H^1$ that $\div u\in H^1(\domain)$. Hence, Green's
  formula yields
  \begin{gather*}
    L(v) = \form(\nabla\div u,v)+\form(\div u,\div v)
    = \forme(v\cdot\n,\div u)_{\d\domain} = 0
    \qquad\forall v\in \ker{\gamma_n}.
  \end{gather*}
  Thus, $L$ vanishes on all elements of $\ker{\gamma_n}$ and the
  theorem is proven.
\end{proof}

\begin{example}
  The trace theorem involves the space $H^{-1/2}(\d\domain)$, which
  requires a short discussion. On one dimensional boundaries, elements
  in $H^{1/2}(\d\domain)$ have continuous representatives. The
  situation in three dimensions is similar, where no jumps across a
  line, for instance between two faces is allowed. Therefore,
  functions in $H^{-1/2}(\d\domain)$ cannot be localized to parts of
  the boundary, for instance the edge of a cell.

  We give an example (modified from \cite[Section
  2.5.1]{BoffiBrezziFortin13}) of this phenomenon.
\end{example}

\begin{Theorem}{Hdiv-helmholtz}
  Let $\domain$ be connected. Let
  \begin{gather}
    \label{eq:darcy:1}
    V_0 = \bigl\{ v\in \Hdiv_0(\domain) \big\vert
    \div v = 0 \bigr\}.
  \end{gather}
  Then,
  \begin{gather}
    L^2(\domain;\R^d) = V_0 \oplus \ortho V,
  \end{gather}
  and
  \begin{gather}
    \ortho V = \bigl\{ v = \nabla q \big\vert
    q\in H^1(\domain) \bigr\}.
  \end{gather}
\end{Theorem}

\begin{proof}
  Let $X=\{ v= \nabla q \vert q\in H^1(\domain)\}$. we have to show
  $\ortho V = X$. Observe that $X$ is closed in $L^2$ since $H^1$ is
  complete. We show that $V_0 = \ortho X$ and thus
  \begin{gather*}
    \ortho {V_0} = \ortho{(\ortho{V_0})} = \overline X = X.
  \end{gather*}
  First, let $u\in V_0$. Then, Green's formula reduces to
  \begin{gather*}
    \form(u,\nabla q) = 0 \qquad\forall q\in H^1(\domain).
  \end{gather*}
  Hence, $V_0\subset \ortho X$. Let now conversely
  $u\in L^2(\domain;\R^d)$ such that the previous identity
  holds. Choosing $q\in C^\infty_{00}(\domain)$ yields $\div u=0$,
  which in turn means $u\in\Hdiv(\domain)$. Therefore, we can use
  Green's formula to obtain $u\cdot\n=0$ on $\d\domain$. This together
  implies $u\in V_0$, proving $\ortho X \subset V_0$.
\end{proof}

\subsection{Well-posedness of the dual mixed formulation}

\begin{intro}
  In order to apply the theorey from
  Chapter~\ref{sec:mixed-wellposedness}, we have to define the
  abstract bilinear forms $a(.,.)$ and $b(.,.)$. We read from the dual
  mixed formulation
  \begin{align*}
    a(u,v) &= \form(K^{-1}u,v) \\
    b(v,q) &= \form(\div v,q).
  \end{align*}
\end{intro}


\begin{Problem}{mixed-inhomogeneous-bc}
  In both the primal and the dual mixed formulation, we ignored
  inhomogeneous essential boundary conditions. Show that the usual
  lifting method applies. Determine the modified equations and the
  spaces needed for the liftings.
\end{Problem}


\begin{Lemma}{darcy-reduced-wellposed}
  Let $V=\Hdiv(\domain)$ and $Q=L^2(\domain)$ with their norms. Let
  \begin{gather}
    \label{eq:darcy:12}
    V_0 = \ker{B} = \bigl\{v\in V\big\vert
    \form(\div v,q) =0 \;\forall q\in Q\bigr\}.
  \end{gather}
  Assume there exist constants $\ellipa$ and $\norm a$ such that
  \begin{gather}
    \label{eq:darcy:13}
    \ellipa \abs{\xi}^2
    \le \xi^T K^{-1} \xi \le \norm a \abs{\xi}^2
    \qquad\forall \xi\in\R^d.
  \end{gather}
  Then, there holds
  \begin{xalignat}2
    \label{eq:darcy:14}
    a(u,v) & \le \norm a \norm{u}_V \norm{v}_V
    & \forall u,v&\in V\\
    \label{eq:darcy:15}
    a(u,u) & \ge \ellipa \norm{u}_V^2
    & \forall u &\in \ker B.
  \end{xalignat}
\end{Lemma}

\begin{remark}
  Differing from the Stokes problem, ellipticity of $a(.,.)$ cannot be
  extended to the whole space $V$. This is going to be the major
  difference between this chapter and the previous.
\end{remark}

\begin{Lemma}{darcy-infsup}
  Let $V=\Hdiv(\domain)$ and $Q=L^2(\domain)$ with their norms.  Then,
  the inf-sup condition
  \begin{gather}
    \inf_{q\in Q} \sup_{v\in V} \frac{b(v,q)}{\norm{v}_V\norm{q}_Q}
    \ge \beta
  \end{gather}
  holds with a constant $\beta$ depending on the domain.
\end{Lemma}

\begin{proof}
  We can use the construction leading to
  \blockref{Corollary}{stokes-iso}. In fact, since the norm of $\Hdiv$
  is weaker than the one of $H^1$, the same function $v$ can be chosen
  in the Stokes inf-sup condition~\eqref{eq:stokes:1}, yielding a
  constant $\beta$ not worse than for Stokes.
\end{proof}

Combining these lemmas, we obtain

\begin{Theorem}{darcy-well-posed}
  Under the assumptions on \blockref{Lemma}{darcy-reduced-wellposed},
  the \putindex{dual mixed formulation} is well-posed.
\end{Theorem}

\section{Discretization of dual mixed problems}

\subsection{Conforming subspaces of $\Hdiv(\domain)$}

\begin{intro}
  Our goal in this section is the derivation of general criteria
  applying to the approximation of $\Hdiv(\domain)$ by piecewise
  polynomial functions. This affects in particular continuity
  conditions and the properties of $\ker{B_h}$.
  
  As be before, we will assume that all families of meshes $\mesh_h$
  for $h\to 0$ are shape-regular. We will also assume that meshes are
  regular, unless otherwise stated.
\end{intro}

\begin{Lemma}{normal-continuity}
  Let $\mesh_h$ be a subdivision of the domain $\domain$. Let the
  space $V_h$ be cell-wise polynomial. We have $V_h \subset
  \Hdiv(\domain)$ if and only if on each interior face $\face$
  between two cells $\cell_1$ and $\cell_2$ holds
  \begin{gather}
    v_1 \cdot \n_1 + v_2 \cdot\n_2 = 0.
  \end{gather}
  Here, $v_1$ and $v_2$ are the traces of the functions on $\face$
  from each cell.
\end{Lemma}

\begin{proof}
  Since $V_h$ is by definition finite dimensional, all norms are
  bounded. It remains to show that the distributional divergence is
  in $L^2(\domain)$, that is, all its contributions which are Borel
  measures of faces vanish. To this end, let $\phi\in
  C^\infty_{00}(\domain)$ be a test function such that its support
  does not have a nonempty intersection with any face except
  $\face$. Then, we have by Green's formula for $u\in V_h$
  \begin{gather*}
    \form(\div u,\phi) = -\form(u,\nabla \phi)
    + \forme(v_1 \cdot \n_1 + v_2 \cdot\n_2, \phi)_F
  \end{gather*}
  We have $\div u \in L^2(\domain)$ if and only if the face term
  vanishes.
\end{proof}

\begin{intro}
  In \blockref{Lemma}{darcy-reduced-wellposed}, we saw that the
  bilinear form $a(.,.)$ is elliptic only on the kernel of
  $B$. Indeed, for the simplest case with $K\equiv 1$, we conclude
  that if the uniform estimate for $v_h\in \ker{B_h}$
  \begin{gather*}
    \norm{v_h}^2_{L^2}
    \ge \ellipa \norm{v_h}^2_{\Hdiv}
    = \ellipa \bigl(\norm{v_h}^2_{L^2} + \norm{\div v_h}^2_{L^2}\bigr),
  \end{gather*}
  necessary for quasi-bestapproximation requires a constant $c$
  independent of $h$ such that
  \begin{gather*}
    \norm{\div v_h}^2_{L^2} \le c \norm{v_h}^2_{L^2}
    \qquad\forall v_h\in\ker{B_h}.
  \end{gather*}
  The inverse estimate is insufficient by two powers of $h$, such that
  this is actually a hard condition. Therefore, we focus on
  methods where
  \begin{gather}
    \label{eq:darcy:16}
    \ker{B_h} \subset \ker B.
  \end{gather}
\end{intro}

%%% Local Variables:
%%% mode: latex
%%% TeX-master: "main"
%%% End:
