
\begin{example}
  Let $\phi:\ell^2(\R)\to\ell^2(\R)$ be such that
  \begin{gather*}
    v =
    \begin{pmatrix}
      x_1 \\ x_2 \\ \vdots \\ x_k \\ \vdots
    \end{pmatrix}
    \mapsto \phi(v) =
    \begin{pmatrix}
      x_1 \\ 4x_2 \\ \vdots \\ k^2x_k \\ \vdots
    \end{pmatrix}.
  \end{gather*}
  Clearly, $\phi$ is linear. But if we consider the sequence of
  vectors $v_n = \{\delta_{nk}\}$, we see that $v_n \mapsto n^2 v_n$ and
  thus, while the sequence is bounded in $\ell^2(\R)$, its image is
  not.
  
  Moreover, take now the sequence of sequences
  \begin{gather*}
    v_n = \left\{1,\tfrac12,\dots,\tfrac1n,0,\dots,0\right\}
    \quad \mapsto \quad \phi(v_n) = \{1,2,\dots,n,0,\dots,0\}.
  \end{gather*}
  The sequence $v_n$ converges to a limit $v \in \ell^2(\R)$, while
  the sequence $\phi(v_n)$ diverges. While $\phi(v_n)$ is defined for
  all $v_n$, it is not bounded for the limit $v$.
\end{example}

\begin{remark}
  By virtue of completeness of the space, whenever a linear operator
  is not bounded, it must be undefined for some vectors. We could
  exclude such operators from our considerations, but we would
  severely limit the theory we want to develop. Instead, we will
  accept the fact, that we have to extend the notion of a linear
  mapping $\phi: V\to W$ to a linear operator $\phi: V\to W$, which
  may not be defined on all of $V$. The following definition fixes
  this problem somewhat.
\end{remark}

\begin{Definition}{operator-domain}
  Let $\phi: V\to W$ be a linear operator. Then, the \define{domain}
  of $\phi$ is
  \begin{gather*}
    \mathcal D(\phi) = \bigl\{ v\in V \big|
    \phi(v) \in W \bigr\}.
  \end{gather*}
  Here, $\phi(v) \in W$ implies that $\phi(v)$ is also well defined.
\end{Definition}



%%% Local Variables:
%%% mode: latex
%%% TeX-master: "main"
%%% End:
