\svnid{$Id$}

\begin{remark}
  This part of the notes deals with preconditioning of symmetric
  operators, or those, which have a dominating symmetric part. The
  theory of preconditioning methods for nonsymmetric and in particular
  non-normal operators is currently barely developed and thus cannot
  be covered by these notes. 
\end{remark}

\begin{example}
  While the methods developed in this chapter are fairly general, we
  introduce a specific model problem as a simple benchmark case. To
  this end, we consider the Dirichlet problem: find $u\in V =
  H^1_0(\Omega)$ such that
  \begin{gather}
    \label{eq:itintro:1}
    a(u,v) \equiv \int_\Omega \nabla u\cdot \nabla v \dx
    = \int_\Omega f v \dx \equiv f(v),
    \qquad \forall v\in V.
  \end{gather}
  The finite dimensional linear systems of equations are derived from
  finite element discretizations on quasi-uniform meshes of cells with
  maximal diameter $h$, yielding a sequence of spaces $V_h$, on which
  linear systems are introduced by the same weak
  form~\eqref{eq:itintro:1}.
\end{example}

\begin{notation}
  With a bilinear form $a(.,.)$ on $V\times V$ we associate the
  operator $A: V\to V^*$ by
  \begin{gather}
    \label{eq:itintro:2}
    \scal(Au,v) = a(u,v), \quad \forall v\in V,
  \end{gather}
  where $\scal(.,.):V^*\times V \to \R$ is the canonical bilinear form of an
  element of $V^*$ and an element of $V$, namely
  \begin{gather}
    \label{eq:richardson:9}
    \scal(f,v) = f(v), \quad \forall f\in V^*, v\in V.
  \end{gather}
  
  We will tacitly assume that operators $A$, $B$, etc.\ are defined by
  equation~\eqref{eq:itintro:2} and the bilinear forms $a(.,.)$,
  $b(.,.)$, etc., respectively, if they are not defined otherwise.
  
  After choosing a basis for a finite dimensional space $V_n$ or a
  Schauder basis for the space $V$ (assuming $V$ separable), say
  $\{\phi_i\}$, we can define a (possibly infinite-dimensional) matrix
  $\mat A$ associated with the bilinear form $a(.,.)$ with the entries
  \begin{gather*}
    a_{ij} = a(\phi_j, \phi_i).
  \end{gather*}
  
  If we restrict the bilinear forms to a finite dimensional subspace
  $V_n$, we denote the matrices $\mat A$ restricted to this subspace
  by $\mat A_n$. Accordingly, we define the bounds
  \begin{gather}
    \label{eq:richardson:8}
    \Lambda_n = \max_{u\in V_n}\frac{a(u,u)}{\norm{u}_V^2},
    \qquad
    \lambda_n = \min_{u\in V_n}\frac{a(u,u)}{\norm{u}_V^2}.
  \end{gather}
\end{notation}

%%% Local Variables: 
%%% mode: latex
%%% TeX-master: "main"
%%% End: 
