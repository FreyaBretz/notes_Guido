\svnid{$Id$}

\section{The Richardson iteration}

\begin{intro}
  As a first example and prototype for all other iterative methods we
  consider Richardson's method, which for matrices and vectors in
  $\R^n$ reads
  \begin{gather}
    \label{eq:richardson:1}
    \vec x^{(k+1)}
    = \vec x^{(k)}
    - \omega_k \bigl(\mat A \vec x^{(k)} - \vec b \bigr).
  \end{gather}
  $\omega_k$ is a relaxation parameter, which can be chosen a priori
  or can be changed in every step. We will for simplicity assume
  $\omega_k = \omega$.
  
  It can be shown easily, that if $\mat A$ is symmetric, positive definite,
  the method converges if $\omega$ is sufficiently small. More
  precisely, let
  \begin{gather}
    \label{eq:richardson:3}
    \lambda
    = \min_{x\in \R^n} \frac{\vec x^T\mat A\vec x}{\vec x^T\vec x},
    \qquad\text{and}\qquad
    \Lambda = \max_{x\in \R^n} \frac{\vec x^T\mat A\vec x}{\vec x^T\vec x}.
  \end{gather}
  Then, the method converges for $0 < \omega < 2\Lambda$. The optimal
  choice is
  \begin{gather}
    \label{eq:richardson:2}
    \omega = \frac{2}{\lambda+\Lambda},
  \end{gather}
  which leads to a contraction rate of
  \begin{gather}
    \label{eq:richardson:4}
    \rho = \frac{\Lambda-\lambda}{\Lambda+\lambda} =
    \frac{\kappa-1}{\kappa+1} = 1 -\frac2\kappa + \mathcal
    O(\kappa^{-2}),
  \end{gather}
  where $\kappa = \Lambda/\lambda$ is the spectral condition number. 
\end{intro}

\begin{intro}
  The analysis of finite element methods shows that it is beneficial
  to give up the focus on finite dimensional spaces and rather use
  theory that applies to separable Hilbert spaces. If results can
  obtained in this context, they can easily be restricted to finite
  dimensional subspaces and thus become uniform with respect to the
  mesh parameter. Thus, we will first reformulate Richardson's method
  for this case and then derive convergence estimates.
\end{intro}

\begin{intro}
  Elements of an abstract Hilbert space $V$ will be denoted by
  $u,v,w$, etc. On the other hand, coefficient vectors in $\R^n$ are
  denoted by letters $\vec x,\vec y,\vec z$, etc.
\end{intro}

\begin{definition}
  Let $V$ be a Hilbert space with inner product $\scal(.,.)_V$. Let
  $a(.,.)$ be a second bilinear form on $V$. Then, for any right hand
  side $f\in V^*$ and any start vector $u^{(0)}\in V$,
  \define{Richardson's method} is defined by the iteration
  \begin{gather}
    \label{eq:richardson:5}
    \scal(u^{(k+1)},v)_V = \scal(u^{(k)},v)_V
    - \omega_k \bigl(a(u^{(k)},v) - f(v)\bigr), \qquad \forall v\in V.
  \end{gather}
  $\omega_k$ is a suitable \putindex{relaxation parameter}, chosen
  such that the method converges.
\end{definition}

\begin{note}
  The scalar products in~\eqref{eq:richardson:5} become necessary,
  since different from the case in $\R^n$, the result of applying the
  bilinear form $a(.,.)$ to $u^{(k)}$ in the first argument yields a
  linear form on $V$. In order to convert this to a vector in $V$, we
  have to apply the isomorphism induced by the \putindex{Riesz
    representation theorem}.
\end{note}

\begin{theorem}
  Let the bilinear form $a(.,.)$ be bounded and elliptic on $V\times
  V$, namely, let there exist positive constants $\Lambda$ and $\lambda$ such
  that for all $u,v\in V$ there holds
  \begin{gather}
    \label{eq:richardson:6}
    a(u,v) \le \Lambda \norm{u}_V \norm{v}_V,
    \qquad
    a(u,u) \ge \lambda \norm{u}_V^2.
  \end{gather}
  Then, the Richardson iteration~\eqref{eq:richardson:5} converges for
  $\omega_k = \omega \in (0, 2 \Lambda)$. The optimal
  choice is $\omega$ according to~\eqref{eq:richardson:2}, in which
  case the convergence rate is given by~\eqref{eq:richardson:4}.
\end{theorem}

\begin{proof} The proof is for instance used in the proof of the
  Lax/Milgram lemma.
\end{proof}

\begin{note}
  It is clear that the boundedness and ellipticity
  estimates~\eqref{eq:richardson:6} hold for any finite dimensional
  subspace $V_n\subset V$, and thus the convergence
  estimate~\eqref{eq:richardson:4} becomes independent of $n$.
\end{note}  

\begin{definition}
  After choosing a basis for a finite dimensional space $V_n$ or a
  Schauder basis for the space $V$ (assuming $V$ separable), say
  $\{\phi_i\}$, we can define a (possibly infinite-dimensional) matrix
  $\mat A$ associated with the bilinear form $a(.,.)$ with the entries
  \begin{gather*}
    a_{ij} = a(\phi_j, \phi_i).
  \end{gather*}
  Similarly, we define the matrix $\mat B$ associated with the $V$-inner
  product through
  \begin{gather*}
    b_{ij} = \scal(\phi_j,\phi_i)_V.
  \end{gather*}
  We note that $\mat B$ is the matrix for the operator converting a vector
  $u\in V$ to its dual vector $u^*\in V^*$, and $\mat B^{-1}$ is
  the matrix associated with the Riesz isomorphism from $V^*$ to $V$.
  
  If we restrict the bilinear forms to a finite dimensional subspace
  $V_n$, we denote the matrices $\mat A$ and $\mat B$ restricted to
  this subspace by $\mat A_n$ and $\mat B_n$,
  respectively. Accordingly, we define the bounds
  \begin{gather}
    \label{eq:richardson:8}
    \Lambda_n = \max_{u\in V_n}\frac{a(u,u)}{\norm{u}_V^2},
    \qquad
    \lambda_n = \min_{u\in V_n}\frac{a(u,u)}{\norm{u}_V^2}.
  \end{gather}
\end{definition}

\begin{note}
  When we apply Richardson's method as in~\eqref{eq:richardson:5} on a
  computer, each step involves a multiplication with the matrix $A$,
  but an inversion of the matrix $B$, corresponding to the iteration
  \begin{gather*}
    \mat B \vec x^{(k+1)}
    = \mat B \vec x^{(k)}
    - \omega_k \bigl(\mat A \vec x^{(k)} - \vec b \bigr),
  \end{gather*}
  or equivalently,
  \begin{gather}
    \label{eq:richardson:7}
    \vec x^{(k+1)}
    = \vec x^{(k)}
    - \omega_k \mat B^{-1}\bigl(\mat A \vec x^{(k)} - \vec b \bigr).
  \end{gather}
  The iteration in~\eqref{eq:richardson:7} is commonly referred to as
  preconditioned Richardson iteration and $\mat B^{-1}$ as the
  \putindex{preconditioner}. Note that by introducing the iteration in
  its weak form~\eqref{eq:richardson:5}, the preconditioner arrives
  naturally and with necessity.
  
  The goal of this chapter is finding preconditioners $\mat B^{-1}$, or
  equivalently inner products $\scal(.,.)_V$, such that the bilinear
  form $a(.,.)$ is bounded and the condition number
  $\kappa = \Lambda/\lambda$ is small.
\end{note}

\begin{example}
  Let us take for example $V = H^1_0(\Omega)$ and
  \begin{gather*}
    a(u,v) = \int_{\Omega} \nabla u \cdot \nabla v\dx.
  \end{gather*}
  By the Poincaré-Friedrichs inequality, $a(.,.)$ is an inner product
  on $V$ and thus we can choose $\scal(.,.)_V = a(.,.)$. In
  particular, $\lambda = \Lambda = 1$ and the optimal choice is
  $\omega = 1$. Then, Richardson's iteration becomes
  \begin{gather*}
        a(u^{(k+1)},v)_V = a(u^{(k)},v)_V
    - \omega_k \bigl(a(u^{(k)},v) - f(v)\bigr) =  f(v), \qquad \forall v\in V,
  \end{gather*}
  which converges in a single step, but we have to solve the original
  equation for $u$. Thus, either the inversion of the matrix $A_n$ is
  trivial on each finite dimensional subspace $V_n$, or the method is
  useless. With usual finite element bases, the latter is true.
\end{example}

\begin{example}
  In the other extreme, we would like to use the $\R^n$ or $L^2$
  inner product on $V_n$ or $V$, such that the Riesz isomorphism is
  easily computable. But then, the bilinear form $a(.,.)$ is unbounded
  on $V$. Thus, while for each finite $n$, the condition number
  $\kappa_n = \Lambda_n/\lambda_n$ exists, it converges to infinity if
  $n\to\infty$.
\end{example}

\begin{todo}
  Show that the condition number grows like $1/h^2$ for finite element
  methods.
\end{todo}


%%% Local Variables: 
%%% mode: latex
%%% TeX-master: "main"
%%% End: 
