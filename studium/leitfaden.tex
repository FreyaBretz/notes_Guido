\documentclass{article}
\uspeackage[utf8]{inputenc}
\usepackage[german]{babel}
\title{Gedanken zum Studienanfang}
\author{Guido Kanschat}
\begin{document}
\maketitle

\section{Zur Begriffsbildung}

Aufgaben und Teilnehmer des Universitätsbetriebs haben andere Namen als wir sie aus der Schule gewohnt sind. Aus diesen Bezeichnungen kann man bereits Unterschiede in den Methoden und Haltungen der Akteure ableiten. Deswegen ist es sinnvoll, sich diese Begriffe zu vergegenwärtigen.

\begin{description}

\item[studium,] Lat., Subst., n., die Beschäftigung, insbesondere die angenehme Beschäftigung mit einem Gegenstand des Interesses, im Gegensatz zu \textbf{otium}, der Arbeit aus Pflicht und Nützlichkeit. Es ist also die intellektuelle Auseinandersetzung, das durchaus anspruchsvolle Spiel, mit dem gewählten Fachgebiet. Die Durchdringung des Fachwissens passiert also nicht durch das Zuhören in der Vorlesung, sondern durch die eigenständige, aktive Auseinandersetzung mit dem Stoff Kreise der \textbf{comilitones}. Die Wahl des Fachgebietes folgt damit dem eigenen, inneren Interesse, nicht einem äußeren Zwang. Dies steht nicht im Gegensatz zur Tatsache, dass vor der Beschäftigung mit den Aspekten des Fachs, die von eigentlichem Interesse sind, die Erlernung und Absorption von Grundlagen nötig ist. Umgekehrt sollte aber auch für diese Grundlagen hinreichendes Interesse bestehen.

\item[professor,] Lat., Subst., m., von \textbf{proficisci}, Verb, bekennen, inbesondere eine Auffassung öffentlich vertreten. Im Gegensatz zu einem Lehrer, der fremdbestimmten Unterrichtsstoff vorträgt, wird ein Professor in der Regel Fakten darbieten, an deren notwendige Vermittlung er glaubt.

\end{description}

\end{document}

%%% Local Variables: 
%%% mode: latex
%%% TeX-master: t
%%% End: 
