\begin{Definition}{sparse-matrix}
  We call a matrix \textbf{sparse}\index{sparse matrix}, if the number
  of nonzero entries is much less than the total number of entries,
  typically $\bigo(n)$ instead of $\bigo(n^2)$. The distribution of nonzero entries is called \define{sparsity pattern}.

  More abstractly, the application of a sparse matrix to a vector
  requires considerably less than $n^2$ operations.
\end{Definition}

\begin{Example*}{csr}{Compressed row storage (CSR)}
  A sparse matrix can be stored using two integer fields defining the
  sparsity pattern and a field of floating point values for its
  entries. Let \lstinline!n! be the dimension of the matrix and
  \lstinline!n_nonzero! the total number of nonzero entries.
  \begin{lstlisting}[language=Python]
    import numpy as np
    row_start = np.empty(n, dtype=np.uint)
    column = np.empty(n_nonzero, dtype=np.uint)
    entries = np.emtpy(n_nonzero, dtype=np.double)
  \end{lstlisting}

  The operation $\vy=\mata\vx$ is then implemented as
  \begin{lstlisting}[language=Python]
    for i in range(0,n):
    y[i] = 0.
    for j in range (row_start[i],row_start[i+1):
      y[i] += entries[j]*x[column[j]]
  \end{lstlisting}  
\end{Example*}

\begin{Remark}{algorithmic-matrix}
  In cases where the sparsity pattern and the entries of a matrix are
  known at compile time, a compressed stored matrix can be substituted
  by an algorithm performing the multiplication.

  Thus, we start viewing matrices more as linear operators than as
  rectangular schemes of numbers.
\end{Remark}
%%% Local Variables:
%%% mode: latex
%%% TeX-master: "main"
%%% End:
