\begin{intro}
  The following results can be found in any book on linear
  algeba. Thus, we will just keep the arguments short. There will be a
  focus on normal matrices justified by results on conditioning of
  eigenvalue problems later on.

  Thus, spectral theory based on module theory will not be needed in
  this class. The spectral theorem for normal matrices on the other
  hand is fairly simple and can be proved without too much overhead.
\end{intro}

\begin{Definition}{eigenvalue}
  An \define{eigenvalue} of a matrix $\mata\in \C^{n\times n}$ is a
  complex number $\lambda$ such that the matrix $\mata-\lambda\id$ is
  singular.

  The set of all eigenvalues of $\mata$ is called the
  \define{spectrum} $\sigma(\mata)$.

  The \define{eigenspace} for $\lambda$ is the kernel of
  $A-\lambda\id$, that is, the set
\begin{gather}
    \esp{\lambda} = \bigl\{
    \vv \in \C^n \;\big\vert\;
    \mata\vv = \lambda\vv \bigr\}.
\end{gather}
The \define{geometric multiplicity} of $\lambda$ is the dimension of
$\esp{\lambda}$.


An \define{eigenvector} for $\lambda$ is a (normed) vector in
$\esp\lambda$. We refer to an eigenvector $\lambda$ and a
corresponding eigenvector $\vv$ as \define{eigenpair}.
\end{Definition}

\begin{Definition}{eigenvalue-algebraic}
  An \define{eigenvalue} of a matrix $\mata\in \C^{n\times n}$ is a root of the characteristic polynomial $\chi(\lambda) = \det(\mata-\lambda\id)$.
  
  The \define{algebraic multiplicity} of an eigenvalue is the multiplicity of the corresponding root of the characteristic polynomial.
\end{Definition}

\begin{Lemma}{eigenvalue-equivalent}
  The two definitions of an eigenvalue are consistent.
\end{Lemma}

\begin{Theorem}{eigenvalue-count}
  Every matrix in $\C^{n\times n}$ has at most $n$ eigenvalues. The algebraic multiplicities of all eigenvalues add up to $n$.
\end{Theorem}

\begin{proof}
  The ``at most'' follows from the fact that a polynomial contains
  linear factors $x-\lambda_i$ for each of its roots
  $\lambda_i$. Thus, if the characteristic polynomial has $k$ roots it
  has at least degree $k$. On the other hand, the characteristic
  polynomial has degree $n$, such that $k\le n$.

  The second statement is due to the fact that every polynomial over
  $\C$ is a product of linear factors.
\end{proof}

\begin{remark}
  The last theorem is not true in $\R$, as it is not algebraically
  closed. Thus, even a real matrix may have complex eigenvalues and
  eigenvectors. Therefore, all results in this chapter will be on
  complex matrices, but some simplifications for real matrices will be
  pointed out.
\end{remark}

\begin{Definition}{eigenvalue-simple}
  An eigenvalue is \define{simple}, if its algebraic and geometric multiplicity are one. It is \define{semi-simple}, if its algebraic and geometric multiplicities are equal.
\end{Definition}

\begin{remark}
  It is possible to refine the concept of eigenvalues and eigenvectors by distinguishing between right eigenvalues and vectors solving
  \begin{gather}
      \mata \vv = \lambda \vv,
  \end{gather}
  and left eigenvalues and vectors solving
  \begin{gather}
    \vu \mata = \lambda \vu,
  \end{gather}
  where $\vu$ is now a row vector. By taking the transpose of this equation\footnote{Here, we refer to the real transpose obtained by simply exchanging indices, not the complex conjugate transpose.},
  \begin{gather}
    \label{eq:evp:1}
    \mata^T \vu^T = \lambda \vu^T,
  \end{gather}
  we see that $\vu$ is a left eigenvector of $\mata$ if and only if
  $\vu^T$ is a right eigenvector of $\mata^T$.
\end{remark}

\begin{Lemma}{eigenvalues-conjugate}
  For every eigenpair $(\lambda,\vv)$ of $\mata\in\C^{n\times n}$, there is an eigenpair $(\overline\lambda,\overline{\vv})$ of $\mata^*$, and vice versa.
\end{Lemma}

\begin{proof}
  The determinant does not change when the matrix is transposed, therefore
  \begin{gather}
    \chi(\mata^T)
    = \det(\mata^T-\lambda \id)
    = \det(\mata-\lambda \id)
    = \chi(\mata).
  \end{gather}
  Thus, the eigenvalues of $\mata$ and of $\mata^T$ coincide, which
  by~\eqref{eq:evp:1} implies the result, if we not only transpose,
  but also take the complex conjugate.
\end{proof}

\subsection{Normal and Hermitian matrices}

\begin{Definition}{normal-Hermitian}
  A matrix $\mata\in\C^{n\times n}$ is called \define{normal} if there holds
  \begin{gather}
      A^*A = AA^*.
  \end{gather}
  It is called \define{Hermitian} or \define{complex symmetric}, if there holds
  \begin{gather}
      A=A^*.
  \end{gather}
\end{Definition}

\begin{Theorem*}{normal-diagonalizable}{Spectral theorem for normal matrices}
  A matrix $\mata\in\C^{n\times n}$ is normal if and only if it is diagonalizable by a unitary matrix.
  
  It is normal if and only if there exists an orthonormal basis of eigenvectors.
\end{Theorem*}

\begin{proof}
  
\end{proof}

\begin{Corollary*}{symmetric-diagonalizable}{Spectral theorem for Hermitian matrices}
  A Hermitian matrix $\mata\in\C^{n\times n}$ is diagonalizable with
  an orthogonal basis of eigenvectors and real eigenvalues.  
\end{Corollary*}

\begin{proof}
  Hermitian matrices are a special case of normal matrices, such that
  \slideref{Theorem}{normal-diagonalizable} applies.
\end{proof}

%%% Local Variables:
%%% mode: latex
%%% TeX-master: "main"
%%% End:
