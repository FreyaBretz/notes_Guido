\documentclass[ignorenonframetext,notheorems,hidelinks,aspectratio=1610]{beamer}
%\usetheme[compress]{JuanLesPins}
\usetheme{JuanLesPins}
\usecolortheme{iwr}
\usepackage{mathsim}
\lstset{language=Python}
\usetikzlibrary{snakes}
\tikzset{velox/.style={color=black,draw,fill=red,thick,%
    shape=diamond,aspect=.4,
    inner sep=1.3pt,transform shape}}
\tikzset{veloy/.style={color=black,draw,fill=red,thick,%
    shape=diamond,aspect=2.5,
    inner sep=1.3pt,transform shape}}
\tikzset{veloxy/.style={color=black,draw,fill=red,thick,%
    shape=star,star points=4,star point ratio=2.2,
    inner sep=1.3pt,transform shape}}
\tikzset{pressure/.style={color=black,draw,fill=cyan,thick,%
    shape=circle,inner sep=2pt,transform shape}}
\tikzset{velo/.style={transform shape,double=red,arrows={-Stealth[open,fill=red]}}}

%% Macros for drawing degrees of freedom for different shapes/elements.
%% Arguments are always:
%%   #1: Starting point
%%   #2: End point
%%   #3: polynomial degree
%%   #4: node settings

\tikzset{pics/edgenormal/.style args={#1/#2/#3/#4}{%
    code={%
      \draw #1 -- #2
      node foreach \x [evaluate=\x as \xval] in {1,...,#3} [#4,sloped,pos=\xval/(#3+1)] {};
      }
}}


%% Macros for drawing degrees of freedom for different shapes/elements.
%% Arguments are always:
%%   #1: polynomial degree
%%   #2: node settings

\tikzset{pics/tripile/.style args={#1/#2}{%
    code={%
      \coordinate (top) at (0,#1);
      \foreach \i in{0,...,#1}
      \foreach \j in{0,...,\i}
      {
        \tikzmath{
          \y = .3*(2/3*#1-\i)*cos(30);
          \x = .3*(\i/2-\j);
        }
        \node[#2] at (\x,\y) {};
      }
    }
}}

\tikzset{pics/tensor/.style args={#1/#2/#3}{%
    code={%
      \coordinate (top) at (0,#1);
      \foreach \i in{0,...,#1}
      \foreach \j in{0,...,#2}
      {
        \tikzmath{
          \y = 2*(\i+1)/(#1+2);
          \x = 2*(\j+1)/(#2+2);
        }
        \node[#3] at (\x,\y) {};
      }
    }
}}

\tikzset{pics/pfem/.style args={#1/#2}{%
    code={%
      \tikzmath{ \ytop=2*cos(30); }
      \coordinate (top) at (0,\ytop);

      \foreach \i in{0,...,#1}
      \foreach \j in{0,...,\i}
      {
        \tikzmath{
          \y = \ytop-\ytop*\i/#1;
          \x = 2*(\i/2-\j)/#1+1;
        }
        \node[#2] at (\x,\y) {};
      }
    }
}}

\tikzset{pics/qfem/.style args={#1/#2}{%
    code={%
      \foreach \i in{0,...,#1}
      \foreach \j in{0,...,#1}
      {
        \tikzmath{
          \y = 2-2*\i/#1;
          \x = 2-2*\j/#1;
        }
        \node[#2] at (\x,\y) {};
      }
    }
}}

%%% Local Variables:
%%% mode: latex
%%% TeX-master: "all"
%%% End:

\def\esp#1{V_{#1}}

\usepackage{times}
\usepackage{xr}
\externaldocument{main}
\usepackage{mfirstuc}
\usepackage{mathtools}  
\mathtoolsset{showonlyrefs}

\newcommand{\rd}{\operatorname{rd}}

\def\footnote#1{}
\def\putindex#1{#1}
\title{Numerical Linear Algebra}
\author{Guido Kanschat}
\date{\today}
\begin{document}
\frame{\maketitle}
\frame{\frametitle{Overview}\tableofcontents[hideallsubsections]}
\section{Dense Algebraic Eigenvalue Problems}
\frame{\sectoc}

\subsection{Mathematical background}
\subsubsection{Definition of EVP}
\frame {\input {blocks/Definition-eigenvalue.tex}}
\frame {\input {blocks/Definition-eigenvalue-algebraic.tex}
  \pause
  \input {blocks/Lemma-eigenvalue-equivalent.tex}}
\frame {\input {blocks/Theorem-eigenvalue-count.tex}
  \pause
  \input {blocks/Definition-eigenvalue-simple.tex}}
\frame {\input {blocks/Lemma-eigenvalues-conjugate.tex}}

\subsubsection{Bases and matrices}
\frame{\subtoc}
\frame {\input {blocks/Notation-column-vectors.tex}}
\frame {\input {blocks/Notation-matrix-linear-combination.tex}}
\frame {\input {blocks/Lemma-change-of-basis.tex}}
\frame {\input {blocks/Corollary-change-of-basis.tex}}
\frame {\input {blocks/Definition-similar-matrix.tex}
  \input {blocks/Lemma-similarity-equivalence.tex}}
\frame {\input {blocks/Lemma-matrix-basis-change.tex}}
\frame {\input {blocks/Theorem-Jordan-canonical-form.tex}}
\frame {\input {blocks/Definition-diagonalizable.tex}
  \input {blocks/Theorem-matrix-functions.tex}}
\frame {\input {blocks/Theorem-simultaneous-diagonalization.tex}}

\subsubsection{Normal and Hermitian matrices}
\frame{\subtoc}

\frame {\input {blocks/Definition-sesqui.tex}}
\frame {\input {blocks/Definition-inner-product.tex}
  \input {blocks/Example-Euclidean-ip.tex}}
\frame {\input {blocks/Definition-conjugate-matrix.tex}}
\frame {\input {blocks/Definition-orthonormal-unitary.tex}}

\frame {\input {blocks/Definition-normal-Hermitian.tex}}
\frame {\input {blocks/Lemma-Hermitian-eigenvalues-real.tex}}
\frame {\input {blocks/Theorem-Hermitian-diagonalizable.tex}}
\frame {\input {blocks/Corollary-symmetric-diagonalizable.tex}}
\frame {\input {blocks/Theorem-normal-diagonalizable.tex}}

\subsection{Well-posedness and bounds}
\frame{\subtoc}
\subsubsection{Bounds on eigenvalues}
\frame {\input {blocks/Lemma-bound-by-norm.tex}}
\frame {\input {blocks/Lemma-pre-gershgorin.tex}}
\frame {\input {blocks/Theorem-gershgorin.tex}}
\subsubsection{The Rayleigh quotient}
\frame {\input {blocks/Definition-rayleigh-quotient.tex}}
\frame {\input {blocks/Theorem-minmax.tex}}

\subsubsection{Conditioning of the EVP}

\frame {\input {blocks/Example-characteristic-polynomial.tex}}
\frame {\input {blocks/Example-conditioning-Jordan-block.tex}}
\frame {\input {blocks/Theorem-Jordan-block-ill-conditioned.tex}}
\frame {\input {blocks/Theorem-bauer-fike.tex}}
\frame {\input {blocks/Corollary-conditioning-eigenvalues-normal.tex}}
\frame {\input {blocks/Theorem-conditioning-eigenvalue-single.tex}}

\frame {\input {blocks/Example-conditioning-eigenvectors.tex}}
\frame {\input {blocks/Remark-conditioning-eigenvectors.tex}}

\subsection{Vector iterations}
\frame{\subtoc}
\frame {\input {blocks/Algorithm-vector-iteration.tex}}
\frame {\input {blocks/Theorem-vector-iteration.tex}}
\frame {\input {blocks/Remark-vector-iteration.tex}}
\frame {\input {blocks/Algorithm-shifted-vector-iteration.tex}}
\frame {\input {blocks/Algorithm-inverse-iteration.tex}}
\frame {\input {blocks/Algorithm-Rayleigh-iteration.tex}}

\subsection{Subspace iterations, the QR-iteration}
\frame{\subtoc}

\subsubsection{Definition of the methods}
\frame {\input {blocks/Algorithm-subspace-iteration.tex}}

\frame {\input {blocks/Definition-qr-decomposition.tex}}
\frame {\input {blocks/Lemma-qr-columns.tex}}
\frame {\input {blocks/Theorem-qr-existence.tex}}
\frame {\input {blocks/Algorithm-qr-iteration.tex}}

\subsubsection{Analysis}
\frame {\input {blocks/Theorem-schur-canonical.tex}}
\frame {\input {blocks/Lemma-schur-canonical-1.tex}}
\frame {\input {blocks/Lemma-schur-canonical-2.tex}}

\frame {\input {blocks/Definition-spectral-radius.tex}}
\frame {\input {blocks/Lemma-spectral-radius.tex}}

\frame {\input {blocks/Algorithm-subspace-iteration.tex}}
\frame {\input {blocks/Theorem-convergence-subspace-iteration.tex}}

\frame {\input {blocks/Definition-householder-transformation.tex}}
\frame {\input {blocks/Lemma-householder-symmetry.tex}}
\frame {\input {blocks/Lemma-householder-qr.tex}}

\frame {\input {blocks/Algorithm-qr-iteration.tex}}
\frame {\input {blocks/Lemma-qr-1.tex}}
\frame {\input {blocks/Theorem-convergence-qr-iteration.tex}}

\subsubsection{Implementation issues}
\frame {\input {blocks/Definition-hessenberg.tex}}
\frame {\input {blocks/Theorem-Hessenberg-qr.tex}}
\frame {\input {blocks/Corollary-Hessenberg-qr.tex}}
\frame {\input {blocks/Theorem-Hessenberg-householder.tex}}
\frame {\input {blocks/Definition-givens.tex}}
\frame {\input {blocks/Lemma-givens-computation.tex}}
\frame {\input {blocks/Algorithm-qr-method.tex}}

\frame {\input {blocks/Theorem-hessenberg-qr-convergence.tex}}
\frame {\input {blocks/Definition-projection-distance.tex}}
\frame {\input {blocks/Theorem-qr-reduction.tex}
  \input {blocks/Definition-hessenberg-unreduced.tex}}

\subsubsection{Shitfs and deflation}

\frame {\input {blocks/Algorithm-shifted-qr-iteration.tex}
  \input {blocks/Lemma-shifted-qr-convergence.tex}}
\frame {\input {blocks/Example-rayleigh-shift.tex}}
\frame {\input {blocks/Definition-wilkinson-shift.tex}}
\frame {\input {blocks/Remark-wilkinson-shift.tex}}
\frame {\input {blocks/Algorithm-qr-deflation.tex}}
\frame {\input {blocks/Theorem-real-schur-form.tex}}
\frame {\input {blocks/Remark-francis-qr.tex}}
\frame {\input {blocks/Remark-real-symmetric-qr.tex}}

\subsubsection{Singular Value Decomposition}

\frame {\input {blocks/Definition-svd.tex}
  \input {blocks/Theorem-svd.tex}}
\frame {\input {blocks/Corollary-svd-rank.tex}
  \input {blocks/Corollary-svd-inverse.tex}}
\frame {\input {blocks/Remark-svd-geometry.tex}}
\frame {\input {blocks/Lemma-svd-ata.tex}}
\frame {\input {blocks/Theorem-implicit-Q.tex}}

\section{Solving Large Sparse Linear Systems}
\frame{\sectoc}
\subsection{Motivation and sparse matrices}
\frame {\input {blocks/Definition-sparse-matrix.tex}}
\frame {\input {blocks/Example-csr.tex}}
\frame {\input {blocks/Remark-algorithmic-matrix.tex}}

\subsection{Basic iterations}
\frame{\subtoc}
\frame {\input {blocks/Definition-jacobi.tex}
  \input {blocks/Definition-gauss-seidel.tex}}
\frame {\input {blocks/Definition-richardson-iteration.tex}}
\frame {\input {blocks/Definition-matrix-iteration.tex}}
\frame {\input {blocks/Theorem-bfpt.tex}}
\frame {\input {blocks/Corollary-matrix-norm-convergence.tex}}
\frame {\input {blocks/Example-matrix-norm-convergence.tex}}
\frame {\input {blocks/Theorem-matrix-radius-convergence.tex}}

\subsection{Krylov-space methods}
\frame{\subtoc}

\section{Large Sparse Eigenvalue Problems}
\frame{\sectoc}


\section*{Bibliography}
\frame{\bibliographystyle{alpha}
\bibliography{all}}

%\frame {\input {blocks/Satz-trigonometrische-interpolation.tex}}

\end{document}

%%% Local Variables:
%%% mode: latex
%%% TeX-master: t
%%% End:
