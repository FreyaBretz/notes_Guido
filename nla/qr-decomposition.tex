\begin{intro}
  The facts in this section can be found in~\cite[Chapter
  5]{GolubVanLoan83}. In particular, the QR factorization in section
  5.2, Housholder transformations and Givens rotations in 5.1 there.
\end{intro}

\subsection{Definition and existence}
\begin{Definition}{qr-decomposition}
  The \define{QR factorization} of a matrix $\mata\in\C^{m\times n}$
  with $m\ge n$ is the product
  \begin{gather}
    \mata = \matq\matr,
  \end{gather}
  such that $\matq \in\C^{m\times n}$ is unitary and
  $\matr\in \C^{n\times n}$ is upper triangular.
\end{Definition}

\begin{Lemma}{qr-columns}
  Let $\mata = \matq\matr$. Then, the column vectors of $\mata$ and
  $\matq$ admit the relation
  \begin{gather}
    \va_k = \sum_{i=1}^k r_{ik} \vq_i.
  \end{gather}
  If $r_{ii}\neq 0$ for $i=1,\dots,k$, this relation is uniquely
  invertible. In particular,
  \begin{gather}
    \spann{\vq_1,\dots,\vq_k}
    = 
    \spann{\va_1,\dots,\va_k}.
  \end{gather}
\end{Lemma}


\begin{Theorem}{qr-existence}
  Every matrix $\mata\in\C^{m\times n}$ with $m\ge n$ of full rank
  admits a QR factorization. It is unique under the condition that for
  all $i$ there holds $r_{ii} > 0$.
\end{Theorem}

\subsection{Householder transformations}

\subsection{Givens rotation}

%%% Local Variables:
%%% mode: latex
%%% TeX-master: "main"
%%% End:
