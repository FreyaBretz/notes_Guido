
\subsection{Definition and existence}
\begin{Definition}{qr-decomposition}
  The \define{QR decomposition} of a matrix $\mata\in\C^{m\times n}$
  with $m\ge n$ is the product
  \begin{gather}
    \mata = \matq\matr,
  \end{gather}
  such that $\matq \in\C^{m\times n}$ is unitary and
  $\matr\in \C^{n\times n}$ is upper triangular.
\end{Definition}

\begin{Lemma}{qr-columns}
  Let $\mata = \matq\matr$. Then, the column vectors of $\mata$ and
  $\matq$ admit the relation
  \begin{gather}
    \va_k = \sum_{i=1}^k r_{ik} \vq_i.
  \end{gather}
  If $r_{ii}\neq 0$ for $i=1,\dots,k$, this relation is uniquely
  invertible. In particular,
  \begin{gather}
    \operatorname{span}\{\vq_1,\dots,\vq_k\}
    = 
    \operatorname{span}\{\va_1,\dots,\va_k\}.
  \end{gather}
\end{Lemma}

\begin{Theorem}{qr-existence}
  Every matrix $\mata\in\C^{m\times n}$ with $m\ge n$ of full rank
  admits a QR decomposition. It is unique under the condition that for
  all $i$ there holds $r_{ii} > 0$.
\end{Theorem}

\subsection{Householder transformations}

\begin{Definition}{householder-transformation}
  The \define{Householder transformation}
  associated with a vector $\vw\in\C^n$ is
  \begin{gather}
    \matq_w = \id - 2\frac{\vw\vw^*}{\vw^*\vw}
  \end{gather}
  It is also called \define{Householder matrix} or, particularly in
  the real case, \define{Householder reflection}.
\end{Definition}

\begin{Lemma}{householder-symmetry}
  For any vector $\vw\in\C^n$ the Householder transformation
  $\matq_{\vw}$ is Hermitian and orthogonal, that is,
  \begin{gather}
    \matq^{-1} = \matq^* = \matq.
  \end{gather}
\end{Lemma}

\begin{Lemma}{householder-qr}
  For any vector $\vy\in\C^n$ there are vectors $\vw_\phi\in\C^n$ such
  that $\matq_{\vw_\phi} \vy$ is a multiple of $\ve_1$.

  The vector of choice for numerical stability is
  \begin{gather}
    \vw = \vy + e^{i\phi} \norm{\vy}_2\ve_1,
  \end{gather}
  where $\phi$ is the phase of $y_1$.
\end{Lemma}

\begin{proof}
  The statement of the lemma says that for a suitable vector $\vw$ there is a complex number
  $\alpha$ such that $\matq_{\vw} \vy = \alpha \ve_1$. Since
  $\matq_{\vw}$ preserves the Euclidean norm, we already know
  \begin{gather}
    \abs{\alpha} = \norm{\vy}_2.
  \end{gather}
  There holds
  \begin{gather}
    \alpha\ve_1 = \matq_{\vw} \vy
    = \vy - 2 \frac{\vw\vw^*\vy}{\vw^*\vw}
    = \vy - w \frac{\vw^*\vy}{\vw^*\vw}\vw.
  \end{gather}
  Thus, $\vw$ is in the span of $\vy-\alpha \ve_1$. Since we divide by
  its norm, its length does not matter and we let
  \begin{gather}
    \vw_\phi =
    \begin{pmatrix}
      y_1 - e^{i\phi} \norm{\vy}_2\\y_2\\\vdots\\y_n
    \end{pmatrix},
    \qquad
    \matq_{\vw_\phi}\vy =
    \begin{pmatrix}
      e^{i\phi} \norm{\vy}_2\\0\\\vdots\\0
    \end{pmatrix}.
  \end{gather}
  Since the computation of the first component of $\vw_\phi$ is prone to loss of significance, we choose $\phi$ as the phase of $-y_1$.
\end{proof}

%%% Local Variables:
%%% mode: latex
%%% TeX-master: "main"
%%% End:
