\subsection{Derivation from orthogonal subspace iteration}
\begin{remark}
  \label{par:qr:intro}
  If the \putindex{orthogonal subspace iteration} converges, then
  there is an orthogonal matrix $\matq$ such that
  $\lim_{k \to \infty} \matq_k = \matq$.  From the two assignments of
  the algorithm, we get
  \begin{gather}
    \lim_{k \to \infty} \matq_k^* \mata \matq_k = \lim_{k \to \infty} \matq_k^* \maty_k = \lim_{k \to \infty}
    \matq_k^*\matq_{k+1}\matr_{k+1} = \matr,
  \end{gather}
  where $\matr\leftarrow\matr_k$, that is, the matrices
  $\mata_k = \matq_k^*\mata\matq_k$ converge to an upper triangular
  matrix with the converged eigenvalues on the diagonal. Our next goal
  ist the modification of the orthogonal subspace iteration, such that
  we compute the sequence $\mata_k$ directly, without the intermediate
  $\maty_k$. To this end, we assume $m=n$, that is, we compute the
  whole spectrum of $\mata$. By lines 2 and 3 of the algorithm, we have
  \begin{gather*}
    \mata_{k-1} = \matq_{k-1}^*\mata\matq_{k-1} = \matq_{k-1}^* \maty_{k-1} = \left(\matq_{k-1}^*\matq_{k}\right)\matr_k,
  \end{gather*}
  where we see that on the right we have obtained the QR factorization of $\mata_{k-1}$.
  On the other hand,
  \begin{align*}
    \mata_k
    &= \matq_k^*\mata\matq_k\\
    &= \matq_k^*\mata\matq_{k-1}\matq_{k-1}^*\matq_k\\
    &= \matq_k^*\maty_{k-1} \matq_{k-1}^*\matq_k\\
    &= \matr_k \left(\matq_{k-1}^*\matq_k\right).
  \end{align*}
  Thus, we see that $\mata_k$ is obtained by multiplying the QR
  factorization of $\mata_{k-1}$ in reverse order. Note that $\matq_k$
  in the following algorithm differs from $\matq_k$ in the previous!
\end{remark}

\begin{Algorithm*}{qr-iteration}{QR iteration}
  \begin{algorithmic}[1]
    \Require $\mata_0\in\Cnn$.
    \For{$k=1,\dots$ until convergence}
    \State $\matq_k\matr_k \gets \mata_{k-1}$ \Comment{QR factorization}
    \State $\mata_{k} = \matr_k\matq_k$
    \EndFor
  \end{algorithmic}
\end{Algorithm*}

\begin{Lemma}{qr-Schur}
  If convergent, the QR iteration converges to the Schur canonical
  form of the matrix $\mata$ with eigenvalues sorted according to
  their modulus.
\end{Lemma}

\begin{proof}
  See~\ref{par:qr:intro}.
\end{proof}

\begin{Lemma}{qr-1}
  The matrices $\mata_k$ of the QR-iteration with $\mata_0 = \mata$
  have the following properties:
  \begin{enumerate}
  \item If $\matq_0=\id$, $\mata_{k} = \matq_k^*\mata_{k-1}\matq_k = \matq_k^*\dots\matq_0^*\mata\matq_0\dots\matq_k$.
  \item $\mata^k=\matq_1\dots\matq_k\matr_k\dots\matr_1$.
  \item If $\mata$ is normal, so is $\mata_k$ for any $k$.
  \item If $\mata$ is complex symmetric, so is $\mata_k$ for any $k$.
  \end{enumerate}
\end{Lemma}

\begin{proof}
  For the first relation, we observe
  $\mata_0 = \mata = \matq_0^*\mata\matq_0$ and compute
  \begin{gather}
    \mata_{k} = \matr_k\matq_k = \matq_k^* \matq_k \matr_k \matq_k = \matq_k^* \mata_{k-1} \matq_k.
  \end{gather}
  The second, we prove by induction, where
  $\mata = \mata_0 = \matq_1\matr_1$ follows directly from the first
  step of the algorithm. For the induction step, we abbreviate
  \begin{gather}
    \matp_k = \matq_1\dots\matq_k,\qquad \matu_k = \matr_k\dots\matr_1.
  \end{gather}
  Assuming $\mata^k = \matp_k\matu_k$, we obtain
  \begin{gather}
    \mata^{k+1} = \mata\matp_k\matu_k = \matp_k\mata_{k+1}\matu_k
    = \matp_k\matq_{k+1}\matr_{k+1}\matu_k = \matp_{k+1}\matu_{k+1}.
  \end{gather}
\end{proof}

%%%%%%%%%%%%%%%%%%%%%%%%%%%%%%%%%%%%%%%%%%%%%%%%%%%%%%%%%%%%%%%%%%%%%%
\subsection{Hessenberg matrices}
%%%%%%%%%%%%%%%%%%%%%%%%%%%%%%%%%%%%%%%%%%%%%%%%%%%%%%%%%%%%%%%%%%%%%%
\begin{intro}
  In each step of the QR-iteration, a QR factorization of the matrix
  is needed, which requires $\bigo(n^3)$ operations. Thus, the
  complexity of the iteration is highly unfavorable. The following
  discussion will provide us with means to reduce the complexity of
  the QR factorization to $\bigo(n^2)$, in the symmetric case even to
  $\bigo(n)$.
\end{intro}

\begin{Definition}{hessenberg}
  A matrix is in \define{Hessenberg form} or is a \define{Hessenberg
    matrix}, if all its entries below the first subdiagonal are zero. Visually,
  \begin{gather}
    H =
    \begin{pmatrix}
      *&*&*&*&*&*\\
      *&*&*&*&*&*\\
      0&*&*&*&*&*\\
      0&0&*&*&*&*\\
      0&0&0&*&*&*\\
      0&0&0&0&*&*
    \end{pmatrix}
  \end{gather}
  A symmetric or Hermitian Hessenberg matrix is \define{tridiagonal}.
\end{Definition}

\begin{Algorithm*}{Hessenberg-qr-1}{Explicit Hessenberg QR step}
  \index{Hessenberg QR!explicit}
  \begin{algorithmic}[1]
    \Require $\matH\in\Cnn$ in Hessenberg form
    \For{$k=1,\dots,n-1$}
    \Comment{factorization $\matq\matr = \matH$, $\matr$ stored in $\matH$}
    \State $\givens_{k,k+1} \gets$ Givens rotation for $h_{kk},h_{k+1,k}$
    \State $\matH\gets \givens^*_{k,k+1}\matH$
    \EndFor
    \For{$k=1,\dots,n-1$}
    \Comment{$\matH = \matr\matq$}
    \State $\matH\gets \matH\givens_{k,k+1}$
    \EndFor
  \end{algorithmic}
\end{Algorithm*}

\begin{example}
  At the example of a 4-by-4-matrix, we show how this algorithm works.
  \begin{enumerate}
  \item Apply a Givens rotation from the left which eliminates the value $Y$ from the matrix. It affects the two top rows.

    {\small\begin{tikzpicture}
      \node
[matrix of math nodes,left delimiter=(,right delimiter=)] (m)  at (0,0)
{
  X&X&X&X\\
  Y&X&X&X\\
  0&X&X&X\\
  0&0&X&X\\
};

\node
[matrix of math nodes,left delimiter=(,right delimiter=)] (r)  at (5,0)
{
  X&X&X&X\\
  0&X&X&X\\
  0&X&X&X\\
  0&0&X&X\\
};

\draw[->,thick] (2,0) -- node[above] {$\matg_{12}^*\matH$} (3,0);

\begin{pgfonlayer}{bg}
  \node[fit={(m-1-1.north west) (m-2-4.south east)},fill=yellow,inner sep=0,rounded corners=1mm]{};
  \node[fit={(m-1-1.north west) (m-2-1.south east)},fill=green,inner sep=0,rounded corners=1mm]{};
\end{pgfonlayer}

;
    \end{tikzpicture}}

  \item Do the same with the second and third rows.

    {\small\begin{tikzpicture}
        \matrix
[matrix of math nodes,left delimiter=(,right delimiter=)] (m)
{
  X&X&X&X\\
  0&X&X&X\\
  0&Y&X&X\\
  0&0&X&X\\
};

\node
[matrix of math nodes,left delimiter=(,right delimiter=)] (r)  at (5,0)
{
  X&X&X&X\\
  0&X&X&X\\
  0&0&X&X\\
  0&0&X&X\\
};

\draw[->,thick] (2,0) -- node[above] {$\matg_{23}^*\matH$} (3,0);

\begin{pgfonlayer}{bg}
  \node[fit={(m-2-2.north west) (m-3-4.south east)},fill=yellow,inner sep=0,rounded corners=1mm]{};
  \node[fit={(m-2-2.north west) (m-3-2.south east)},fill=green,inner sep=0,rounded corners=1mm]{};
\end{pgfonlayer}
;
      \end{tikzpicture}}

    {\small\begin{tikzpicture}
        \matrix
[matrix of math nodes,left delimiter=(,right delimiter=)] (m)
{
  X&X&X&X\\
  0&X&X&X\\
  0&0&X&X\\
  0&0&Y&X\\
};

\node
[matrix of math nodes,left delimiter=(,right delimiter=)] (r)  at (5,0)
{
  X&X&X&X\\
  0&X&X&X\\
  0&0&X&X\\
  0&0&0&X\\
};

\draw[->,thick] (2,0) -- node[above] {$\matg_{34}^*\matH$} (3,0);

\begin{pgfonlayer}{bg}
  \node[fit={(m-3-3.north west) (m-4-4.south east)},fill=yellow,inner sep=0,rounded corners=1mm]{};
  \node[fit={(m-3-3.north west) (m-4-3.south east)},fill=green,inner sep=0,rounded corners=1mm]{};
\end{pgfonlayer}
;
      \end{tikzpicture}}

    Note that columns with two zero entries remain unchanged and will not have to be processed.
  \item Now the matrix is upper triangular and the transformation was
    \begin{gather*}
      \matq^* = \givens_{34}^*\givens_{23}^*\givens_{12}^*.
    \end{gather*}

  \item When we go back, we apply Givens rotations from the right, thus affecting columns of the matrices.

    {\footnotesize\begin{tikzpicture}\matrix
[matrix of math nodes,left delimiter=(,right delimiter=)] (m)
{
  X&X&X&X\\
  0&X&X&X\\
  0&0&X&X\\
  0&0&0&X\\
};
\begin{pgfonlayer}{bg}
  \node[fit={(m-1-1.north west) (m-2-2.south east)},fill=yellow,inner sep=0,rounded corners=1mm]{};
\end{pgfonlayer}
;\end{tikzpicture}
        \begin{tikzpicture}\matrix
[matrix of math nodes,left delimiter=(,right delimiter=)] (m)
{
  X&X&X&X\\
  X&X&X&X\\
  0&0&X&X\\
  0&0&0&X\\
};
\begin{pgfonlayer}{bg}
  \node[fit={(m-1-2.north west) (m-3-3.south east)},fill=yellow,inner sep=0,rounded corners=1mm]{};
\end{pgfonlayer}
;\end{tikzpicture}
        \begin{tikzpicture}\matrix
[matrix of math nodes,left delimiter=(,right delimiter=)] (m)
{
  X&X&X&X\\
  X&X&X&X\\
  0&X&X&X\\
  0&0&0&X\\
};
\begin{pgfonlayer}{bg}
  \node[fit={(m-1-3.north west) (m-4-4.south east)},fill=yellow,inner sep=0,rounded corners=1mm]{};
\end{pgfonlayer}
;\end{tikzpicture}
        \begin{tikzpicture}\matrix
[matrix of math nodes,left delimiter=(,right delimiter=)] (m)
{
  X&X&X&X\\
  X&X&X&X\\
  0&X&X&X\\
  0&0&X&X\\
};
;\end{tikzpicture}}
  \end{enumerate}
\end{example}

\begin{Algorithm*}{Hessenberg-qr-2}{Implicit Hessenberg QR step}
  \begin{algorithmic}[1]
    \Require $\matH\in\Cnn$ in Hessenberg form
    \State $\givens_{1,2} \gets$ Givens rotation for $h_{11},h_{21}$
    \State $\matH \gets \givens^*_{1,2}\matH \givens_{1,2}$
    \For{$k=2,\dots,n-1$}
    \State $\givens_{k,k+1} \gets$ Givens rotation for $h_{k,k-1},h_{k+1,k-1}$
    \State $\matH \gets \givens^*_{k,k+1}\matH \givens_{k,k+1}$
    \EndFor
  \end{algorithmic}
\end{Algorithm*}

\begin{example}
  At the example of a 4-by-4-matrix, we show how this algorithm works.
  \begin{enumerate}
  \item Apply a Givens rotation from the left which eliminates the value $Y$ from the matrix. It affects the two top rows. By
    applying the Givens rotation from the left an additional non zero entry below the subdiagonal is created.

    {\small\begin{tikzpicture}
      \node
[matrix of math nodes,left delimiter=(,right delimiter=)] (m)  at (0,0)
{
  X&X&X&X\\
  Y&X&X&X\\
  0&X&X&X\\
  0&0&X&X\\
};

\node
[matrix of math nodes,left delimiter=(,right delimiter=)] (r)  at (4.5,0)
{
  X&X&X&X\\
  0&X&X&X\\
  0&X&X&X\\
  0&0&X&X\\
};
\node
[matrix of math nodes,left delimiter=(,right delimiter=)] (e)  at (9,0)
{
  X&X&X&X\\
  X&X&X&X\\
  X&X&X&X\\
  0&0&X&X\\
};

\draw[->,thick] (1.75,0) -- node[above] {$\matg_{12}^*\matH$} (2.75,0);
\draw[->,thick] (6.25,0) -- node[above] {$\matH \matg_{12}$} (7.25,0);

\begin{pgfonlayer}{bg}
  \node[fit={(m-1-1.north west) (m-2-4.south east)},fill=yellow,inner sep=0,rounded corners=1mm]{};
  \node[fit={(m-1-1.north west) (m-2-1.south east)},fill=green,inner sep=0,rounded corners=1mm]{};
  \node[fit={(r-1-1.north west) (r-4-2.south east)},fill=yellow,inner sep=0,rounded corners=1mm]{};
\end{pgfonlayer}

;
    \end{tikzpicture}}
\item Apply a Givens rotation from the left to restore Hessenberg form of the first column.
  
    {\small\begin{tikzpicture}
        
\node
[matrix of math nodes,left delimiter=(,right delimiter=)] (m)  at (0,0)
{
  X&X&X&X\\
  X&X&X&X\\
  Y&X&X&X\\
  0&0&X&X\\
};

\node
[matrix of math nodes,left delimiter=(,right delimiter=)] (r)  at (4.5,0)
{
  X&X&X&X\\
  X&X&X&X\\
  0&X&X&X\\
  0&0&X&X\\
};
\node
[matrix of math nodes,left delimiter=(,right delimiter=)] (e)  at (9,0)
{
  X&X&X&X\\
  X&X&X&X\\
  0&X&X&X\\
  0&X&X&X\\
};

\draw[->,thick] (1.75,0) -- node[above] {$\matg_{23}^*\matH$} (2.75,0);
\draw[->,thick] (6.25,0) -- node[above] {$\matH \matg_{23}$} (7.25,0);

\begin{pgfonlayer}{bg}
  \node[fit={(m-2-1.north west) (m-3-4.south east)},fill=yellow,inner sep=0,rounded corners=1mm]{};
  \node[fit={(m-2-1.north west) (m-3-1.south east)},fill=green,inner sep=0,rounded corners=1mm]{};
  \node[fit={(r-1-2.north west) (r-4-3.south east)},fill=yellow,inner sep=0,rounded corners=1mm]{};
\end{pgfonlayer}

;
      \end{tikzpicture}}
    
  \item Do the same with the second column.
    
    {\small\begin{tikzpicture}
        \node
[matrix of math nodes,left delimiter=(,right delimiter=)] (m)  at (0,0)
{
  X&X&X&X\\
  X&X&X&X\\
  0&X&X&X\\
  0&Y&X&X\\
};

\node
[matrix of math nodes,left delimiter=(,right delimiter=)] (r)  at (4.5,0)
{
  X&X&X&X\\
  X&X&X&X\\
  0&X&X&X\\
  0&0&X&X\\
};
\node
[matrix of math nodes,left delimiter=(,right delimiter=)] (e)  at (9,0)
{
  X&X&X&X\\
  X&X&X&X\\
  0&X&X&X\\
  0&0&X&X\\
};

\draw[->,thick] (1.75,0) -- node[above] {$\matg_{23}^*\matH$} (2.75,0);
\draw[->,thick] (6.25,0) -- node[above] {$\matH \matg_{23}$} (7.25,0);

\begin{pgfonlayer}{bg}
  \node[fit={(m-3-1.north west) (m-4-4.south east)},fill=yellow,inner sep=0,rounded corners=1mm]{};
  \node[fit={(m-3-2.north west) (m-4-2.south east)},fill=green,inner sep=0,rounded corners=1mm]{};
  \node[fit={(r-1-3.north west) (r-4-4.south east)},fill=yellow,inner sep=0,rounded corners=1mm]{};
\end{pgfonlayer}

;
      \end{tikzpicture}}
    Note that columns with two zero entries remain unchanged and will not have to be processed.
  \end{enumerate}
\end{example}

\begin{remark}
  This algorithm is called \define{bulge chasing} with the following
  image in mind. After the application of the first rotation, there is
  a bulge protruding down from the Hessenberg form in the furst
  column. This bulge is then ``chased'' down row by row until it
  leaves the matrix at the bottom.
\end{remark}

\begin{Definition}{hessenberg-unreduced}
  A Hessenberg matrix is called \define{unreduced} if all entries on
  the first subdiagonal are nonzero. It is called \define{reduced}
  otherwise.
\end{Definition}

\begin{todo}
  In the next semester, use $\parallel$!
\end{todo}

\begin{Theorem*}{implicit-Q}{Implicit Q Theorem}
  Let $\mata\in\Cnn$ arbitrary, let $\matq,\matv\in\Cnn$ unitary such that
  \begin{gather}
    \matq^*\mata\matq = \matH,\qquad \matv^*\mata\matv = \matg,
  \end{gather}
  where $\matH$ and $\matg$ are Hessenberg matrices. Let $k$ denote
  the smallest integer such that $h_{k+1,k} = 0$, or $k=n$ if $\matH$
  is unreduced. Assume $\vv_1 = e^{i\phi_0}\vq_1$. Then,
  $\vv_j = e^{i\phi_j}\vq_j$ and $\abs{h_{j+1,j}} = \abs{g_{j+1,j}}$
  for $j=1,\dots,k-1$. if $k<n$, then $g_{k+1,k} = 0$.
\end{Theorem*}

\begin{proof}
  We define the matrix $\matw = \matv^*\matq$, which is orthogonal as the product of orthogonal matrices. There holds
  \begin{gather}
    \matg\matw = \matv^*\mata\matv\matv^*\matq = \matv^*\mata\matq
    = \matv^*\matq\matq^*\mata\matq = \matw\matH.
  \end{gather}
  Spelling out column $j$ of this product, we obtain
  \begin{gather}
    \label{eq:qr:implicitq-1}
    \matg\vw_j = \sum_{k=1}^{j+1} h_{kj} \vw_{k}.
  \end{gather}
  We use this equality to show by induction over the columns that the
  entries $w_{ij}$ of $\matw$ are zero for $i>j$. In other words,
  $\matw$ is upper triangular.  For $j=1$, we obtain from the
  orthogonality of $\matq$ and $\matv$ and the parallelity of $\vq_1$
  and $\vv_1$ that $\vw_1 = e^{i\phi} \ve_1$ for some argument $\phi$.

  Now let the statement be proven for all columns of $\matw$ up to
  column $j$. Then, from~\eqref{eq:qr:implicitq-1} we obtain
  \begin{gather}
    h_{j+1,j}\vw_{j+1} = \matg\vw_j - \sum_{k=1}^j h_{kj} \vw_{k}.
  \end{gather}
  For each vector in the sum, there holds $(\vw_{k})_i = 0$ for
  $i>j$. Since $\matg$ is Hessenberg and applied to $\matw_j$, the
  last possibly nonzero entry of the product is in position $j+1$,
  what we wanted to show.

  Since every orthogonal matrix is normal and due to
  \slideref{Problem}{normal-triangular-diagonal} every triangular
  normal matrix is diagonal, the matrix $(\vw_1,\dots,\vw_k)$ is diagonal with
  diaognal entries of the form $w_{jj} = e^{i\phi_j}$ with some
  arguments $\phi_j$. Hence, $\vv_j = e^{-i\phi_j} \vq_j$ for $j=1,\dots,k$.

  There holds for $j<k$
  \begin{gather}
    \abs{h_{j+1,j}} = \abs{\ve_{j+1}^*\matH\ve_j} = \abs{\vq_{j+1}^*\mata\vq_j}
    = \abs{\vv_{j+1}^*\mata\vv_j} = \abs{g_{j+1,j}}.
  \end{gather}

  If $k<n$, that is, $h_{k+1,k}=0$, it remains to show that
  \begin{multline}
    g_{k+1,k} = \ve_{k+1}^*\matg\ve_k = e^{i\phi_k}\ve_{k+1}^*\matg\matw\ve_k
    =  e^{i\phi_k}\ve_{k+1}^*\matw\matH\ve_k\\
    =  e^{i\phi_k}\ve_{k+1}^* \sum_{j=1}^k h_{jk} \matw\ve_j
    =  \sum_{j=1}^k e^{i\phi_*} h_{jk} \ve_{k+1}^* \ve_j = 0.
  \end{multline}
\end{proof}

\begin{Definition}{essentially-equal}
  The \putindex{Implicit Q Theorem} says that two Hessenberg forms of $\mata$ with the same initial reduction vector are \define{essentially equal} in the sense that they only differ by the diagonal scaling $\matg = \matd^{-1}\matH\matd$ where $\matd=\diag(d_1,\dots,d_n)$ matrix with $\abs{d_{i}} = 1$.
\end{Definition}

\begin{Corollary}{Hessenberg-qr-equivalence}
  The two versions of the Hessenberg QR step are essentially equal.
\end{Corollary}

\begin{Problem}{Hessenberg-qr-effort}
  \begin{enumerate}
  \item How many operations do the two versions of the Hessenberg QR step require?
  \item Show that if $\matH$ is Hermitian, the result of the
    Hessenberg QR step is Hermitian as well.
  \end{enumerate}
\end{Problem}

\begin{Corollary}{Hessenberg-qr}
  The complexity of each step of a QR-iteration for Hessenberg matrices is $\bigo(n^2)$. For tridiagonal (complex) symmetric matrices, it is $\bigo(n)$.
\end{Corollary}

\begin{Theorem}{Hessenberg-householder}
  Every matrix $\mata\in\Cnn$ is unitarily similar to a Hessenberg matrix $\matH$, that is,
  \begin{gather}
    \matH = \matq^* \mata \matq.
  \end{gather}
  The matrix $\matq$ can be obtained by $n-2$ \putindex{Householder
    reflection}s.
\end{Theorem}

\begin{Algorithm*}{qr-method}{The QR-Method}
  Compute the spectrum of a matrix $\mata\in\Cnn$ by
  \begin{enumerate}
  \item Use $n-2$ Householder transformations to transform $\mata$ to
    Hessenberg form
    \begin{gather}
     \matH = \matq^*\mata\matq.
   \end{gather}
 \item QR-iteration: let $\matH_{0}=\matH$ and perform the implicit Hessenberg QR step until convergence
 \item Store Householder vectors as well as $r$ and $c$ for each
   Givens rotation \textbf{only} if the eigenvectors are desired in the end.
  \end{enumerate}
\end{Algorithm*}

\begin{Problem}{Hermitian-tridiagonal}
  Show that every (complex) Hermitian matrix is orthogonally similar
  to a symmetric tridiagonal matrix with real entries.
\end{Problem}

\subsection{Deflation and shifts}

\begin{intro}
  The goal of this section is the development and justification of a
  method which accelerates convergence of the QR-iteration and
  reducing the effort at the same time. It is based on shifts, like
  for the simple or inverse power method. But, shifts are much more
  powerful here, since we compute not only ``converging subspace'',
  but also its complement. The presentation follows
  mostly~\cite{GolubVanLoan83}.
\end{intro}

\begin{Theorem}{qr-reduction}
  Let the matrix $\matH^{(k)}\in\Cnn$ in the QR iteration be of the
  form
  \begin{gather}
    \matH^{(k)} =
    \begin{pmatrix}
      \matH_{11} & \mata_{12}\\0 & \matH_{22}
    \end{pmatrix}
  \end{gather}
  with Hessenberg matrices $\matH_{11}\in\C^{p\times p}$,
  $\matH_{22}\in \C^{n-p\times n-p}$ and an arbitrary matrix
  $\mata_{12}\in \C^{p\times n-p}$. Then, the matrix $\matq^{(k)}$
  decouples into two diagonal blocks and $\matH^{(k+1)}$ has the same
  form. Thus, the iteration decouples into two separate iterations. If
  $p=n-1$, then $h_{nn}$ approximates an eigenvalue.
\end{Theorem}

\begin{Algorithm*}{qr-deflation}{Deflation}
  After each step of the shifted QR-iteration monitor the subdiagonal
  elements of $\matH^{(k)}$. Whenever
  \begin{gather}
    \abs{h_{j,j-1}} \le \eps \bigl(\abs{h_{j-1,j-1}}+\abs{h_{jj}}\bigr)
  \end{gather}
  set $h_{j,j-1}=0$.

  If this happens in the last row, consider $h_{nn}=\lambda_n$
  converged and proceed with a matrix of dimension $n-1\times n-1$.

  If this happens in the center of the matrix, proceed with both
  remaining diagonal blocks separately.

  Note that deflation changes the matrix in a ``nonorthogonal'' way
  and thus changes the eigenvalues. Their accuracy will be determined
  by the parameter $\eps$ in the end.
\end{Algorithm*}

\begin{remark}
  The purpose of deflation is removing Schur vectors from the
  iteration. Thus, if one of the Schur vectors for a multiple
  eigenvalue has converged, the remaining iterations will deal with
  reduced multiplicity. Deflation will also help us to deal with the
  requirement that all eigenvalues must have different modulus, but
  this is solved below in combination with shifts.
\end{remark}

\begin{Algorithm*}{shifted-qr-iteration}{QR iteration with shift}
  \begin{algorithmic}[1]
    \Require $\matH_0 in\Cnn$, Hessenberg, unreduced
    \For {$k=1,\ldots$ until convergence}
    \State $\matq_k\matr_k = \matH_{k-1} - \sigma_k\id$\Comment{QR factorization}
    \State $\matH_{k} = \matr_k\matq_k + \sigma_k\id$
    \EndFor
  \end{algorithmic}
  There is an implicit form of the shifted QR step which follows
  exactly the version outlined for the unshifted case.
\end{Algorithm*}

\begin{Lemma}{shifted-qr-similarity}
  The matrices $\matH_k$ generated by the QR iteration with shifts
  admit the recurrence relation
  \begin{gather}
    \matH_k = \matq_k^*\matH_{k-1}\matq.
  \end{gather}
\end{Lemma}  

\begin{proof}
  The proof is almost identical to \slideref{Lemma}{qr-1}. There holds
  \begin{multline}
    \matH_k = \matr_k\matq_k + \sigma_k \id
    = \matq_k^*\matq_k\matr_k\matq_k + \sigma_k \id\\
    =\matq_k^*\left(\matH_{k-1}-\sigma_k\id\right)\matq_k + \sigma_k \id
    =\matq_k^*\matH_{k-1}\matq_k.
  \end{multline}
\end{proof}

\begin{Lemma*}{perfect-shift}{Perfect shift}
  Let $\matH\in\Cnn$ be an unreduced Hessenberg matrix with eigenvalue
  $\sigma$. Let $\matq\matr = \matH - \sigma\id$ be a QR factorization
  and $\widetilde\matH = \matr\matq+\sigma\id$. Then,
  $\tilde h_{n,n-1}=0$ and $\tilde h_{nn} =\sigma$.
\end{Lemma*}

\begin{proof}
  See also~\cite[Theorem 7.5.1]{GolubVanLoan83}.  Since $\matH$ is
  unreduced, its first $n-1$ columns are linearly independent. Hence,
  if $\matq\matr=\matH-\sigma\id$ is a QR factorization, then
  $r_{ii} \neq 0$ for $i=1,\dots,n-1$.

  Since $\matH-\sigma\id$ is singular, we conclude $r_{nn}=0$. Thus,
  the last row of $\matr\matq$ is zero and the statement holds.
\end{proof}

\begin{intro}
  Obviously, if we knew an eigenvalue, we could deflate right
  away. Thus, the next step in the development of the algorithm is the
  determination of a \define{shift strategy} which drives $h_{n,n-1}$
  to zero by approximating the last eigenvalue.

  Such a shift strategy selects a new shift parameter $\sigma_k$ in
  every step of the algorithm. The shift strategies differ in the
  approximation of the eigenvalue which is closest to $h_{nn}$.
\end{intro}

\begin{Example*}{rayleigh-shift}{Rayleigh shift}
  The Rayleigh quotient for the smallest eigenvalue by magnitude
  converges to $h_{nn}$, as
  \begin{gather}
    \ve_n^* H^{(k)} \ve_n = h_{nn}^{(k)}
  \end{gather}
  and $\vq_n$ is orthogonal to all eigenvectors for eigenvalues of
  greater magnitude. Therefore, using $\sigma_k = h_{nn}^{(k)}$ seems
  a good idea, and often is. But it is not reliable, as in the example
  \begin{gather}
    H =
    \begin{pmatrix}
      0 & 1 \\ 1 & 0
    \end{pmatrix}.
  \end{gather}
\end{Example*}

\begin{Definition*}{wilkinson-shift}{Wilkinson shift}
  Let
  \begin{gather}
    \matm =
    \begin{pmatrix}
      h_{n-1,n-1}^{(k)}&h_{n-1,n}^{(k)}\\h_{n,n-1}^{(k)}&h_{nn}^{(k)}
    \end{pmatrix}.
  \end{gather}
  Then, for $\sigma_k$ use the eigenvalue of $\matm$ which is closer
  to $h_{nn}^{(k)}$.
\end{Definition*}

\begin{Remark}{wilkinson-shift}
  The Wilkinson shift is reliable and the $h_{n,n-1}$ and $h_{nn}$
  converge to zero and the smallest eigenvalue by magnitude,
  respectively. They converge at least quadratically and cubically in
  the symmetric case~\cite[Section 8.2]{GolubVanLoan83}.
\end{Remark}

\begin{Example}{wilkinson-failure}
  Consider the orthogonal matrix
  \begin{gather}
    \mata =
    \begin{pmatrix}
      0&0&1\\1&0&0\\0&1&0
    \end{pmatrix}.
  \end{gather}
  The lower right block has a single eigenvalue zero, such that the
  Wilkinson shift and the Rayleigh shift for this matrix are zero. The eigenvalues of $\mata$ are
  \begin{gather}
    \sigma(\mata) = \left\{1, -\tfrac12 \pm \sqrt{\tfrac34}i\right\},
  \end{gather}
  which all have the same modulus. Thus, the algorithm will not converge with either shift.
\end{Example}

\begin{Algorithm}{exceptional-shift}
  If no deflation has ocurred for a given number of iteration steps of
  the shifted QR iteration, the chosen shift strategy has failed.

  In this case, perform a single step with a random shift parameter, a
  so-called \define{exceptional shift}.
\end{Algorithm}

\begin{remark}
  From the necessity to introduce exceptional shifts, we realize that
  a convergence result for the shifted QR iteration is hard to
  obtain. Nevertheless, from the result for the orthogonal subspace
  iteration, we have
  \begin{gather}
    h_{j+1,j}^{(k)} = \bigo \left(\abs*{\frac{\lambda_{j+1}-\sigma}{\lambda_j-\sigma}}^k\right).
  \end{gather}
  Hence, the situation is not hopeless and typically, the algorithm
  converges again after an exceptional shift.

  We will see soon, that the situation is significantly better in the
  case of Hermitian matrices.
\end{remark}

\begin{Algorithm*}{qr-step-deflation}{QR step with deflation}
  In each step of the QR iteration, first set
  \begin{gather}
    h_{i,i-1} = 0, \qquad \text{where}\quad
    \abs{h_{i,i-1}} \le \eps \left(\abs{h_{i,i}}+\abs{h_{i-1,i-1}}_{\vphantom{g}}\right).
  \end{gather}

  Then, partition the matrix
  $\matH$ as
  \begin{gather}
    \matH =
    \begin{pmatrix}
      \matH_{11} & \matH_{12} & \matH_{13} \\
      & \matH_{22} & \matH_{23}\\
      && \matH_{33}
    \end{pmatrix},
  \end{gather}
  where $\matH_{22}\in\R^{q\times q}$ and
  $\matH_{33}\in\R^{p\times p}$ are chosen maximal such that
  $\matH_{33}$ is upper \putindex{triangular} and $\matH_{22}$
  is unreduced.

  The shifted QR step is then applied to $\matH_{22}$ only. If the
  eigenvectors are not computed, even the transformations of
  $\matH_{12}$ and $\matH_{23}$ can be eliminated.
\end{Algorithm*}

\begin{remark}
  When implementing \slideref{Algorithm}{qr-step-deflation}, we have
  to be able to run a QR step on a submatrix. Allocating new memory
  and copying the submatrix should be avoided since it comes at
  considerable cost.

  This means on the other hand, that the matrix $\matH_{22}$ will not
  be stored as a consecutive array of $q\times q$ numbers. Depending
  on whether the entries are sorted in row-major or column-major
  order, there will either be a gap between each consecutive element
  of a column or between the last element of one column and the first
  element of the next.

  Thus, the QR step operations must allow for a \define{stride}
  between rows or columns. This can be achieved either by storing the
  matrix as a sequence of column vectors, or by using strided versions
  of the algorithms.
\end{remark}

\begin{Example*}{daxpy}{DAXPY}
  The BLAS function daxpy computes
  \begin{gather}
    \vx \gets a \vx + \vy.
  \end{gather}
  It has the signature
  \begin{lstlisting}[language=c]
    int daxpy(int N, double A,
              double* X, int INCX,
              double* Y, int INCY);
  \end{lstlisting}
  The values \lstinline!INCX! and \lstinline!INCY! are used to
  implement stride, possibly with different values in both vectors.
\end{Example*}

%%% Local Variables:
%%% mode: latex
%%% TeX-master: "main"
%%% End:
