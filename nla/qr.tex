\subsection{Definition of the methods}

\begin{Algorithm*}{subspace-iteration}{Orthogonal subspace iteration}

  Let $\mata\in\Cnn$, $\matx_0 \in \C^{n\times m}$.\\
  For $k=0,\ldots$ until convergence repeat
  \begin{itemize}
  \item $\matz_k = \mata \matx_k$.
  \item $\matq_k\matr_k = \matz_k$ (QR factorization)
  \item $\matx_{k+1} = \matq_k$
  \end{itemize}
\end{Algorithm*}

\begin{Algorithm*}{qr-iteration}{QR iteration}
  
  Let $\mata_1 = \mata\in\Cnn$.\\
  For $k=1,\ldots$ until convergence repeat
  \begin{itemize}
  \item $\matq_k\matr_k = \mata_k$ (QR factorization)
  \item $\mata_{k+1} = \matr_k\matq_k$
  \end{itemize}
\end{Algorithm*}

\subsection{Analysis}
\begin{Theorem*}{schur-canonical}{Schur canonical form}
  For every matrix $\mata\in\Cnn$ there are a unitary matrix
  $\matq\in\Cnn$ and an upper triangular matrix $\matr\in\Cnn$ such
  that
  \begin{gather}
    \mata = \matq \matr \matq^*.
  \end{gather}
  The diagonal entries of $\matr$ are the eigenvalues of $A$. The
  column vectors of $\matq$ are called \define{Schur vectors}.
\end{Theorem*}

\begin{Lemma}{schur-canonical-1}
  For any $k\le n$ the span of the Schur vectors
  $\vq_1,\dots,\vq_k$ is invariant under the action of $\mata$.

  For $\matq_k = (\vq_1\dots\vq_k)$ and $R_k$ the upper left $k\times k$ block of $\matr$, there holds
  \begin{gather}
    \mata\matq_k = \matq_k \matr_k.
  \end{gather}
\end{Lemma}

\begin{Lemma}{schur-canonical-2}
  The Schur vectors depend on the order chosen for the eigenvalues,
  and in case of geometric multiplicity, the eigenvectors,
  respectively. They are determined up to factors $e^{i\phi}$
\end{Lemma}

\begin{Theorem}{convergence-subspace-iteration}
  Let $\mata\in\Cnn$ and
  \begin{gather}
    \abs{\lambda_1} >
    \abs{\lambda_2}>\dots>\abs{\lambda_m}>\abs{\lambda}
  \end{gather}
  for all
  remaining eigenvalues $\lambda\in\sigma(\mata)$. Let
  $\matq = (\vq_1\dots\vq_m)$ be the Schur vectors associated with the
  first $m$ eigenvalues and $\esp{1,\dots,j}$ be the space spanned by
  the first $j$ eigenvectors and $P_j$ the orthogonal projector
  onto this space. Let the set of start vectors of the orthogonal subspace iteration
  $\matx_0 = (\vx_1\dots\vx_m)$ be chosen that
  \begin{gather}
    \operatorname{span}\{P_1 \vx_1,\dots,P_j\vx_j\} = \esp{1,\dots,j},\qquad j=1,\dots,m.
  \end{gather}
  Then, the $j$-th column of $\matx_k$ converges to $\vq_j$ for $j=1,\dots,m$ up to a factor $e^{i\phi}$.
\end{Theorem}

\begin{Lemma}{qr-1}
  The matrices $\mata_k$ of the QR-iteration have the following properties:
  \begin{enumerate}
  \item $\mata_{k+1} = \matq_k^*\mata_k\matq_k = \matq_k^*\dots\matq_1^*A\matq_1\dots\matq_k$.
  \item $\mata^k=\matq_1\dots\matq_k\matr_k\dots\matr_1$.
  \item If $\mata$ is normal, so is $\mata_k$ for any $k$.
  \item If $\mata$ is symmetric, so is $\mata_k$ for any $k$.
  \end{enumerate}
\end{Lemma}

\begin{Theorem}{convergence-qr-iteration}
  Let $\mata\in\Cnn$ symmetric and
  \begin{gather}
    \abs{\lambda_1} >
    \abs{\lambda_2}>\dots>\abs{\lambda_n} \neq 0.
  \end{gather}
  Then, the sequences of the QR-iteration admit the following estimates:
  \begin{align}
    \lim_{k\to\infty} \matq_k &= \id,\\
    \lim_{k\to\infty} \matr_k &= \Lambda,\\
    a_{ij}^{(k)} &= \bigo \left(\abs*{\frac{\lambda_i}{\lambda_j}}^k\right),
                   \qquad i>j.
  \end{align}
  Here, $\Lambda$ is the diagonal matrix of eigenvalues of $\mata$ and
  $a_{ij}^{(k)}$ are the entries of $\mata_k$.
\end{Theorem}

\subsection{Implementation issues}
\begin{intro}
  In each step of the QR-iteration, a QR-decomposition of the matrix
  is needed, which requires $\bigo(n^3)$ operations. Thus, the
  complexity of the iteration is highly unfavorable. The following
  discussion will provide us with means to reduce the complexity of
  the QR-decomposition to $\bigo(n^2)$, in the symmetric case even to
  $\bigo(n)$.
\end{intro}

\begin{Definition}{hessenberg}
  A matrix is in \define{Hessenberg form} or is a \define{Hessenberg
    matrix}, if all its entries below the first subdiagonal are zero. Visually,
  \begin{gather}
    H = 
    \begin{pmatrix}
      *&*&*&*&*&*&*\\
      *&*&*&*&*&*&*\\
      0&*&*&*&*&*&*\\
      0&0&*&*&*&*&*\\
      0&0&0&*&*&*&*\\
      0&0&0&0&*&*&*
    \end{pmatrix}
  \end{gather}
  A symmetric or Hermitian Hessenberg matrix is \define{tridiagonal}.
\end{Definition}

\begin{Theorem}{Hessenberg-qr}
  The QR-decomposition of a Hessenberg matrix $\matH$ can be obtained
  by $n-1$ givens rotations. The matrix $\matr\matq$ is again in
  Hessenberg form. For a (complex) symmetric matrix $\matH$, the
  matrix $\matr\matq$ is even tridiagonal and (complex) symmetric.
\end{Theorem}

\begin{Corollary}{Hessenberg-qr}
  The complexity of each step of a QR-iteration for Hessenberg matrices is $\bigo(n^2)$. For tridiagonal (complex) symmetric matrices, it is $\bigo(n)$.
\end{Corollary}

\begin{Theorem}{Hessenberg-householder}
  Every matrix $\mata\in\Cnn$ is unitarily similar to a Hessenberg matrix $\matH$, that is,
  \begin{gather}
    \matH = \matq \mata \matq^*.
  \end{gather}
  The matrix $\matq$ can be obtained by $n-2$ \putindex{Householder
    reflections}.
\end{Theorem}

\begin{Algorithm*}{qr-method}{The QR-Method}
  Compute the spectrum of a matrix $\mata\in\Cnn$ by
  \begin{enumerate}
  \item Use $n-2$ Householder transformations to transform $\mata$ to
    Hessenberg form
    \begin{gather}
     \matH = \matq\mata\matq^*.
   \end{gather}
 \item QR-iteration: let $\matH_{0}=\matH$ and perform until convergence
   \begin{align}
     \Omega^{(k)}_{1,2}\times\dots\times\Omega^{(k)}_{n-1,n} \matr &= \matH_k\\
     \matH_{k+1} &= \matr \Omega^{(k)}_{1,2}\times\dots\times\Omega^{(k)}_{n-1,n}.
   \end{align}
 \item Store Householder vectors as well as $r$ and $c$ for each
   Givens rotation if the eigenvectors are desired in the end.
  \end{enumerate}
\end{Algorithm*}

\subsection{Shifts and deflation}


%%% Local Variables:
%%% mode: latex
%%% TeX-master: "main"
%%% End:
