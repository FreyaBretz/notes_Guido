\begin{Algorithm*}{inverse-iteration}{The inverse power method}
    \begin{algorithmic}[1]
    \Require $\mata\in\Cnn$, $\sigma\in\C$, initial vector $\vv^{(0)}\in\C^n$
    \For{$k=1,\dots$ until convergence}
    \State Solve $(\mata-\sigma\id) \vy = \vv^{(k-1)}$
    \State $\vv^{(k)} = \frac{\vy}{\norm{\vy}}$
    \State $\alpha_k = R_\mata(\vv^{(k)})$
    \EndFor
  \end{algorithmic}
\end{Algorithm*}

\begin{remark}
  Applying the inverse iteration, we have to solve an almost singular
  linear system, which in itself may be the cause of round-off
  errors. Nevertheless, we have some tools helping us to see that this
  is not a hopeless endeavor.

  First, we can use the Hessenberg form $\matH$ of the matrix $\mata$
  in the inverse iteration. This requires computing the orthogonal
  transformation, when we compute it, but it makes a factorization
  here much cheaper.

  Next, the perfect shift (\slideref{Lemma}{perfect-shift} and
  \slideref{Lemma}{perfect-shift-sym}) indicates that in the singular
  case, the QR factorization of a singular Hessenberg matrix is
  possible, if the matrix is unreduced.

  If the Hessenberg matrix $\matH$ is reduced, then it is
  straightforward that eigenvectors of the diagonal blocks are
  eigenvectors of the whole matrix if padded with zeros. Hence, it is
  sufficient to compute a factorization of the unreduced diagonal
  blocks.

  Finally, there are results showing that the error of the solution
  produced by the QR and even the LU factorization is mostly in
  direction of the eigenvectors related to small eigenvalues, which is
  not an issue here due to normalization.
\end{remark}

\begin{remark}
  The last entry $r_{nn}$ of the QR factorization
  $QR = \matH-\tilde\lambda\id$ with the approximate eigenvalue
  $\tilde\lambda$ is of the size
  \begin{gather}
    \abs{r_{nn}} \approx \abs{\lambda-\tilde\lambda}
  \end{gather}
  for some exact eigenvalue $\lambda$. Hence, if the approximation is
  very good, the corresponding vector $\vq_n$ will be strongly
  amplified and corrected by the subsequent process of backward
  elimination. Since the convergence speed is dominated by
  \begin{gather}
    \frac{\abs{\lambda-\tilde\lambda}}{\abs{\mu-\tilde\mu}},
  \end{gather}
  for all other eigenvalues $\mu$ it will be very fast, if the
  eigenvalues are well separated. When it is very slow, we are in a
  regime where the eigenvector problem itself is not well-conditioned.
\end{remark}
