\begin{Definition}{sesqui}
  A \define{sesquilinear form} on a complex vector space $V$ is a mapping
  \begin{gather}
      a\colon V\times V \to \C,
  \end{gather}
  which is \define{linear} and \define{semi-linear} in the first and second argument, respectively. That is, for all $\vu,\vv,\vw\in V$ and $\alpha\in\C$ holds
  \begin{xalignat}2
  a(\vu+\vv,\vw) &= a(\vu,\vw)+a(\vv,\vw),
  & a(\alpha \vu,\vw) &= \alpha a(\vu,\vw),\\
  a(\vu,\vv+\vw) &= a(\vu,\vv)+a(\vu,\vw),
  & a(\vu,\alpha \vw) &= \overline\alpha a(\vu,\vw).
  \end{xalignat}
  It is \define{Hermitian} or \define{complex symmetric}, if in addition there holds
  \begin{gather}
      \label{eq:inner:symmetry}
      a(\vu,\vv) = \overline{a(\vv,\vu)}.
  \end{gather}
\end{Definition}

\begin{remark}
  A sesquilinear form can be defined equivalently as being semi-linear in the second argument and linear in the first. Both versions can be found in the literature and we follow the convention in the book of Saad.
  
  The author prefers the term ``complex symmetric'' over the common and pompous ``Hermitian'', since it is the natural extension of symmetry to complex vector spaces, as the following asserts. The author would even prefer the term ``symmetric'', which is unfortunately understood in the real way in other publications. In particular,~\eqref{eq:inner:symmetry} implies that $a(\vu,\vu)$ is always real, such that the following makes sense:
\end{remark}

\begin{Definition}{inner-product}
  An \define{inner product} on the complex vector space $V$ is a sesqui-linear form $\scal(\cdot,\cdot)$ on $V$ which in addition is definite, namely, for all $\vv\in V$ there holds
  \begin{gather}
      \scal(\vv,\vv) \ge 0,
  \end{gather}
  and $\scal(\vv,\vv)=0$ implies $\vv=0$.
\end{Definition}

\begin{todo}
  Euclidean inner product
\end{todo}

\begin{todo}
  Define conjugate $\mata^*$
\end{todo}

\begin{todo}
  Define orthonormal set, unitary matrix
\end{todo}


%%% Local Variables:
%%% mode: latex
%%% TeX-master: "main"
%%% End:
