\subsubsection{The conjugate gradient method}
\frame{\subsubtoc}

\frame {\input {blocks/Theorem-arnoldi-linear-system.tex}}
\frame {\input {blocks/Theorem-arnoldi-linear-residual.tex}}

\frame {\input {blocks/Lemma-lanczos-linear.tex}}
\frame {\input {blocks/Lemma-lanczos-incremental.tex}}
\frame {\input {blocks/Lemma-lanczos-orthogonality.tex}}
\frame {\input {blocks/Algorithm-cg.tex}}
\frame {\input {blocks/Lemma-cg-orthogonality.tex}}
\frame {\input {blocks/Corollary-cg-Lanczos.tex}}

\frame {\input {blocks/Theorem-projection-orthogonal-optimal.tex}}
\frame {\input {blocks/Lemma-krylov-polynomial.tex}}

\frame {\input {blocks/Theorem-cg-optimality.tex}
  \pause
  \input {blocks/Corollary-cg-vs-descent}}
\frame {\input {blocks/Corollary-cg-optimality-spectrum.tex}}

\frame {\input {blocks/Definition-chebyshev-polynomials.tex}}
\frame {\input {blocks/Lemma-chebyshev-representation.tex}}
\frame {\input {blocks/Lemma-chebyshev-abscissae.tex}}
% \frame {\input {blocks/Theorem-chebyshev-minimal-1.tex}}
\frame {\input {blocks/Theorem-chebyshev-growth.tex}}
\frame {\input {blocks/Corollary-chebyshev-minimal-2.tex}}
\frame {\input {blocks/Corollary-cg-condition-number.tex}}
\begin{frame}
  \begin{columns}
    \begin{column}{.48\textwidth}
      \input {blocks/Algorithm-steepest-descent-algol.tex}
    \end{column}
    \begin{column}{.48\textwidth}
      \input {blocks/Algorithm-cg.tex}
    \end{column}
  \end{columns}
\end{frame}

%%% Local Variables:
%%% mode: latex
%%% TeX-master: "slides"
%%% End:
