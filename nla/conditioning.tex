In this section, we study the conditioning of finding eigenvalues and eigenvectors. While we will not cover the full theory, we will provide examples for ill-posed problems as well as exemplary proofs for well-posedness.

In all cases, we will investigate the change of eigenvalues or eigenvectors whne the matrix $\mata$ is perturbed by a small matrix $\mate$ of norm $\epsilon$.

\begin{Example}{conditioning-Jordan-block}
  Consider the matrix
  \begin{gather}
  \mata_\epsilon =
      \begin{pmatrix}
        0&1\\
        &0&1\\
        &&\ddots&\ddots\\
        &&&0&1\\
        \epsilon &&&&0
      \end{pmatrix}
      \in\C^{n\times n},
  \end{gather}
  For $\epsilon=0$, it has a single eigenvalue of geometric multiplicity one and algebraic multiplicity $n$.
  
  For $\epsilon>0$, it has $n$ simple eigenvalues
  \begin{gather}
      \lambda_j = \sqrt[n]{\epsilon} \,e^{2\frac jni\pi}
  \end{gather}
\end{Example}

\begin{proof}
  For $\epsilon=0$, the matrix is the generic Jordan-block of an eigenvalue which is not semi-simple, thus the ill-posedness of this example implies the ill-posedness for not semi-simple eigenvalues in the general case. Note that this statement holds notwithstanding that special perturbations may be benign.
  
  The characteristic polynomial of this matrix is
  \begin{gather}
      \chi(\lambda) = \det(\mata-\lambda\id)
      = \det\begin{pmatrix}
      -\lambda&1\\  
        &\ddots&\ddots\\
        &&-\lambda&1\\
        \epsilon &&&-\lambda
      \end{pmatrix}.
  \end{gather}
  Applying Laplace expansion to the first column yields
  \begin{gather}
      \chi(\lambda)
      = -\lambda \det\begin{pmatrix}
        -\lambda&1\\
        &\ddots&\ddots\\
        &&-\lambda&1\\
        &&&-\lambda
      \end{pmatrix}
      + (-1)^{n+1} \epsilon\det\begin{pmatrix}
        1 \\
        -\lambda &1\\
        &\ddots&\ddots\\
        &&-\lambda&1
      \end{pmatrix},
  \end{gather}
  where both matrices are of dimension $n-1$. Since they are triangular, recursion of Laplace expansion is particularly simple and yields the product of the diagonal elements. Thus
  \begin{gather}
      \chi(\lambda) = (-1)^n \lambda^n
      + (-1)^{n+1} \epsilon.
  \end{gather}
  Its roots are determined by the condition
  \begin{gather}
      \lambda^n = \epsilon.
  \end{gather}
  Thus, $\lambda$ can be computed as an $n$th root of unity times the (real) $n$th root of $\epsilon$.
\end{proof}

\begin{Theorem}{Jordan-block-ill-conditioned}
  The eigenvalue problem for eigenvalues which are not semi-simple is
  in general ill-posed.
\end{Theorem}

\begin{proof}
  The analysis in \slideref{Example}{conditioning-Jordan-block} is
  generic in the sense that it applies to nonzero eigenvalues and also
  to matrices which are similar to such a block. Thus, we can conclude
  that for every matrix $\mata$ which is similar to a matrix with a
  nontrivial Jordan block for eigenvalue $\lambda$, there is a
  perturbation $\mate$ such that the derivative of the function
  $\lambda(\epsilon) = \lambda(A+\epsilon\mate)$ at zero is unbounded.
\end{proof}



%%% Local Variables:
%%% mode: latex
%%% TeX-master: "main"
%%% End:
