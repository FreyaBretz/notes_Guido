\begin{Notation}{column-vectors}
  For a matrix $\mata\in \C^{n\times k}$ with entries $a_{ij}$, we denote its column vectors by $\va_j$ such that
  \begin{gather}
      (\va_j)_i = a_{ij}.
  \end{gather}
  Vice versa, given vectors $\vv_1,\dots,\vv_k$ in $\C^n$, we denote the matrix generated by those column vectors as
  \begin{gather}
    \matv = \bigl( \vv_1,\dots,\vv_k\bigr).
  \end{gather}
  The \define{range}\index{ran} of a matrix is the span of its column vectors,
  \begin{gather}
    \ran\matv = \spann{\vv_1,\dots,\vv_k}.
  \end{gather}
\end{Notation}

\begin{Notation}{matrix-linear-combination}
  Let $\vv_1,\dots,\vv_k$ be a set of vectors in $\C^n$ and let the matrix $\matv=(\vv_1,\dots,\vv_k) \in \C^{n\times k}$. Then, for vectors $\va\in\C^k$ there holds
  \begin{gather}
      \vu = \sum_{i=1}^k a_i \vv_i
      \qquad \Longleftrightarrow \qquad
      \vu = \matv \va.
  \end{gather}
  Thus, we can write linear combinations as matrix vector products.
  We will abuse notation in a way that we refer to a matrix $\matv$
  used in this way as a set of vectors or as a basis.
\end{Notation}

\begin{Notation}{colon-notation}
  A common way of specifying parts of a matrix is the so-called \define{colon notation}. We define
  \begin{gather}
    \mata(i:j,k:l) =
    \begin{pmatrix}
      a_{ik}&\cdots&a_{il}\\
      \vdots&&\vdots\\
      a_{jk}&\cdots&a_{jl}
    \end{pmatrix}.
  \end{gather}
  If in one dimension we span the whole matrix, we omit the indices,
  such that $\mata(i,:)$ is the $i$-th row and $\mata(:,k)$ is the $k$-th
  column of the matrix.

  We also use the notation $a_{i:j,k:l}$ instead.
\end{Notation}

\begin{Definition}{invariant-subspace}
  An \define{invariant subspace} $V\subset\C^n$ for the matrix
  $\mata\in\Cnn$ is defined by the property
  \begin{gather}
    \mata\vv \in V \qquad \forall\;\vv\in V,
  \end{gather}
  or equivalently $\mata V \subseteq V$.
  For a matrix with the block structure
  \begin{gather}
    \begin{pmatrix}
      \mata & \matb\\ 0 & \matc,
    \end{pmatrix}
  \end{gather}
  where $\mata\in\C^{k\times k}$ and $\matc\in\C^{(n-k)\times (n-k)}$,
  the vectors $\ve_{k+1},\dots,\ve_{n}$ span an invariant subspace.
\end{Definition}

\begin{Lemma*}{change-of-basis}{Change of basis}
  Let $\matv = (\vv_1,\dots, \vv_n)$ and $\matw = (\vw_1,\dots, \vw_n)$ be
  bases of $\C^n$. Let $\vu\in \C^n$ be given by coefficient vectors
  $\vx,\vy\in \C^n$, such that
  \begin{gather}
    \vu = \matv\vx = \matw\vy.
  \end{gather}
  Then,
  \begin{gather}
    \vx = \matc \vy,
  \end{gather}
  where the columns of $\matc\in \C^{n\times n}$ are the coefficient
  vectors of the basis vectors $\vw_j$ with respect to the basis
  vectors $\matv$.
\end{Lemma*}

\begin{Corollary}{change-of-basis}
  Let $\vv\in \C^n$ and $\matw$ be a basis for $\C^n$. Then, the
  coefficient vector $\vy$ of a vector $\vx$ with respect to the
  basis $\matw$ is obtained by
  \begin{gather}
    \vy = \matw^{-1} \vx.
  \end{gather}
\end{Corollary}

%%% Local Variables:
%%% mode: latex
%%% TeX-master: "main"
%%% End:
