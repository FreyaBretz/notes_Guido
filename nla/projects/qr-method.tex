\documentclass[a4paper]{article}
\usepackage{amsmath}
%\usepackage{background}

\addtolength{\textwidth}{2cm}
\addtolength{\oddsidemargin}{-1cm}

\setlength{\parindent}{0pt}
\setlength{\parskip}{1ex plus 1ex}

\begin{document}
\begin{center}
  \textbf{\Large Programming-Exam Numerical Linear Algebra}
  
  \textbf{Winter 2022/23}
\end{center}

\subsection*{Assignment}
Minimum: implement the QR-method following Algorithm 1.4.17 in the notes and
verify your implementation.

Gold standard: implement the Francis-QR-iteration with double shifts
and deflation in Algorithm 1.5.15.
\subsection*{Dates}

The program(s) should be submitted electronically by Feb 12th,
2023. Oral presentations will be scheduled for the week of Feb 13th, or later upon request.

\subsection*{Minimal requirements}
\begin{itemize}
\item The program must run with several example matrices without
  crashing and you must be able to change parameters like the matrix
  size
\item The program must be subdivided into functions of well-defined
  purpose
\item You must be able to describe how you verify the correctness of
  your program
\end{itemize}

If your program following algorithm 1.4.17 works for several matrices and you can explain it, you will pass the exam, but only with the minimal passing grade (4,0).

\subsection*{Improving your grade}
You can improve your grade by putting more effort into the
implementation. You can consider items from the following list, but
you can also think of other improvements:
\begin{enumerate}
\item Use implicit single shifts to accelerate convergence (Algorithm 1.5.6)
\item Use double shifts to avoid complex arithmetic (Theorem 1.5.13)
\item Use deflation when the subdiagonal element in the last row is
  sufficiently small
\item Use deflation, if any other subdiagonal element is small (somewhat more difficult)
\item Compute not only eigenvalues but also eigenvectors
\item Well-chosen tests for correctness
\item Investigation into convergence
\item Write very well structured code
\item Well-prepared jupyter notebooks with code, accompanying text, and results
\item Object-oriented programming: it is possible to implement
  algorithm 1.4.17 in a way, that $\boldmath H$ is either Hessenberg or
  tridiagonal, either real or complex, but encapsulate the differences
  inside the QR-decompositions applied in every step.
\end{enumerate}

Note that combining for instance options 1 and 2 is not really helpful. Choose wisely!

If you implement the QR iteration based on Algorithm 1.5.15 observing items 2, 3, 4, 6, 7, and 8 of the previous list, and you can explain your work, the result will be a top grade (1,0). But, going all the way may not be necessary.

\subsection*{Additional rules}
\begin{itemize}
\item You can prepare the assignment by yourself or in a team of two
  students.
\item There will be short oral tests with the authors of each program
  in order to verify authorship and help determining a grade. You will
  have to demonstrate the running program. Both students of a team
  will have to contribute equally!
\item Please submit at the oral test a signed declaration: ``I/We have
  prepared the assignment myself/ourselves and I/we have only used the
  sources declared in comments to the program''.
\end{itemize}
\end{document}