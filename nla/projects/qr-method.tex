\documentclass[a4paper]{article}
\usepackage{amsmath}
\usepackage{background}

\addtolength{\textwidth}{2cm}
\addtolength{\oddsidemargin}{-1cm}


\begin{document}
\begin{center}
  \textbf{\Large Programming-Exam Numerical Linear Algebra}
  
  \textbf{Winter 2021/22}
\end{center}

\subsection*{Assignment}
Implement the QR-method following Algorithm 1.4.14 in the notes and
verify your implementation. Choose one of the following options
\begin{enumerate}
\item General complex matrices with transformation to Hessenberg form
\item General real matrices with transformation to Hessenberg form and
  double shifts for complex eigenvalues
\item Symmetric real matrices with transformation to tridiagonal form
  with optimized storage
\end{enumerate}

\subsection*{Dates}

The program(s) should be submitted electronically by Feb 12th,
2022. Oral presentations will be scheduled for the week of Feb 14th.

\subsection*{Minimal requirements}
\begin{itemize}
\item The program must run without crashing and you must be able to
  change parameters like the matrix size
\item The program must be subdivided into functions of well-defined
  purpose
\item You must be able to describe how you verify the correctness of
  your program
\end{itemize}

If your program works for several matrices and you can explain it, you will pass the exam, but only with the minimal passing grade.

\subsection*{Improving your grade}
You can improve your grade by putting more effort into the
implementation. You can consider items from the following list, but
you can also think of other improvements:
\begin{itemize}
\item Use shifts to accelerate convergence (Wilkinson shifts)
\item Use double shifts to avoid complex arithmetic (necessary for option 2)
\item Use deflation when the subdiagonal element in the last row is
  sufficiently small (implementation may be hard if not option 3)
\item Use deflation, if any other subdiagonal element is small (seems very hard)
\item Compute not only eigenvalues but also eigenvectors
\item Add SVD capability to option 3
\item Apply your program to several suitable matrices from the ``Matrix Market''\footnote{\texttt{https://math.nist.gov/MatrixMarket/}}. Use suitable library functions for reading the matrices.
\item Well-chosen tests for correctness
\item Investigation into convergence
\item Write very well structured code
\item Well-prepared jupyter notebooks with code, accompanying text, and results
\item Object-oriented programming: is it possible to implement
  algorithm 1.4.14 in a way, that $\boldmath H$ is either Hessenberg or
  tridiagonal, either real or complex, but encapsulate the differences
  inside the QR-decompositions applied in every step.
\end{itemize}

\subsection*{Additional rules}
\begin{itemize}
\item You can prepare the assignment by yourself or in a team of two
  students.
\item There will be short oral tests with the authors of each program
  in order to verify authorship and help determining a grade. You will
  have to demonstrate the running program. Both students of a team
  will have to contribute equally!
\item Please submit at the oral test a signed declaration: ``I/We have
  prepared the assignment myself/ourselves and I/we have only used the
  sources declared in comments to the program''.
\end{itemize}
\end{document}