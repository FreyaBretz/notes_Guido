
\subsection{Simple iterations}

\begin{Algorithm*}{vector-iteration}{Vector iteration}
  
\end{Algorithm*}

\begin{Theorem}{vector-iteration}
  Lat $\mata\in\Cnn$ be diagonalizable such that $\lambda_1$ is the
  unique eigenvalue with maximal modulus. Let furthermore the
  component of $v^{(0)}$ in direction of the first eigenvector be
  nonzero. Then, the factors $\alpha_k$ and vectors $v^{(k)}$ of the
  vector iteration converge to the eigenvalue $\lambda_1$ and its
  associated eigenvector. Moreover, there holds
  \begin{align}
    \abs{\alpha_{k+1}-\lambda_1}
    &\le \frac{\abs{\lambda_1}}{\abs{\lambda_2}} \abs{\alpha_{k}-\lambda_1}\\
    \norm{v^{(k+1)}-u_1}
    &\le \frac{\abs{\lambda_1}}{\abs{\lambda_2}} \norm{v^{(k)}-u_1}
  \end{align}
\end{Theorem}

\begin{Remark}{vector-iteration}
  The proof actually requires, that the entry defining $\alpha_k$
  remains the same during the iteration, at least during the steps
  used for detecting convergence.

  The result does not actually require that $\mata$ is diagonalizable,
  as long as $\lambda_1$ is single and of largest modulus.
\end{Remark}

\begin{Algorithm*}{shifted-vector-iteration}{Shifted vector iteration}
  The vector iteration can be applied to the matrix $\mata-\sigma\id$
  for some $\sigma\in\C$.

  Then, $\alpha_k$ converges to the eigenvalue $\lambda$ such that
  $\lambda-\sigma$ has largest modulus. $v^{(k)}$ converges to an
  eigenvector for this eigenvalue.
\end{Algorithm*}

\begin{Algorithm*}{inverse-iteration}{The inverse power method}
  
\end{Algorithm*}

\begin{Algorithm*}{Rayleigh-iteration}{The Rayleigh quotient iteration}
  
\end{Algorithm*}

\subsection{Deflation}

%%% Local Variables:
%%% mode: latex
%%% TeX-master: "main"
%%% End:
