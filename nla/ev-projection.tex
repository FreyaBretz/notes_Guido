\begin{Definition}{galerkin-ev}
  Let $\mata\in\Rnn$. Then, the solutions to the eigenvalue problem
  \begin{align}
    \tilde\vx &\in K,\\
    \mata\tilde\vx - \mu\tilde\vx&\perp L.
  \end{align}
  are called the \define{Galerkin
    approximation} of the eigenvalue value problem in a subspace $K$ orthogonal to a subspace
  $L$. In the case $K=L$, it is also called the \define{Rayleigh-Ritz approximation}.
\end{Definition}

\begin{Algorithm*}{rayleigh-ritz}{Rayleigh-Ritz method}
  \begin{enumerate}
  \item Compute an orthonormal basis $\matq_m$ for the subspace $K$ and let
    \begin{gather}
      \matb_m = \matq_m^T\mata\matq_m \in \R^{m\times m}.
    \end{gather}
  \item Compute the eigenvalues of $\matb_m$ and select $k\le m$ ``desired'' eigenvalues
    \begin{gather}
      \mu_1, \dots,\mu_k.
    \end{gather}
  \item Compute the eigenvectors $\vy_1,\dots,\vy_k\in\R^m$ of $\matb$
    and the corresponding approximate eigenvectors of $\mata$:
    \begin{gather}
      \tilde\vu_i = \matq_m \vy_i,\qquad i=1,\dots,k.
    \end{gather}
  \end{enumerate}
\end{Algorithm*}

\begin{Definition}{ritz-values}
  Let $\mata$ be a matrix and $\matb_m = \matq_m^*\mata\matq_m$ be its
  orthogonal projection to the subspace spanned by the orthonormal basis
  $\matq_m$. Then, we refer to the eigenvalues of $\matb_m$ as
  \define{Ritz values}.
  For each eigenvector $\vy_i$ of $\matb_m$, we refer to the vector
  $\vx_i = \matq_m\vy_i$ as \define{Ritz vector}.
\end{Definition}

\begin{Lemma}{ritz-invariant}
  Let $V\subset\R^n$ be an invariant subspace under the action of
  $\mata$. Then, the Ritz values and associated Ritz vectors of the
  Rayleigh-Ritz approximation are exact eigenpairs of $\mata$.
\end{Lemma}

\begin{proof}
  Let $\vq_1,\dots,\vq_m$ be an orthonormal basis for the invariant
  subspace $V$, and let $\matb_m = \matq_m^T\mata\matq$ be the projected
  matrix. For an eigenvector $\vy\in\R^m$ to eigenvalue $\lambda$ of
  $\matb_m$, compute the Ritz vector $\vv = \matq_m\vy$. Then,
  \begin{gather}
    \lambda \vv = \lambda\matq_m\vy = \matq_m\matb\vy
    = \matq_m\matq_m^T\mata\matq_m\vy = \matq_m\matq_m^T\mata\vv.
  \end{gather}
  Since $\matq_m\matq_m^T$ is the orthogonal projector to $V$, the
  term on the right is equal to $\mata\vv$ if and only if $V$ is
  invariant.
\end{proof}

% \begin{remark}
%   A variation of \slideref{Algorithm}{rayleigh-ritz} computes the
%   Schur vectors of $\matb$ instead of the eigenvectors. For symmetric
%   matrices, they are actually the same vectors, but for
%   non-diagonalizable matrices, this variant is preferred as it does
%   not require cumbersome computation of invariant subspaces.
% \end{remark}

\begin{intro}
  We now focus on the Rayleigh-Ritz method for real, symmetric
  matrices. The results transfer immediately to the complex symmetric
  (Hermitian) case, but we avoid formulating with complex numbers.
\end{intro}

\begin{Lemma}{Courant-Fischer-Ritz}
  Let the eigenvalues of a symmetric matrix $\mata\in\Rnn$ be ordered such that
  \begin{gather}
    \lambda_1\ge \lambda_2\ge \dots \ge \lambda_n.
  \end{gather}
  Let $\mu_1,\dots,\mu_m$ be the Ritz values in the subspace
  $V$. Then, there holds
  \begin{xalignat}2
    \lambda_k &\ge \mu_i, &i&=1,\dots,m,\\
    \lambda_{n-i}& \le \mu_{m-i}, &i&=0,\dots,m-1.
  \end{xalignat}
\end{Lemma}

\begin{proof}
  Due to the Courant-Fischer min-max theorem,
  \slideref{Theorem}{minmax}, we can characterize the Ritz values as
  \begin{alignat}2
    \mu_i &= \max_{\substack{W \subset V\\\dim W = i}} \min_{\vx\in W} R_\mata(\vx)
    &\le \max_{\substack{W \subset \R^n\\\dim W = i}} \min_{\vx\in W} R_\mata(\vx)
    &= \lambda_i\\
    \mu_{m-i}&= \min_{\substack{W \subset V\\\dim W = i}} \max_{\vx\in W} R_\mata(\vx)
    &\le \min_{\substack{W \subset \R^n\\\dim V = i}} \max_{\vx\in W} R_\mata(\vx)
    &= \lambda_{n-i}.
  \end{alignat}
\end{proof}

\begin{Lemma}{ritz-rayleigh-estimate}
  Let $(\lambda,\vu)$ be an eigenpair of the symmetric matrix
  $\mata\in\Rnn$. Let $P_V$ be the orthogonal projection to the
  subspace $V$. Then, the \putindex{Rayleigh quotient}
  $R_\mata(P_V \vu)$ admits the estimate
  \begin{gather}
    \abs*{\lambda-R_\mata(P_V \vu)}
    \le \norm{\mata-\lambda\id}
    \frac{\norm{\vu-P_V\vu}^2}{\norm{P_V\vu}^2}
    = \norm{\mata-\lambda\id} \tan^2\theta
    ,
  \end{gather}
  where $\theta$ is the angle between $V$ and the subspace spanned by
  the eigenvector $\vu$.
\end{Lemma}

\begin{proof}
  See also~\cite[Lemma 4.1]{Saad11}.

  Let $\lambda$ be an eigenvalue of $\mata$. We have
  \begin{gather}
    \lambda-R_\mata(P_K \vu)
    = \frac{\scal({(\lambda\id-\mata)} P_V\vu,P_V\vu)}{\scal(P_V\vu,P_V\vu)}.
  \end{gather}
  Furthermore, if $\vu$ is an eigenvector for $\lambda$, then
  \begin{gather}
    (\lambda\id-\mata)P_V\vu = (\lambda\id-\mata) \bigl[u-(\id-P_V)\vu\bigr]
    = (\mata-\lambda\id) (\id-P_V)\vu,
  \end{gather}
  and
  \begin{gather}
    0 = \scal(P_V\vu,{(\mata-\lambda\id)}u)
    = \scal({(\mata-\lambda\id)}P_V\vu,\vu).
  \end{gather}
  Hence,
  \begin{gather}
    \lambda-R_\mata(P_K \vu)
    = \frac{\scal({(\mata-\lambda\id)} {(\id-P_V)}\vu,{(\id-P_V)}\vu)}{\scal(P_V\vu,P_V\vu)}.
  \end{gather}
  
\end{proof}

\begin{Lemma}{ritz-estimate-sine}
  Let $(\lambda,\vu)$ be an eigenpair of the symmetric matrix
  $\mata\in\Rnn$, and let $(\mu,\tilde\vu)$ be the eigenpair
  of the projected problem such that $\mu$ is the closest
  Ritz value to $\lambda$.  Then,
  \begin{gather}
    \abs{\lambda-\mu}
    \le \norm{\mata-\lambda\id} \sin^2\theta,
  \end{gather}
  where $\theta$ is the angle between $\vu$ and $\tilde\vu$.
\end{Lemma}

\begin{proof}
  Note that
  \begin{gather}
    \mu-\lambda
    = \scal(\mata\tilde \vu,\tilde\vu)
    - \lambda\scal(\tilde \vu,\tilde\vu)
    = \scal({(\mata-\lambda)} \tilde\vu,\tilde\vu).
  \end{gather}
  Let now $\alpha = \scal(\vu,\tilde\vu) = \cos\theta$. Exploiting
  $(\mata-\lambda\id)\vu=0$, we write
  \begin{gather}
    \mu-\lambda
    = \scal({(\mata-\lambda\id)}{(\tilde\vu - \alpha\vu)},\tilde\vu)
    = \scal({(\mata-\lambda\id)}{(\tilde\vu - \alpha\vu)},\tilde\vu - \alpha\vu).
  \end{gather}
  Hence, by Bunyakovsky-Cauchy-Schwarz,
  \begin{align}
    \abs{\mu-\lambda}
    &\le \norm{(\mata-\lambda\id)(\tilde\vu - \alpha\vu)}\norm{\tilde\vu - \alpha\vu}\\
    &\le \norm{\mata-\lambda\id}\norm{\tilde\vu - \alpha\vu}^2\\
    &= \norm{\mata-\lambda\id} \sin^2\theta.
  \end{align}
\end{proof}


\begin{Algorithm*}{ev-projection}{Projected subspace iteration}
  \begin{algorithmic}[1]
    \Require $\matx^{(0)} = (\vx_1,\dots,\vx_m)$
    \For {$k=1,\dots$ until convergence}
    \State $\matw \gets \mata\matx^{(k-1)}$
    \State $\matq \gets ONB(\matw)$ \Comment{QR factorization}
    \State $\matb \gets \matq^*\mata\matq$ \Comment{Projected matrix}
    \State $\maty \gets EV(\matb)$ \Comment{Eigenvectors}
    \State $\matx^{(k)} \gets \matq\maty$
    \EndFor
  \end{algorithmic}
\end{Algorithm*}

\begin{Theorem}{ritz-convergence}
  Let $D_m(\mata)$ be the space spanned by the dominant $m$
  eigenvectors $\vv_1,\dots,\vv_m$ of $\mata$ and let $\matp_D$ be the
  orthogonal projector onto it. Let $S_k$ be the subspace spanned by
  the basis $\matx^{(k)}$.
  
  If $\matp_D S_0 = D_m(\mata)$, then there holds: for each
  eigenvector $\vv_i, i=1,\dots,m$, there is a vector
  $\vy_i\in S_0$ such that $\matp\vy_i = \vv_i$. Furthermore,
  \begin{gather}
    \norm{\vv_i - \matp_{S_k} \vv_i}_2 \le \norm{\vv_i-\vy_i}
    \left(\abs*{\frac{\lambda_{m+1}}{\lambda_i}}+\epsilon_k\right)^k,
  \end{gather}
  where $\matp_{S_k}$ is the orthogonal projector onto $S_k$ and
  $\epsilon_k\to 0$ as $k \to \infty$.
\end{Theorem}

\begin{proof}
  See~\cite[Theorem 5.2]{Saad11}
\end{proof}

\begin{remark}
  The estimates are all with respect to the $m+1$-st eigenvalue. This
  differs from the standard subspace iteration in the first chapter,
  where we always deal with the separation to the following
  eigenvalue.
  
  Thus, it is a reasonable approach to choose $m > n_{\text{ev}}$,
  such that for all relevant eigenvalues the quotient
  $\lambda_{m+1}/\lambda_i$ is sufficiently small. This is
  particularly true if there is a known gap in the spectrum.

  With respect to this property, this method seems superior to the
  orthogonal subspace iteration. Be aware though that a Schur
  decomposition by QR factorization or the computation of eigenvalues
  and eigenvectors of $\matb$ relies on methods from the first
  chapter.
\end{remark}



%%% Local Variables:
%%% mode: latex
%%% TeX-master: "main"
%%% End:
