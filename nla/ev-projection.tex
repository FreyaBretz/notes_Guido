\begin{Definition}{galerkin-ev}
  Let $\mata\in\Rnn$. Then, the solutions to the eigenvalue problem
  \begin{align}
    \tilde\vx &\in K,\\
    \mata\tilde\vx - \mu\tilde\vx&\perp L.
  \end{align}
  are called the \define{Galerkin
    approximation} of the eigenvalue value problem in a subspace $K$ orthogonal to a subspace
  $L$. In the case $K=L$, it is also called the \define{Rayleigh-Ritz approximation}.
\end{Definition}

\begin{Algorithm*}{rayleigh-ritz}{Rayleigh-Ritz method}
  \begin{enumerate}
  \item Compute an orthonormal basis $\matv_m$ for the subspace $K$ and let
    \begin{gather}
      \matb_m = \matv_m^T\mata\matv_m \in \R^{m\times m}.
    \end{gather}
  \item Compute the eigenvalues of $\matb_m$ and select $k\le m$ ``desired'' eigenvalues
    \begin{gather}
      \mu_1, \dots,\mu_k.
    \end{gather}
  \item Compute the eigenvectors $\vy_1,\dots,\vy_k\in\R^m$ of $\matb$
    and the corresponding approximate eigenvectors of $\mata$:
    \begin{gather}
      \tilde\vu_i = \matv_m \vy_i,\qquad i=1,\dots,k.
    \end{gather}
  \end{enumerate}
\end{Algorithm*}

\begin{Definition}{ritz-values}
  Let $\mata$ be a matrix and $\matb = \matv^*\mata\matv$ be its
  projection to the subspace spanned by the orthonormal basis
  $\matv$. Then, we refer to the eigenvalues of $\matb$ as
  \define{Ritz values}.
  For each eigenvector $\vy_i$ of $\matb$, we refer to the vector
  $\vx_i = \matv\vy_i$ as \define{Ritz vector}.
\end{Definition}

\begin{Lemma}{ritz-invariant}
  Let $K\subset\R^n$ be an invariant subspace under the action of
  $\mata$. Then, the Ritz values and associated Ritz vectors of the
  Rayleigh-Ritz approximation are exact eigenpairs of $\mata$.
\end{Lemma}

\begin{remark}
  A variation of \slideref{Algorithm}{rayleigh-ritz} computes the
  Schur vectors of $\matb$ instead of the eigenvectors. For symmetric
  matrices, they are actually the same vectors, but for
  non-diagonalizable matrices, this variant is preferred as it does
  not require cumbersome computation of invariant subspaces.
\end{remark}

\begin{intro}
  We now focus on the Rayleigh-Ritz approximation of real, symmetric
  matrices. The results transfer immediately to the complex symmetric
  (Hermitian) case, but we avoid formulating with complex numbers.
\end{intro}

\begin{Lemma}{Courant-Fischer-Ritz}
  Let the eigenvalues of a symmetric matrix $\mata\in\Rnn$ be ordered such that
  \begin{gather}
    \lambda_1\ge \lambda_2\ge \dots \ge \lambda_n.
  \end{gather}
  Let $\mu_1,\dots,\mu_m$ be their Rayleigh-Ritz approximation in the
  subspace $K$. Then, there holds
  \begin{gather}
    \lambda_i \ge \mu_i, \qquad i=1,\dots,m.
  \end{gather}
\end{Lemma}

\begin{Lemma}{ritz-rayleigh-estimate}
  Let $(\lambda,\vu)$ be an eigenpair of the symmetric matrix
  $\mata\in\Rnn$. Let $\P_K$ be the orthogonal projection to the
  subspace $K$. Then, the \putindex{Rayleigh quotient}
  $R_\mata(P_K \vu)$ admits the estimate
  \begin{gather}
    \abs*{\lambda-R_\mata(P_K \vu)}
    \le \norm{\mata-\lambda\id}
    \frac{\norm{\vu-P_K\vu}^2}{\norm{P_K\vu}^2}.
  \end{gather}
\end{Lemma}


\begin{Algorithm*}{ev-projection}{Projected subspace iteration}
  \begin{algorithmic}[1]
    \Require $\matx^{(0)} = (\vx_1,\dots,\vx_m)$
    \For {$k=1,\dots$ until convergence}
    \State $\matw \gets \mata\matx^{(k-1)}$
    \State $\matv \gets ONB(\matw)$ \Comment{QR factorization}
    \State $\matb \gets \matv^*\mata\matv$ \Comment{Projected matrix}
    \State $\maty \gets EV(\matb)$ \Comment{Eigenvectors}
    \State $\matx^{(k)} \gets \matv\maty$
    \EndFor
  \end{algorithmic}
\end{Algorithm*}

\begin{Theorem}{ritz-convergence}
  Let $\esp{1,\dots,m}$ be the space spanned by the first (by
  magnitude of eigenvalue) $m$ eigenvectors and let $\matp$ be the
  orthogonal projector onto it. Let $S_k$ be the subspace spanned by
  the basis $\matx^{(k)}$.
  
  If $\matp S_0 = \esp{1,\dots,m}$, then there holds: for each
  eigenvector $\vv_i$ of $\mata$, $i=1,\dots,m$, there is a vector
  $\vy_i\in S_0$ such that $\matp\vy_i = \vv_i$. Furthermore,
  \begin{gather}
    \norm{\vv_i - \matp_{S_k} \vv_i}_2 \le \norm{\vv_i-\vy_i}
    \left(\abs*{\frac{\lambda_{m+1}}{\lambda_i}}+\epsilon_k\right)^k,
  \end{gather}
  where $\matp_{S_k}$ is the orthogonal projector onto $S_k$ and
  $\epsilon_k\to 0$ as $k \to \infty$.
\end{Theorem}

\begin{proof}
  See~\cite[Theorem 5.2]{Saad11}
\end{proof}

\begin{remark}
  The estimates are all with respect to the $m+1$-st eigenvalue. This
  differs from the standard subspace iteration in the first chapter,
  where we always deal with the separation to the following
  eigenvalue.
  
  Thus, it is a reasonable approach to choose $m > n_{\text{ev}}$,
  such that for all relevant eigenvalues the quotient
  $\lambda_{m+1}/\lambda_i$ is sufficiently small. This is
  particularly true if there is a known gap in the spectrum.

  With respect to this property, this method seems superior to the
  orthogonal subspace iteration. Be aware though that a Schur
  decomposition by QR factorization or the computation of eigenvalues
  and eigenvectors of $\matb$ relies on methods from the first
  chapter.
\end{remark}



%%% Local Variables:
%%% mode: latex
%%% TeX-master: "main"
%%% End:
