\begin{Theorem*}{schur-canonical}{Schur canonical form}
  For every matrix $\mata\in\Cnn$ there exists a unitary matrix
  $\matq\in\Cnn$ and an upper triangular matrix $\matr\in\Cnn$ such
  that
  \begin{gather}
    \mata = \matq \matr \matq^*.
  \end{gather}
  The diagonal entries of $\matr$ are the eigenvalues of $A$. The
  column vectors of $\matq$ are called \define{Schur vectors}.
\end{Theorem*}

\begin{proof}
    We prove the theorem by induction over $n$. For $n = 1$, the statement is obvious.

    Now, let $n>1$. Every matrix has at least one eigenpair, say
    $(\lambda_1,\vq_1)$. We complete $\vq_n$ to an orthonormal basis
    $\matx = (\vq_1,\vw_1,\dots,\vw_{n-1,})$ of $\C^n$.  Since
    $\mata\matq_1 = \lambda_1\matq_1$, there holds
  \begin{gather}
    \matx^* \mata \matx =
    \begin{pmatrix}
      \lambda_1 & \widetilde\va \\
      0 & \widetilde A
    \end{pmatrix},
  \end{gather}
  with a vector $\widetilde\va\in\C^{n-1}$.  Since $\widetilde A$ is
  of dimension $n-1$, we can use the induction argument to deduce that
  there is an orthonormal basis $\vq_1,\dots,\vq_{n-1}\in\C^{n-1}$
  which transforms it by similarity to an upper triangular
  matrix. Introducing $\widetilde \matq = \vq_1,\dots,\vq_{n-1}$, we obtain
  \begin{gather}
    \begin{pmatrix}
      \lambda_1& *& *\\
      &\ddots& * \\
      &&\lambda_n
    \end{pmatrix}
    = \matq^*\mata\matq,
    \qquad\text{with}\qquad
    \matq = \matx
    \begin{pmatrix}
      1 &\\ &\widetilde \matq
    \end{pmatrix}
  \end{gather}
  Since $\matx$ and and $\widetilde \matq$ are unitary, so is $\matq$, such
  that we have proven that there is a unitary similarity
  transformation to an upper triangular matrix.

  The eigenvalues of a triangular matrix are necessarily its diagonal
  elements. Therefore, $\lambda_1,\dots,\lambda_n$ are indeed the
  eigenvalues of $\mata$. Note though that the columns of $\matq$ are
  not necessarily its eigenvectors!
\end{proof}

\begin{Lemma}{schur-canonical-1}
  For any $k\le n$ the span of the Schur vectors
  $\vq_1,\dots,\vq_k$ is invariant under the action of $\mata$.

  For $\matq_k = (\vq_1\dots\vq_k)$ and $\matr_k$ the upper left $k\times k$ block of $\matr$, there holds
  \begin{gather}
    \mata\matq_k = \matq_k \matr_k.
  \end{gather}
\end{Lemma}

\begin{Lemma}{schur-canonical-2}
  The Schur vectors depend on the order chosen for the eigenvalues,
  and in case of geometric multiplicity, the eigenvectors,
  respectively. They are determined up to factors $e^{i\phi}$
\end{Lemma}

\begin{Definition}{dominant-invariant-subspace}
  Let $\mata\in\Cnn$ be a matrix with Schur decomposition
  $\mata = \matq^*\matr\matq$ where the eigenvalues are such that
  \begin{gather}
    \abs{\lambda_1}\ge\abs{\lambda_2}\ge\dots\ge\abs{\lambda_n}.
  \end{gather}
  If $\abs{\lambda_r}>\abs{\lambda_{r+1}}$, we define the
  \define{dominant invariant subspace} of dimension $r$ as
  \begin{gather}
    D_r(\mata) = \spann{\vq_1,\dots,\vq_r}.
  \end{gather}
\end{Definition}

%%% Local Variables:
%%% mode: latex
%%% TeX-master: "main"
%%% End:
