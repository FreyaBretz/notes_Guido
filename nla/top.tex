\lstset{language=Python,numbers=left,resetmargins=true,xleftmargin=8pt,basicstyle=\small,numberstyle=\scriptsize}
\usetikzlibrary{svg.path}
\excludecomment{solution}
\tikzset{velox/.style={color=black,draw,fill=red,thick,%
    shape=diamond,aspect=.4,
    inner sep=1.3pt,transform shape}}
\tikzset{veloy/.style={color=black,draw,fill=red,thick,%
    shape=diamond,aspect=2.5,
    inner sep=1.3pt,transform shape}}
\tikzset{veloxy/.style={color=black,draw,fill=red,thick,%
    shape=star,star points=4,star point ratio=2.2,
    inner sep=1.3pt,transform shape}}
\tikzset{pressure/.style={color=black,draw,fill=cyan,thick,%
    shape=circle,inner sep=2pt,transform shape}}
\tikzset{velo/.style={transform shape,double=red,arrows={-Stealth[open,fill=red]}}}

%% Macros for drawing degrees of freedom for different shapes/elements.
%% Arguments are always:
%%   #1: Starting point
%%   #2: End point
%%   #3: polynomial degree
%%   #4: node settings

\tikzset{pics/edgenormal/.style args={#1/#2/#3/#4}{%
    code={%
      \draw #1 -- #2
      node foreach \x [evaluate=\x as \xval] in {1,...,#3} [#4,sloped,pos=\xval/(#3+1)] {};
      }
}}


%% Macros for drawing degrees of freedom for different shapes/elements.
%% Arguments are always:
%%   #1: polynomial degree
%%   #2: node settings

\tikzset{pics/tripile/.style args={#1/#2}{%
    code={%
      \coordinate (top) at (0,#1);
      \foreach \i in{0,...,#1}
      \foreach \j in{0,...,\i}
      {
        \tikzmath{
          \y = .3*(2/3*#1-\i)*cos(30);
          \x = .3*(\i/2-\j);
        }
        \node[#2] at (\x,\y) {};
      }
    }
}}

\tikzset{pics/tensor/.style args={#1/#2/#3}{%
    code={%
      \coordinate (top) at (0,#1);
      \foreach \i in{0,...,#1}
      \foreach \j in{0,...,#2}
      {
        \tikzmath{
          \y = 2*(\i+1)/(#1+2);
          \x = 2*(\j+1)/(#2+2);
        }
        \node[#3] at (\x,\y) {};
      }
    }
}}

\tikzset{pics/pfem/.style args={#1/#2}{%
    code={%
      \tikzmath{ \ytop=2*cos(30); }
      \coordinate (top) at (0,\ytop);

      \foreach \i in{0,...,#1}
      \foreach \j in{0,...,\i}
      {
        \tikzmath{
          \y = \ytop-\ytop*\i/#1;
          \x = 2*(\i/2-\j)/#1+1;
        }
        \node[#2] at (\x,\y) {};
      }
    }
}}

\tikzset{pics/qfem/.style args={#1/#2}{%
    code={%
      \foreach \i in{0,...,#1}
      \foreach \j in{0,...,#1}
      {
        \tikzmath{
          \y = 2-2*\i/#1;
          \x = 2-2*\j/#1;
        }
        \node[#2] at (\x,\y) {};
      }
    }
}}

%%% Local Variables:
%%% mode: latex
%%% TeX-master: "all"
%%% End:


\title{Numerical Linear Algebra}
\author{Guido Kanschat}
\date{\today}

\def\esp#1{V_{#1}}

\begin{document}
\maketitle
\tableofcontents
\chapter{Dense Algebraic Eigenvalue Problems}
\begin{intro}
  We refer to problems in linear algebra which allow storing the
  complete matrix as \define{dense linear algebra}. They are typically
  characterized by dimensions into the hundreds, possibly thousands,
  and any entry in the matrix may have a nonzero value. Such matrices
  are typically stored as a rectangular or quadratic array of numbers,
  and we can perform manipulations based on the matrix entry.

  In contrast, we will turn to \define{sparse linear algebra} in the
  later chapters, where dimensions got up to millions and billions
  ($10^9$). Since currently no computer on earth can store a matrix
  with $10^{18}$ entries, those matrices will be characterized by the
  fact that each row only contains very few nonzero entries, or that
  the matrix is not stored, but only available algorithmically in the
  form of a function performing the action $\vx\mapsto \mata
  \vx$. Thus, access to and manipulation of matrix entries is not
  possible, and we have to focus on methods only using the properties
  of the matrix as a linear mapping.
\end{intro}

\section{Mathematical background}
\subsection{Definition of Eigenvalue Problems}
\begin{intro}
  The following results can be found in any book on linear
  algeba. Thus, we will just keep the arguments short. There will be a
  focus on normal matrices justified by results on conditioning of
  eigenvalue problems later on.

  Thus, spectral theory based on module theory will not be needed in
  this class. The spectral theorem for normal matrices on the other
  hand is fairly simple and can be proved without too much overhead.
\end{intro}

\begin{Definition}{eigenvalue}
  An \define{eigenvalue} of a matrix $\mata\in \C^{n\times n}$ is a
  complex number $\lambda$ such that the matrix
  \begin{gather}
   \mata-\lambda\id 
  \end{gather}
  is singular.

  The set of all eigenvalues of $\mata$ is called the
  \define{spectrum} $\sigma(\mata)$.

  The \define{eigenspace} for $\lambda$ is the kernel of
  $A-\lambda\id$, that is, the set
\begin{gather}
    \esp{\lambda} = \bigl\{
    \vv \in \C^n \;\big\vert\;
    \mata\vv = \lambda\vv \bigr\}.
\end{gather}
The \define{geometric multiplicity} of $\lambda$ is the dimension of
$\esp{\lambda}$.


An \define{eigenvector} for $\lambda$ is a (normed) vector in
$\esp\lambda$. We refer to an eigenvector $\lambda$ and a
corresponding eigenvector $\vv$ as \define{eigenpair}.
\end{Definition}

\begin{Definition}{eigenvalue-algebraic}
  An \define{eigenvalue} of a matrix $\mata\in \C^{n\times n}$ is a root of the characteristic polynomial $\chi(\lambda) = \det(\mata-\lambda\id)$.
  
  The \define{algebraic multiplicity} of an eigenvalue is the multiplicity of the corresponding root of the characteristic polynomial.
\end{Definition}

\begin{Lemma}{eigenvalue-equivalent}
  The two definitions of an eigenvalue are consistent.
\end{Lemma}

\begin{Theorem}{eigenvalue-count}
  Every matrix in $\C^{n\times n}$ has at most $n$ eigenvalues. The algebraic multiplicities of all eigenvalues add up to $n$.
\end{Theorem}

\begin{proof}
  The ``at most'' follows from the fact that a polynomial contains
  linear factors $x-\lambda_i$ for each of its roots
  $\lambda_i$. Thus, if the characteristic polynomial has $k$ roots it
  has at least degree $k$. On the other hand, the characteristic
  polynomial has degree $n$, such that $k\le n$.

  The second statement is due to the fact that every polynomial over
  $\C$ is a product of linear factors.
\end{proof}

\begin{remark}
  The last theorem is not true in $\R$, as it is not algebraically
  closed. Thus, even a real matrix may have complex eigenvalues and
  eigenvectors. Therefore, all results in this chapter will be on
  complex matrices, but some simplifications for real matrices will be
  pointed out.
\end{remark}

\begin{Definition}{eigenvalue-simple}
  An eigenvalue is \define{simple}, if its algebraic and geometric multiplicity are one. It is \define{semi-simple}, if its algebraic and geometric multiplicities are equal.
\end{Definition}

\begin{remark}
  It is possible to refine the concept of eigenvalues and eigenvectors by distinguishing between right eigenvalues and vectors solving
  \begin{gather}
      \mata \vv = \lambda \vv,
  \end{gather}
  and left eigenvalues and vectors solving
  \begin{gather}
    \vu \mata = \lambda \vu,
  \end{gather}
  where $\vu$ is now a row vector. By taking the transpose of this equation\footnote{Here, we refer to the real transpose obtained by simply exchanging indices, not the complex conjugate transpose.},
  \begin{gather}
    \label{eq:evp:1}
    \mata^T \vu^T = \lambda \vu^T,
  \end{gather}
  we see that $\vu$ is a left eigenvector of $\mata$ if and only if
  $\vu^T$ is a right eigenvector of $\mata^T$.
\end{remark}

\begin{Lemma}{eigenvalues-conjugate}
  Every eigenvalue $\lambda$ of $\mata\in\C^{n\times n}$ is also an eigenvalue of $\mata^T$.
\end{Lemma}

\begin{proof}
  The determinant does not change when the matrix is transposed, therefore
  \begin{gather}
    \chi(\mata^T)
    = \det(\mata^T-\lambda \id)
    = \det(\mata-\lambda \id)
    = \chi(\mata).
  \end{gather}
  Thus, the eigenvalues of $\mata$ and of $\mata^T$ coincide.
\end{proof}

\subsection{Normal and Hermitian matrices}

\begin{Definition}{normal-Hermitian}
  A matrix $\mata\in\C^{n\times n}$ is called \define{normal} if there holds
  \begin{gather}
      A^*A = AA^*.
  \end{gather}
  It is called \define{Hermitian} or \define{complex symmetric}, if there holds
  \begin{gather}
      A=A^*.
  \end{gather}
\end{Definition}

\begin{Theorem*}{normal-diagonalizable}{Spectral theorem for normal matrices}
  A matrix $\mata\in\C^{n\times n}$ is normal if and only if it is diagonalizable by a unitary matrix.
  
  It is normal if and only if there exists an orthonormal basis of eigenvectors.
\end{Theorem*}

\begin{proof}
  
\end{proof}

\begin{Corollary*}{symmetric-diagonalizable}{Spectral theorem for Hermitian matrices}
  A Hermitian matrix $\mata\in\C^{n\times n}$ is diagonalizable with
  an orthogonal basis of eigenvectors and real eigenvalues.  
\end{Corollary*}

\begin{proof}
  Hermitian matrices are a special case of normal matrices, such that
  \slideref{Theorem}{normal-diagonalizable} applies.
\end{proof}

%%% Local Variables:
%%% mode: latex
%%% TeX-master: "main"
%%% End:

\section{Well-posedness of the EVP and bounds on eigenvalues}
\subsection{Bounds on eigenvalues}

\begin{Lemma}{bound-by-norm}
  Let $\norm{\cdot}$ be a vector norm and denote by the same symbol
  a consistent norm for matrices. Then, for any matrix $\mata\in\Cnn$
  and for any eigenvalue $\lambda\in\sigma(\mata)$ there holds
  \begin{gather}
    \abs{\lambda} \le \norm{\mata}.
  \end{gather}
\end{Lemma}

\begin{Lemma}{pre-gershgorin}
  Let $\mata,\matb\in\Cnn$ and let $\norm{\cdot}$ be an operator norm
  on the space of matrices corresponding to a vector norm denoted by
  $\norm{\cdot}$ as well. Then, for any eigenvalue
  $\lambda\in\sigma(\mata)$ such that $\lambda\not\in\sigma(\matb)$
  there holds
  \begin{gather}
    \norm*{(\lambda\id-\matb)^{-1}(A-B)} \ge 1.
  \end{gather}
\end{Lemma}


\begin{Theorem*}{gershgorin}{Gershgorin circle theorem}
  All eigenvalues of a matrix $\mata\in\Cnn$ are contained in the
  union of the \define{Gershgorin Circle}s
  \begin{gather}
    G_j = \left\{ z\in \C \middle| \abs{z-a_{jj}} \le \sum_{k\neq j} \abs{a_{jk}}\right\}.
  \end{gather}
  Furthermore, if there is a subset of $m$ circles disjoint from the
  other circles, then this subset contains $m$ eigenvalues.
\end{Theorem*}

\subsection{The Rayleigh quotient}

\begin{Definition}{rayleigh-quotient}
  For a matrix $\mata\in\Cnn$ and a vector $\vx\in\C^n$, the
  \define{Rayleigh quotient} is defined as
  \begin{gather}
    R_\mata(\vx) = \frac{\scal(\mata\vx,\vx)}{\scal(\vx,\vx)}.
  \end{gather}
\end{Definition}

\begin{Theorem*}{minmax}{Courant-Fischer min-max theorem}
  Let $\mata\in\Cnn$ be Hermitian with eigenvalues
  $\lambda_1 \le \lambda_2\le\dots\le \lambda_n$. Then, for $k=1,\dots,n$
  \begin{align}
    \lambda_k
    &= \min_{\substack{V \subset \C^n\\\dim V = k}} \max_{\vx\in V} R_\mata(\vx),\\
    &= \max_{\substack{V \subset \C^n\\\dim V = n-k+1}} \min_{\vx\in V} R_\mata(\vx).
  \end{align}
  In particular,
  \begin{gather}
    \lambda_{\min}(\mata) = \min_{\vx\in\C^n} R_\mata(\vx),
    \qquad
    \lambda_{\max}(\mata) = \max_{\vx\in\C^n} R_\mata(\vx).
  \end{gather}
\end{Theorem*}

\subsection{Conditioning of the eigenvalue problem}

In this section, we study the conditioning of finding eigenvalues and
eigenvectors. While we will not cover the full theory, we will provide
examples for ill-posed problems as well as exemplary proofs for
well-posedness.

In all cases, we will investigate the change of eigenvalues or
eigenvectors when the matrix $\mata$ is perturbed by a small matrix
$\mate$ of norm $\epsilon$.

\begin{Definition}{conditioning-eigenvalue}
  Let $\mata\in\Cnn$ and let $\mata+\mate$ be a small perturbation of $\mata$. Then, we define the \define{absolute conditioning}\index{conditioning!absolute} of the eigenvalue problem as follows: let $\mu$ be an eigenvalue of $\mata+\mate$, then, there is an eigenvalue $\lambda$ of $\mata$ and a constant $C_{\text{abs}}$ called \define{condition number}, such that
  \begin{gather*}
    \abs{\lambda-\mu} \le C_{\text{abs}} \norm{E},
  \end{gather*}
  for a suitable matrix norm. The \define{relative conditioning}\index{conditioning!relative} is defined by
  the estimate
  \begin{gather*}
    \frac{\abs{\lambda-\mu}}{\lambda}
    \le C_{\text{rel}} \frac{\norm{E}}{\norm{A}}.
  \end{gather*}
\end{Definition}

\begin{Example}{characteristic-polynomial}
  Take a matrix of dimension 20 with eigenvalues $1,2,\ldots,20$. Its
  characteristic polynomial is
  \begin{gather}
    \chi(\lambda) = (\lambda-1)\dots(\lambda-20).
  \end{gather}
  The coefficient in front of $\lambda^{20}$ is one, the constant term is $20! > 10^{19}$.
  We perturbe it in the form
  \begin{gather}
    \tilde \chi(\lambda) = \chi(\lambda) - 10^{-23}\lambda^{19}.
  \end{gather}
  Their greatest roots are
  \begin{gather}
    \begin{array}{l@{\qquad}l@{\,}c@{\,}l}
      \multicolumn{1}{c}{\chi}&
      \multicolumn{3}{c}{\tilde \chi}\\
      20&20.847\\
      19,18&19.502&\pm&1.940i\\
      17,16&16.731&\pm&2.813i\\
      15,14&13.992&\pm&2.519i\\
    \end{array}
  \end{gather}
  {\tiny Source: \cite{DeuflhardHohmann08}}
\end{Example}

\begin{Example}{conditioning-Jordan-block}
  Consider the matrix
  \begin{gather}
  \mata_\epsilon =
      \begin{pmatrix}
        0&1\\
%        &0&1\\
        &\ddots&\ddots\\
        &&0&1\\
        \epsilon &&&0
      \end{pmatrix}
      \in\C^{n\times n},
  \end{gather}
  For $\epsilon=0$, it has a single eigenvalue of geometric multiplicity one and algebraic multiplicity $n$.

  For $\epsilon>0$, it has $n$ simple eigenvalues
  \begin{gather}
      \lambda_j = \sqrt[n]{\epsilon} \,e^{2\frac jni\pi}
  \end{gather}
\end{Example}

\begin{proof}
  For $\epsilon=0$, the matrix is the generic Jordan-block of an eigenvalue which is not semi-simple, thus the ill-posedness of this example implies the ill-posedness for not semi-simple eigenvalues in the general case. Note that this statement holds notwithstanding that special perturbations may be benign.

  The characteristic polynomial of this matrix is
  \begin{gather}
      \chi(\lambda) = \det(\mata-\lambda\id)
      = \det\begin{pmatrix}
      -\lambda&1\\
        &\ddots&\ddots\\
        &&-\lambda&1\\
        \epsilon &&&-\lambda
      \end{pmatrix}.
  \end{gather}
  Applying Laplace expansion to the first column yields
  \begin{gather}
      \chi(\lambda)
      = -\lambda \det\begin{pmatrix}
        -\lambda&1\\
        &\ddots&\ddots\\
        &&-\lambda&1\\
        &&&-\lambda
      \end{pmatrix}
      + (-1)^{n+1} \epsilon\det\begin{pmatrix}
        1 \\
        -\lambda &1\\
        &\ddots&\ddots\\
        &&-\lambda&1
      \end{pmatrix},
  \end{gather}
  where both matrices are of dimension $n-1$. Since they are triangular, recursion of Laplace expansion is particularly simple and yields the product of the diagonal elements. Thus
  \begin{gather}
      \chi(\lambda) = (-1)^n \lambda^n
      + (-1)^{n+1} \epsilon.
  \end{gather}
  Its roots are determined by the condition
  \begin{gather}
      \lambda^n = \epsilon.
  \end{gather}
  Thus, $\lambda$ can be computed as an $n$th root of unity times the (real) $n$th root of $\epsilon$.
\end{proof}

\begin{Theorem}{Jordan-block-ill-conditioned}
  The eigenvalue problem for eigenvalues which are not semi-simple is
  in general ill-posed.
\end{Theorem}

\begin{proof}
  The analysis in \slideref{Example}{conditioning-Jordan-block} is
  generic in the sense that it applies to nonzero eigenvalues and also
  to matrices which are similar to such a block. Thus, we can conclude
  that for every matrix $\mata$ which is similar to a matrix with a
  nontrivial Jordan block for eigenvalue $\lambda$, there is a
  perturbation $\mate$ such that the derivative of the function
  $\lambda(\epsilon) = \lambda(A+\epsilon\mate)$ at zero is unbounded.
\end{proof}

\begin{Theorem*}{bauer-fike}{Bauer-Fike}
  Let $\mata\in \Cnn$ be diagonalizable with matrix of eigenvectors
  $\matv \in \Cnn$ and diagonal matrix
  $\matlambda = \diag(\lambda_1\dots,\lambda_n)$. Let $\mata+\mate$ be
  a perturbation of $\mata$. Then, for any eigenvalue $\mu$ of
  $\mata+\mate$, there is an eigenvalue $\lambda_i$ of $\mata$ such
  that
  \begin{gather}
    \abs{\mu-\lambda_i} \le \cond_2(\matv) \norm{\mate}_2.
  \end{gather}
\end{Theorem*}

\begin{proof}
  Wikipedia
\end{proof}

\begin{Corollary}{conditioning-eigenvalues-normal}
  The eigenvalue problem of a normal matrix $\mata\in\Cnn$ is
  well-conditioned in the sense that for every eigenvalue $\mu$ of the
  perturbed matrix $\mata+\mate$, there is an eigenvalue $\lambda$ of
  $\mata$ such that
  \begin{gather}
    \abs{\mu-\lambda} \le \norm{E}_2.
  \end{gather}
\end{Corollary}

The Bauer-Fike theorem provides a general estimate for diagonalizable
matrices in terms of the condition number of the matrix of
eigenvectors. The following theorem is less general, since it only
applies to simple eigenvalues, but it provides geometric intuition of
the issue.

\begin{Theorem}{conditioning-eigenvalue-single}
  Let $\mata_\epsilon = \mata+\epsilon\mate\in\Cnn$ be a perturbation
  of $\mata\in\Cnn$. Let $\lambda(0)$ be a simple
  eigenvalue of $\mata$. Then, there exists a uniquely defined
  differentiable continuation $\lambda(\epsilon)$ for small $\epsilon$
  such that $\lambda(\epsilon) \in \sigma(\mata_\epsilon)$ and with
  its left and right eigenvectors $\vu$, and $\vv$, respectively, there
  holds
  \begin{gather}
    \abs*{\tfrac{d}{d\epsilon} \lambda(0)}
    \le \norm{E}_2\frac{\norm{\vu}_2\norm{\vv}_2}{\abs{\scal(\vu,\vv)}}
    = \norm{E}_2 \frac1{\cos\angle(\vu,\vv)}.
  \end{gather}
\end{Theorem}

\begin{Problem}{almost-parallel}
  Consider the matrix 
  \begin{gather} 
    M = 
    \begin{pmatrix}
      \eta & 1\\  \eta &\eta
    \end{pmatrix}
   \end{gather}
   with $|\eta| << 1$.\\
  Explain why the problem of finding eigenvectors is \textit{not} well-posed in this example.
\end{Problem}

\subsection{Conditioning of eigenvectors and eigenspaces}

\begin{intro}
  Positive results on the conditioning of eigenvectors require
  additional tools which go beyond the exposition planned for this
  class. We will thus only discuss this question at hand of an example
  and conclude a rule of thumb.
\end{intro}

\begin{Example}{conditioning-eigenvectors}
  Consider the two matrices
  \begin{gather}
    A =
    \begin{pmatrix}
      1-\epsilon & 0\\ 0 & 1+\epsilon
    \end{pmatrix},
    \qquad
    B =
    \begin{pmatrix}
      1&\epsilon\\\epsilon&1
    \end{pmatrix}.
  \end{gather}
  Their eigenvalues are $1-\epsilon$ and $1+\epsilon$, but their
  eigenvectors differ by an angle of $\pi/4$ independent of
  $\epsilon$.
\end{Example}

\begin{Remark}{conditioning-eigenvectors}
  The problem of finding eigenvectors for tight clusters of
  eigenvalues is ill-posed. Nevertheless, finding the invariant
  subspace associated to all eigenvalues in such a cluster is
  well-posed.

  Conditioning of the eigenvector problem depends on the separation of
  eigenvalues.
\end{Remark}

%%% Local Variables:
%%% mode: latex
%%% TeX-master: "main"
%%% End:


\section{Vector iterations}

\section{Simultaneous vector iterations and the QR method}

%\section{Polynomräume}

% \begin{Satz}{nullstellen}
%   Ein reelles Polynom vom Grad $n$ hat höchstens $n$ Nullstellen oder es ist das Nullpolynom.
% \end{Satz}

% \begin{proof}
%   Für $n=1$ handelt es sich um ein lineares Polynom und die Aussage
%   des Satzes ist unmittelbar klar. Sei nun $p$ ein Polynom strikt vom
%   Grad $n>1$ mit Nullstelle $x_0$. Dann gibt es nach dem euklidischen
%   Algorithmus zur Division mit Rest ein Polynom $q$ vom Grad $n-1$ und
%   eine Konstante $c$, so dass
%   \begin{gather}
%     p(x) = (x-x_0)q(x)+c.
%   \end{gather}
%   Daraus folgt $p(x_0) = c$, so dass folgt $c=0$. Wir können dieses
%   Verfahren für alle weiteren Nullstellen $x_1,\dots,x_m$ wiederholen
%   und erhalten
%   \begin{gather}
%     p(x) =  r(x) \prod_{k=0}^m (x-x_i),
%   \end{gather}
%   wobei $r(x)$ ein Polynom vom Grad $n-m$ sein muss, da $p$ vom Grad
%   $n$ ist. Insbesondere muss gelten $m\le n$.
% \end{proof}

% \begin{Korollar}{polynome-identisch}
%   Zwei reelle Polynome vom Grad $n$ sind identisch, wenn sie in
%   mindestens $n+1$ Punkten übereinstimmen. 
% \end{Korollar}


\begin{Lemma}{monome-linear-unabhaengig}
  Die Menge der Monome $\{x^0, x^1,\dots,x^n\}$ ist linear unabhängig.
\end{Lemma}

\begin{proof}
  Sei $p$ ein Polynom vom Grad $n$, also
  \begin{gather}
     p(x) = a_nx^n+a_{n-1}x^{n-1}+\dots+a_1x+a_0
   \end{gather}
   $p$ ist also gerade eine Linearkombination der Monome.  Zu zeigen
   ist, dass aus der Eigenschaft $p \equiv 0$ folgt, dass alle
   Koeffizienten verschwinden, also
  \begin{gather}
    p(x) \equiv 0
    \quad\Rightarrow\quad a_n = \dots = a_0 = 0.
  \end{gather}
  Zu diesem Zweck berechnen wir die $n$-te Ableitung von $p$ und
  erhalten, da mit $p$ auch alle seine Ableitungen identisch
  verschwinden,
  \begin{gather}
    n! a_n = 0.
  \end{gather}
  Daraus schließen wir $a_n = 0$. Nun gilt für die $(n-1)$-te Ableitung
  \begin{gather}
    n! a_n x + (n-1)! a_{n-1} = (n-1)! a_{n-1} = 0.
  \end{gather}
  Auf diese Weise schließen wir rekursiv bis $a_0$, dass alle Koeffizienten verschwinden. Damit ist das Lemma bewiesen.
\end{proof}

\begin{Satz}{polynomraum}
  Die Polynome vom maximalen Grad $n$ bilden einen Vektorraum der
  Dimension $n+1$.  Wir bezeichnen ihn mit $\P_n$.
\end{Satz}

\begin{proof}
  Es ist leicht nachzurechnen, dass sowohl die Summe, als auch reelle
  Vielfache von Polynomen wieder Polynome sind. Insbesondere erhöhen
  beide Operationen den Grad nicht. Damit ist $\P_n$ ein
  Vektorraum. Er wird per definitionem von den Monomen vom Grad bis zu
  $n$ erzeugt. Da diese nach
  \slideref{Lemma}{monome-linear-unabhaengig} linear unabhängig sind,
  bilden sie eine Basis und die Dimension von $\P_n$ ist $n+1$.
\end{proof}

\begin{Quiz}{Polynomräume}
  Gegeben beliebige Werte $x_j\in\R$ mit $j=1,\dots,n$. Die Menge der
  Polynome $p_i$ definiert durch
  \begin{align*}
    p_0(x) &= 1\\
    p_i(x) &= \prod_{j=1}^i (x-x_j),\qquad i=1,\dots,n
  \end{align*}
  \begin{enumerate}[A]
  \item ist linear unabhängig
  \item ist linear abhängig
  \item ist ein Erzeugendensystem für $\P_n$
  \item ist eine Basis von $\P_n$
  \end{enumerate}
\end{Quiz}
\section{Skalarprodukt und Orthogonalität}
\begin{Definition}{skalarprodukt}
  Sei $V$ ein reeller Vektorraum. Eine Abbildung
  $a\colon V \times V \to \R$ heißt \define{Bilinearform}, wenn für
  $u,v,w\in V$ und $\lambda,\mu\in \R$ gilt
  \begin{align}
    a(\lambda u + \mu v,w) &= \lambda a(u,w) + \mu a(v,w)\\
    a(w,\lambda u + \mu v) &= \lambda (w,u) + \mu a(w,v).
  \end{align}
  Eine Bilinearform heißt \define{symmetrisch}, wenn für $u,v\in V$ gilt
  \begin{gather}
    a(u,v) = a(v,u).
  \end{gather}
  Sie heißt \define{positiv semi-definit}, wenn $a(u,u) \ge 0$ für alle
  $u\in V$ und \define{positiv definit}, wenn zusätzlich
  \begin{gather}
    a(u,u) = 0 \quad \Longrightarrow \quad u=0.
  \end{gather}
  Eine symmetrische, positiv definite Bilinearform heißt
  \define{Skalarprodukt}, in der Regel notiert als $\scal(\cdot,\cdot)$.
\end{Definition}

%%%%%%%%%%%%%%%%%%%%%%%%%%%%%%%%%%%%%%%%%%%%%%%%%%%%%%%%%%%%%%%%%%%%%%
\begin{Lemma*}{bcs}{Bunjakowski-Cauchy-Schwarzsche Ungleichung}
  Sei $\scal(\cdot,\cdot)$ ein Skalarprodukt auf $V$.  Für zwei beliebige Elemente $u,v\in V$ gilt
  \begin{gather}
    \abs{\scal(u,v)} \le \sqrt{\scal(u,u)} \, \sqrt{\scal(v,v)}.
  \end{gather}
  Gleichheit gilt genau dann, wenn $u$ und $v$ kollinear sind, also
  $v=\alpha u$ mit einem skalaren Faktor $\alpha$.
\end{Lemma*}

\begin{proof}
  Zunächst zeigen wir nur die Ungleichung: Für $v=0\in V$ ist sie
  offensichtlich.
  
  Seien nun $v,u \in V$ keine Nullvektoren. Für beliebige $\mu, \lambda \in \R$
  gilt wegen der Bilinearität 
  \begin{gather}
   0 \le \scal(\lambda u + \mu v,\lambda u +  \mu v)
    = \lambda^{2} \scal(u,u)+2 \mu \lambda \scal(u,v) +\mu^{2} \scal(v,v)
  \end{gather}
  Setze $\lambda := \scal(v,v) \neq 0$
  \begin{gather}
   0 \le \scal(v,v)^{2} \scal(u,u) + 2\mu \scal(v,v)\scal(u,v) +\mu^{2}\scal(v,v)
  \end{gather}
  Dividiere durch$\scal(v,v)$
  \begin{gather}
   0 \le \scal(v,v) \scal(u,u) + 2\mu \scal(u,v) +\mu^{2}
  \end{gather}
  Setze nun $\mu := -\scal(u,v)$
  \begin{gather}
    0 \le \scal(v,v) \scal(u,u) -2\scal(u,v)^{2}+\scal(u,v)^{2}
  \end{gather}
  Daraus folgt
  \begin{gather}
    \scal(u,v)^{2} \le \scal(u,u) \scal(v,v)
  \end{gather}
  und mit der Monotonie der Quadratfunktion die Ungleichung.

  Nun bleibt die Äquivalenz für die Gleichheit zu zeigen.
  Für $v=0$ ist dies wieder trivial erfüllt. Seien zunächst $u,v$ linear abhängig, also zum Beispiel $u=av$.
  Dann gilt mit der Abkürzung $f(v) = \sqrt{\scal(v,v)}$
  \begin{gather}
    \abs{\scal(u,v)} = \abs{\scal(av,v)}
    = \abs{a} \cdot f(v) \cdot f(v)
    = f(av) \cdot f(v) =f(u) \cdot f(v).
  \end{gather}

  Gelte nun umgekehrt $\scal(u,v) = \sqrt{\scal(u,u)}\sqrt{\scal(v,v)}$.
  Es folgt
  \begin{gather}
     \scal(v,v) \scal(u,u) -2\scal(u,v)^{2}+\scal(u,v)^{2} = 0.
  \end{gather}
  Setze $\mu = \scal(u,v)\neq 0 $ und
  $\lambda = \scal(v,v)\neq 0$. Dann erhält man
  \begin{gather}
    \lambda \scal(u,u) - 2 \mu \scal(u,v) + \mu^2 = 0.
  \end{gather}
  Multipliplikation mit $\scal(v,v)$ ergibt
  \begin{gather}
   \lambda^2 \scal(u,u)+2\mu \scal(u,v)\scal(v,v) +\mu^{2}\scal(v,v) = 0 = \scal(\lambda u-\mu v,\lambda u-\mu v).
  \end{gather}
  
  Wegen der Definitheit folgt nun
  $\lambda u + \mu v = 0$ und da $\mu$ und $\lambda$ ungleich Null sind gilt,
  dass $ u,v$ linear abhängig sind
\end{proof}

%%%%%%%%%%%%%%%%%%%%%%%%%%%%%%%%%%%%%%%%%%%%%%%%%%%%%%%%%%%%%%%%%%%%%%
\begin{Lemma}{hilbertnorm}
  Sei $V$ ein reeller Vektorraum mit Skalarprodukt
  $\scal(\cdot,\cdot)$. Dann ist durch
  \begin{gather}
    \norm{u} = \sqrt{\scal(u,u)}
  \end{gather}
  auf $V$ eine Norm definiert. Ein reeller Vektorraum $V$ mit
  Skalarprodukt und zugehöriger Norm heißt \define{euklidischer
    Vektorraum}.
\end{Lemma}

\begin{proof}
  Das Skalarprodukt ist nicht negativ, daher ist die Abbildung $\norm{\cdot}\colon V \to \R$ wohldefiniert.
  Wir müssen nun die Normeigenschaften nachrechnen. Sei dazu $u \in V$. Es gilt
  \begin{enumerate}
  \item Nichtnegativität und Definitheit folgen sofort aus den entsprechenden Eigenschaften des Skalarprodukts.
  \item Homogenität
  \begin{gather}
    \norm{\lambda u} = \sqrt{\scal(\lambda u,\lambda u)}
    =\sqrt{\lambda^{2}\scal(u,u)}
    = \abs{\lambda}\sqrt{\scal(u,u)}
    =\abs{\lambda}\norm{u}
  \end{gather}
  \item Deiecksungleichung
  \begin{gather}
    \begin{aligned}
      \norm{u+v}^{2}
      &= \scal(u+v,u+v)\\
      &= \scal(u,u)+ 2 \scal(u,v) + \scal(v,v)\\
      \label{eq:orthopoly:1}
      &\le \scal(u,u)+ 2 \norm{u} \, \norm{v} + \scal(v,v)\\
      &=\norm{u}^{2}+ 2 \norm{u} \, \norm{v}+ \norm{v}^{2}\\
      & =(\norm{u}+\norm{v})^{2}\\
    \end{aligned}
  \end{gather}
  Daraus folgt durch Wurzelziehen auf beiden Seiten $\norm{u+v} \le \norm{u}+\norm{v}$.
  Für die Abschätzung in Zeile~\eqref{eq:orthopoly:1} haben wir die
  Bunyakovsky-Cauchy-Schwarz-Ungleichung aus \slideref{Lemma}{bcs} verwendet.
  \end{enumerate}
\end{proof}

\begin{Lemma*}{l2-norm}{$L^2$-Skalarprodukt}
  Auf dem Raum $V=\P_n$ der reellen Polynome vom Grad bis zu $n$ ist durch
  \begin{gather}
    \scal(p,q) = \int_{-1}^1 p(x)q(x)\dx
  \end{gather}
  ein Skalarprodukt definiert. Dieses wird $L^2$ Skalarprodukt genannt.
\end{Lemma*}

\begin{proof}
  Hier gilt es zu prüfen, ob die Abbildung auch die vier Eigenschaften eines
  Skalarprodukts erfüllt.\\
  Seien  $p,q,g \in \P_n$ in diesem Beweis.\\
  Da wir schon von einem Skalarprodukt ausgehen, empfiehlt es sich
  zuerst die Symmetrie zu zeigen.
  \begin{gather}
    \scal(p,q) =  \int_{-1}^1 p(x)q(x)\dx = \int_{-1}^1 q(x)p(x)\dx
    =\scal(q,p)
  \end{gather}
  Wenn wir nun zeigen, dass es eine Bilinearform ist müssen wir nur noch eine
  Identität zeigen, da wir schon wissen, dass die Symmetrieeigenschaft
  erfüllt ist.
  \begin{gather}
    \begin{aligned}
    \scal(\lambda p + \mu q, g)
    &= \int_{-1}^1 (\lambda p(x)+ \mu q(x))g(x)\dx\\
   & = \int_{-1}^1 \lambda p(x)g(x)+ \mu q(x)g(x)\dx \\
   &= \int_{-1}^1 \lambda  p(x)g(x)\dx + \int_{-1}^1 \mu q(x)g(x)\dx \\
   &= \lambda \int_{-1}^1 p(x)g(x)\dx + \mu  \int_{-1}^1 q(x)g(x)\dx \\
   &= \lambda \scal(p,g) + \mu \scal(q,g)
    \end{aligned}
  \end{gather}
  Da wir die Symmetrie vorher gezeigt haben, gilt Linearität auch
  im zweiten Argument.\\
  
  Als letztes zeigen wir, dass die Abbildung positiv definit ist.
  \begin{gather}
    0 = \scal(p,p) = \int_{-1}^1 p(x)p(x)\dx =\int_{-1}^1 p(x)^{2}\dx
  \end{gather}
  Aus den Integraleigenschaften folgt
  \begin {gather}
    0 = p(x)^{2} \quad \forall x
  \end{gather}
  Dies kann nur der Fall sein, wenn $p \equiv 0$ ist.\\
  Somit haben wir nachgerechnet, dass es sich um Skalarprodukt handelt.
  \end{proof}

\begin{Definition}{l2-norm}
  Nach dem \slideref{Lemma}{hilbertnorm} können wir mit diesem Skalarprodukt eine Norm auf $\P_n$
  definierten. Diese Norm wird als die $L^2$ Norm bezeichnet.
  \begin{gather}
    \norm{f}_{L^2} = \sqrt{\scal(f,f)_{L^2}} = \int_{-1}^1 f(x)^2 dx
  \end{gather}
  \end{Definition}

\begin{Definition}{orthogonal}
  Zwei Vektoren $u,v\in V$ heißen \define{orthogonal}, wenn
  \begin{gather}
    \scal(u,v) = 0.
  \end{gather}
  Ein Vektor $u\in V$ ist orthogonal zum Untervektorraum $W\subset V$, wenn
  \begin{gather}
    \scal(u,v) = 0\quad\forall v\in W.
  \end{gather}
\end{Definition}

\begin{Notation}{euklidischer-vr}
  Von nun an bezeichnet $V$ immer einen endlichdimensionalen, reellen,
  euklidischen Vektorraum.
\end{Notation}

\begin{Lemma*}{pythagoras}{Pythagoras}
  Seien zwei Vektoren $u\in V$ und $v\in V$ orthogonal zueinander. Dann gilt
  \begin{gather}
    \norm{u+v}^{2} = \norm{u}^{2} + \norm{v}^{2}
  \end{gather}
\end{Lemma*}

\begin{proof}
  Seien $u,v \in V$. Es gilt $ 0 = \scal(u,v)$
   \begin{gather}
    \norm{u+v}^{2} = \scal(u+v,u+v)
    %=\scal(u+v,u)+\scal(u+v,v)
    %=\scal(u,u)+\scal(v,u)+\scal(u,v)+\scal(v,v)
    =\norm{u}^{2} + \norm{v}^{2} +2\scal(u,v) = \norm{u}^{2} + \norm{v}^{2}
  \end{gather}
\end{proof}

\section{Bestapproximation und orthogonale Projektion}
\begin{Definition}{bestapproximation}
  Sei $A\subset V$ ein affiner Unterraum eines euklidischen
  Vektorraums. Dann ist die Bestapproximation $u_b\in A$ eines Vektors
  $u\in V$ in $A$ definiert durch die Beziehung
  \begin{gather}
    \norm{u-u_b} = \min_{v\in A} \norm{u-v}.
  \end{gather}
\end{Definition}

\begin{Satz}{bestapproximation}
  Sei $w \in V$ und $W \subset V$.
  Sei $A=w+W$ ein nichtleerer, affiner Unterraum von $V$. Dann
  existiert die Bestapproximation nach
  \slideref{Definition}{bestapproximation} und ist eindeutig
  bestimmt. Es gilt die notwendige und hinreichende Bedingung
  \begin{gather}
    \scal(u-u_b,v) = 0 \quad \forall v\in W.
  \end{gather}
  Das heißt $ u_b$ ist Bestapproximation genau dann wenn $u-u_b$
  orthogonal zu $W$ bzgl. des Skalarprodukts $\scal(\cdot,\cdot)$ ist.
\end{Satz}

\begin{proof}
  Der Beweis gliedert sich in drei Teile. Zuert wird die Äquivalenz
  gezeigt danach zeigen wir die Eindeutigkeit und zum Schluss
  erst die Existenz.\\ \\
 $\glqq \Rightarrow \glqq$
  Sei $ u_b \in A$ die Bestapproximation des Vektors $ u \in V$\\
  Wir defnieren nun eine Funktion:
  \begin{gather}
    F_v(x):= \norm{u-u_b-xv}^{2}, x \in \R,  v\in A
  \end{gather}

  Diese Funktion besitzt ein Minimum bei x=0. Folglich gilt
  \begin{gather}
    \left. \frac{d}{dx} F(x) \right|_{x=0}
    =\left. \frac{d}{dx}\norm{u-u_b-xv}^{2} \right|_{x=0}=0
  \end{gather}
    
  Dies kann weiter umgeformt werden zu
  $\scal(u-u_b-xv,v)|_{x=0}=0 \ \forall v\in A$ und folglich zu
  \begin{gather}
   \scal(u-u_b,v)=0 \ \forall v\in A
  \end{gather}

  $\grqq \Leftarrow \glqq$
  Nun erfüllt $u_b\in A$ die Bedingung.\\
  Dann gilt mit einem beliebigen $v\in A$:
  \begin{gather}
   \norm{u-u_b}^{2}=\scal(u-u_b,u-u_b)\\
   = \scal(u-u_b,u-v)+\scal(u-u_b,v-u_b)\\
   \le \norm{u-u_b}\cdot\norm{u-v}\\
  \end{gather}
  Daraus folgt $\norm{u-u_b} \le \inf_{v\in A}{\norm{u-v}}$\\
  Damit erfüllt $u_b$ eben die Definiton der Bestapproximation\\ \\
  Nun zur Eindeutigkeit:\\
  Seien $u_b$ und $u_d \in$ A zwei Bestapproximationen.
  Dann gilt notwendigerweise
  \begin{gather}
   \scal(u-u_b,v) = 0 = \scal(u-u_d,v) \quad \forall v\in A
  \end{gather}
  Dies wird umgeformt zu
  \begin{gather}
  \scal(u-u_d,v)-\scal(u-u_b,v)=0 \quad  \forall v\in A \\
  \scal(u_b-u_d,v) = 0 \quad \forall v \in A
  \end{gather}
  Wähle nun $v:=u_b-u_d \in A$. Dies ergibt
  $\norm{u_b-u_d}^{2} =0$ und somit folgt $u_b = u_d$\\
  Die Existenz:\\
  Der endliche dimensionale Teilraum A$\subseteq$V besitzt eine Basis
  $(b_1,\dots, b_n)$ mit $n:=dim V$. Die gesuchte Approximation
  $u_b\in A$ lässt sich
  durch die Basis in folgender Form darstellen
  \begin{gather}
   u_b = \sum_{k=1}^n a_k b_k
  \end{gather}
  Dies wird in die notwendige Orthogonalitätsbedingung
  \slideref{Satz}{bestapproximation} eingesetzt.
  \begin{gather}
   \scal(u-\sum_{k=1}^n a_k b_k,v)=\scal(u,v)-\sum_{k=1}^n a_k\scal(b_k,v)=0
   \quad \forall v\in A
   \end{gather}
 Dies ist bei der Wahl von $v:=b_i \quad i=1,\dots,n$ äquivalent zu dem
 linearen $n$x$n$ Gleichungssystem.
 \begin{gather}
   \sum_{k=1}^n\scal(b_k,b_i) a_k= \scal(u,b_i) \quad i=1,\dots,n
 \end{gather}
 Definiere nun $A,x,b$ wie folgt
 \begin{gather}
  A:=(\scal(b_k,b_i))_{i,k=1}^n \quad x:=(a_k)_{k=1}^n\quad b:=(\scal(u,b_i))_{i=1}^n
 \end{gather}
 Dadurch lässt sich das LGS in der Form $Ax=b$ schreiben.
 Betrachte nun folgendes
 \begin{gather}
  x^{T}Ax =\sum_{i,k=1}^n a_i a_k\scal(b_k,b_i)=\scal(u_b,u_b)\ge 0
 \end{gather}
 $A$ ist folglich positiv definit. Das Gleichungssystem $Ax=b$ ist also für
 jede rechte Seite $b$, das heißt für jedes $u \in V$ eindeutig lösbar.
 Folglich bestimmt die Orthogonalitätsbedingung eindeutig ein Element
 $u_b \in A$, welches dann die Bestapproximation von $u$ ist.
 
\end{proof}

\begin{Definition}{komplement-projektion}
  Sei $W \subset V$ ein Untervektorraum. Dann gilt
  $V = W \oplus W^\perp$, wobei das \define{orthogonale Komplement}
  $W^\perp$ eindeutig definiert ist durch
  \begin{gather}
    W^\perp = \bigl\{ v\in V \big| \scal(v,w) = 0 \quad\forall w\in W\bigr\}.
  \end{gather}
  Die Lösung der Bestapproximationsaufgabe bezeichnen wir mit
  \begin{gather}
    \Pi_W u = u_b\in W
  \end{gather}
  und nennen es die \define{orthogonale Projektion} von $u\in V$ auf $W$.
\end{Definition}

\begin{Lemma}{komp-projekt-wohldefiniert}
  Das orthogonale Komplement und die orthogonale Projektion sind wohldefiniert.
\end{Lemma}

\begin{proof}
  \slideref{Satz}{bestapproximation}.
\end{proof}

\begin{Beispiel}{polynom-bestapproximation}
  Die Aufgabe der Gaußschen Ausgleichsrechnung lautet: finde zu einer
  gegebenen Funktion $f$ das Polynom vom Grad höchstens $n$, das auf
  dem Intervall $[-1,1]$ den mittleren quadratischen Abstand
  minimiert, also $p\in \P_n$ mit
  \begin{gather}
    \int_{-1}^1 \bigl(f(x)-p(x)\bigr)^2 \dx
    = \min_{q\in \P_n} \int_{-1}^1 \bigl(f(x)-q(x)\bigr)^2 \dx.
  \end{gather}
  Die Lösung erfüllt
  \begin{gather}
    \int_{-1}^1 p(x)q(x) \dx = \int_{-1}^1 f(x)q(x) \dx
    \qquad\forall q\in \P_n.
  \end{gather}
\end{Beispiel}

\begin{remark}
  Mit unserem Wissen über die $L^2$ Norm aus \slideref{Definition}{l2-norm} erkennen wir, dass sich
  die Gaußsche Ausgleichsrechnung auch über die Norm formulieren lässt.
  \begin{gather}
    \norm{f-p}_{L^2}^2 = \min_{q\in \P_n} \norm{f-q}_{L^2}^2
  \end{gather}
\end{remark}

\section{Orthogonale Basen}

\begin{Lemma}{gram-system}
  Wählt man eine Basis $\{\phi_i\}$ für $W$, so transformiert wird die
  Orthogonalitätsbedingung in \slideref{Satz}{bestapproximation} zum
  linearen Gleichungssystem
  \begin{gather}
    \matg\vx = \vb.
  \end{gather}
  Hier sind $\vx$ der Koeffizientenvektor der Lösung $u_b$, $\matg$ die
  \define{Gramsche Matrix} und $\vb$ die rechte Seite gegeben durch
\begin{gather}
  g_{ij} = \scal(\phi_i,\phi_j), \qquad
  b_i = \scal(u,\phi_i).
\end{gather}
\end{Lemma}

\begin{remark}
  Das Gleichungssystem hängt nur von der Wahl einer Basis in $W$ ab,
  nicht in $V$.
\end{remark}

\begin{Definition}{ortho-system}
  Eine Menge von Vektoren $\{\phi_1,\dots,\phi_n\}\subset V$ bildet
  ein \define{Orthogonalsystem}, wenn
  \begin{gather*}
    \scal(\phi_i,\phi_j) = 0
    \qquad \forall 1\le i < j \le n.
  \end{gather*}
  Sie ist ein \define{Orthonormalsystem}, wenn zusätzlich
  $\norm{\phi_i} = 1$ für alle Elemente gilt. Ein Orthonormalsystem, das eine Basis bildet, heißt \define{Orthonormalbasis} (\define{ONB}).
\end{Definition}

\begin{Lemma}{ortho-lu}
  Jedes Orthogonalsystem ist linear unabhängig.
\end{Lemma}

\begin{Lemma*}{parseval}{Parsevalsche Gleichung}
  Sei $\{\phi_i\}$ für $i=1,\dots,n$ eine ONB von $V$. dann gilt für
  jedes $v\in V$ mit der Basisdarstellung
  \begin{gather}
    v = \sum_{i=1}^n x_i \phi_i
  \end{gather}
  die Identität
  \begin{gather}
    \norm{v}^2 = \sum_{i=1}^n x_i^2.
  \end{gather}
\end{Lemma*}
\begin{Lemma}{least-squares-orthogonal}
  Bezüglich einer ONB ist die Gramsche Matrix die
  Einheitsmatrix. Damit berechnen sich die Einträge des
  Koeffizientenvektors $\vx$ in \slideref{Lemma}{gram-system} durch
  die einfache Formel
  \begin{gather}
    x_i = b_i = \scal(u,\phi_i).
  \end{gather}
\end{Lemma}

\begin{Theorem*}{gram-schmidt}{Gram-Schmidt-Verfahren}
  Jede linear unabhängige Menge von Vektoren
  $\{v_1,\dots,v_n\}\subset V$ wird mit dem folgenden Verfahren in ein
  Orthonormalsystem $\{\phi_1,\dots,\phi_n\}\subset V$ umgeformt:
  \begin{gather}
    \begin{aligned}
      \phi_1 &= \tfrac1{\norm{v_1}} \,v_1\\
      w_j &= v_j - \sum_{i=1}^{j-1} \scal(v_j,\phi_i)\,\phi_i
      & \quad \phi_j &= \tfrac1{\norm{w_j}}\, w_j
      &\quad j=2,\dots,n
    \end{aligned}
  \end{gather}
  Für alle $1\le k \le n$ gilt
  \begin{gather}
    \operatorname{span}\{\phi_1,\dots,\phi_k\}
    =
    \operatorname{span}\{v_1,\dots,v_k\}
  \end{gather}
\end{Theorem*}

\begin{proof}
  Per Induktion über $n$ zeigen wir Orthogonalität und Normierung.\\

  $Indukionsanfang$ Sei $n=1$.\\
  Wird nur ein Vektor aus dem Raum gewählt, so erfüllt dieser
  die Orthogonalitätsbedingung, da er der einzige Vektor im System ist.
  Wird dieser Vektor zusätzlich normiert erhält man ein Orthonormalsystem.\\
  
  $Induktionsschritt$ Das Verfahren gelte
  für $\{v_1,\dots,v_{n-1}\}$ Vektoren aus V. \\
  $n-1 \rightarrow n$\\
  Sei $(\phi_1,\dots,\phi_{n-1})$ ein Orthonormalsystem\\
  Annahme $w_n$ nicht wohldefiniert. Dann gilt
  \begin{gather}
    w_n = v_n -\sum_{i=1}^{n-1}\scal(v_n,\phi_i)\,\phi_i = 0
  \end{gather}
  In diesem Fall sind $(v_1,\dots,v_n)$ linear abhängig.
  Das ist ein Widerspruch zur Voraussetzung,
  dass $(v_1,\dots,v_n)$ linear unabhängig sind.\\
  $w_n$ wird nun normiert über $\frac{1}{\norm{w_n}} \cdot w_n =\phi_n $.
  Nun zur Orthogonalität:
  \begin{gather}
    \scal(\phi_n,\phi_j)=\scal(v_n,v_j)-
    \sum_{i=1}^{n-1}\scal(v_n,\phi_i)
    \,\underbrace{\scal(\phi_i,\phi_j)}_{=\delta_{ij}}  = 0
    \quad j=1,\dots,n-1
  \end{gather}
\end{proof}

\begin{Algorithmus*}{gram-schmidt}{Gram-Schmidt}
  \lstinputlisting{code/gram-schmidt.py}
\end{Algorithmus*}

\begin{remark}
  Um den Code zu verstehen ist es ratsam ihn zu Beginn einmal durchzugehen und sich Stellen zu
  merke an denen Zuweisungen getätigt werden. Ebenso sollte der Code mit einem einfachen Beispiel
  probiert werden, um die Parallelen zum Verfahren besser zu erkennen.\\  
   1 Es wird eine Funktion mit dem Namen $gram schmidt$. Dieser Funktion wird eine Matrix $v$
      übergeben. In dieser Matrix stehen die Vektoren $v_1$ bis $v_n$ in Spalten. \\ 
   2 Es wird $n$ die Länge der Zeilen(Anzahl der Vektoren) zugewiesen und
      $m$ wird die Länge der Spalten (Anzahl der Einträge im Vektor) zugewiesen. \\
   3 Beginn des GS Verfahrens. Es geht von Vektor 1 bis Vektor $n$ \\
   4 Initialsieren eines Vektors $delta$ der Länge $m$ mit Nullen als Einträge\\
   5 Es wird eine weitere Schleife begonnen in der ein Index i über alle bisher orthogonalisierten
      Vektoren läuft. Dies entspricht der Summe aus dem Verfahren. \\
   6 $r$ ist das Skalarprodukt aus dem Vektor $v_j$ und einem bereits orthogonalisierten Vektor.
      Die Vektoren befinden sich in der Matrix $v$ und über diesen Befehl wird darauf zugegriffen.\\
   7 delta = delta $+$ Skalarprodukt $*$ dem orthogonalen Vektor\\  
   8 Hier wird die zweite for-Schleife wieder verlassen! Es reicht tatsächlich das Einrücken.
      Die Summe wird vom Vektor $v_j$ abgezogen, somit $v_j$ orthogonalisiert und wieder in der
      Matrix $v$ an der richtigen Stelle zugewiesen.  \\
   9  Die Norm von $v_j$ wird berechnet.\\  
   10 Wie im Verfahren wird in  dieser Zeile $v_j$ normiert und in der Matrix $v$ an der Stelle des
       früheren $v_j$ zugewiesen.
\end{remark}

\begin{Beispiel}{gram-schmidt}
  Wir wählen für Polynome das $L^2$-Skalarprodukt aus
  \slideref{Lemma}{l2-norm} und die Basis $\{1,x,\dots,x^{n-1}\}$
  für $\P_{n-1}$. Wir verwenden die Iplementation in
  \slideref{Algorithmus}{gram-schmidt} und messen den Erfolg nach der
  Größe der Nebendiagonaleinträge der Gramschen Matrix.
  \begin{center}
    \begin{tabular}{c|c}
      $n$ & $\max_{i\neq j} \abs{g_{ij}}$ \\
      \hline
      5 & $8.9\cdot 10^{-16}$ \\
      10 & $9.1\cdot 10^{-12}$ \\
      15 & $1.2\cdot 10^{-7}$ \\
      20 & $0.23$
    \end{tabular}
  \end{center}
\end{Beispiel}

\begin{Algorithmus*}{mgs}{Modifizierter Gram-Schmidt}
  \lstinputlisting{code/modified-gram-schmidt.py}  
\end{Algorithmus*}

\begin{remark}
  In diesem Programm wurde der Zwischenschritt über das delta ausgelassen, was mögliche Rundungsfehler
  verringert.\\
  4 Hier wird direkt $r$ mit Nullen initialisiert.\\
  7 Der Vektor $v_j$ wird hier orthogonalisiert, ohne dass die Summe aus dem Verfahren in delta
    zwischengespeichert wird und direkt wieder an entsprechender Stelle in der
    Matrix $v$ zugewiesen.\\
\end{remark}

\begin{Beispiel}{gs-mgs}
  In dieser Tabelle wiederholen wir die Zahlen
  $\max_{i\neq j} \abs{g_{ij}}$ aus \slideref{Beispiel}{gram-schmidt}
  und stellen sie den entsprechenden Ergebnissen des modifizierten
  Verfahrens in \slideref{Algorithmus}{mgs} gegenüber.
  \begin{center}
    \begin{tabular}{c|cc}
      $n$ &  Gram-Schmidt & modifiziert\\
      \hline
      5 & $8.9\cdot 10^{-16}$ & $1.3\cdot 10^{-16}$ \\
      10 & $9.1\cdot 10^{-12}$ & $2.9\cdot 10^{-12}$ \\
      15 & $1.2\cdot 10^{-7}$ & $2.7\cdot 10^{-9}$ \\
      20 & $0.23$ & $3.9\cdot 10^{-5}$
    \end{tabular}
  \end{center}
\end{Beispiel}

\begin{remark}
  Wir sehen, dass die Wahl der Implementation eines Rechenverfahrens
  bei mathematischer Äquivalenz durchaus erheblichen Einfluss auf das
  Ergebnis haben kann. Dieses Phänomen werden wir in
  \Cref{sec:stability} näher untersuchen. Zunächst diskutieren wir
  aber eine weitere Variante der Erzeugung orthogonaler Basen in
  Polynomräumen.
\end{remark}

\section{Drei-Term-Rekursion}

\begin{Satz*}{dreiterm}{Dreiterm-Rekursion}
  Zu jedem Skalarprodukt $\scal(\cdot,\cdot)$ auf dem Raum der
  stetigen Funktionen gibt es genau eine Folge von orthogonalen
  Polynomen $p_k\in \P_k$ mit führendem Koeffizienten eins. Sie
  genügen der Dreiterm-Rekursionsformel
  \begin{gather}
    p_k(x) = (x-a_k)p_{k-1}(x) - b_k p_{k-2}(x),
    \qquad k=1,2,\ldots
  \end{gather}
  mit Startwerten $p_{-1} \equiv 0$ und $p_0 \equiv 1$. Die
  Koeffizienten sind
  \begin{gather}
    a_k = \frac{\scal(x p_{k-1},p_{k-1})}{\scal(p_{k-1},p_{k-1})}
    \qquad\text{und}\qquad
    b_k = \frac{\scal(p_{k-1},p_{k-1})}{\scal(p_{k-2},p_{k-2})}.
  \end{gather}
\end{Satz*}

\begin{proof}
  Siehe \cite[Satz 6.2]{DeuflhardHohmann08}
\end{proof}

\begin{Bemerkung}{dreiterm-normierung}
  Der Beweis ergibt, eigentlich die ``Eindeutigkeit einer Orthogonalfolge bis auf Normierung''. Tatsächlich werden in der Literatur immer wieder veschiedene Normierungen benutzt. Beispiele sind:
  \begin{enumerate}
  \item Führender Koeffizient eins, $p_k = x^k + \dots$
  \item $\norm{p_k} = 1$
  \item $p_k(1) = 1$
  \end{enumerate}
\end{Bemerkung}

\begin{Definition}{legendre-polynome}
  Die \define{Legendre-Polynome} $L_k$ sind definiert durch
  die Dreiterm-Rekursion
  \begin{gather}
    L_{k} = \tfrac{2k-1}{k}x L_{k-1}(x) - \tfrac{k-1}{k} L_{k-2}(x).
  \end{gather}
  Sie sind orthogonal bezüglich des $L^2$-Skalarprodukts in
  \slideref{Lemma}{l2-norm}.
\end{Definition}

\begin{Beispiel}{least-squares-legendre}
  Das Problem der Gaußschen Ausgleichsrechnung war: Zu einer gegebenen
  Funktion $f$ finde $p\in \P_n$, so dass
  \begin{gather}
    \norm{f-q}_{L^2}^2
    = \min_{q\in\P_n} \norm{f-q}_{L^2}^2.
  \end{gather}
  Mit Hilfe der Legendre-Polynome können wir nun die Lösung explizit angeben als
  \begin{gather}
    p(x) = \sum_{i=0}^n \alpha_i L_i(x)
    \qquad\text{mit}\qquad
    \alpha_i = \frac1{\norm{L_i}^2}\int_{-1}^1 f L_i(x)\dx.
  \end{gather}
\end{Beispiel}

\begin{Definition}{chebyshev-polynome}
  Die \define{Tschebyscheff-Polynome} $T_k$ sind definiert durch
  die Dreiterm-Rekursion
  \begin{gather}
    T_{k} = 2x T_{k-1}(x) - T_{k-2}(x).
  \end{gather}
  Sie sind orthogonal bezüglich des Skalarprodukts
  \begin{gather}
    \scal(p,q) = \int_{-1}^1 \tfrac1{\sqrt{1-x^2}} \,p(x)q(x)\dx.
  \end{gather}
\end{Definition}

%%% Local Variables:
%%% mode: latex
%%% TeX-master: "main"
%%% End:


\chapter{Solving Large Sparse Linear Systems}

\chapter{Large Sparse Eigenvalue Problems}

\appendix
\chapter{Basics from Linear Algebra}
\section{Bases and matrices}
\subsection{Matrix notation for bases}
\begin{Notation}{column-vectors}
  For a matrix $\mata\in \C^{n\times k}$ with entries $a_{ij}$, we denote its column vectors by $\va_j$ such that
  \begin{gather}
      (\va_j)_i = a_{ij}.
  \end{gather}
  Vice versa, given vectors $\vv_1,\dots,\vv_k$ in $\C^n$, we denote the matrix generated by those column vectors as
  \begin{gather}
      \matv = \bigl( \vv_1,\dots,\vv_k\bigr).
  \end{gather}
\end{Notation}

\begin{Notation}{matrix-linear-combination}
  Let $\vv_1,\dots,\vv_k$ be a set of vectors in $\C^n$ and let the matrix $\matv=(\vv_1,\dots,\vv_k) \in \C^{n\times k}$. Then, for vectors $\va\in\C^k$ there holds
  \begin{gather}
      \vu = \sum_{i=1}^k a_i \vv_i
      \qquad \Longleftrightarrow \qquad
      \vu = \matv \va.
  \end{gather}
  Thus, we can write linear combinations as matrix vector products.
  We will abuse notation in a way that we refer to a matrix $\matv$
  used in this way as a set of vectors or as a basis.
\end{Notation}

\begin{Lemma*}{change-of-basis}{Change of basis}
  Let $\matv = (\vv_1,\dots, \vv_n)$ and $\matw = (\vw_1,\dots, \vw_n)$ be
  bases of $\C^n$. Let $\vu\in \C^n$ be given by coefficient vectors
  $\vx,\vy\in \C^n$, such that
  \begin{gather}
    \vu = \matv\vx = \matw\vy.
  \end{gather}
  Then,
  \begin{gather}
    \vx = \matc \vy,
  \end{gather}
  where the columns of $\matc\in \C^{n\times n}$ are the coefficient
  vectors of the basis vectors $\vw_j$ with respect to the basis
  vectors $\matv$.
\end{Lemma*}

\begin{Corollary}{change-of-basis}
  Let $\vv\in \C^n$ and $\matw$ be a basis for $\C^n$. Then, the
  coefficient vector $\vy$ of a vector $\vx$ with respect to the
  basis $\matw$ is obtained by
  \begin{gather}
    \vy = \matw^{-1} \vx.
  \end{gather}
\end{Corollary}

%%% Local Variables:
%%% mode: latex
%%% TeX-master: "main"
%%% End:

\subsection{Similarity transformations}
\begin{Definition}{similar-matrix}
  Two matrices $\mata,\matb\in\C^{n\times n}$ are called \define{similar}, if there is a nonsingular matrix $\matv\in\C^{n\times n}$, such that
  \begin{gather}
      \mata = \matv \matb \matv^{-1}.
  \end{gather}
  We call the mapping
  \begin{gather}
      \matb \mapsto \matv \matb \matv^{-1}
  \end{gather}
  a \define{similarity transformation}.
\end{Definition}

\begin{Lemma}{similarity-equivalence}
  Similarity is an equivalence relation, that is, it is reflexive,
  symmetric, and transitive.
\end{Lemma}

\begin{Lemma}{matrix-basis-change}
  Let $\phi\colon \C^n\to \C^n$ be a linear mapping represented with
  respect to the canonical basis by the matrix $\mata$, such that
  \begin{gather}
    \phi(\vx) = \mata \vx.
  \end{gather}
  Then, the matrix $\matb = \matv^{-1}\mata\matv$ represents $\phi$
  with respect to the basis $\matv$, namely, if $\vu = \matv \vy$ and
  $\phi(\vu) = \matv \vz$, then
  \begin{gather}
    \vz = \matb \vy.
  \end{gather}
\end{Lemma}

\begin{Theorem*}{Jordan-canonical-form}{Jordan canonical form}
  Every matrix $A\in\C^{n\times n}$ is similar to a matrix with block structure
  \begin{gather}
    \begin{pmatrix}
      J_1\\&J_2\\&&\ddots\\&&&J_k
    \end{pmatrix},
  \end{gather}
  where each block has the form
  \begin{gather}
    J_i = \begin{pmatrix}
      \lambda_i&1\\&\ddots&\ddots\\
      &&\lambda_i&1\\
      &&&\lambda_i
    \end{pmatrix}.
  \end{gather}
\end{Theorem*}


\begin{Definition}{diagonalizable}
  A matrix is called \define{diagonalizable}, if it is similar to a diagonal matrix.
  Equivalently, a diagonalizable matrix has a basis of eigenvectors.
\end{Definition}

\begin{Theorem}{matrix-functions}
  Let $\mata\in \C^{n\times n}$ be diagonalizable such that
  \begin{gather}
    \mata = \matv \matlambda \matv^{-1},
    \qquad \matlambda = \diag(\lambda_1,\dots,\lambda_n).
  \end{gather}
  Then, for any analytic function $f\colon \C \to \C$, the matrix
  $f(\mata)$ is well defined by
  \begin{gather}
    f(\mata) = \matv f(\matlambda) \matv^{-1},
    \qquad f(\matlambda) = \diag\bigl(f(\lambda_1),\dots,f(\lambda_n)\bigr),
  \end{gather}
  if all eigenvalues are in the domain of convergence of the Tayor
  series of $f$.
\end{Theorem}

\begin{proof}
  First, we observe that
  \begin{gather}
    \mata^2 = \matv \matlambda \matv^{-1} \matv \matlambda \matv^{-1}
    = \matv \matlambda^2 \matv^{-1}.
  \end{gather}
  The square of the diagonal matrix $\matlambda$ is easily computed.
  By induction $\mata^k =  \matv \matlambda^k \matv^{-1}$.

  Let now $f(x) = \sum a_k x^k$ be the Taylor series of $f$. Then,
  \begin{gather}
    \begin{split}
      f(\mata)
      &= \sum_{k=0}^\infty a_k  \matv \matlambda^k \matv^{-1}\\
      &= \matv \left(\sum_{k=0}^\infty a_k \matlambda^k\right) \matv^{-1}\\
      &= \matv \diag\left(
        \sum_{k=0}^\infty a_k \lambda_1^k,\dots,
        \sum_{k=0}^\infty a_k\lambda_n^k
      \right)  \matv^{-1}.
    \end{split}
  \end{gather}
  All limits are well defined since we assume that all eigenvalues are
  in the domain of convergence.
\end{proof}

\begin{Theorem}{simultaneous-diagonalization}
  Two diagonalizable matrices $\mata, \matb\in \C^{n\times n}$ can be
  diagonalized by the same set of eigenvectors if and only if they
  commute. namely $\mata\matb = \matb\mata$.
\end{Theorem}

\begin{todo}
  Move this up next time
\end{todo}

\begin{Lemma}{similarity-eigenvalues}
  The eigenvalues of a matrix are invariant under similarity transformations.
\end{Lemma}

%%% Local Variables:
%%% mode: latex
%%% TeX-master: "main"
%%% End:

\section{Inner products and orthogonality}
\begin{Definition}{sesqui}
  A \define{sesquilinear form} on a complex vector space $V$ is a mapping
  \begin{gather}
      a\colon V\times V \to \C,
  \end{gather}
  which is \define{linear} and \define{semi-linear} in the first and second argument, respectively. That is, for all $\vu,\vv,\vw\in V$ and $\alpha\in\C$ holds
  \begin{xalignat}2
  a(\vu+\vv,\vw) &= a(\vu,\vw)+a(\vv,\vw),
  & a(\alpha \vu,\vw) &= \alpha a(\vu,\vw),\\
  a(\vu,\vv+\vw) &= a(\vu,\vv)+a(\vu,\vw),
  & a(\vu,\alpha \vw) &= \overline\alpha a(\vu,\vw).
  \end{xalignat}
  It is \define{Hermitian} or \define{complex symmetric}, if in addition there holds
  \begin{gather}
      \label{eq:inner:symmetry}
      a(\vu,\vv) = \overline{a(\vv,\vu)}.
  \end{gather}
\end{Definition}

\begin{remark}
  A sesquilinear form can be defined equivalently as being semi-linear in the second argument and linear in the first. Both versions can be found in the literature and we follow the convention in the book of Saad.
  
  The author prefers the term ``complex symmetric'' over the common and pompous ``Hermitian'', since it is the natural extension of symmetry to complex vector spaces, as the following asserts. The author would even prefer the term ``symmetric'', which is unfortunately understood in the real way in other publications. In particular,~\eqref{eq:inner:symmetry} implies that $a(\vu,\vu)$ is always real, such that the following makes sense:
\end{remark}

\begin{Definition}{inner-product}
  An \define{inner product} on the complex vector space $V$ is a sesqui-linear form $\scal(\cdot,\cdot)$ on $V$ which in addition is definite, namely, for all $\vv\in V$ there holds
  \begin{gather}
      \scal(\vv,\vv) \ge 0,
  \end{gather}
  and $\scal(\vv,\vv)=0$ implies $\vv=0$.
\end{Definition}

\begin{Example*}{Euclidean-ip}{Euclidean inner product}
  The bilinear form
  \begin{gather}
    \scal(\vx,\vy) = \sum_{i=1}^n x_i \overline{y}_i
  \end{gather}
  defines an inner product on $\C^n$.
\end{Example*}

\begin{Definition}{conjugate-matrix}
  The conjugate $\mata^*\in C^{n\times m}$ of a matrix $\mata\in C^{m\times n}$
  is defined by the relation
  \begin{gather}
    \scal(\vx,\mata^*\vy)_{\C^n} = \scal(\mata\vx,\vy)_{\C^m},
    \qquad\forall \vx\in\C^n,\;\vy\in\C^m.
  \end{gather}
  The conjugate with respect to the Euclidean inner product is
  $\overline\mata^T$, often abbreviated as $\mata^H$.
\end{Definition}

\begin{Definition}{rothonormal-unitary}
  Given an inner product $\scal(\cdot,\cdot)$ in $\C^n$. A set of
  vectors $\vv_1,\dots,\vv_k$ is called \define{orthonormal}, if
  \begin{gather}
    \scal(\vv_i,\vv_j) = \delta_{ij}, \qquad i,j=1,\dots,k.
  \end{gather}
  A matrix $\matq\in C^{n\times n}$ with orthonormal columns is called
  \define{unitary}. There holds
  \begin{gather}
    \matq^*\matq = \id.
  \end{gather}
\end{Definition}


%%% Local Variables:
%%% mode: latex
%%% TeX-master: "main"
%%% End:

\section{Projections}

\bibliographystyle{alpha}
\bibliography{all}
\printindex

%%% Local Variables:
%%% mode: latex
%%% TeX-master: "main"
%%% End:
