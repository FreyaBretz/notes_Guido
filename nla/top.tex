\usepackage{algpseudocode}
\lstset{language=Python,numbers=left,resetmargins=true,xleftmargin=8pt,basicstyle=\small,numberstyle=\scriptsize}
\usetikzlibrary{svg.path}
\excludecomment{solution}
\tikzset{velox/.style={color=black,draw,fill=red,thick,%
    shape=diamond,aspect=.4,
    inner sep=1.3pt,transform shape}}
\tikzset{veloy/.style={color=black,draw,fill=red,thick,%
    shape=diamond,aspect=2.5,
    inner sep=1.3pt,transform shape}}
\tikzset{veloxy/.style={color=black,draw,fill=red,thick,%
    shape=star,star points=4,star point ratio=2.2,
    inner sep=1.3pt,transform shape}}
\tikzset{pressure/.style={color=black,draw,fill=cyan,thick,%
    shape=circle,inner sep=2pt,transform shape}}
\tikzset{velo/.style={transform shape,double=red,arrows={-Stealth[open,fill=red]}}}

%% Macros for drawing degrees of freedom for different shapes/elements.
%% Arguments are always:
%%   #1: Starting point
%%   #2: End point
%%   #3: polynomial degree
%%   #4: node settings

\tikzset{pics/edgenormal/.style args={#1/#2/#3/#4}{%
    code={%
      \draw #1 -- #2
      node foreach \x [evaluate=\x as \xval] in {1,...,#3} [#4,sloped,pos=\xval/(#3+1)] {};
      }
}}


%% Macros for drawing degrees of freedom for different shapes/elements.
%% Arguments are always:
%%   #1: polynomial degree
%%   #2: node settings

\tikzset{pics/tripile/.style args={#1/#2}{%
    code={%
      \coordinate (top) at (0,#1);
      \foreach \i in{0,...,#1}
      \foreach \j in{0,...,\i}
      {
        \tikzmath{
          \y = .3*(2/3*#1-\i)*cos(30);
          \x = .3*(\i/2-\j);
        }
        \node[#2] at (\x,\y) {};
      }
    }
}}

\tikzset{pics/tensor/.style args={#1/#2/#3}{%
    code={%
      \coordinate (top) at (0,#1);
      \foreach \i in{0,...,#1}
      \foreach \j in{0,...,#2}
      {
        \tikzmath{
          \y = 2*(\i+1)/(#1+2);
          \x = 2*(\j+1)/(#2+2);
        }
        \node[#3] at (\x,\y) {};
      }
    }
}}

\tikzset{pics/pfem/.style args={#1/#2}{%
    code={%
      \tikzmath{ \ytop=2*cos(30); }
      \coordinate (top) at (0,\ytop);

      \foreach \i in{0,...,#1}
      \foreach \j in{0,...,\i}
      {
        \tikzmath{
          \y = \ytop-\ytop*\i/#1;
          \x = 2*(\i/2-\j)/#1+1;
        }
        \node[#2] at (\x,\y) {};
      }
    }
}}

\tikzset{pics/qfem/.style args={#1/#2}{%
    code={%
      \foreach \i in{0,...,#1}
      \foreach \j in{0,...,#1}
      {
        \tikzmath{
          \y = 2-2*\i/#1;
          \x = 2-2*\j/#1;
        }
        \node[#2] at (\x,\y) {};
      }
    }
}}

%%% Local Variables:
%%% mode: latex
%%% TeX-master: "all"
%%% End:


\title{Numerical Linear Algebra}
\author{Guido Kanschat}
\date{\today}

\def\esp#1{V_{#1}}

\begin{document}
\maketitle
\tableofcontents
\chapter{Dense Algebraic Eigenvalue Problems}
\begin{intro}
  We refer to problems in linear algebra which allow storing the
  complete matrix as \define{dense linear algebra}. They are typically
  characterized by dimensions into the hundreds, possibly thousands,
  and any entry in the matrix may have a nonzero value. Such matrices
  are typically stored as a rectangular or quadratic array of numbers,
  and we can perform manipulations based on the matrix entry.

  In contrast, we will turn to \define{sparse linear algebra} in the
  later chapters, where dimensions got up to millions and billions
  ($10^9$). Since currently no computer on earth can store a matrix
  with $10^{18}$ entries, those matrices will be characterized by the
  fact that each row only contains very few nonzero entries, or that
  the matrix is not stored, but only available algorithmically in the
  form of a function performing the action $\vx\mapsto \mata
  \vx$. Thus, access to and manipulation of matrix entries is not
  possible, and we have to focus on methods only using the properties
  of the matrix as a linear mapping.
\end{intro}

\section{Mathematical background}
\subsection{Definition of Eigenvalue Problems}
\begin{intro}
  The following results can be found in any book on linear
  algeba. Thus, we will just keep the arguments short. There will be a
  focus on normal matrices justified by results on conditioning of
  eigenvalue problems later on.

  Thus, spectral theory based on module theory will not be needed in
  this class. The spectral theorem for normal matrices on the other
  hand is fairly simple and can be proved without too much overhead.
\end{intro}

\begin{Definition}{eigenvalue}
  An \define{eigenvalue} of a matrix $\mata\in \C^{n\times n}$ is a
  complex number $\lambda$ such that the matrix
  \begin{gather}
   \mata-\lambda\id 
  \end{gather}
  is singular.

  The set of all eigenvalues of $\mata$ is called the
  \define{spectrum} $\sigma(\mata)$.

  The \define{eigenspace} for $\lambda$ is the kernel of
  $A-\lambda\id$, that is, the set
\begin{gather}
    \esp{\lambda} = \bigl\{
    \vv \in \C^n \;\big\vert\;
    \mata\vv = \lambda\vv \bigr\}.
\end{gather}
The \define{geometric multiplicity} of $\lambda$ is the dimension of
$\esp{\lambda}$.


An \define{eigenvector} for $\lambda$ is a (normed) vector in
$\esp\lambda$. We refer to an eigenvector $\lambda$ and a
corresponding eigenvector $\vv$ as \define{eigenpair}.
\end{Definition}

\begin{Definition}{eigenvalue-algebraic}
  An \define{eigenvalue} of a matrix $\mata\in \C^{n\times n}$ is a root of the characteristic polynomial $\chi(\lambda) = \det(\mata-\lambda\id)$.
  
  The \define{algebraic multiplicity} of an eigenvalue is the multiplicity of the corresponding root of the characteristic polynomial.
\end{Definition}

\begin{Lemma}{eigenvalue-equivalent}
  The two definitions of an eigenvalue are consistent.
\end{Lemma}

\begin{Theorem}{eigenvalue-count}
  Every matrix in $\C^{n\times n}$ has at most $n$ eigenvalues. The algebraic multiplicities of all eigenvalues add up to $n$.
\end{Theorem}

\begin{proof}
  The ``at most'' follows from the fact that a polynomial contains
  linear factors $x-\lambda_i$ for each of its roots
  $\lambda_i$. Thus, if the characteristic polynomial has $k$ roots it
  has at least degree $k$. On the other hand, the characteristic
  polynomial has degree $n$, such that $k\le n$.

  The second statement is due to the fact that every polynomial over
  $\C$ is a product of linear factors.
\end{proof}

\begin{remark}
  The last theorem is not true in $\R$, as it is not algebraically
  closed. Thus, even a real matrix may have complex eigenvalues and
  eigenvectors. Therefore, all results in this chapter will be on
  complex matrices, but some simplifications for real matrices will be
  pointed out.
\end{remark}

\begin{Definition}{eigenvalue-simple}
  An eigenvalue is \define{simple}, if its algebraic and geometric multiplicity are one. It is \define{semi-simple}, if its algebraic and geometric multiplicities are equal.
\end{Definition}

\begin{remark}
  It is possible to refine the concept of eigenvalues and eigenvectors by distinguishing between right eigenvalues and vectors solving
  \begin{gather}
      \mata \vv = \lambda \vv,
  \end{gather}
  and left eigenvalues and vectors solving
  \begin{gather}
    \vu \mata = \lambda \vu,
  \end{gather}
  where $\vu$ is now a row vector. By taking the transpose of this equation\footnote{Here, we refer to the real transpose obtained by simply exchanging indices, not the complex conjugate transpose.},
  \begin{gather}
    \label{eq:evp:1}
    \mata^T \vu^T = \lambda \vu^T,
  \end{gather}
  we see that $\vu$ is a left eigenvector of $\mata$ if and only if
  $\vu^T$ is a right eigenvector of $\mata^T$.
\end{remark}

\begin{Lemma}{eigenvalues-conjugate}
  Every eigenvalue $\lambda$ of $\mata\in\C^{n\times n}$ is also an eigenvalue of $\mata^T$.
\end{Lemma}

\begin{proof}
  The determinant does not change when the matrix is transposed, therefore
  \begin{gather}
    \chi(\mata^T)
    = \det(\mata^T-\lambda \id)
    = \det(\mata-\lambda \id)
    = \chi(\mata).
  \end{gather}
  Thus, the eigenvalues of $\mata$ and of $\mata^T$ coincide.
\end{proof}

\subsection{Normal and Hermitian matrices}

\begin{Definition}{normal-Hermitian}
  A matrix $\mata\in\C^{n\times n}$ is called \define{normal} if there holds
  \begin{gather}
      A^*A = AA^*.
  \end{gather}
  It is called \define{Hermitian} or \define{complex symmetric}, if there holds
  \begin{gather}
      A=A^*.
  \end{gather}
\end{Definition}

\begin{Theorem*}{normal-diagonalizable}{Spectral theorem for normal matrices}
  A matrix $\mata\in\C^{n\times n}$ is normal if and only if it is diagonalizable by a unitary matrix.
  
  It is normal if and only if there exists an orthonormal basis of eigenvectors.
\end{Theorem*}

\begin{proof}
  
\end{proof}

\begin{Corollary*}{symmetric-diagonalizable}{Spectral theorem for Hermitian matrices}
  A Hermitian matrix $\mata\in\C^{n\times n}$ is diagonalizable with
  an orthogonal basis of eigenvectors and real eigenvalues.  
\end{Corollary*}

\begin{proof}
  Hermitian matrices are a special case of normal matrices, such that
  \slideref{Theorem}{normal-diagonalizable} applies.
\end{proof}

%%% Local Variables:
%%% mode: latex
%%% TeX-master: "main"
%%% End:

\section{Well-posedness of the EVP and bounds on eigenvalues}
\subsection{Bounds on eigenvalues}

\begin{Lemma}{bound-by-norm}
  Let $\norm{\cdot}$ be a vector norm and denote by the same symbol
  a consistent norm for matrices. Then, for any matrix $\mata\in\Cnn$
  and for any eigenvalue $\lambda\in\sigma(\mata)$ there holds
  \begin{gather}
    \abs{\lambda} \le \norm{\mata}.
  \end{gather}
\end{Lemma}

\begin{Lemma}{pre-gershgorin}
  Let $\mata,\matb\in\Cnn$ and let $\norm{\cdot}$ be an operator norm
  on the space of matrices corresponding to a vector norm denoted by
  $\norm{\cdot}$ as well. Then, for any eigenvalue
  $\lambda\in\sigma(\mata)$ such that $\lambda\not\in\sigma(\matb)$
  there holds
  \begin{gather}
    \norm*{(\lambda\id-\matb)^{-1}(A-B)} \ge 1.
  \end{gather}
\end{Lemma}


\begin{Theorem*}{gershgorin}{Gershgorin circle theorem}
  All eigenvalues of a matrix $\mata\in\Cnn$ are contained in the
  union of the \define{Gershgorin Circle}s
  \begin{gather}
    G_j = \left\{ z\in \C \middle| \abs{z-a_{jj}} \le \sum_{k\neq j} \abs{a_{jk}}\right\}.
  \end{gather}
  Furthermore, if there is a subset of $m$ circles disjoint from the
  other circles, then this subset contains $m$ eigenvalues.
\end{Theorem*}

\subsection{The Rayleigh quotient}

\begin{Definition}{rayleigh-quotient}
  For a matrix $\mata\in\Cnn$ and a vector $\vx\in\C^n$, the
  \define{Rayleigh quotient} is defined as
  \begin{gather}
    R_\mata(\vx) = \frac{\scal(\mata\vx,\vx)}{\scal(\vx,\vx)}.
  \end{gather}
\end{Definition}

\begin{Theorem*}{minmax}{Courant-Fischer min-max theorem}
  Let $\mata\in\Cnn$ be Hermitian with eigenvalues
  $\lambda_1 \le \lambda_2\le\dots\le \lambda_n$. Then, for $k=1,\dots,n$
  \begin{align}
    \lambda_k
    &= \min_{\substack{V \subset \C^n\\\dim V = k}} \max_{\vx\in V} R_\mata(\vx),\\
    &= \max_{\substack{V \subset \C^n\\\dim V = n-k+1}} \min_{\vx\in V} R_\mata(\vx).
  \end{align}
  In particular,
  \begin{gather}
    \lambda_{\min}(\mata) = \min_{\vx\in\C^n} R_\mata(\vx),
    \qquad
    \lambda_{\max}(\mata) = \max_{\vx\in\C^n} R_\mata(\vx).
  \end{gather}
\end{Theorem*}

\subsection{Conditioning of the eigenvalue problem}

In this section, we study the conditioning of finding eigenvalues and
eigenvectors. While we will not cover the full theory, we will provide
examples for ill-posed problems as well as exemplary proofs for
well-posedness.

In all cases, we will investigate the change of eigenvalues or
eigenvectors when the matrix $\mata$ is perturbed by a small matrix
$\mate$ of norm $\epsilon$.

\begin{Definition}{conditioning-eigenvalue}
  Let $\mata\in\Cnn$ and let $\mata+\mate$ be a small perturbation of $\mata$. Then, we define the \define{absolute conditioning}\index{conditioning!absolute} of the eigenvalue problem as follows: let $\mu$ be an eigenvalue of $\mata+\mate$, then, there is an eigenvalue $\lambda$ of $\mata$ and a constant $C_{\text{abs}}$ called \define{condition number}, such that
  \begin{gather*}
    \abs{\lambda-\mu} \le C_{\text{abs}} \norm{E},
  \end{gather*}
  for a suitable matrix norm. The \define{relative conditioning}\index{conditioning!relative} is defined by
  the estimate
  \begin{gather*}
    \frac{\abs{\lambda-\mu}}{\lambda}
    \le C_{\text{rel}} \frac{\norm{E}}{\norm{A}}.
  \end{gather*}
\end{Definition}

\begin{Example}{characteristic-polynomial}
  Take a matrix of dimension 20 with eigenvalues $1,2,\ldots,20$. Its
  characteristic polynomial is
  \begin{gather}
    \chi(\lambda) = (\lambda-1)\dots(\lambda-20).
  \end{gather}
  The coefficient in front of $\lambda^{20}$ is one, the constant term is $20! > 10^{19}$.
  We perturbe it in the form
  \begin{gather}
    \tilde \chi(\lambda) = \chi(\lambda) - 10^{-23}\lambda^{19}.
  \end{gather}
  Their greatest roots are
  \begin{gather}
    \begin{array}{l@{\qquad}l@{\,}c@{\,}l}
      \multicolumn{1}{c}{\chi}&
      \multicolumn{3}{c}{\tilde \chi}\\
      20&20.847\\
      19,18&19.502&\pm&1.940i\\
      17,16&16.731&\pm&2.813i\\
      15,14&13.992&\pm&2.519i\\
    \end{array}
  \end{gather}
  {\tiny Source: \cite{DeuflhardHohmann08}}
\end{Example}

\begin{Example}{conditioning-Jordan-block}
  Consider the matrix
  \begin{gather}
  \mata_\epsilon =
      \begin{pmatrix}
        0&1\\
%        &0&1\\
        &\ddots&\ddots\\
        &&0&1\\
        \epsilon &&&0
      \end{pmatrix}
      \in\C^{n\times n},
  \end{gather}
  For $\epsilon=0$, it has a single eigenvalue of geometric multiplicity one and algebraic multiplicity $n$.

  For $\epsilon>0$, it has $n$ simple eigenvalues
  \begin{gather}
      \lambda_j = \sqrt[n]{\epsilon} \,e^{2\frac jni\pi}
  \end{gather}
\end{Example}

\begin{proof}
  For $\epsilon=0$, the matrix is the generic Jordan-block of an eigenvalue which is not semi-simple, thus the ill-posedness of this example implies the ill-posedness for not semi-simple eigenvalues in the general case. Note that this statement holds notwithstanding that special perturbations may be benign.

  The characteristic polynomial of this matrix is
  \begin{gather}
      \chi(\lambda) = \det(\mata-\lambda\id)
      = \det\begin{pmatrix}
      -\lambda&1\\
        &\ddots&\ddots\\
        &&-\lambda&1\\
        \epsilon &&&-\lambda
      \end{pmatrix}.
  \end{gather}
  Applying Laplace expansion to the first column yields
  \begin{gather}
      \chi(\lambda)
      = -\lambda \det\begin{pmatrix}
        -\lambda&1\\
        &\ddots&\ddots\\
        &&-\lambda&1\\
        &&&-\lambda
      \end{pmatrix}
      + (-1)^{n+1} \epsilon\det\begin{pmatrix}
        1 \\
        -\lambda &1\\
        &\ddots&\ddots\\
        &&-\lambda&1
      \end{pmatrix},
  \end{gather}
  where both matrices are of dimension $n-1$. Since they are triangular, recursion of Laplace expansion is particularly simple and yields the product of the diagonal elements. Thus
  \begin{gather}
      \chi(\lambda) = (-1)^n \lambda^n
      + (-1)^{n+1} \epsilon.
  \end{gather}
  Its roots are determined by the condition
  \begin{gather}
      \lambda^n = \epsilon.
  \end{gather}
  Thus, $\lambda$ can be computed as an $n$th root of unity times the (real) $n$th root of $\epsilon$.
\end{proof}

\begin{Theorem}{Jordan-block-ill-conditioned}
  The eigenvalue problem for eigenvalues which are not semi-simple is
  in general ill-posed.
\end{Theorem}

\begin{proof}
  The analysis in \slideref{Example}{conditioning-Jordan-block} is
  generic in the sense that it applies to nonzero eigenvalues and also
  to matrices which are similar to such a block. Thus, we can conclude
  that for every matrix $\mata$ which is similar to a matrix with a
  nontrivial Jordan block for eigenvalue $\lambda$, there is a
  perturbation $\mate$ such that the derivative of the function
  $\lambda(\epsilon) = \lambda(A+\epsilon\mate)$ at zero is unbounded.
\end{proof}

\begin{Theorem*}{bauer-fike}{Bauer-Fike}
  Let $\mata\in \Cnn$ be diagonalizable with matrix of eigenvectors
  $\matv \in \Cnn$ and diagonal matrix
  $\matlambda = \diag(\lambda_1\dots,\lambda_n)$. Let $\mata+\mate$ be
  a perturbation of $\mata$. Then, for any eigenvalue $\mu$ of
  $\mata+\mate$, there is an eigenvalue $\lambda_i$ of $\mata$ such
  that
  \begin{gather}
    \abs{\mu-\lambda_i} \le \cond_2(\matv) \norm{\mate}_2.
  \end{gather}
\end{Theorem*}

\begin{proof}
  Wikipedia
\end{proof}

\begin{Corollary}{conditioning-eigenvalues-normal}
  The eigenvalue problem of a normal matrix $\mata\in\Cnn$ is
  well-conditioned in the sense that for every eigenvalue $\mu$ of the
  perturbed matrix $\mata+\mate$, there is an eigenvalue $\lambda$ of
  $\mata$ such that
  \begin{gather}
    \abs{\mu-\lambda} \le \norm{E}_2.
  \end{gather}
\end{Corollary}

The Bauer-Fike theorem provides a general estimate for diagonalizable
matrices in terms of the condition number of the matrix of
eigenvectors. The following theorem is less general, since it only
applies to simple eigenvalues, but it provides geometric intuition of
the issue.

\begin{Theorem}{conditioning-eigenvalue-single}
  Let $\mata_\epsilon = \mata+\epsilon\mate\in\Cnn$ be a perturbation
  of $\mata\in\Cnn$. Let $\lambda(0)$ be a simple
  eigenvalue of $\mata$. Then, there exists a uniquely defined
  differentiable continuation $\lambda(\epsilon)$ for small $\epsilon$
  such that $\lambda(\epsilon) \in \sigma(\mata_\epsilon)$ and with
  its left and right eigenvectors $\vu$, and $\vv$, respectively, there
  holds
  \begin{gather}
    \abs*{\tfrac{d}{d\epsilon} \lambda(0)}
    \le \norm{E}_2\frac{\norm{\vu}_2\norm{\vv}_2}{\abs{\scal(\vu,\vv)}}
    = \norm{E}_2 \frac1{\cos\angle(\vu,\vv)}.
  \end{gather}
\end{Theorem}

\begin{Problem}{almost-parallel}
  Consider the matrix 
  \begin{gather} 
    M = 
    \begin{pmatrix}
      \eta & 1\\  \eta &\eta
    \end{pmatrix}
   \end{gather}
   with $|\eta| << 1$.\\
  Explain why the problem of finding eigenvectors is \textit{not} well-posed in this example.
\end{Problem}

\subsection{Conditioning of eigenvectors and eigenspaces}

\begin{intro}
  Positive results on the conditioning of eigenvectors require
  additional tools which go beyond the exposition planned for this
  class. We will thus only discuss this question at hand of an example
  and conclude a rule of thumb.
\end{intro}

\begin{Example}{conditioning-eigenvectors}
  Consider the two matrices
  \begin{gather}
    A =
    \begin{pmatrix}
      1-\epsilon & 0\\ 0 & 1+\epsilon
    \end{pmatrix},
    \qquad
    B =
    \begin{pmatrix}
      1&\epsilon\\\epsilon&1
    \end{pmatrix}.
  \end{gather}
  Their eigenvalues are $1-\epsilon$ and $1+\epsilon$, but their
  eigenvectors differ by an angle of $\pi/4$ independent of
  $\epsilon$.
\end{Example}

\begin{Remark}{conditioning-eigenvectors}
  The problem of finding eigenvectors for tight clusters of
  eigenvalues is ill-posed. Nevertheless, finding the invariant
  subspace associated to all eigenvalues in such a cluster is
  well-posed.

  Conditioning of the eigenvector problem depends on the separation of
  eigenvalues.
\end{Remark}

%%% Local Variables:
%%% mode: latex
%%% TeX-master: "main"
%%% End:


\section{Vector iterations}
%%%%%%%%%%%%%%%%%%%%%%%%%%%%%%%%%%%%%%%%%%%%%%%%%%%%%%%%%%%%%%%%%%%%%%

\begin{Problem}{intro-problem-vector-iterations}
	Consider the following matrix
	\begin{gather*}
	\mata =
	\begin{pmatrix}
	\cos\phi & -\sin\phi\\
	\sin\phi &  \cos\phi
	\end{pmatrix}^T
	\begin{pmatrix}
	1 & \\
	& c
	\end{pmatrix}
	\begin{pmatrix}
	\cos\phi & -\sin\phi\\
	\sin\phi &  \cos\phi
	\end{pmatrix}
	\end{gather*}
	with parameters $\phi\in[0,2\pi]$ and $c\in(0,1)$.
	\begin{enumerate}
		\item Determine (think, don't compute!) the eigenvalues and eigenvectors of $\mata$.
		\item (Programming) Write a program which computes the sequence
		$\vx^{(n)}\in\R^2$ defined as
		\begin{align*}
		\vx^{(n)} &= \mata \vx^{(n-1)}, \\
		\vx^{(0)} &= \vx^{*},
		\end{align*}
		for $\vx^{*} = (1,\ 0)^T$, $c = 0.1$, and
		$\phi=\frac\pi4$. Try playing with different values of those
		parameters.
		\item Is there a limit of $\vx^{(n)}$? What is about the case
		$c=1$?
		\item Compute the limit: $\lim_{n\to\infty}\mata^n$.
	\end{enumerate}
\end{Problem}

%%%%%%%%%%%%%%%%%%%%%%%%%%%%%%%%%%%%%%%%%%%%%%%%%%%%%%%%%%%%%%%%%%%%%%
\subsection{Simple iterations}
%%%%%%%%%%%%%%%%%%%%%%%%%%%%%%%%%%%%%%%%%%%%%%%%%%%%%%%%%%%%%%%%%%%%%%

\begin{intro}
  The eigenspace of a dominant eigenvalue seems to get amplified
  whenever we multiply with the matrix $\mata$. Hence, we should be a
  able to approximate this eigenspace by a simple iteration.
\end{intro}

\begin{Algorithm*}{vector-iteration}{Vector iteration (power method)}
  \begin{algorithmic}[1]
    \Require $\mata\in\Cnn$, initial vector $\vv^{(0)}\in\C^n$ with $\norm{\vv}=1$
    \For{$k=1,\dots$ until convergence}
    \State $\vy^{(k)} \gets \mata \vv^{(k-1)}$
    \State $\vv^{(k)} = \frac{\vy^{(k)}}{\norm{\vy^{(k)}}}$
    \State $\alpha_k = \frac{\bigl(\vy^{(k)}\bigr)_i}{\bigl(\vv^{(k)}\bigr)_i}$ where $\abs*{\bigl(\vv^{(k)}\bigr)_i} = \norm*{\vv^{(k)}}_\infty $
    \EndFor
%    \State $\alpha = \frac{\vy^{(k_\max +1)}}{\vv^{(k_{\max})}_i}$
%    where $\abs{\vv^{(k_{\max})}_i}$ is maximal
  \end{algorithmic}
\end{Algorithm*}

\begin{Theorem}{vector-iteration}
  Let $\mata\in\Cnn$ be diagonalizable such that $\lambda_1$ is the
  unique dominant eigenvalue. Let furthermore the
  component of $v^{(0)}$ in direction of the first eigenvector be
  nonzero. Then, the factors
  \[\alpha_k := \frac{\vy^{(k+1)}_i}{\vv_i^{(k)}} \; \text{where} \; |\vv_i^{(k)|}| \; \text{is maximal}\]
  and vectors $v^{(k)}$ of the
  vector iteration converge to the eigenvalue $\lambda_1$ and its
  associated eigenvector. Moreover, there holds
    %\State $\alpha = \frac{\vy^{(k_\max +1)}}{\vv^{(k_{\max})}_i}$
    %where $\abs{\vv^{(k_{\max})}_i}$ is maximal
  \begin{align}
    \abs{\alpha_{k+1}-\lambda_1}
    &\le \frac{\abs{\lambda_1}}{\abs{\lambda_2}} \abs{\alpha_{k}-\lambda_1}\\
    \norm{v^{(k+1)}-u_1}
    &\le \frac{\abs{\lambda_1}}{\abs{\lambda_2}} \norm{v^{(k)}-u_1}
  \end{align}
\end{Theorem}

\begin{remark}
  The previous theorem actually holds eveon for matrices, which are
  not diagonalizable, but then the proof becomes more involved. Since
  we are not interested in eigenvalues, which are not semi-simple, we
  decided for the simpler version.

  The proof can be easily improved to cover the case that $\lambda_1$
  is not a single eigenvalue. In exact arithmetic, there will be no
  change of the result. When computing with floating point numbers,
  there will be a small glitch. It is left to the reader to find it.
\end{remark}

\begin{remark}
  The algorithm can be improved in several ways. First, we saw in the
  proof that it generates a converging sequence of vectors. Threfore,
  we can save the effort of computing eigenvalues inside the iteration
  and postpone it until the end.

  Second, if $\mata$ is a Hermitian matrix, the computation of the
  eigenvalue can be improved by usig the Rayleigh quotient.

  Using both statements, we obtain an optimized version of the vector
  iteration.
\end{remark}

\begin{Algorithm*}{vector-iteration-rayleigh}{Power method (optimized)}
  \begin{algorithmic}[1]
    \Require $\mata\in\Cnn$ Hermitian, initial vector $\vv^{(0)}\in\C^n$ with $\norm{\vv}=1$
    \For{$k=1,\dots$ until convergence}
    \State $\vy^{(k)} \gets \mata \vv^{(k-1)}$
    \State $\vv^{(k)} = \frac{\vy^{(k)}}{\norm{\vy^{(k)}}}$
    \EndFor
    \State $\lambda_1 \approx R_\mata(\vv^{(k_{\max})} = \scal(\mata\vv^{(k_{\max})},\vv^{(k_{\max})})$
  \end{algorithmic}
\end{Algorithm*}


\begin{todo}
  GM:
  Why is it required, that the eigenvalue is single? That is not included in the previous Theorem? \\
  Espacially a diagonalizable matrix can have multiple eigenvalues
\end{todo}
\begin{Remark}{vector-iteration}
  The proof actually requires, that the entry defining $\alpha_k$
  remains the same during the iteration, at least during the steps
  used for detecting convergence.

  The result does not actually require that $\mata$ is diagonalizable,
  as long as $\lambda_1$ is single and of largest modulus.
\end{Remark}

\begin{todo}
  GM:
  Fixed a Symbol
\end{todo}
\begin{Lemma}{Rayleigh-approximation}
  Let $(\lambda,\vv)$ be an eigenpair of the Hermitian matrix
  $\mata\in\Cnn$, and let $\vw\in\C^n$. Then, there is a constant $C$ depending only on the matrix $\mata$ such that there holds
  \begin{gather}
    \abs{R_\mata(\vw)-\lambda} \le C \norm{\vw-\vv}^2,
  \end{gather}
  where $R_\mata(\vw)$ is the \putindex{Rayleigh quotient} from
  \slideref{Definition}{rayleigh-quotient}.
\end{Lemma}

\begin{Algorithm*}{shifted-vector-iteration}{Shifted vector iteration}
  The vector iteration can be applied to the matrix $\mata-\sigma\id$
  for some $\sigma\in\C$.

  Then, $\alpha_k$ converges to the eigenvalue $\lambda$ such that
  $\lambda-\sigma$ has largest modulus. $v^{(k)}$ converges to an
  eigenvector for this eigenvalue.
\end{Algorithm*}

\begin{todo}
  GM: added the shifted matrix polynomial
\end{todo}
\begin{Definition}{shifted-matrix-polynomial}{Shifted Matrix Polynomial}
  Let \(\mata \in \C^{n \times n}\) be a square matrix.
  The \define{matrix polynomial} of degree \(m\), with shift parameter \(\rho_1, \ldots, \rho_m\) is defined by
  \[ \mathbf{p(A)} := \left( A - \varrho_1 \id \right) \dots \left( A - \rho_m\right).\]
\end{Definition}

%%%%%%%%%%%%%%%%%%%%%%%%%%%%%%%%%%%%%%%%%%%%%%%%%%%%%%%%%%%%%%%%%%%%%%
\subsection{Subspace iterations}
%%%%%%%%%%%%%%%%%%%%%%%%%%%%%%%%%%%%%%%%%%%%%%%%%%%%%%%%%%%%%%%%%%%%%%

\begin{intro}
  
\end{intro}

\begin{todo}
  GM:
  Moved the inverse iteration back after the QR Iteration, and instead inserted the subspace iteration.\\
  Insert an exercise to show why the shifted matrix polynomial is required.
\end{todo}

\begin{Algorithm*}{subspace-iteration-polynomial}{Orthogonal subspace iteration}
  \begin{algorithmic}[1]
    \Require $\mata\in\Cnn$ and choose $\matx_0 \in \C^{n\times m}$.
    \For {$k=1,\ldots$ until convergence}
    \State $\matq_k\matr_k \gets \matx_{k-1}$ \Comment{QR factorization}
    \State $\matx_{k} \gets p(\mata) \matq_{k}$
    \EndFor
    \State {$\lambda_i \approx \vq_{k_{\max},i}^* \mata \vq_{k_{\max},i},\qquad i=1,\dots,m$}\
    \Comment{Rayleigh quotient}
  \end{algorithmic}
\end{Algorithm*}



%%% Local Variables:
%%% mode: latex
%%% TeX-master: "main"
%%% End:


\section{Subspace iterations and the QR method}
\subsection{Definition of the methods}

\begin{Algorithm*}{subspace-iteration}{Orthogonal subspace iteration}

  Let $\mata\in\Cnn$, $\matx_0 \in \C^{n\times m}$.\\
  For $k=0,\ldots$ until convergence repeat
  \begin{itemize}
  \item $\matz_k = \mata \matx_k$.
  \item $\matq_k\matr_k = \matz_k$ (QR factorization)
  \item $\matx_{k+1} = \matq_k$
  \end{itemize}
\end{Algorithm*}

\begin{Algorithm*}{qr-iteration}{QR iteration}
  
  Let $\mata_1 = \mata\in\Cnn$.\\
  For $k=1,\ldots$ until convergence repeat
  \begin{itemize}
  \item $\matq_k\matr_k = \mata_k$ (QR factorization)
  \item $\mata_{k+1} = \matr_k\matq_k$
  \end{itemize}
\end{Algorithm*}

\subsection{Analysis}
\begin{Theorem*}{schur-canonical}{Schur canonical form}
  For every matrix $\mata\in\Cnn$ there are a unitary matrix
  $\matq\in\Cnn$ and an upper triangular matrix $\matr\in\Cnn$ such
  that
  \begin{gather}
    \mata = \matq \matr \matq^*.
  \end{gather}
  The diagonal entries of $\matr$ are the eigenvalues of $A$. The
  column vectors of $\matq$ are called \define{Schur vectors}.
\end{Theorem*}

\begin{Lemma}{schur-canonical-1}
  For any $k\le n$ the span of the Schur vectors
  $\vq_1,\dots,\vq_k$ is invariant under the action of $\mata$.

  For $\matq_k = (\vq_1\dots\vq_k)$ and $R_k$ the upper left $k\times k$ block of $\matr$, there holds
  \begin{gather}
    \mata\matq_k = \matq_k \matr_k.
  \end{gather}
\end{Lemma}

\begin{Lemma}{schur-canonical-2}
  The Schur vectors depend on the order chosen for the eigenvalues,
  and in case of geometric multiplicity, the eigenvectors,
  respectively. They are determined up to factors $e^{i\phi}$
\end{Lemma}

\begin{Theorem}{convergence-subspace-iteration}
  Let $\mata\in\Cnn$ and
  \begin{gather}
    \abs{\lambda_1} >
    \abs{\lambda_2}>\dots>\abs{\lambda_m}>\abs{\lambda}
  \end{gather}
  for all
  remaining eigenvalues $\lambda\in\sigma(\mata)$. Let
  $\matq = (\vq_1\dots\vq_m)$ be the Schur vectors associated with the
  first $m$ eigenvalues and $\esp{1,\dots,j}$ be the space spanned by
  the first $j$ eigenvectors and $P_j$ the orthogonal projector
  onto this space. Let the set of start vectors of the orthogonal subspace iteration
  $\matx_0 = (\vx_1\dots\vx_m)$ be chosen that
  \begin{gather}
    \operatorname{span}\{P_1 \vx_1,\dots,P_j\vx_j\} = \esp{1,\dots,j},\qquad j=1,\dots,m.
  \end{gather}
  Then, the $j$-th column of $\matx_k$ converges to $\vq_j$ for $j=1,\dots,m$ up to a factor $e^{i\phi}$.
\end{Theorem}

\begin{Lemma}{qr-1}
  The matrices $\mata_k$ of the QR-iteration have the following properties:
  \begin{enumerate}
  \item $\mata_{k+1} = \matq_k^*\mata_k\matq_k = \matq_k^*\dots\matq_1^*A\matq_1\dots\matq_k$.
  \item $\mata^k=\matq_1\dots\matq_k\matr_k\dots\matr_1$.
  \item If $\mata$ is normal, so is $\mata_k$ for any $k$.
  \item If $\mata$ is symmetric, so is $\mata_k$ for any $k$.
  \end{enumerate}
\end{Lemma}

\begin{Theorem}{convergence-qr-iteration}
Removed
\end{Theorem}

\subsection{Implementation issues}
\begin{intro}
  In each step of the QR-iteration, a QR-decomposition of the matrix
  is needed, which requires $\bigo(n^3)$ operations. Thus, the
  complexity of the iteration is highly unfavorable. The following
  discussion will provide us with means to reduce the complexity of
  the QR-decomposition to $\bigo(n^2)$, in the symmetric case even to
  $\bigo(n)$.
\end{intro}

\begin{Definition}{hessenberg}
  A matrix is in \define{Hessenberg form} or is a \define{Hessenberg
    matrix}, if all its entries below the first subdiagonal are zero. Visually,
  \begin{gather}
    H = 
    \begin{pmatrix}
      *&*&*&*&*&*\\
      *&*&*&*&*&*\\
      0&*&*&*&*&*\\
      0&0&*&*&*&*\\
      0&0&0&*&*&*\\
      0&0&0&0&*&*
    \end{pmatrix}
  \end{gather}
  A symmetric or Hermitian Hessenberg matrix is \define{tridiagonal}.
\end{Definition}

\begin{Theorem}{Hessenberg-qr}
  The QR-decomposition of a Hessenberg matrix $\matH$ can be obtained
  by $n-1$ givens rotations. The matrix $\matr\matq$ is again in
  Hessenberg form. For a (complex) symmetric matrix $\matH$, the
  matrix $\matr\matq$ is even tridiagonal and (complex) symmetric.
\end{Theorem}

\begin{Corollary}{Hessenberg-qr}
  The complexity of each step of a QR-iteration for Hessenberg matrices is $\bigo(n^2)$. For tridiagonal (complex) symmetric matrices, it is $\bigo(n)$.
\end{Corollary}

\begin{Theorem}{Hessenberg-householder}
  Every matrix $\mata\in\Cnn$ is unitarily similar to a Hessenberg matrix $\matH$, that is,
  \begin{gather}
    \matH = \matq \mata \matq^*.
  \end{gather}
  The matrix $\matq$ can be obtained by $n-2$ \putindex{Householder
    reflections}.
\end{Theorem}

\begin{Algorithm*}{qr-method}{The QR-Method}
  Compute the spectrum of a matrix $\mata\in\Cnn$ by
  \begin{enumerate}
  \item Use $n-2$ Householder transformations to transform $\mata$ to
    Hessenberg form
    \begin{gather}
     \matH = \matq\mata\matq^*.
   \end{gather}
 \item QR-iteration: let $\matH_{0}=\matH$ and perform until convergence
   \begin{align}
     \Omega^{(k)}_{1,2}\times\dots\times\Omega^{(k)}_{n-1,n} \matr &= \matH_k\\
     \matH_{k+1} &= \matr \Omega^{(k)}_{1,2}\times\dots\times\Omega^{(k)}_{n-1,n}.
   \end{align}
 \item Store Householder vectors as well as $r$ and $c$ for each
   Givens rotation if the eigenvectors are desired in the end.
  \end{enumerate}
\end{Algorithm*}

\begin{Theorem}{hessenberg-qr-convergence}
    Let $\matH\in\Cnn$ be a Hessenberg matrix with eigenvalues such that
  \begin{gather}
    \abs{\lambda_1} >
    \abs{\lambda_2}>\dots>\abs{\lambda_n}.
  \end{gather}
  Then, the sequences of the QR-iteration admit the following estimates:
  \begin{align}
    \dist(\esp{1,\dots,j},\spann{\vq_1^{(k)},\dots,\vq_j^{(k)}}) &= \bigo \left(\abs*{\frac{\lambda_{j+1}}{\lambda_j}}^k\right),
    \\
    h_{j+1,j}^{(k)} &= \bigo \left(\abs*{\frac{\lambda_{j+1}}{\lambda_j}}^k\right)
                      .
  \end{align}
  Here, $h_{ij}^{(k)}$ are the entries of $\matH_k$.
\end{Theorem}

\begin{proof}
  See~\cite[Theorem 7.3-1]{GolubVanloan83}.
\end{proof}

\subsection{Shifts and deflation}

\begin{intro}
  The goal of this section is the development and justification of a
  method which accelerates convergence of the QR-iteration and
  reducing the effort at the same time. It is based on shifts, like
  for the simple or inverse power method. But, shifts are much more
  powerful here, since we compute not only ``converging subspace'',
  but also its complement.
\end{intro}

\begin{Theorem}{qr-reduction}
  Let the matrix $\matH^{(k)}\in\Cnn$ in the QR iteration be of the
  form
  \begin{gather}
    \matH^{(k)} =
    \begin{pmatrix}
      \matH_{11} & \mata_{12}\\0 & \matH_{22}
    \end{pmatrix}
  \end{gather}
  with Hessenberg matrices $\matH_{11}\in\C^{p\times p}$,
  $\matH_{22}\in \C^{n-p\times n-p}$ and an arbitrary matrix
  $\mata_{12}\in \C^{p\times n-p}$. Then, the matrix $\matq^{(k)}$
  decouples into two diagonal blocks and $\matH^{(k+1)}$ has the same
  form. Thus, the iteration decouples into two separate iterations.
\end{Theorem}

\begin{Definition}{hessenberg-unreduced}
  A Hessenberg matrix is called \define{unreduced} if all entries on
  the first subdiagonal are nonzero. It is called \define{reduced}
  otherwise.
\end{Definition}

\begin{Algorithm*}{shifted-qr-iteration}{QR iteration with shift}
  Let $\matH_1 = \matq_0^*\mata\matq_0\in\Cnn$.\\
  For $k=1,\ldots$ until convergence repeat
  \begin{itemize}
  \item $\matq_k\matr_k = \matH_k - \sigma\id$ (QR factorization)
  \item $\mata_{k+1} = \matr_k\matq_k + \sigma\id$
  \end{itemize}
\end{Algorithm*}




\subsection{Methods in real arithmetic}

\begin{Theorem*}{real-schur-form}{The real Schur form}
  For every matrix $\mata\in \Rnn$ there is an orthogonal matrix
  $\matq\in\Rnn$ and a matrix $\matr\in\Rnn$ such that
  \begin{gather}
    \mata = \matq\matr\matq^*,
    \qquad
    \matr =
    \begin{pmatrix}
      R_{11} &* & *&*\\
      &R_{22}&*&*\\
      &&\ddots&*\\
      &&& R_{jj}
    \end{pmatrix},
  \end{gather}
  where the diagonal blocks are either of dimension one containing the
  real eigenvalues or of dimension 2 for complex conjugate eigenvalue
  pairs.
\end{Theorem*}


%%% Local Variables:
%%% mode: latex
%%% TeX-master: "main"
%%% End:


%\section{Polynomräume}

% \begin{Satz}{nullstellen}
%   Ein reelles Polynom vom Grad $n$ hat höchstens $n$ Nullstellen oder es ist das Nullpolynom.
% \end{Satz}

% \begin{proof}
%   Für $n=1$ handelt es sich um ein lineares Polynom und die Aussage
%   des Satzes ist unmittelbar klar. Sei nun $p$ ein Polynom strikt vom
%   Grad $n>1$ mit Nullstelle $x_0$. Dann gibt es nach dem euklidischen
%   Algorithmus zur Division mit Rest ein Polynom $q$ vom Grad $n-1$ und
%   eine Konstante $c$, so dass
%   \begin{gather}
%     p(x) = (x-x_0)q(x)+c.
%   \end{gather}
%   Daraus folgt $p(x_0) = c$, so dass folgt $c=0$. Wir können dieses
%   Verfahren für alle weiteren Nullstellen $x_1,\dots,x_m$ wiederholen
%   und erhalten
%   \begin{gather}
%     p(x) =  r(x) \prod_{k=0}^m (x-x_i),
%   \end{gather}
%   wobei $r(x)$ ein Polynom vom Grad $n-m$ sein muss, da $p$ vom Grad
%   $n$ ist. Insbesondere muss gelten $m\le n$.
% \end{proof}

% \begin{Korollar}{polynome-identisch}
%   Zwei reelle Polynome vom Grad $n$ sind identisch, wenn sie in
%   mindestens $n+1$ Punkten übereinstimmen. 
% \end{Korollar}


\begin{Lemma}{monome-linear-unabhaengig}
  Die Menge der Monome $\{x^0, x^1,\dots,x^n\}$ ist linear unabhängig.
\end{Lemma}

\begin{proof}
  Sei $p$ ein Polynom vom Grad $n$, also
  \begin{gather}
     p(x) = a_nx^n+a_{n-1}x^{n-1}+\dots+a_1x+a_0
   \end{gather}
   $p$ ist also gerade eine Linearkombination der Monome.  Zu zeigen
   ist, dass aus der Eigenschaft $p \equiv 0$ folgt, dass alle
   Koeffizienten verschwinden, also
  \begin{gather}
    p(x) \equiv 0
    \quad\Rightarrow\quad a_n = \dots = a_0 = 0.
  \end{gather}
  Zu diesem Zweck berechnen wir die $n$-te Ableitung von $p$ und
  erhalten, da mit $p$ auch alle seine Ableitungen identisch
  verschwinden,
  \begin{gather}
    n! a_n = 0.
  \end{gather}
  Daraus schließen wir $a_n = 0$. Nun gilt für die $(n-1)$-te Ableitung
  \begin{gather}
    n! a_n x + (n-1)! a_{n-1} = (n-1)! a_{n-1} = 0.
  \end{gather}
  Auf diese Weise schließen wir rekursiv bis $a_0$, dass alle Koeffizienten verschwinden. Damit ist das Lemma bewiesen.
\end{proof}

\begin{Satz}{polynomraum}
  Die Polynome vom maximalen Grad $n$ bilden einen Vektorraum der
  Dimension $n+1$.  Wir bezeichnen ihn mit $\P_n$.
\end{Satz}

\begin{proof}
  Es ist leicht nachzurechnen, dass sowohl die Summe, als auch reelle
  Vielfache von Polynomen wieder Polynome sind. Insbesondere erhöhen
  beide Operationen den Grad nicht. Damit ist $\P_n$ ein
  Vektorraum. Er wird per definitionem von den Monomen vom Grad bis zu
  $n$ erzeugt. Da diese nach
  \slideref{Lemma}{monome-linear-unabhaengig} linear unabhängig sind,
  bilden sie eine Basis und die Dimension von $\P_n$ ist $n+1$.
\end{proof}

\begin{Quiz}{Polynomräume}
  Gegeben beliebige Werte $x_j\in\R$ mit $j=1,\dots,n$. Die Menge der
  Polynome $p_i$ definiert durch
  \begin{align*}
    p_0(x) &= 1\\
    p_i(x) &= \prod_{j=1}^i (x-x_j),\qquad i=1,\dots,n
  \end{align*}
  \begin{enumerate}[A]
  \item ist linear unabhängig
  \item ist linear abhängig
  \item ist ein Erzeugendensystem für $\P_n$
  \item ist eine Basis von $\P_n$
  \end{enumerate}
\end{Quiz}
\section{Skalarprodukt und Orthogonalität}
\begin{Definition}{skalarprodukt}
  Sei $V$ ein reeller Vektorraum. Eine Abbildung
  $a\colon V \times V \to \R$ heißt \define{Bilinearform}, wenn für
  $u,v,w\in V$ und $\lambda,\mu\in \R$ gilt
  \begin{align}
    a(\lambda u + \mu v,w) &= \lambda a(u,w) + \mu a(v,w)\\
    a(w,\lambda u + \mu v) &= \lambda (w,u) + \mu a(w,v).
  \end{align}
  Eine Bilinearform heißt \define{symmetrisch}, wenn für $u,v\in V$ gilt
  \begin{gather}
    a(u,v) = a(v,u).
  \end{gather}
  Sie heißt \define{positiv semi-definit}, wenn $a(u,u) \ge 0$ für alle
  $u\in V$ und \define{positiv definit}, wenn zusätzlich
  \begin{gather}
    a(u,u) = 0 \quad \Longrightarrow \quad u=0.
  \end{gather}
  Eine symmetrische, positiv definite Bilinearform heißt
  \define{Skalarprodukt}, in der Regel notiert als $\scal(\cdot,\cdot)$.
\end{Definition}

%%%%%%%%%%%%%%%%%%%%%%%%%%%%%%%%%%%%%%%%%%%%%%%%%%%%%%%%%%%%%%%%%%%%%%
\begin{Lemma*}{bcs}{Bunjakowski-Cauchy-Schwarzsche Ungleichung}
  Sei $\scal(\cdot,\cdot)$ ein Skalarprodukt auf $V$.  Für zwei beliebige Elemente $u,v\in V$ gilt
  \begin{gather}
    \abs{\scal(u,v)} \le \sqrt{\scal(u,u)} \, \sqrt{\scal(v,v)}.
  \end{gather}
  Gleichheit gilt genau dann, wenn $u$ und $v$ kollinear sind, also
  $v=\alpha u$ mit einem skalaren Faktor $\alpha$.
\end{Lemma*}

\begin{proof}
  Zunächst zeigen wir nur die Ungleichung: Für $v=0\in V$ ist sie
  offensichtlich.
  
  Seien nun $v,u \in V$ keine Nullvektoren. Für beliebige $\mu, \lambda \in \R$
  gilt wegen der Bilinearität 
  \begin{gather}
   0 \le \scal(\lambda u + \mu v,\lambda u +  \mu v)
    = \lambda^{2} \scal(u,u)+2 \mu \lambda \scal(u,v) +\mu^{2} \scal(v,v)
  \end{gather}
  Setze $\lambda := \scal(v,v) \neq 0$
  \begin{gather}
   0 \le \scal(v,v)^{2} \scal(u,u) + 2\mu \scal(v,v)\scal(u,v) +\mu^{2}\scal(v,v)
  \end{gather}
  Dividiere durch$\scal(v,v)$
  \begin{gather}
   0 \le \scal(v,v) \scal(u,u) + 2\mu \scal(u,v) +\mu^{2}
  \end{gather}
  Setze nun $\mu := -\scal(u,v)$
  \begin{gather}
    0 \le \scal(v,v) \scal(u,u) -2\scal(u,v)^{2}+\scal(u,v)^{2}
  \end{gather}
  Daraus folgt
  \begin{gather}
    \scal(u,v)^{2} \le \scal(u,u) \scal(v,v)
  \end{gather}
  und mit der Monotonie der Quadratfunktion die Ungleichung.

  Nun bleibt die Äquivalenz für die Gleichheit zu zeigen.
  Für $v=0$ ist dies wieder trivial erfüllt. Seien zunächst $u,v$ linear abhängig, also zum Beispiel $u=av$.
  Dann gilt mit der Abkürzung $f(v) = \sqrt{\scal(v,v)}$
  \begin{gather}
    \abs{\scal(u,v)} = \abs{\scal(av,v)}
    = \abs{a} \cdot f(v) \cdot f(v)
    = f(av) \cdot f(v) =f(u) \cdot f(v).
  \end{gather}

  Gelte nun umgekehrt $\scal(u,v) = \sqrt{\scal(u,u)}\sqrt{\scal(v,v)}$.
  Es folgt
  \begin{gather}
     \scal(v,v) \scal(u,u) -2\scal(u,v)^{2}+\scal(u,v)^{2} = 0.
  \end{gather}
  Setze $\mu = \scal(u,v)\neq 0 $ und
  $\lambda = \scal(v,v)\neq 0$. Dann erhält man
  \begin{gather}
    \lambda \scal(u,u) - 2 \mu \scal(u,v) + \mu^2 = 0.
  \end{gather}
  Multipliplikation mit $\scal(v,v)$ ergibt
  \begin{gather}
   \lambda^2 \scal(u,u)+2\mu \scal(u,v)\scal(v,v) +\mu^{2}\scal(v,v) = 0 = \scal(\lambda u-\mu v,\lambda u-\mu v).
  \end{gather}
  
  Wegen der Definitheit folgt nun
  $\lambda u + \mu v = 0$ und da $\mu$ und $\lambda$ ungleich Null sind gilt,
  dass $ u,v$ linear abhängig sind
\end{proof}

%%%%%%%%%%%%%%%%%%%%%%%%%%%%%%%%%%%%%%%%%%%%%%%%%%%%%%%%%%%%%%%%%%%%%%
\begin{Lemma}{hilbertnorm}
  Sei $V$ ein reeller Vektorraum mit Skalarprodukt
  $\scal(\cdot,\cdot)$. Dann ist durch
  \begin{gather}
    \norm{u} = \sqrt{\scal(u,u)}
  \end{gather}
  auf $V$ eine Norm definiert. Ein reeller Vektorraum $V$ mit
  Skalarprodukt und zugehöriger Norm heißt \define{euklidischer
    Vektorraum}.
\end{Lemma}

\begin{proof}
  Das Skalarprodukt ist nicht negativ, daher ist die Abbildung $\norm{\cdot}\colon V \to \R$ wohldefiniert.
  Wir müssen nun die Normeigenschaften nachrechnen. Sei dazu $u \in V$. Es gilt
  \begin{enumerate}
  \item Nichtnegativität und Definitheit folgen sofort aus den entsprechenden Eigenschaften des Skalarprodukts.
  \item Homogenität
  \begin{gather}
    \norm{\lambda u} = \sqrt{\scal(\lambda u,\lambda u)}
    =\sqrt{\lambda^{2}\scal(u,u)}
    = \abs{\lambda}\sqrt{\scal(u,u)}
    =\abs{\lambda}\norm{u}
  \end{gather}
  \item Deiecksungleichung
  \begin{gather}
    \begin{aligned}
      \norm{u+v}^{2}
      &= \scal(u+v,u+v)\\
      &= \scal(u,u)+ 2 \scal(u,v) + \scal(v,v)\\
      \label{eq:orthopoly:1}
      &\le \scal(u,u)+ 2 \norm{u} \, \norm{v} + \scal(v,v)\\
      &=\norm{u}^{2}+ 2 \norm{u} \, \norm{v}+ \norm{v}^{2}\\
      & =(\norm{u}+\norm{v})^{2}\\
    \end{aligned}
  \end{gather}
  Daraus folgt durch Wurzelziehen auf beiden Seiten $\norm{u+v} \le \norm{u}+\norm{v}$.
  Für die Abschätzung in Zeile~\eqref{eq:orthopoly:1} haben wir die
  Bunyakovsky-Cauchy-Schwarz-Ungleichung aus \slideref{Lemma}{bcs} verwendet.
  \end{enumerate}
\end{proof}

\begin{Lemma*}{l2-norm}{$L^2$-Skalarprodukt}
  Auf dem Raum $V=\P_n$ der reellen Polynome vom Grad bis zu $n$ ist durch
  \begin{gather}
    \scal(p,q) = \int_{-1}^1 p(x)q(x)\dx
  \end{gather}
  ein Skalarprodukt definiert. Dieses wird $L^2$ Skalarprodukt genannt.
\end{Lemma*}

\begin{proof}
  Hier gilt es zu prüfen, ob die Abbildung auch die vier Eigenschaften eines
  Skalarprodukts erfüllt.\\
  Seien  $p,q,g \in \P_n$ in diesem Beweis.\\
  Da wir schon von einem Skalarprodukt ausgehen, empfiehlt es sich
  zuerst die Symmetrie zu zeigen.
  \begin{gather}
    \scal(p,q) =  \int_{-1}^1 p(x)q(x)\dx = \int_{-1}^1 q(x)p(x)\dx
    =\scal(q,p)
  \end{gather}
  Wenn wir nun zeigen, dass es eine Bilinearform ist müssen wir nur noch eine
  Identität zeigen, da wir schon wissen, dass die Symmetrieeigenschaft
  erfüllt ist.
  \begin{gather}
    \begin{aligned}
    \scal(\lambda p + \mu q, g)
    &= \int_{-1}^1 (\lambda p(x)+ \mu q(x))g(x)\dx\\
   & = \int_{-1}^1 \lambda p(x)g(x)+ \mu q(x)g(x)\dx \\
   &= \int_{-1}^1 \lambda  p(x)g(x)\dx + \int_{-1}^1 \mu q(x)g(x)\dx \\
   &= \lambda \int_{-1}^1 p(x)g(x)\dx + \mu  \int_{-1}^1 q(x)g(x)\dx \\
   &= \lambda \scal(p,g) + \mu \scal(q,g)
    \end{aligned}
  \end{gather}
  Da wir die Symmetrie vorher gezeigt haben, gilt Linearität auch
  im zweiten Argument.\\
  
  Als letztes zeigen wir, dass die Abbildung positiv definit ist.
  \begin{gather}
    0 = \scal(p,p) = \int_{-1}^1 p(x)p(x)\dx =\int_{-1}^1 p(x)^{2}\dx
  \end{gather}
  Aus den Integraleigenschaften folgt
  \begin {gather}
    0 = p(x)^{2} \quad \forall x
  \end{gather}
  Dies kann nur der Fall sein, wenn $p \equiv 0$ ist.\\
  Somit haben wir nachgerechnet, dass es sich um Skalarprodukt handelt.
  \end{proof}

\begin{Definition}{l2-norm}
  Nach dem \slideref{Lemma}{hilbertnorm} können wir mit diesem Skalarprodukt eine Norm auf $\P_n$
  definierten. Diese Norm wird als die $L^2$ Norm bezeichnet.
  \begin{gather}
    \norm{f}_{L^2} = \sqrt{\scal(f,f)_{L^2}} = \int_{-1}^1 f(x)^2 dx
  \end{gather}
  \end{Definition}

\begin{Definition}{orthogonal}
  Zwei Vektoren $u,v\in V$ heißen \define{orthogonal}, wenn
  \begin{gather}
    \scal(u,v) = 0.
  \end{gather}
  Ein Vektor $u\in V$ ist orthogonal zum Untervektorraum $W\subset V$, wenn
  \begin{gather}
    \scal(u,v) = 0\quad\forall v\in W.
  \end{gather}
\end{Definition}

\begin{Notation}{euklidischer-vr}
  Von nun an bezeichnet $V$ immer einen endlichdimensionalen, reellen,
  euklidischen Vektorraum.
\end{Notation}

\begin{Lemma*}{pythagoras}{Pythagoras}
  Seien zwei Vektoren $u\in V$ und $v\in V$ orthogonal zueinander. Dann gilt
  \begin{gather}
    \norm{u+v}^{2} = \norm{u}^{2} + \norm{v}^{2}
  \end{gather}
\end{Lemma*}

\begin{proof}
  Seien $u,v \in V$. Es gilt $ 0 = \scal(u,v)$
   \begin{gather}
    \norm{u+v}^{2} = \scal(u+v,u+v)
    %=\scal(u+v,u)+\scal(u+v,v)
    %=\scal(u,u)+\scal(v,u)+\scal(u,v)+\scal(v,v)
    =\norm{u}^{2} + \norm{v}^{2} +2\scal(u,v) = \norm{u}^{2} + \norm{v}^{2}
  \end{gather}
\end{proof}

\section{Bestapproximation und orthogonale Projektion}
\begin{Definition}{bestapproximation}
  Sei $A\subset V$ ein affiner Unterraum eines euklidischen
  Vektorraums. Dann ist die Bestapproximation $u_b\in A$ eines Vektors
  $u\in V$ in $A$ definiert durch die Beziehung
  \begin{gather}
    \norm{u-u_b} = \min_{v\in A} \norm{u-v}.
  \end{gather}
\end{Definition}

\begin{Satz}{bestapproximation}
  Sei $w \in V$ und $W \subset V$.
  Sei $A=w+W$ ein nichtleerer, affiner Unterraum von $V$. Dann
  existiert die Bestapproximation nach
  \slideref{Definition}{bestapproximation} und ist eindeutig
  bestimmt. Es gilt die notwendige und hinreichende Bedingung
  \begin{gather}
    \scal(u-u_b,v) = 0 \quad \forall v\in W.
  \end{gather}
  Das heißt $ u_b$ ist Bestapproximation genau dann wenn $u-u_b$
  orthogonal zu $W$ bzgl. des Skalarprodukts $\scal(\cdot,\cdot)$ ist.
\end{Satz}

\begin{proof}
  Der Beweis gliedert sich in drei Teile. Zuert wird die Äquivalenz
  gezeigt danach zeigen wir die Eindeutigkeit und zum Schluss
  erst die Existenz.\\ \\
 $\glqq \Rightarrow \glqq$
  Sei $ u_b \in A$ die Bestapproximation des Vektors $ u \in V$\\
  Wir defnieren nun eine Funktion:
  \begin{gather}
    F_v(x):= \norm{u-u_b-xv}^{2}, x \in \R,  v\in A
  \end{gather}

  Diese Funktion besitzt ein Minimum bei x=0. Folglich gilt
  \begin{gather}
    \left. \frac{d}{dx} F(x) \right|_{x=0}
    =\left. \frac{d}{dx}\norm{u-u_b-xv}^{2} \right|_{x=0}=0
  \end{gather}
    
  Dies kann weiter umgeformt werden zu
  $\scal(u-u_b-xv,v)|_{x=0}=0 \ \forall v\in A$ und folglich zu
  \begin{gather}
   \scal(u-u_b,v)=0 \ \forall v\in A
  \end{gather}

  $\grqq \Leftarrow \glqq$
  Nun erfüllt $u_b\in A$ die Bedingung.\\
  Dann gilt mit einem beliebigen $v\in A$:
  \begin{gather}
   \norm{u-u_b}^{2}=\scal(u-u_b,u-u_b)\\
   = \scal(u-u_b,u-v)+\scal(u-u_b,v-u_b)\\
   \le \norm{u-u_b}\cdot\norm{u-v}\\
  \end{gather}
  Daraus folgt $\norm{u-u_b} \le \inf_{v\in A}{\norm{u-v}}$\\
  Damit erfüllt $u_b$ eben die Definiton der Bestapproximation\\ \\
  Nun zur Eindeutigkeit:\\
  Seien $u_b$ und $u_d \in$ A zwei Bestapproximationen.
  Dann gilt notwendigerweise
  \begin{gather}
   \scal(u-u_b,v) = 0 = \scal(u-u_d,v) \quad \forall v\in A
  \end{gather}
  Dies wird umgeformt zu
  \begin{gather}
  \scal(u-u_d,v)-\scal(u-u_b,v)=0 \quad  \forall v\in A \\
  \scal(u_b-u_d,v) = 0 \quad \forall v \in A
  \end{gather}
  Wähle nun $v:=u_b-u_d \in A$. Dies ergibt
  $\norm{u_b-u_d}^{2} =0$ und somit folgt $u_b = u_d$\\
  Die Existenz:\\
  Der endliche dimensionale Teilraum A$\subseteq$V besitzt eine Basis
  $(b_1,\dots, b_n)$ mit $n:=dim V$. Die gesuchte Approximation
  $u_b\in A$ lässt sich
  durch die Basis in folgender Form darstellen
  \begin{gather}
   u_b = \sum_{k=1}^n a_k b_k
  \end{gather}
  Dies wird in die notwendige Orthogonalitätsbedingung
  \slideref{Satz}{bestapproximation} eingesetzt.
  \begin{gather}
   \scal(u-\sum_{k=1}^n a_k b_k,v)=\scal(u,v)-\sum_{k=1}^n a_k\scal(b_k,v)=0
   \quad \forall v\in A
   \end{gather}
 Dies ist bei der Wahl von $v:=b_i \quad i=1,\dots,n$ äquivalent zu dem
 linearen $n$x$n$ Gleichungssystem.
 \begin{gather}
   \sum_{k=1}^n\scal(b_k,b_i) a_k= \scal(u,b_i) \quad i=1,\dots,n
 \end{gather}
 Definiere nun $A,x,b$ wie folgt
 \begin{gather}
  A:=(\scal(b_k,b_i))_{i,k=1}^n \quad x:=(a_k)_{k=1}^n\quad b:=(\scal(u,b_i))_{i=1}^n
 \end{gather}
 Dadurch lässt sich das LGS in der Form $Ax=b$ schreiben.
 Betrachte nun folgendes
 \begin{gather}
  x^{T}Ax =\sum_{i,k=1}^n a_i a_k\scal(b_k,b_i)=\scal(u_b,u_b)\ge 0
 \end{gather}
 $A$ ist folglich positiv definit. Das Gleichungssystem $Ax=b$ ist also für
 jede rechte Seite $b$, das heißt für jedes $u \in V$ eindeutig lösbar.
 Folglich bestimmt die Orthogonalitätsbedingung eindeutig ein Element
 $u_b \in A$, welches dann die Bestapproximation von $u$ ist.
 
\end{proof}

\begin{Definition}{komplement-projektion}
  Sei $W \subset V$ ein Untervektorraum. Dann gilt
  $V = W \oplus W^\perp$, wobei das \define{orthogonale Komplement}
  $W^\perp$ eindeutig definiert ist durch
  \begin{gather}
    W^\perp = \bigl\{ v\in V \big| \scal(v,w) = 0 \quad\forall w\in W\bigr\}.
  \end{gather}
  Die Lösung der Bestapproximationsaufgabe bezeichnen wir mit
  \begin{gather}
    \Pi_W u = u_b\in W
  \end{gather}
  und nennen es die \define{orthogonale Projektion} von $u\in V$ auf $W$.
\end{Definition}

\begin{Lemma}{komp-projekt-wohldefiniert}
  Das orthogonale Komplement und die orthogonale Projektion sind wohldefiniert.
\end{Lemma}

\begin{proof}
  \slideref{Satz}{bestapproximation}.
\end{proof}

\begin{Beispiel}{polynom-bestapproximation}
  Die Aufgabe der Gaußschen Ausgleichsrechnung lautet: finde zu einer
  gegebenen Funktion $f$ das Polynom vom Grad höchstens $n$, das auf
  dem Intervall $[-1,1]$ den mittleren quadratischen Abstand
  minimiert, also $p\in \P_n$ mit
  \begin{gather}
    \int_{-1}^1 \bigl(f(x)-p(x)\bigr)^2 \dx
    = \min_{q\in \P_n} \int_{-1}^1 \bigl(f(x)-q(x)\bigr)^2 \dx.
  \end{gather}
  Die Lösung erfüllt
  \begin{gather}
    \int_{-1}^1 p(x)q(x) \dx = \int_{-1}^1 f(x)q(x) \dx
    \qquad\forall q\in \P_n.
  \end{gather}
\end{Beispiel}

\begin{remark}
  Mit unserem Wissen über die $L^2$ Norm aus \slideref{Definition}{l2-norm} erkennen wir, dass sich
  die Gaußsche Ausgleichsrechnung auch über die Norm formulieren lässt.
  \begin{gather}
    \norm{f-p}_{L^2}^2 = \min_{q\in \P_n} \norm{f-q}_{L^2}^2
  \end{gather}
\end{remark}

\section{Orthogonale Basen}

\begin{Lemma}{gram-system}
  Wählt man eine Basis $\{\phi_i\}$ für $W$, so transformiert wird die
  Orthogonalitätsbedingung in \slideref{Satz}{bestapproximation} zum
  linearen Gleichungssystem
  \begin{gather}
    \matg\vx = \vb.
  \end{gather}
  Hier sind $\vx$ der Koeffizientenvektor der Lösung $u_b$, $\matg$ die
  \define{Gramsche Matrix} und $\vb$ die rechte Seite gegeben durch
\begin{gather}
  g_{ij} = \scal(\phi_i,\phi_j), \qquad
  b_i = \scal(u,\phi_i).
\end{gather}
\end{Lemma}

\begin{remark}
  Das Gleichungssystem hängt nur von der Wahl einer Basis in $W$ ab,
  nicht in $V$.
\end{remark}

\begin{Definition}{ortho-system}
  Eine Menge von Vektoren $\{\phi_1,\dots,\phi_n\}\subset V$ bildet
  ein \define{Orthogonalsystem}, wenn
  \begin{gather*}
    \scal(\phi_i,\phi_j) = 0
    \qquad \forall 1\le i < j \le n.
  \end{gather*}
  Sie ist ein \define{Orthonormalsystem}, wenn zusätzlich
  $\norm{\phi_i} = 1$ für alle Elemente gilt. Ein Orthonormalsystem, das eine Basis bildet, heißt \define{Orthonormalbasis} (\define{ONB}).
\end{Definition}

\begin{Lemma}{ortho-lu}
  Jedes Orthogonalsystem ist linear unabhängig.
\end{Lemma}

\begin{Lemma*}{parseval}{Parsevalsche Gleichung}
  Sei $\{\phi_i\}$ für $i=1,\dots,n$ eine ONB von $V$. dann gilt für
  jedes $v\in V$ mit der Basisdarstellung
  \begin{gather}
    v = \sum_{i=1}^n x_i \phi_i
  \end{gather}
  die Identität
  \begin{gather}
    \norm{v}^2 = \sum_{i=1}^n x_i^2.
  \end{gather}
\end{Lemma*}
\begin{Lemma}{least-squares-orthogonal}
  Bezüglich einer ONB ist die Gramsche Matrix die
  Einheitsmatrix. Damit berechnen sich die Einträge des
  Koeffizientenvektors $\vx$ in \slideref{Lemma}{gram-system} durch
  die einfache Formel
  \begin{gather}
    x_i = b_i = \scal(u,\phi_i).
  \end{gather}
\end{Lemma}

\begin{Theorem*}{gram-schmidt}{Gram-Schmidt-Verfahren}
  Jede linear unabhängige Menge von Vektoren
  $\{v_1,\dots,v_n\}\subset V$ wird mit dem folgenden Verfahren in ein
  Orthonormalsystem $\{\phi_1,\dots,\phi_n\}\subset V$ umgeformt:
  \begin{gather}
    \begin{aligned}
      \phi_1 &= \tfrac1{\norm{v_1}} \,v_1\\
      w_j &= v_j - \sum_{i=1}^{j-1} \scal(v_j,\phi_i)\,\phi_i
      & \quad \phi_j &= \tfrac1{\norm{w_j}}\, w_j
      &\quad j=2,\dots,n
    \end{aligned}
  \end{gather}
  Für alle $1\le k \le n$ gilt
  \begin{gather}
    \operatorname{span}\{\phi_1,\dots,\phi_k\}
    =
    \operatorname{span}\{v_1,\dots,v_k\}
  \end{gather}
\end{Theorem*}

\begin{proof}
  Per Induktion über $n$ zeigen wir Orthogonalität und Normierung.\\

  $Indukionsanfang$ Sei $n=1$.\\
  Wird nur ein Vektor aus dem Raum gewählt, so erfüllt dieser
  die Orthogonalitätsbedingung, da er der einzige Vektor im System ist.
  Wird dieser Vektor zusätzlich normiert erhält man ein Orthonormalsystem.\\
  
  $Induktionsschritt$ Das Verfahren gelte
  für $\{v_1,\dots,v_{n-1}\}$ Vektoren aus V. \\
  $n-1 \rightarrow n$\\
  Sei $(\phi_1,\dots,\phi_{n-1})$ ein Orthonormalsystem\\
  Annahme $w_n$ nicht wohldefiniert. Dann gilt
  \begin{gather}
    w_n = v_n -\sum_{i=1}^{n-1}\scal(v_n,\phi_i)\,\phi_i = 0
  \end{gather}
  In diesem Fall sind $(v_1,\dots,v_n)$ linear abhängig.
  Das ist ein Widerspruch zur Voraussetzung,
  dass $(v_1,\dots,v_n)$ linear unabhängig sind.\\
  $w_n$ wird nun normiert über $\frac{1}{\norm{w_n}} \cdot w_n =\phi_n $.
  Nun zur Orthogonalität:
  \begin{gather}
    \scal(\phi_n,\phi_j)=\scal(v_n,v_j)-
    \sum_{i=1}^{n-1}\scal(v_n,\phi_i)
    \,\underbrace{\scal(\phi_i,\phi_j)}_{=\delta_{ij}}  = 0
    \quad j=1,\dots,n-1
  \end{gather}
\end{proof}

\begin{Algorithmus*}{gram-schmidt}{Gram-Schmidt}
  \lstinputlisting{code/gram-schmidt.py}
\end{Algorithmus*}

\begin{remark}
  Um den Code zu verstehen ist es ratsam ihn zu Beginn einmal durchzugehen und sich Stellen zu
  merke an denen Zuweisungen getätigt werden. Ebenso sollte der Code mit einem einfachen Beispiel
  probiert werden, um die Parallelen zum Verfahren besser zu erkennen.\\  
   1 Es wird eine Funktion mit dem Namen $gram schmidt$. Dieser Funktion wird eine Matrix $v$
      übergeben. In dieser Matrix stehen die Vektoren $v_1$ bis $v_n$ in Spalten. \\ 
   2 Es wird $n$ die Länge der Zeilen(Anzahl der Vektoren) zugewiesen und
      $m$ wird die Länge der Spalten (Anzahl der Einträge im Vektor) zugewiesen. \\
   3 Beginn des GS Verfahrens. Es geht von Vektor 1 bis Vektor $n$ \\
   4 Initialsieren eines Vektors $delta$ der Länge $m$ mit Nullen als Einträge\\
   5 Es wird eine weitere Schleife begonnen in der ein Index i über alle bisher orthogonalisierten
      Vektoren läuft. Dies entspricht der Summe aus dem Verfahren. \\
   6 $r$ ist das Skalarprodukt aus dem Vektor $v_j$ und einem bereits orthogonalisierten Vektor.
      Die Vektoren befinden sich in der Matrix $v$ und über diesen Befehl wird darauf zugegriffen.\\
   7 delta = delta $+$ Skalarprodukt $*$ dem orthogonalen Vektor\\  
   8 Hier wird die zweite for-Schleife wieder verlassen! Es reicht tatsächlich das Einrücken.
      Die Summe wird vom Vektor $v_j$ abgezogen, somit $v_j$ orthogonalisiert und wieder in der
      Matrix $v$ an der richtigen Stelle zugewiesen.  \\
   9  Die Norm von $v_j$ wird berechnet.\\  
   10 Wie im Verfahren wird in  dieser Zeile $v_j$ normiert und in der Matrix $v$ an der Stelle des
       früheren $v_j$ zugewiesen.
\end{remark}

\begin{Beispiel}{gram-schmidt}
  Wir wählen für Polynome das $L^2$-Skalarprodukt aus
  \slideref{Lemma}{l2-norm} und die Basis $\{1,x,\dots,x^{n-1}\}$
  für $\P_{n-1}$. Wir verwenden die Iplementation in
  \slideref{Algorithmus}{gram-schmidt} und messen den Erfolg nach der
  Größe der Nebendiagonaleinträge der Gramschen Matrix.
  \begin{center}
    \begin{tabular}{c|c}
      $n$ & $\max_{i\neq j} \abs{g_{ij}}$ \\
      \hline
      5 & $8.9\cdot 10^{-16}$ \\
      10 & $9.1\cdot 10^{-12}$ \\
      15 & $1.2\cdot 10^{-7}$ \\
      20 & $0.23$
    \end{tabular}
  \end{center}
\end{Beispiel}

\begin{Algorithmus*}{mgs}{Modifizierter Gram-Schmidt}
  \lstinputlisting{code/modified-gram-schmidt.py}  
\end{Algorithmus*}

\begin{remark}
  In diesem Programm wurde der Zwischenschritt über das delta ausgelassen, was mögliche Rundungsfehler
  verringert.\\
  4 Hier wird direkt $r$ mit Nullen initialisiert.\\
  7 Der Vektor $v_j$ wird hier orthogonalisiert, ohne dass die Summe aus dem Verfahren in delta
    zwischengespeichert wird und direkt wieder an entsprechender Stelle in der
    Matrix $v$ zugewiesen.\\
\end{remark}

\begin{Beispiel}{gs-mgs}
  In dieser Tabelle wiederholen wir die Zahlen
  $\max_{i\neq j} \abs{g_{ij}}$ aus \slideref{Beispiel}{gram-schmidt}
  und stellen sie den entsprechenden Ergebnissen des modifizierten
  Verfahrens in \slideref{Algorithmus}{mgs} gegenüber.
  \begin{center}
    \begin{tabular}{c|cc}
      $n$ &  Gram-Schmidt & modifiziert\\
      \hline
      5 & $8.9\cdot 10^{-16}$ & $1.3\cdot 10^{-16}$ \\
      10 & $9.1\cdot 10^{-12}$ & $2.9\cdot 10^{-12}$ \\
      15 & $1.2\cdot 10^{-7}$ & $2.7\cdot 10^{-9}$ \\
      20 & $0.23$ & $3.9\cdot 10^{-5}$
    \end{tabular}
  \end{center}
\end{Beispiel}

\begin{remark}
  Wir sehen, dass die Wahl der Implementation eines Rechenverfahrens
  bei mathematischer Äquivalenz durchaus erheblichen Einfluss auf das
  Ergebnis haben kann. Dieses Phänomen werden wir in
  \Cref{sec:stability} näher untersuchen. Zunächst diskutieren wir
  aber eine weitere Variante der Erzeugung orthogonaler Basen in
  Polynomräumen.
\end{remark}

\section{Drei-Term-Rekursion}

\begin{Satz*}{dreiterm}{Dreiterm-Rekursion}
  Zu jedem Skalarprodukt $\scal(\cdot,\cdot)$ auf dem Raum der
  stetigen Funktionen gibt es genau eine Folge von orthogonalen
  Polynomen $p_k\in \P_k$ mit führendem Koeffizienten eins. Sie
  genügen der Dreiterm-Rekursionsformel
  \begin{gather}
    p_k(x) = (x-a_k)p_{k-1}(x) - b_k p_{k-2}(x),
    \qquad k=1,2,\ldots
  \end{gather}
  mit Startwerten $p_{-1} \equiv 0$ und $p_0 \equiv 1$. Die
  Koeffizienten sind
  \begin{gather}
    a_k = \frac{\scal(x p_{k-1},p_{k-1})}{\scal(p_{k-1},p_{k-1})}
    \qquad\text{und}\qquad
    b_k = \frac{\scal(p_{k-1},p_{k-1})}{\scal(p_{k-2},p_{k-2})}.
  \end{gather}
\end{Satz*}

\begin{proof}
  Siehe \cite[Satz 6.2]{DeuflhardHohmann08}
\end{proof}

\begin{Bemerkung}{dreiterm-normierung}
  Der Beweis ergibt, eigentlich die ``Eindeutigkeit einer Orthogonalfolge bis auf Normierung''. Tatsächlich werden in der Literatur immer wieder veschiedene Normierungen benutzt. Beispiele sind:
  \begin{enumerate}
  \item Führender Koeffizient eins, $p_k = x^k + \dots$
  \item $\norm{p_k} = 1$
  \item $p_k(1) = 1$
  \end{enumerate}
\end{Bemerkung}

\begin{Definition}{legendre-polynome}
  Die \define{Legendre-Polynome} $L_k$ sind definiert durch
  die Dreiterm-Rekursion
  \begin{gather}
    L_{k} = \tfrac{2k-1}{k}x L_{k-1}(x) - \tfrac{k-1}{k} L_{k-2}(x).
  \end{gather}
  Sie sind orthogonal bezüglich des $L^2$-Skalarprodukts in
  \slideref{Lemma}{l2-norm}.
\end{Definition}

\begin{Beispiel}{least-squares-legendre}
  Das Problem der Gaußschen Ausgleichsrechnung war: Zu einer gegebenen
  Funktion $f$ finde $p\in \P_n$, so dass
  \begin{gather}
    \norm{f-q}_{L^2}^2
    = \min_{q\in\P_n} \norm{f-q}_{L^2}^2.
  \end{gather}
  Mit Hilfe der Legendre-Polynome können wir nun die Lösung explizit angeben als
  \begin{gather}
    p(x) = \sum_{i=0}^n \alpha_i L_i(x)
    \qquad\text{mit}\qquad
    \alpha_i = \frac1{\norm{L_i}^2}\int_{-1}^1 f L_i(x)\dx.
  \end{gather}
\end{Beispiel}

\begin{Definition}{chebyshev-polynome}
  Die \define{Tschebyscheff-Polynome} $T_k$ sind definiert durch
  die Dreiterm-Rekursion
  \begin{gather}
    T_{k} = 2x T_{k-1}(x) - T_{k-2}(x).
  \end{gather}
  Sie sind orthogonal bezüglich des Skalarprodukts
  \begin{gather}
    \scal(p,q) = \int_{-1}^1 \tfrac1{\sqrt{1-x^2}} \,p(x)q(x)\dx.
  \end{gather}
\end{Definition}

%%% Local Variables:
%%% mode: latex
%%% TeX-master: "main"
%%% End:


\chapter{Solving Large Sparse Linear Systems}

\section{Motivation: discretization of partial differential equations}

\begin{Example*}{page-rank}{The PageRank Algorithm}
  The page rank of a web page is computed from the Google Matrix
  \begin{gather}
    \matp = d (\matl+\vw \mathbf 1^T) + \tfrac{(1-d)}{n} \id,
  \end{gather}
  where $d \in (0,1)$ is a damping parameter, $\mathbf 1^T = (1,\dots,1)$,
  \begin{gather}
    l_{ij} =
    \begin{cases}
      \nicefrac1{c_i} & \text{if $i$ links to $j$}\\0&\text{else} 
    \end{cases},
    \qquad
    w_i =
    \begin{cases}
      1 &\text{if } c_i=1\\
      0&\text{else}
    \end{cases},
  \end{gather}
  and $c_i$ is the number of links from page $i$.

  The number of web pages is estimated at 4.86 billion pages\footnote{\url{https://www.worldwidewebsize.com/}}.
\end{Example*}

\begin{intro}
  Diffusion problems are modelled by the Poisson equation
  \begin{gather}
    -\Delta u = f.
  \end{gather}
\end{intro}

\begin{Example}{7-point-stencil}
  Let there be a sequence of points $x_k$, $k=0,\dots,n$ such that
  $x_k-x_{k-1} = h$ and an approximating function $u_k = u(x_k)$. The
  second derivative of a function $u(x)$ in a point $x_k$ in the
  interior can be approximated by the 3-point stencil
  \begin{gather}
    u''(x_k) = \Delta_h^2 u(x_k) = - \frac{2 u_k - u_{k-1} - u_{k+1}}{h^2}.
  \end{gather}
  This can be generalized to the Laplacian in two dimensions by the \define{5-point stencil}
  \begin{gather}
    \Delta u_{ij} = -\frac{4 u_{ij}- u_{i-1,j} - u_{i+1,j}- u_{i,j-1} - u_{i,j+1}}{h^2},
  \end{gather}
  and to three dimensions by the \define{7-point stencil}
  \begin{multline}
    \Delta u_{ijk} = \tfrac{-1}{h^2}\Bigl(6 u_{ij}
    - u_{i-1,j,k} - u_{i+1,j,k}
    \\
    - u_{i,j-1,k} - u_{i,j+1,k}
    - u_{i,j,k-1} - u_{i,j,k+1}\Bigr).
  \end{multline}
\end{Example}

\begin{Definition}{sparse-matrix}
  We call a matrix \textbf{sparse}\index{sparse matrix} in strict
  sense, if the number of nonzero entries is much less than the total
  number of entries, typically $\bigo(n)$ instead of $\bigo(n^2)$. The
  distribution of nonzero entries is called \define{sparsity pattern}.

  More broadly, a sparse matrix has the property that its application
  to a vector requires considerably less than $n^2$, typically
  $\bigo(n)$ operations. It also can be stored with considerably less
  than $n^2$ floating point numbers.
\end{Definition}

\begin{Example*}{csr}{Compressed row storage (CSR)}
  A sparse matrix can be stored using two integer fields defining the
  sparsity pattern and a field of floating point values for its
  entries. Let \lstinline!n! be the dimension of the matrix and
  \lstinline!n_nonzero! the total number of nonzero entries.
  \begin{lstlisting}[language=Python]
    import numpy as np
    row_start = np.empty(n, dtype=np.uint)
    column = np.empty(n_nonzero, dtype=np.uint)
    entries = np.emtpy(n_nonzero, dtype=np.double)
  \end{lstlisting}

  The operation $\vy=\mata\vx$ is then implemented as
  \begin{lstlisting}[language=Python]
    for i in range(0,n):
    y[i] = 0.
    for j in range (row_start[i],row_start[i+1):
      y[i] += entries[j]*x[column[j]]
  \end{lstlisting}  
\end{Example*}

\begin{Remark}{algorithmic-matrix}
  In cases where the sparsity pattern and the entries of a matrix are
  known at compile time, a compressed stored matrix can be substituted
  by an algorithm performing the multiplication.

  Thus, we start viewing matrices more as linear operators than as
  rectangular schemes of numbers.

  Such algorithms are of high importance on modern hardware, where the
  time and also the energy cost of computations is much lower than of
  moving data from memory.
\end{Remark}

\begin{Example}{sparse-computation-memory}
  The matrix of the 7-point stencil on a grid of $N=n^3$ points has
  dimension $N$. For $n=100$, this is $N=10^6$. The whole matrix has
  $N^2=10^{12}$ entries, but only $7N \approx 10^7$ are nonzero.
  \begin{enumerate}
  \item Storing the full matrix in double precision requires
    \begin{gather}
     8\cdot 10^{12}\text{B} \approx 10\text{TB}. 
    \end{gather}
    Storing relevant information in
    CSR requires
    \begin{gather}
      12\cdot7\cdot 10^6\text{B} \approx 100\text{MB}.
    \end{gather}
  \item On a hypothetical CPU with $10^9$ multiplication/additions per second, multiplying a vector with this matrix takes
    \begin{gather}
      \text{Full: } 1000\text{sec} \approx 20\text{min},
      \qquad
      \text{CSR: } 7\text{msec}.
    \end{gather}
  \end{enumerate}
\end{Example}

\begin{Example}{sparse-computation-factorization}
  Solving a linear system with the matrix of the 7-point stencil on a $100^3$ grid
  \begin{enumerate}
  \item Full Choleski factorization
    \begin{gather}
      \frac16 \left(100^3\right)^3\text{flops}\approx 10^{17}\text{flops}
      \simeq 10^8\text{sec} \approx 3\text{yrs.}
    \end{gather}
%    \pause
  \item Banded Choleski factorization
    \begin{gather}
      \frac16 10^{16} \approx 10^{15}\text{flops}
      \simeq 10^6\text{sec} \approx 10\text{d}.
    \end{gather}
%    \pause
  \item Cramer's Rule with Laplace Expansion: $\approx 10^{65,000}$ years.
  \end{enumerate}
\end{Example}

\begin{Example*}{von-Neumann-series}{von Neumann series}
  If $\norm{\mata}<1$, then
  \begin{gather}
    (\id-\mata)^{-1} = \sum_{k=0}^{\infty} \mata^k.
  \end{gather}
\end{Example*}

%%% Local Variables:
%%% mode: latex
%%% TeX-master: "main"
%%% End:


\section{Basic iterative methods}

\begin{Definition}{jacobi}
  The \define{Jacobi iteration} for a matrix $\mata\in\Rnn$ and a
  right hand side vector $\vb\in \R^n$ generates the iterate
  $\vx^{(k+1)}\in \R^n$ from $\vx^{(k)}\in \R^n$ as follows:
  \begin{gather}
     x^{(k+1)}_i = \frac1{a_{ii}}\left( b_i - \sum_{j\neq i} a_{ij}x^{(k)}_j\right).
  \end{gather}
\end{Definition}

\begin{Definition}{gauss-seidel}
  The \define{Gauss-Seidel iteration} for a matrix $\mata\in\Rnn$ and a
  right hand side vector $\vb\in \R^n$ generates the iterate
  $\vx^{(k+1)}\in \R^n$ from $\vx^{(k)}\in \R^n$ as follows:
  \begin{gather}
    x^{(k+1)}_i = \frac1{a_{ii}}
    \left( b_i
      - \sum_{j< i} a_{ij}x^{(k+1)}_j
      - \sum_{j> i} a_{ij}x^{(k)}_j
  \right).
  \end{gather}
\end{Definition}

\begin{Definition}{richardson-iteration}
  The \define{Richardson iteration} for a matrix $\mata\in\Rnn$ and a
  right hand side vector $\vb\in \R^n$ generates the iterate
  $\vx^{(k+1)}\in \R^n$ from $\vx^{(k)}\in \R^n$ as follows:
  \begin{gather}
    \vx^{(k+1)} = \vx^{(k)} - 
    \omega_k\left( \mata\vx^{(k)} - \vb
  \right).
\end{gather}
The relaxation parameter $\omega_k$ must be chosen carefully to obtain
convergence.
\end{Definition}

\begin{Definition}{matrix-iteration}
  We call a \define{matrix iteration} any iterative method of the structure
  \begin{gather}
    \vx^{(k+1)} = \matm \vx^{(k)} + \vg,
  \end{gather}
  with an \define{iteration matrix} $\matm\in\Rnn$ and an inhomogeneity $\vg$.
\end{Definition}

\begin{Lemma}{Jacobi-gs-matrices}
  Let $\mata = \matd +\matl+\matu$ be the decomposition of $\mata$
  into the diagonal, and the strict upper and lower triangles,
  respectively. Let $\vb$ be the right hand side of the linear system
  $\mata\vx=\vb$. Then, Jacobi iteration has the matrix form
  \begin{gather}
    x^{(k+1)} = \bigl(\id-\matd^{-1}\mata\bigr) x^{(k)} + \matd^{-1} \vb.
  \end{gather}
  The Gauss-Seidel iteration has the matrix form
  \begin{gather}
    x^{(k+1)} = \bigl(\id-(\matd+\matl)^{-1}\mata\bigr) x^{(k)} + (\matd+\matl)^{-1} \vb.
  \end{gather}
\end{Lemma}

\begin{Theorem*}{bfpt}{Banach fixed-point theorem}
  Let $V$ be a vector space with norm $\norm{\cdot}$ and $M$ be a
  closed subset of $V$. Let $F\colon M\to M$ be a \define{contraction},
  that is, there is a \define{contraction number} $q\in[0,1)$ such that
  \begin{gather}
    \norm*{F(\vx)-F(\vy)} \le q \norm*{\vx-\vy}\qquad\forall \vx,\vy\in M.
  \end{gather}
  Then, there is a unique \define{fixed-point} $\vx^*\in M$ with the property
  \begin{gather}
    F(\vx^*) = \vx^*.
  \end{gather}
  The \define{fixed-point iteration}
  \begin{gather}
    \vx^{(k+1)} = F\bigl(\vx^{(k)}\bigr)
  \end{gather}
  converges to $\vx^*$ for any $\vx^{(0)}\in M$.
%and there holds
%  \begin{gather}
%    \norm{\vx^{(k)}-x^*} \le \frac{q^n}{1-q}\norm{\vx^{(1)}-\vx^{(0)}}.
%  \end{gather}
\end{Theorem*}

\begin{Corollary}{bfp-estimates}
  Let $F$ define a fixed-point iteration with contraction number $q<1$
  and let $x^*$ be the unique fixed-point. Then, the following
  estimates hold:
  \begin{align}
    \norm{x^{(k)} - x^*} &\le \frac{q}{1-q} \norm{x^{(k)}-x^{(k-1)}}\\
    \norm{x^{(k)} - x^*} &\le \frac{q^k}{1-q} \norm{x^{(1)}-x^{(0)}}
  \end{align}
\end{Corollary}  

\begin{Corollary}{matrix-norm-convergence}
  Let $\norm{\matm} < 1$ for some operator norm of a vector norm $\norm{\cdot}$ on $\R^n$. Then, the matrix iteration
  \begin{gather}
    \vx^{(k+1)} = \matm \vx^{(k)} + \vg
  \end{gather}
  converges for any initial value $\vx^{(0)}\in\R^n$.
\end{Corollary}

\begin{Example}{convergence-row-sum}
  If the matrix $\mata\in\Rnn$ is irreducibly diagonally dominant, that is,
  \begin{gather}
    \abs{a_{ii}} \ge \sum_{j=1}^n \abs{a_{ij}},\qquad i=1,\dots,n,
  \end{gather}
  the inequality holds strictly in at least one row, and the matrix is
  irreducible in the sense that for any two indices $i$ and $j$ there
  is a chain of nonzero entries
  \begin{gather}
    a_{i,k_{1}}, a_{k_{1},k_{2}}, a_{k_{2},k_{3}},\dots, a_{k_{m},j}.
  \end{gather}
  Then, the Jacobi and Gauss-Seidel methods are contractions in the
  \putindex{row sum norm} and thus convergent.
\end{Example}

\begin{Example}{matrix-norm-convergence}
  Let $\mata\in\R^{2\times 2}$ be a rotation by 45\textdegree combined
  with a scaling,
  \begin{gather}
    \mata = 0.9
    \begin{pmatrix}
      \cos \tfrac\pi4 &\sin\tfrac\pi4\\
      -\sin\tfrac\pi4&\cos\tfrac\pi4
    \end{pmatrix}.
  \end{gather}
  Its spectral norm is
  \begin{gather}
    \norm{\mata}_2 = 0.9,
  \end{gather}
  while
  \begin{gather}
    \norm{\mata}_1 \ge 0.9\sqrt2 > 1.2
  \end{gather}
  which can be seen by mapping the vector $(1,0)^T$. We conclude, that
  a matrix iteration might be a contraction with respect to one norm,
  but not with respect to another.
\end{Example}

\begin{remark}
  The statement of \slideref{Corollary}{matrix-norm-convergence}
  implies, that it is sufficient to find one vector norm such that the
  iteration is a contraction to prove convergence. The example shows
  that this can only be a sufficient condition, but not a necessary
  one.

  After the following Lemma, we will provide a result which provides us
  with a sufficiant \emph{and} necessary condition.
\end{remark}

\begin{Lemma}{norm-spectral-radius}
  For any matrix $\mata\in\Rnn$ and for any $\epsilon>0$ there exist a
  vector norm $\norm{\cdot}_{\mata,\epsilon}$ and associated operator norm
  denoted by the same symbol, such that
  \begin{gather}
    \rho(\mata) \le \norm{\mata}_{\mata,\epsilon} \le \rho(\mata)+\epsilon.
  \end{gather}
\end{Lemma}

\begin{proof}
  See~\cite[Lemma 3.1]{Rannacher18nla}.
\end{proof}

\begin{Theorem}{matrix-radius-convergence}
  The matrix iteration
  \begin{gather}
    \vx^{(k+1)} = \matm \vx^{(k)} + \vg
  \end{gather}
  converges to a fixed-point $x^*$ for any start vector $x^{(0)}$,
  if and only if for the spectral radius there holds
  \begin{gather}
    \rho(\matm) < 1.
  \end{gather}
  Then, there holds asymptotically only
  \begin{gather}
    \operatorname*{lim\,sup}_{k\to\infty} \frac{\norm{x^{(k+1)}-x^*}}{\norm{x^{(k)}-x^*}}
    \le \rho(\matm).
  \end{gather}
\end{Theorem}

\begin{proof}
  See~\cite[Theorem 3.1]{Rannacher18nla}.
\end{proof}

\begin{Remark}{contraction-vs-convergence}
  While \slideref{Theorem}{matrix-radius-convergence} has the
  mathematically more pleasing statement ``if and only if'', its
  assumptions do not establish a contraction. In particular, the norms
  of iterates may grow in early steps of the iteration and only then
  start decreasing, something which is not desired from a practical
  point of view.

  For practical purposes, we typically investigate contraction with
  respect to a suitably chosen norm.
\end{Remark}

\begin{Theorem}{richardson-convergence}
  Let $\mata$ be symmetric, positive definite, such that
  $\sigma(\mata)\subset [\lambda_{\min},\lambda_{\max}]$. Then, the Richardson method converges for
  \begin{gather}
    \omega_k = \omega < \frac2{\lambda_{\max}}.
  \end{gather}
  The optimal relaxation parameter is
  \begin{gather}
    \omega = \frac2{\lambda_{\min}+\lambda_{\max}}.
  \end{gather}
  The contraction number and spectral radius of the iteration matrix are both
  \begin{gather}
    \rho = \frac{\lambda_{max}-\lambda_{\min}}{\lambda_{max}+\lambda_{\min}}
    =\frac{\cond_2(\mata) -1}{\cond_2(\mata) +1}.
  \end{gather}
\end{Theorem}

\begin{proof}
  Homework.
\end{proof}

\begin{Remark}{richardson-convergence}
  The qualitative statement of
  \slideref{Theorem}{richardson-convergence} holds under the weaker
  assumption that $\mata$ is diagonalizable and that the real part of
  all eigenvalues is positive.

  An estimate for the spectral radius can be obtained even without
  assuming that $\mata$ is diagonalizable.
\end{Remark}

\begin{remark}
  Let $e_k$ be some measure of the error of a fixed-point iteration
  with contraction number $\rho$ after $k$ steps. Then, there holds
  \begin{gather}
    e_k \le \rho^k e_0.
  \end{gather}
  Thus, the number of steps needed to obtain a (relative) reduction of
  the error by a prescribed relative tolerance $\epsilon$, that is, to
  achieve $e_k/e_0\le\epsilon$, is
  \begin{gather}
    \label{eq:krylov:steps-convergence}
    k \ge \frac{\log\epsilon}{\log\rho}.
  \end{gather}
  From this formula, we realize that the number of steps of such a
  method grows linearly with the logarithm of the tolerance. Further,
  it is inverse proportional to the logarithm of the contraction
  number $\rho$.

  Note that logarithms with respect to any base can be used
  in~\eqref{eq:krylov:steps-convergence}, since the quotient is
  independent of the base.
\end{remark}

\begin{Definition}{convergence-rate-logarithmic}
  The (logarithmic) \define{convergence rate} of a contraction with
  \putindex{contraction number} $\rho$ is
  \begin{gather}
    r_c = -\log_{10} \rho.
  \end{gather}
  
  An iteration with convergence rate $r_c$ yields an error reduction
  of $10^{-k}$ in $\nicefrac{k}{r_c}$ iteration steps.
\end{Definition}

\begin{remark}
  The logarithmic convergence rate allows us to directly compare two
  iterative methods. Let's say, there are two iterations $M_1$ and
  $M_2$ and the convergence rate of $M_2$ ist twice the rate of
  $M_1$. Then, given an initial vector, $M_1$ will need twice as many
  steps compared to $M_2$ to reach the same accuracy.

  This implies that $M_2$ is favorable, if the effort for each step is
  less than twice the effort for $M_1$. If it is more than twice, then
  $M_1$ is the faster method in spite of the slower convergence.
\end{remark}

\begin{Corollary}{convergence-rate-logarithmic}
  Let there be iterative methods with numerical efforts $n_i$ and
  logarithmic convergence rates $r_i$. Then, the most efficient method
  is the one where $\nicefrac{n_i}{r_i}$ is smallest.
\end{Corollary}

\begin{remark}
  Typically, it is observed that iterative methods will start with
  faster convergence, but then slow down, such that after some steps
  the error reduction is almost equal to the theoretical contraction
  number. This is due to the fact that eigenvectors corresponding to a
  less dominant eigenvalue are reduced fast and become
  irrelevant. Hence, we can also estimate the contraction number from
  a running iteration.

  Given three consecutive vectors, the contraction property tells us
  that
  \begin{gather}
    \norm{x^{(k+1)} - x^{(k)}} \le \rho \norm{x^{(k)}-x^{(k-1)}}.
  \end{gather}
  If we now assume that we are in a regime where this is almost equal, we obtain
  \begin{gather}
    \rho \gtrapprox \frac{\norm*{x^{(k+1)} - x^{(k)}}}{\norm*{x^{(k)} - x^{(k-1)}}}.
  \end{gather}
  This estimate can be improved if it is monitored over serveral steps
  to gain confidence.

  Note that once $\rho$ has been estimated, the error can be estimated by
  \slideref{Corollary}{bfp-estimates}, namely
  \begin{gather}
    \label{eq:iterations:obscon:1}
    \norm{x^{(k)} - x^*} \le \frac{\rho}{1-\rho} \norm{x^{(k)}-x^{(k-1)}}.
  \end{gather}
  Observe how this estimate deteriorates if $\rho$ is close to one.
\end{remark}

\begin{Definition}{convergence-rate-observed}
  The average \define{observed convergence rate} of $k$ steps of a
  method with error measure $e_k$ is
  \begin{gather}
    \overline{r_c} = \frac1k \log_{10}\frac{e_0}{e_k}.
  \end{gather}
\end{Definition}

\begin{remark}
  The averaging over several steps serves the purpose of making the
  estimate more robust. But, as already discussed in the previous
  remark, the actual contraction in the first steps is often much
  better than in later steps. Therefore, it may be advisable not to
  start with the initial error, but with the error at a later step and
  modify the estimate accordingly.

  The measure for the error may be a computation of the actual error
  for model cases where the solution is known. It might also be an
  estimate of the actual error like in
  equation~\eqref{eq:iterations:obscon:1}.
\end{remark}

%%% Local Variables:
%%% mode: latex
%%% TeX-master: "main"
%%% End:


\section{Krylov-space methods}
\input{Krylov}

\chapter{Large Sparse Eigenvalue Problems}

\begin{intro}
  In this chapter we study algorithms for finding a small number
  $n_{\text{ev}}$ of eigenvalues and corresponding eigenvectors of a
  large matrix $\mata\in\Rnn$, where $n$ is large, for instance
  $n>10^6$. Like in the previous chapter, this is achieved by
  projecting thw peoblem into smaller subspaces, where we can solve
  the eigenvalue problem using the methods of the first chapter.
\end{intro}

\section{Projected subspace iteration}
\begin{Definition}{galerkin-ev}
  Let $\mata\in\Rnn$. Then, the solutions to the eigenvalue problem
  \begin{align}
    \tilde\vx &\in K,\\
    \mata\tilde\vx - \mu\tilde\vx&\perp L.
  \end{align}
  are called the \define{Galerkin
    approximation} of the eigenvalue value problem in a subspace $K$ orthogonal to a subspace
  $L$. In the case $K=L$, it is also called the \define{Rayleigh-Ritz approximation}.
\end{Definition}

\begin{Algorithm*}{rayleigh-ritz}{Rayleigh-Ritz method}
  \begin{enumerate}
  \item Compute an orthonormal basis $\matv_m$ for the subspace $K$ and let
    \begin{gather}
      \matb_m = \matv_m^T\mata\matv_m \in \R^{m\times m}.
    \end{gather}
  \item Compute the eigenvalues of $\matb_m$ and select $k\le m$ ``desired'' eigenvalues
    \begin{gather}
      \mu_1, \dots,\mu_k.
    \end{gather}
  \item Compute the eigenvectors $\vy_1,\dots,\vy_k\in\R^m$ of $\matb$
    and the corresponding approximate eigenvectors of $\mata$:
    \begin{gather}
      \tilde\vu_i = \matv_m \vy_i,\qquad i=1,\dots,k.
    \end{gather}
  \end{enumerate}
\end{Algorithm*}

\begin{Definition}{ritz-values}
  Let $\mata$ be a matrix and $\matb = \matv^*\mata\matv$ be its
  projection to the subspace spanned by the orthonormal basis
  $\matv$. Then, we refer to the eigenvalues of $\matb$ as
  \define{Ritz values}.
  For each eigenvector $\vy_i$ of $\matb$, we refer to the vector
  $\vx_i = \matv\vy_i$ as \define{Ritz vector}.
\end{Definition}

\begin{Lemma}{ritz-invariant}
  Let $K\subset\R^n$ be an invariant subspace under the action of
  $\mata$. Then, the Ritz values and associated Ritz vectors of the
  Rayleigh-Ritz approximation are exact eigenpairs of $\mata$.
\end{Lemma}

\begin{remark}
  A variation of \slideref{Algorithm}{rayleigh-ritz} computes the
  Schur vectors of $\matb$ instead of the eigenvectors. For symmetric
  matrices, they are actually the same vectors, but for
  non-diagonalizable matrices, this variant is preferred as it does
  not require cumbersome computation of invariant subspaces.
\end{remark}

\begin{intro}
  We now focus on the Rayleigh-Ritz approximation of real, symmetric
  matrices. The results transfer immediately to the complex symmetric
  (Hermitian) case, but we avoid formulating with complex numbers.
\end{intro}

\begin{Lemma}{Courant-Fischer-Ritz}
  Let the eigenvalues of a symmetric matrix $\mata\in\Rnn$ be ordered such that
  \begin{gather}
    \lambda_1\ge \lambda_2\ge \dots \ge \lambda_n.
  \end{gather}
  Let $\mu_1,\dots,\mu_m$ be their Rayleigh-Ritz approximation in the
  subspace $K$. Then, there holds
  \begin{gather}
    \lambda_i \ge \mu_i, \qquad i=1,\dots,m.
  \end{gather}
\end{Lemma}

\begin{Lemma}{ritz-rayleigh-estimate}
  Let $(\lambda,\vu)$ be an eigenpair of the symmetric matrix
  $\mata\in\Rnn$. Let $\P_K$ be the orthogonal projection to the
  subspace $K$. Then, the \putindex{Rayleigh quotient}
  $R_\mata(P_K \vu)$ admits the estimate
  \begin{gather}
    \abs*{\lambda-R_\mata(P_K \vu)}
    \le \norm{\mata-\lambda\id}
    \frac{\norm{\vu-P_K\vu}^2}{\norm{P_K\vu}^2}.
  \end{gather}
\end{Lemma}


\begin{Algorithm*}{ev-projection}{Projected subspace iteration}
  \begin{algorithmic}[1]
    \Require $\matx^{(0)} = (\vx_1,\dots,\vx_m)$
    \For {$k=1,\dots$ until convergence}
    \State $\matw \gets \mata\matx^{(k-1)}$
    \State $\matv \gets ONB(\matw)$ \Comment{QR factorization}
    \State $\matb \gets \matv^*\mata\matv$ \Comment{Projected matrix}
    \State $\maty \gets EV(\matb)$ \Comment{Eigenvectors}
    \State $\matx^{(k)} \gets \matv\maty$
    \EndFor
  \end{algorithmic}
\end{Algorithm*}

\begin{Theorem}{ritz-convergence}
  Let $\esp{1,\dots,m}$ be the space spanned by the first (by
  magnitude of eigenvalue) $m$ eigenvectors and let $\matp$ be the
  orthogonal projector onto it. Let $S_k$ be the subspace spanned by
  the basis $\matx^{(k)}$.
  
  If $\matp S_0 = \esp{1,\dots,m}$, then there holds: for each
  eigenvector $\vv_i$ of $\mata$, $i=1,\dots,m$, there is a vector
  $\vy_i\in S_0$ such that $\matp\vy_i = \vv_i$. Furthermore,
  \begin{gather}
    \norm{\vv_i - \matp_{S_k} \vv_i}_2 \le \norm{\vv_i-\vy_i}
    \left(\abs*{\frac{\lambda_{m+1}}{\lambda_i}}+\epsilon_k\right)^k,
  \end{gather}
  where $\matp_{S_k}$ is the orthogonal projector onto $S_k$ and
  $\epsilon_k\to 0$ as $k \to \infty$.
\end{Theorem}

\begin{proof}
  See~\cite[Theorem 5.2]{Saad11}
\end{proof}

\begin{remark}
  The estimates are all with respect to the $m+1$-st eigenvalue. This
  differs from the standard subspace iteration in the first chapter,
  where we always deal with the separation to the following
  eigenvalue.
  
  Thus, it is a reasonable approach to choose $m > n_{\text{ev}}$,
  such that for all relevant eigenvalues the quotient
  $\lambda_{m+1}/\lambda_i$ is sufficiently small. This is
  particularly true if there is a known gap in the spectrum.

  With respect to this property, this method seems superior to the
  orthogonal subspace iteration. Be aware though that a Schur
  decomposition by QR factorization or the computation of eigenvalues
  and eigenvectors of $\matb$ relies on methods from the first
  chapter.
\end{remark}



%%% Local Variables:
%%% mode: latex
%%% TeX-master: "main"
%%% End:


\section{The Arnoldi/Lanczos method}

\begin{intro}
  In this section, we investigate Krylov space projection methods in
  more detail. All results hold, with the due simplifications for the
  Lanczos method as well. Note that the \define{Arnoldi factorization}
  in \slideref{Definition}{arnoldi-factorization} is a reformulation
  of \slideref{Theorem}{arnoldi-projection} and of
  \slideref{Lemma}{arnoldi-w}.
\end{intro}

\begin{notation}
  As with all methods for computing eigenvalues of large, sparse
  systems, we will only compute a subset of the eigenvalues. An
  important concept of the methods developed below is the idea ow
  \define{wanted eigenvalues} and \define{unwanted eigenvalues}.

  Given a set of Ritz values or computed eigenvalues, we will always
  assume that we can decide which of them are wanted. The complement
  set is then unwanted.

  Note that there might not be a clear mathematical definition of
  wanted eigenvalues, as long as at any step of the iteration we can
  select them. In particular, a Ritz value flagged as wanted in one
  step of the iteration might be unwanted in the next step, when
  another one fits the criteria better.

  Typical criteria are greatest, smallest, greatest/smallest real
  value, and greatest/smallest by magnitude.
\end{notation}

\begin{Algorithm*}{ev-arnoldi}{Arnoldi for eigenvalues}
  Given an initial vector $\vv_1$ with $\norm{\vv_1}=1$.
  \begin{enumerate}
  \item Compute the Arnoldi basis $\matv_{m+1}$ and the projected matrix
    \begin{gather}
      \matH_m = \matv_m^*\mata\matv_m.
    \end{gather}
  \item Compute the Schur form of $\matH_m$ by QR iteration
    \begin{gather}
      \matr_m = \matq_m^* \matH_m\matq_m.
    \end{gather}
  \item Use the Ritz values $\lambda_i^{(m)}=r_{ii}^{(m)}$ .
  \item Compute the approximate Schur vectors $\matv_m\matq_m$.
  \end{enumerate}
\end{Algorithm*}

\begin{remark}
  The previous algorithm only shows the basic version of Arnoldi's
  method for eigenvalues in complex arithmetic. Modifications which we
  have applied to the QR iteration can be applied here as well. In
  particular,
  \begin{enumerate}
  \item If the matrix $\mata$ is normal, we can compute the
    eigenvalues and eigenvectors of $\matH_m$ in step 2, such that
    step 4 yields the Ritz vectors right away. This is not advisable
    if $\mata$ is not normal, since eigenvector computation might be
    unstable.
  \item The method can be applied in real arithmetic. If then $\mata$
    is nonsymmetric and has compex eigenvalues, $\matr_m$ is the real
    Schur form and the computation of Ritz values is modified
    accordingly.
  \end{enumerate}
\end{remark}

\begin{Definition}{arnoldi-factorization}
  The result of $m$ steps of the Arnoldi algorithm is the
  \define{Arnoldi factorization}
  \begin{gather*}
    \label{eq:ev-arnoldi-projection}
    \mata\matv_m = \matv_m\matH_m + h_{m+1,m}\vv_{m+1}\ve_m^*,
  \end{gather*}
  where $\matv_m\in\C^{n\times m}$ is the matrix of mutually
  orthogonal \define{Arnoldi vectors} and $\matH_m\in\R^{m\times m}$
  is Hessenberg. There holds
  \begin{gather}
    \matv_m^*\vv_{m+1} = 0.
  \end{gather}
\end{Definition}

\begin{Lemma}{ev-arnoldi-a-posteriori}
  Let $(\lambda_i^{(m)},\vy_i^{(m)})$ be an eigenpair of the projected
  matrix $\matH_m$ of Arnoldi's method. Then, the Ritz vector
  $\vu_i^{(m)} = \matv_m \vy_i^{(m)}$ has a residual represented as
  \begin{gather}
    \left(A-\lambda_i^{(m)}\identity\right)\vu_i^{(m)}
    = h_{m+1,m}\ve_m^*\vy_i^{(m)}\vv_{m+1}.
  \end{gather}
  In particular, its norm is
  \begin{gather}
    \norm*{\left(A-\lambda_i^{(m)}\identity\right)\vu_i^{(m)}}_2
    =  \abs{h_{m+1,m}\ve_m^*\vy_i^{(m)}}.
  \end{gather}
  Note that $\ve_m^*\vy_i^{(m)}$ is the $m$-th entry of
  $\vy_i^{(m)}$. In particular, we can estimate the residual of the
  Ritz pair without computing the Ritz vector.
\end{Lemma}

\begin{proof}
  See also~\cite[Proposition 6.8]{Saad11}.
  Multiplying~\eqref{eq:ev-arnoldi-projection} from the
  right with $\vy_i^{(m)}$, we obtain
  \begin{align}
    \mata\matv_m\vy_i^{(m)}
    &= \matv_m\matH_m\vy_i^{(m)} + h_{m+1,m}\vv_{m+1}\ve_m^*\vy_i^{(m)}\\
    &=\lambda_i^{(m)} \matv_m\vy_i^{(m)}
      + h_{m+1,m}\bigl(\ve_m^*\vy_i^{(m)}\bigr)\vv_{m+1},
  \end{align}
  which is the result as soon as we enter $\vu_i^{(m)} = \matv_m \vy_i^{(m)}$.
\end{proof}

\begin{Corollary}{ev-arnoldi-breakdown}
  If the Arnoldi algorithm breaks down, the generated Ritz vectors are
  eigenvectors.
\end{Corollary}

\begin{remark}
  While this breakdown was ``lucky'' for linear solvers, it is not
  entirely so in the case of eigenvalue computations. On the one hand,
  we have the preceding corollary, which states that the computed Ritz
  vectors are actually eigenvectors and the Ritz values are
  eigenvalues of $\mata$. This is good, if all desired eigenpairs are
  in this set.

  On the other hand, a breakdown actually
  breaks the algorithm if the number of desired eigenvalues is higher
  than the dimension of the Krylov space. In this case, we will have
  to insert a new initial vector during the computation of the Krylov
  space.

  As always with numerical computations, a clear breakdown is
  unlikely, but a near breakdown may happen more likely. Detecting
  this is related to loss of orthogonality.
\end{remark}

\begin{Definition}{arnoldi-locking}
  In the case of a breakdown at step $m$ of the Arnoldi algorithm,
  we restart the algorithm after \define{locking} the wanted eigenvalues.

  \begin{enumerate}
  \item Choose the wanted eigenvalues from the computed ones. Store
    the corresponding eigenvectors $\vu_1,\dots,\vu_k$.
  \item Choose a new initial vector $\vv_1$ which is orthogonal to all
    previously converged eigenvectors $\vu_1,\dots,\vu_m$, as the
    eigenvalues $\lambda_{k+1},\dots,\lambda_m$ are unwanted.
  \item Build the Arnoldi basis, but orthogonalize not only with
    respect to the vectors $\vv_1,\dots$, but also with respect to
    $\vu_1,\dots,\vu_k$.
  \end{enumerate}
  This concept can be repeated until the desired number of
  eigenvectors has been found. This may also involved removing
  eigenvectors of previous runs, if their eigenvalues are not wanted
  anymore.
\end{Definition}

\begin{intro}
  The size of the matrix $\matv_k$ of the Arnoldi algorithm is growing
  in each step. Thus, if the algorithm is converging slowly, memory
  and reorthogonalization become a limiting factor. Thus, we have to
  avoid growth of this basis. The two most obvious options have
  considerable drawbacks:
  \begin{enumerate}
  \item Restarting after $m$ steps by choosing a single column vector
    from $\matv_m$ or a single linear combination of them eliminates
    most information on the spectral structure gathered in these $m$
    steps. Indeed, we would have to make sure that the new initial
    vector has dominant components in the wanted eigenspaces and
    small components in the others.
  \item A windowing technique only considering the last $m$ vectors
    from $\matv_k$ might eliminate the desired eigenvectors, which
    then have to reenter in the following steps. Furthermore, the
    spectrum of $\matH_m$ is not related to the spectrum of $\mata$
    anymore, if the basis $\matv_m$ is not orthogonal. Hence, we
    do not consider this option.
  \end{enumerate}
\end{intro}

\begin{intro}
  In the previous chapter, we have seen the restart of GMRES in
  \slideref{Algorithm}{gmres-restart}. When this algorithm is applied
  here, we have to make sure that the initial vector for the restart
  is ``rich'' in components of the wanted eigenspaces, but has small
  components in the unwanted directions. In principle, we could use a
  technique like in the restart after a breakdown, but it comes at a
  high cost. Hence, we consider a different option.
\end{intro}

\begin{Definition}{polynomial-filtering}
  By \define{polynomial filtering} we denote a method for prerocessing
  the initial vector for an Arnoldi process by applying a matrix
  polynomial. Given a vector $\tilde\vv$, the initial vector is obtained by
  \begin{gather}
    \vw = p(\mata) \tilde \vv,\qquad \vv_1 = \frac{\vw}{\norm{\vw}}.
  \end{gather}
  The polynomial should be chosen such that its values are large on
  the wanted eigenvalues and small on the unwanted ones.
\end{Definition}

\begin{intro}
  We will now describe how we can use the Arnoldi factorization to
  implement polynomial filtering in a fairly cheap way. To this end,
  let us assume that the filter polynomial is given in factored form,
  namely
  \begin{gather}
    p(x) = (x-\sigma_p)\dots(x-\sigma_2)(x-\sigma_1).
  \end{gather}
  We apply the first factor not only to the initial vector $\vv_1$,
  but to the whole basis $\matv_m$. Hence, we obtain
  \begin{align}
    (\mata-\sigma_1\id_n)\matv_m
    &= \mata\matv_m - \sigma_1\matv_m\\
    &= \matv_m\matH_m - \sigma_1 \matv_m + h_{m+1,m} \vv_{m+1} \ve_m^*\\
    &= \matv_m (\matH_m-\sigma_1\id_m) + \vv_{m+1} \vw_m^*,
  \end{align}
  where we abbreviated $\vw_{m+1} =  h_{m+1,m} \ve_m$.
  Now, we apply a step of the shifted QR iteration in
  \slideref{Algorithm}{shifted-qr-iteration} to obtain
  \begin{gather}
    \matq_1\matr_1 = \matH_m-\sigma_1\id_m,
    \qquad
    \matH_m^{(1)} = \matr_1\matq_1+\sigma_1\id_m.
  \end{gather}
  Hence,
  \begin{gather}
    \label{eq:ev-arnoldi:1}
    (\mata-\sigma_1\id_n)\matv_m = \matv_m \matq_1\matr_1
    + \vv_{m+1} \vw_{m+1}^*.
  \end{gather}
  Multiplying with $\matq_1$ from the right yields
  \begin{gather}
    (\mata-\sigma_1\id_n)\matv_m\matq_1 = \matv_m \matq_1 \matr_1\matq_1
    + \vv_{m+1} \vw_{m+1}^* \matq_1.
  \end{gather}
  Finally, adding $\sigma_1\matv_m\matq_1$ from both sides, we obtain
  \begin{gather}
    \mata\matv_m\matq_1 = \matv_m \matq_1 \matH_m^{(1)}
    + \vv_{m+1} \vw_{m+1}^* \matq_1.
  \end{gather}
  Letting $\matv_m^{(1)} = \matv_m\matq_1$, which is still an orthogonal set since $\matq_1$ is an orthogonal matrix, and
  $\vw_{m+1}^{(1)} = \matq_1^*\vw_{m+1}$, we have
  \begin{gather}
    \label{eq:ev-arnoldi:2}
    \mata\matv_m^{(1)} = \matv_m^{(1)} \matH_m^{(1)}
    + \vv_{m+1} \bigl[\vw_{m+1}^{(1)}\bigr]^*.
  \end{gather}
  Multiplying~\eqref{eq:ev-arnoldi:1} with
  $\ve_1$, we see that
  \begin{gather}
    (\mata-\sigma_1\id) \vv_1 = \matv_m\matq_1 \matr_1\ve_1
    = r_{11} \matv_m\matq_1 \ve_1 = r_{11} \matv_m^{(1)} \ve_1
    = r_{11} \vv_1^{(1)},
  \end{gather}
  we see that $\vv_1^{(1)}$ is parallel to
  $(\mata-\sigma_1\id) \vv_1$, which means, it is obtained from
  $\vv_1$ by filtering with the polynomial $x-\sigma_1$.
  
  The transformation $\matH_m^{(1)} = \matq_1^*\matH_{m}\matq_1$ can
  be implemented by $m-1$ Givens rotations according to
  \slideref{Algorithm}{Hessenberg-qr-2}. Hence, the last two entries
  of $\vw_{m+1}^{(1)}$ are nonzero, which is the only difference
  between this representation and the Arnoldi factorization. This
  means on the other hand, that the first $m-1$
  columns~\eqref{eq:ev-arnoldi:2} are indeed an Arnoldi factorization
  starting with $\vv_1^{(1)}$.

  We can repeat the same process with
  $\matq_2\matr_2 = \matH_m^{(1)} - \sigma_2\id_m$ to obtain
  \begin{gather}
    \mata\matv_m^{(2)} = \matv_m^{(2)} \matH_m^{(2)}
    + \vv_{m+1} \bigl[\vw_{m+1}^{(2)}\bigr]^*,
  \end{gather}
  where $\matv_m^{(2)} = \matv_m^{(1)}\matq_2$,
  $\matH_m^{(2)} = \matq_2^*\matH_{m}^{(1)}\matq_2$, and
  $\vw_{m+1}^{(2)} = \matq_2^*\vw_{m+1}^{(2)}$.
  With the same arguments as before,
  \begin{gather}
    \vv_1^{(2)} \sim (\mata-\sigma_2\id)\vv_1^{(1)}
    \sim (\mata-\sigma_2\id) (\mata-\sigma_1\id)\vv_1.
  \end{gather}
  Hence, we have constructively proven the following lemma.
\end{intro}

\begin{Lemma}{filtering-restart}
  Let $\matv_m$ and $\matH_m$ be the result of $m$ steps of the
  Arnoldi method with initial vector $\vv_1$. Let $\matH_m^{(p)}$ be
  result of $p$ steps of the shifted QR iteration with shift
  parameters $\sigma_1,\dots,\sigma_p$ and resulting orthogonal
  transformations $\matq_1,\dots,\matq_p$ applied to $\matH_m$.

  Then, the first $m-p$ columns of
  $\matv_m^{(p)} = \matv_m\matq_1\dots\matq_p$ and of $\matH_m^{(p)}$
  correspond to the result of $m-p$ steps of the Arnoldi method with
  initial vector
  \begin{gather}
    \hat\vv_1 = (\mata-\sigma_p\id)\dots(\mata-\sigma_2\id)(\mata-\sigma_1\id)
    \vv_1.
  \end{gather}
\end{Lemma}

\begin{intro}
  The algorithm induced by the preceding lemma allows us to apply
  polynomial filtering to an already existing Arnoldi factorization in
  a fairly cheap manner. The result is a factorization of shorter
  length ``as if'' it had been started with the filtered initial
  vector. That's why the algorithm is called \define{implicit
    restart}.

  The question remaining is the choice of filter parameters
  $\sigma_i$. If these are good approximations to eigenvalues of
  $\mata$, then we expect the filtering to effectively suppress
  eigenspaces for these eigenvalues. The approximations themselves can
  also be obtained from the Arnoldi factorization by using the Ritz
  values.
\end{intro}

% \begin{Lemma}{arnoldi-truncation}
%   Let $m=k+p$ and let the result of $m$ Arnoldi steps be given by
%   \begin{gather}
%     \mata\matv_m = \matv_m\matH_m + \vr_m\ve_m^*.
%   \end{gather}
%   This factorization can be truncated by $p$ shifted QR steps to a
%   Arnoldi factorization
%     \begin{gather}
%     \tilde\mata\tilde\matv_k = \tilde\matv_k\tilde\matH_k + \tilde\vr_k\ve_k^*,
%   \end{gather}
%   where $\tilde\matv_k$ consists of the first $k$ columns of $\matv_m\matq$, $\tilde\matH_k$ is the upper left $k\times k$-block of $\matq^*\matH_m\matq$ and
%   \begin{gather}
%     \tilde\vr_k = \beta_k\matv_m\matq \ve_{k+1} + q_{mk} \vr_m ,
%   \end{gather}
%   and $\matq=\matq_1\dots\matq_p$ is the product of the factors in the
%   shifted QR steps.
% \end{Lemma}

% \begin{proof}
%   See~\cite[Section 4.5.1]{BaiDemmelDongarraRuhevanderVorst00}.
% \end{proof}

\begin{Algorithm*}{implicit-restart}{Implicit restart with locking and purging}
  Given the Arnoldi basis $\matv_m$ and the projected matrix $\matH_m$
  as well as the list of previously locked eigenpairs.
  \begin{enumerate}
  \item Compute the eigenvalues and eigenvectors.
  \item For each converged eigenvalue (\slideref{Lemma}{ev-arnoldi-a-posteriori}), compare to the locked eigenvalues.
  \begin{enumerate}
  \item If the newer one is preferred, \define{purge} the old one from the locked list and \define{lock} the new pair.
  \item If the newer one is not preferred, do not lock it, but remove
    it in the next step
  \end{enumerate}
\item Use all converged eigenvalues and additional unwanted
  eigenvalues of $\matH_m$, a total of $p$ for the implicit restart of
  \slideref{Lemma}{filtering-restart}. The result is an Arnoldi basis
  of length $k=m-p$.
\item Run the necessary $p=m-k$ Arnoldi steps to obtain a Krylov space of
  dimension $m$ again and repeat from step 1.
  \end{enumerate}
\end{Algorithm*}

\begin{Algorithm*}{iram}{Implicitly restarted Arnoldi method}
  Choose the maximal lenxth of the Arnoldi basis $m$, the length of
  the truncated basis $k$ and let $p=m-k$.
  \begin{algorithmic}[1]
    \Require Initial vector $\vv\in\C^n$ with $\norm{\vv}_2=1$.
    \State Compute Arnoldi factorization
    $\mata\matv_m = \matv_m\matH_m + \vr_m\ve_m^*$
    \Repeat
    \State Compute eigenvalues and eigenvectors of $\matH_m$
    \State Lock and purge eigenvectors of $\mata$.
    \State Select shifts $\sigma_1,\dots,\sigma_p$
    \For {$j=1,\dots, p$}
    \State $\matH_m \gets \matq_j^*\matH_m\matq_j$
    \Comment{$\matq_j\matr_j = \matH_m - \sigma_j\id$}
    \EndFor
    \State $\matq \gets \matq_1\dots\matq_p$
    \State $\matv_k = \bigl[\matv_m\matq\bigr](:,1:k)$
    \State $\matH_k = \matH_m(1:k,1:k)$
    \State $\vr_k= [\matH_m]_{k+1,k}\vv_{k+1} + q_{mk}\vr_m$
    \State Continue $p$ steps Arnoldi factorization
    $\mata\matv_m = \matv_m\matH_m + \vr_m\ve_m^*$
    \Until convergence
  \end{algorithmic}
\end{Algorithm*}

\begin{remark}
  Like with the Francis QR iteration, the goal of the implicitly
  restarted Arnoldi method is not so much global convergence of the
  matrix $\matH_m$ to an upper triangular matrix, but the consecutive
  locking of sufficiently many eigenvectors.
  Hence, the method has converged, if the desired number $M$ of wanted
  eigenpairs have been locked.

  The selection of wanted eigenpairs is due to a relative criterion in
  a given set of eigenvalues. Hence, once $M$ eigenpairs have been
  locked, we cannot be sure that we have all eigenvalues in the wanted
  set. As an example, if we want the 5 greatest eigenvalues, we may
  not have caught number 3, but have included number 6 instead. In
  order to address this problem, the algorithm is run even after
  enough eigenpairs have been locked and is terminated if no purging
  has occurred for 2--3 iterations.

  Mathematically, there cannot be a proof that we have obtained all
  eigenvalues fitting the criterion. But evidence shows that round-off
  errors make it very likely that the wanted eigenvalues are found.
\end{remark}

\begin{remark}
  Locking and purging of eigenpairs requires bookkeeping of two sets
  of vectors, the locked eigenvectors $\vu_1,\dots,\vu_\ell$, and the
  Arnoldi basis $\vv_1,\dots,\vv_n$, which is again split into the two
  sets of vectors to be kept after truncation and the remainder.

  Standard implementations use a single array of vectors and store the
  locked vectors in the beginning, then the Arnoldi basis. Hence, the
  matrix $\matH_k$ after truncation has the form
  \begin{gather}
    \matH_k =
    \begin{pmatrix}
      \tau_1\\&\ddots\\
      &&\tau_\ell\\
      &&& \tilde \matH_k
    \end{pmatrix},
  \end{gather}
  where the Hessenberg matrix $\tilde \matH_k$ is of dimension
  $k-\ell$.  Hence, $p+k = m-\ell$. In particular, $k$ in the
  algorithm must be chosen not less than the number of desired
  eigenpairs.

  Alternatively, the locked vectors can be stored separately. Then,
  the size of the truncated basis can be chosen independently
  according to considerations of memory consumption and convergence.
\end{remark}

\begin{remark}
  We have so far neglected two important cases. First, if the initial
  vector of the Arnoldi method does not have any component in the
  direction of a desired eigenvector, the method in exact arithmetic
  will not be able to produce this eigenvector. Here, roundoff errors
  come to our rescue. They will produce spurious components in
  direction of such an eigenvector and due to the amplification in
  each step, will become dominant.
  
  Second, if $\mata$ has multiple eigenvalues,
  \slideref{Algorithm}{iram} will only identify a single vector in its
  eigenspace. This is due to the fact that the algorithm cannot
  distinguish between vectors in the same eigenspace and operates on
  each of them equally. Nevertheless, once we lock such an
  eigenvector, it is treated differently and a possible second
  eigenvector for the same eigenvalue may come up in subsequent
  iterations due to roundoff errors.
\end{remark}

\begin{remark}
  \slideref{Algorithm}{iram} is formulated under the assumption, that
  the Arnoldi factorization in the last line does not suffer a
  breakdown or near breakdown. In the first case, we should apply
  locking and purging according to
  \slideref{Definition}{arnoldi-locking} right away. In the second
  case, we expect that the orthogonalization is not accurate and
  reorthogonalization will be necessary.
\end{remark}

\begin{Algorithm*}{arnoldi-reorthogonalization}{Arnoldi with reorthogonalization}
  \algtext*{EndIf}
  \begin{algorithmic}[1]
    \Require $\mata\in\Rnn$; $\quad\vv_1\in\R^n, \norm{\vv_1}_2 = 1$
    \For{$j=1,\dots,m$}
    \State $\vw_j \gets \mata\vv_j$
    \State $\vh_{j} \gets \matv_j^*\vw_j$
    \State $\vw_j \gets \vw_j - \matv_j\vh_{j}$
    \If{$\norm{\vw_j} < \eta \norm{\vh_j}$} \Comment{Reorthogonalization}
    \State $\vg = \matv_j^*\vw_j$
    \State $\vw_j \gets \vw_j - \matv_j\vg$
    \State $\vh \gets \vh + \vg$.
    \EndIf
    \If{$\norm{\vw_j} < \eta \norm{\vg}$}\Comment{Failure}
    \State Choose random $\vv_{j+1}$
    \State $h_{j+1,j} \gets 0$
    \Else
    \State $h_{j+1,j} \gets \norm{\vw_j}_2$
    \State $\vv_{j+1} = \nicefrac{\vw_j}{h_{j+1,j}}$
    \EndIf
    \EndFor
  \end{algorithmic}
\end{Algorithm*}

\begin{remark}
  In exact arithmetic, $\matv_j\vh_j$ and $\vw_j$ are the projections
  of $\mata\vv_j$ onto the range of $\matv_j$ and its orthogonal
  complement, respectively. Thus, if $\theta$ is the angle between
  $\mata\vv_j$ and the range of $\matv_j$, we have
  \begin{gather}
    \cot\theta = \frac{\norm{\vw_j}}{\norm{\vh_j}}.
  \end{gather}
  Thus, the condition in line 5 of the algorithm enforces a
  reorthogonalization, if the angle $\theta$ is too small.

  \slideref{Algorithm}{arnoldi-reorthogonalization} is taken from the
  literature. After a student had implemented it, I have had several
  discussions and it is not clear whether the criterion for breakdown
  is applied in the right way.
\end{remark}

%%% Local Variables:
%%% mode: latex
%%% TeX-master: "main"
%%% End:



\appendix
\chapter{Basics from Linear Algebra}
\section{Bases and matrices}
\subsection{Matrix notation for bases}
\begin{Notation}{column-vectors}
  For a matrix $\mata\in \C^{n\times k}$ with entries $a_{ij}$, we denote its column vectors by $\va_j$ such that
  \begin{gather}
      (\va_j)_i = a_{ij}.
  \end{gather}
  Vice versa, given vectors $\vv_1,\dots,\vv_k$ in $\C^n$, we denote the matrix generated by those column vectors as
  \begin{gather}
      \matv = \bigl( \vv_1,\dots,\vv_k\bigr).
  \end{gather}
\end{Notation}

\begin{Notation}{matrix-linear-combination}
  Let $\vv_1,\dots,\vv_k$ be a set of vectors in $\C^n$ and let the matrix $\matv=(\vv_1,\dots,\vv_k) \in \C^{n\times k}$. Then, for vectors $\va\in\C^k$ there holds
  \begin{gather}
      \vu = \sum_{i=1}^k a_i \vv_i
      \qquad \Longleftrightarrow \qquad
      \vu = \matv \va.
  \end{gather}
  Thus, we can write linear combinations as matrix vector products.
  We will abuse notation in a way that we refer to a matrix $\matv$
  used in this way as a set of vectors or as a basis.
\end{Notation}

\begin{Lemma*}{change-of-basis}{Change of basis}
  Let $\matv = (\vv_1,\dots, \vv_n)$ and $\matw = (\vw_1,\dots, \vw_n)$ be
  bases of $\C^n$. Let $\vu\in \C^n$ be given by coefficient vectors
  $\vx,\vy\in \C^n$, such that
  \begin{gather}
    \vu = \matv\vx = \matw\vy.
  \end{gather}
  Then,
  \begin{gather}
    \vx = \matc \vy,
  \end{gather}
  where the columns of $\matc\in \C^{n\times n}$ are the coefficient
  vectors of the basis vectors $\vw_j$ with respect to the basis
  vectors $\matv$.
\end{Lemma*}

\begin{Corollary}{change-of-basis}
  Let $\vv\in \C^n$ and $\matw$ be a basis for $\C^n$. Then, the
  coefficient vector $\vy$ of a vector $\vx$ with respect to the
  basis $\matw$ is obtained by
  \begin{gather}
    \vy = \matw^{-1} \vx.
  \end{gather}
\end{Corollary}

%%% Local Variables:
%%% mode: latex
%%% TeX-master: "main"
%%% End:

\subsection{Similarity transformations}
\begin{Definition}{similar-matrix}
  Two matrices $\mata,\matb\in\C^{n\times n}$ are called \define{similar}, if there is a nonsingular matrix $\matv\in\C^{n\times n}$, such that
  \begin{gather}
      \mata = \matv \matb \matv^{-1}.
  \end{gather}
  We call the mapping
  \begin{gather}
      \matb \mapsto \matv \matb \matv^{-1}
  \end{gather}
  a \define{similarity transformation}.
\end{Definition}

\begin{Lemma}{similarity-equivalence}
  Similarity is an equivalence relation, that is, it is reflexive,
  symmetric, and transitive.
\end{Lemma}

\begin{Lemma}{matrix-basis-change}
  Let $\phi\colon \C^n\to \C^n$ be a linear mapping represented with
  respect to the canonical basis by the matrix $\mata$, such that
  \begin{gather}
    \phi(\vx) = \mata \vx.
  \end{gather}
  Then, the matrix $\matb = \matv^{-1}\mata\matv$ represents $\phi$
  with respect to the basis $\matv$, namely, if $\vu = \matv \vy$ and
  $\phi(\vu) = \matv \vz$, then
  \begin{gather}
    \vz = \matb \vy.
  \end{gather}
\end{Lemma}

\begin{Theorem*}{Jordan-canonical-form}{Jordan canonical form}
  Every matrix $A\in\C^{n\times n}$ is similar to a matrix with block structure
  \begin{gather}
    \begin{pmatrix}
      J_1\\&J_2\\&&\ddots\\&&&J_k
    \end{pmatrix},
  \end{gather}
  where each block has the form
  \begin{gather}
    J_i = \begin{pmatrix}
      \lambda_i&1\\&\ddots&\ddots\\
      &&\lambda_i&1\\
      &&&\lambda_i
    \end{pmatrix}.
  \end{gather}
\end{Theorem*}


\begin{Definition}{diagonalizable}
  A matrix is called \define{diagonalizable}, if it is similar to a diagonal matrix.
  Equivalently, a diagonalizable matrix has a basis of eigenvectors.
\end{Definition}

\begin{Theorem}{matrix-functions}
  Let $\mata\in \C^{n\times n}$ be diagonalizable such that
  \begin{gather}
    \mata = \matv \matlambda \matv^{-1},
    \qquad \matlambda = \diag(\lambda_1,\dots,\lambda_n).
  \end{gather}
  Then, for any analytic function $f\colon \C \to \C$, the matrix
  $f(\mata)$ is well defined by
  \begin{gather}
    f(\mata) = \matv f(\matlambda) \matv^{-1},
    \qquad f(\matlambda) = \diag\bigl(f(\lambda_1),\dots,f(\lambda_n)\bigr),
  \end{gather}
  if all eigenvalues are in the domain of convergence of the Tayor
  series of $f$.
\end{Theorem}

\begin{proof}
  First, we observe that
  \begin{gather}
    \mata^2 = \matv \matlambda \matv^{-1} \matv \matlambda \matv^{-1}
    = \matv \matlambda^2 \matv^{-1}.
  \end{gather}
  The square of the diagonal matrix $\matlambda$ is easily computed.
  By induction $\mata^k =  \matv \matlambda^k \matv^{-1}$.

  Let now $f(x) = \sum a_k x^k$ be the Taylor series of $f$. Then,
  \begin{gather}
    \begin{split}
      f(\mata)
      &= \sum_{k=0}^\infty a_k  \matv \matlambda^k \matv^{-1}\\
      &= \matv \left(\sum_{k=0}^\infty a_k \matlambda^k\right) \matv^{-1}\\
      &= \matv \diag\left(
        \sum_{k=0}^\infty a_k \lambda_1^k,\dots,
        \sum_{k=0}^\infty a_k\lambda_n^k
      \right)  \matv^{-1}.
    \end{split}
  \end{gather}
  All limits are well defined since we assume that all eigenvalues are
  in the domain of convergence.
\end{proof}

\begin{Theorem}{simultaneous-diagonalization}
  Two diagonalizable matrices $\mata, \matb\in \C^{n\times n}$ can be
  diagonalized by the same set of eigenvectors if and only if they
  commute. namely $\mata\matb = \matb\mata$.
\end{Theorem}

\begin{todo}
  Move this up next time
\end{todo}

\begin{Lemma}{similarity-eigenvalues}
  The eigenvalues of a matrix are invariant under similarity transformations.
\end{Lemma}

%%% Local Variables:
%%% mode: latex
%%% TeX-master: "main"
%%% End:

\section{Inner products and orthogonality}
\begin{Definition}{sesqui}
  A \define{sesquilinear form} on a complex vector space $V$ is a mapping
  \begin{gather}
      a\colon V\times V \to \C,
  \end{gather}
  which is \define{linear} and \define{semi-linear} in the first and second argument, respectively. That is, for all $\vu,\vv,\vw\in V$ and $\alpha\in\C$ holds
  \begin{xalignat}2
  a(\vu+\vv,\vw) &= a(\vu,\vw)+a(\vv,\vw),
  & a(\alpha \vu,\vw) &= \alpha a(\vu,\vw),\\
  a(\vu,\vv+\vw) &= a(\vu,\vv)+a(\vu,\vw),
  & a(\vu,\alpha \vw) &= \overline\alpha a(\vu,\vw).
  \end{xalignat}
  It is \define{Hermitian} or \define{complex symmetric}, if in addition there holds
  \begin{gather}
      \label{eq:inner:symmetry}
      a(\vu,\vv) = \overline{a(\vv,\vu)}.
  \end{gather}
\end{Definition}

\begin{remark}
  A sesquilinear form can be defined equivalently as being semi-linear in the second argument and linear in the first. Both versions can be found in the literature and we follow the convention in the book of Saad.
  
  The author prefers the term ``complex symmetric'' over the common and pompous ``Hermitian'', since it is the natural extension of symmetry to complex vector spaces, as the following asserts. The author would even prefer the term ``symmetric'', which is unfortunately understood in the real way in other publications. In particular,~\eqref{eq:inner:symmetry} implies that $a(\vu,\vu)$ is always real, such that the following makes sense:
\end{remark}

\begin{Definition}{inner-product}
  An \define{inner product} on the complex vector space $V$ is a sesqui-linear form $\scal(\cdot,\cdot)$ on $V$ which in addition is definite, namely, for all $\vv\in V$ there holds
  \begin{gather}
      \scal(\vv,\vv) \ge 0,
  \end{gather}
  and $\scal(\vv,\vv)=0$ implies $\vv=0$.
\end{Definition}

\begin{Example*}{Euclidean-ip}{Euclidean inner product}
  The bilinear form
  \begin{gather}
    \scal(\vx,\vy) = \sum_{i=1}^n x_i \overline{y}_i
  \end{gather}
  defines an inner product on $\C^n$.
\end{Example*}

\begin{Definition}{conjugate-matrix}
  The conjugate $\mata^*\in C^{n\times m}$ of a matrix $\mata\in C^{m\times n}$
  is defined by the relation
  \begin{gather}
    \scal(\vx,\mata^*\vy)_{\C^n} = \scal(\mata\vx,\vy)_{\C^m},
    \qquad\forall \vx\in\C^n,\;\vy\in\C^m.
  \end{gather}
  The conjugate with respect to the Euclidean inner product is
  $\overline\mata^T$, often abbreviated as $\mata^H$.
\end{Definition}

\begin{Definition}{rothonormal-unitary}
  Given an inner product $\scal(\cdot,\cdot)$ in $\C^n$. A set of
  vectors $\vv_1,\dots,\vv_k$ is called \define{orthonormal}, if
  \begin{gather}
    \scal(\vv_i,\vv_j) = \delta_{ij}, \qquad i,j=1,\dots,k.
  \end{gather}
  A matrix $\matq\in C^{n\times n}$ with orthonormal columns is called
  \define{unitary}. There holds
  \begin{gather}
    \matq^*\matq = \id.
  \end{gather}
\end{Definition}


%%% Local Variables:
%%% mode: latex
%%% TeX-master: "main"
%%% End:

\section{Projections}
\begin{Definition}{projection}
  A matrix $\matp\in\Cnn$ is called a \define{projection matrix} if
  \begin{gather}
    \matp^2 = \matp.
  \end{gather}

  It is called \define{orthogonal projection matrix}\index{orthogonal}
  if in addition
  \begin{gather}
    \matp^* = \matp.
  \end{gather}

  The associated operators are called (orthogonal) projection or
  (orthogonal) \define{projector}. If a projector is not orthogonal,
  it is also called \define{oblique}.
  \index{projector!orthogonal}
  \index{projector!oblique}
\end{Definition}

\begin{Lemma}{projection-range}
  Let $\matp\in\Cnn$ be an orthogonal projection matrix and
  $V = \range \matp$. Then, for any $\vx\in\C^n$ there holds
  \begin{gather}
    (\id-\matp)\vx \in V^\perp.
  \end{gather}
  
  Furthermore, the orthogonal projection $\matp$ is uniquely
  determined by its range.
\end{Lemma}

\begin{proof}
  Let $\vx,\vy\in \C^n$ arbitrary. Then,
  \begin{gather}
    \scal(\matp\vy,{(\id-\matp)}\vx) = \scal(\vy,\matp{(\id-\matp)}\vx)
    = \scal(\vv,{(\matp-\matp^2)}\vx) = \scal(\vv,{(\matp-\matp)}\vx) = 0.
  \end{gather}
  Since $V=\range \matp$, this implies that $(\id-\matp)\vx \in V^\perp$.

  Let now $\matp_1,\matp_2$ be orthogonal projectors. Then, for any
  $\vx\in\C^n$ there holds
  \begin{align}
    \norm{(\matp_1-\matp_2)\vx}_2^2
    &= \scal({(\matp_1-\matp_2)}\vx,{(\matp_1-\matp_2)}\vx)\\
    &= \scal(\matp_1\vx,{(\matp_1-\matp_2)}\vx) - \scal(\matp_2\vx,{(\matp_1-\matp_2)}\vx)\\
    &= \scal(\matp_1\vx,{(\id-\matp_2)}\vx)
      + \scal(\matp_2\vx,{(\id-\matp_1)}\vx).
  \end{align}
  if now $\range{\matp_1}=\range{\matp_2}$, then both inner products
  vanish. Thus, the operator norm of $\matp_1-\matp_2$ is zero and the
  projectors must be equal.
\end{proof}

\begin{Definition}{projection-distance}
  Let $U,V\subset \C^n$ be two subspaces with orthogonal projectors
  $\matp_U$ and $\matp_V$, respectively. Then, we define their
  \define{distance} as\index{dist}
  \begin{gather}
    \dist(U,V) = \norm{\matp_U-\matp_V}_2.
  \end{gather}
\end{Definition}

\begin{Example}{projection-distance}
  Since the orthogonal projection is uniquely defined by its range, there holds
  \begin{gather}
    \dist(U,U) = \norm{\matp_U-\matp_U}_2 = 0.
  \end{gather}
  If $U$ contains a vector $\vx$  which is orthogonal to $V$, then $\matp_U\vx=\vx$ and $\matp_V\vx = 0$, thus
  \begin{gather}
    \norm{\matp_U-\matp_V}_2 = \sup_{\vx\in\C^n} \frac{\norm{\matp_U\vx-\matp_V\vx}}{\norm{\vx}} \ge 1.
  \end{gather}
\end{Example}

% \begin{Example}{projection-distance}
%   Let $U = \spann \vu$ and $V=\spann\vv$ with
%   $\norm{\vu} = \norm{\vv} = 1$. Then,
%   \begin{align}
%     \dist(U,V)^2
%     &= \sup_{\vx\in\C^n}\frac{\norm{\matp_U\vx-\matp_V\vx}^2}{\norm{\vx}^2}\\
%     &= \sup_{\vx\in\C^n}\frac{\norm{\scal(\vx,\vu)\vu - \scal(\vx,\vv)\vv}^2}{\norm{\vx}^2}\\
%     &= \sup_{\vx\in\C^n}\frac{\scal(\vx,\vu)^2+\scal(\vx,\vv)^2 - 2\scal(\vx,\vu)\scal(\vx,\vv)\scal(\vu,\vv)}{\norm{\vx}^2}\\
%   \end{align}
% \end{Example}


\begin{Lemma}{projection-complement}
  If $\matp\in\Cnn$ is a projection operator, so is
  $\id-\matp$, and there holds
  \begin{gather}
    \ker{\matp} = \range{\id-\matp}.
  \end{gather}
  Furthermore, there holds
  \begin{gather}
    \C^n = \ker{\matp} \oplus \range{\matp}.
  \end{gather}
\end{Lemma}

\begin{proof}
  \cite[Section 1.12.1]{Saad00}.
\end{proof}

\begin{Lemma}{projection-spaces}
  Any pair of subspaces $N,R\subset \C^n$ such that $\C^n = N\oplus R$
  uniquely defines a projector $\matp$, such that $N = \ker\matp$ and
  $R=\range\matp$. There holds for $\vx\in\C^n$
  \begin{align}
    \matp \vx &\in R\\
    \vx - \matp \vx &\in N.
  \end{align}
  We say $\matp$ projects onto $R$ along $N$.
\end{Lemma}

\begin{proof}
  \cite[Section 1.12.1]{Saad00}.
\end{proof}

\begin{Definition}{projection-spaces-orthogonal}
  Given two subspaces $R,S\subset\C^n$ of equal dimension, a
  projector $\matp$ is said to project onto $R$ orthogonal to $S$, if
  for any $\vx\in\C^n$ there holds
  \begin{align}
    \matp\vx &\in R\\
    \vx-\matp\vx &\perp S.
  \end{align}
\end{Definition}

\begin{Lemma}{projection-spaces-orthogonal}
  Let $R,S\subset\C^n$ be two subspaces of equal dimension. Then, the
  projection of any vector $\vx\in\C^n$ onto $R$ orthogonal to $S$ is
  uniquely defined if and only if no vector in $R$ is orthogonal to
  $S$.
\end{Lemma}

\begin{proof}
  \cite[Lemma 1.36]{Saad00}.
\end{proof}

\begin{Definition}{biorthogonal}
  Two sets of vectors $\{\vv_1,\dots,\vv_m\}$ and  $\{\vw_1,\dots,\vw_m\}$ are called \define{biorthogonal}, if
  \begin{gather}
    \scal(\vv_i,\vw_j) = \delta_{ij}, \qquad i,j=1,\dots,m.
  \end{gather}
  In matrix notation
  \begin{gather}
    \matw^*\matv = \matv^*\matw = \id
  \end{gather}
\end{Definition}

\begin{Lemma}{projection-basis}
  Let $R,S\subset\C^n$ be two subspaces of equal dimension and let
  $\matp$ be the projector onto $R$ orthogonal to $S$. Let
  $\{\vv_1,\dots,\vv_m\}$ and $\{\vw_1,\dots,\vw_m\}$ be biorthogonal
  and bases of $R$ and $S$, respectively. Then,
  \begin{gather}
    \matp = \matv\matw^*.
  \end{gather}
\end{Lemma}

\begin{proof}
  \cite[Section 1.12.2]{Saad00}.
\end{proof}


%%% Local Variables:
%%% mode: latex
%%% TeX-master: "main"
%%% End:


\chapter{Basics from Numerical Analysis}

\section{QR decomposition}
\begin{intro}
  The facts in this section can be found in~\cite[Chapter
  5]{GolubVanLoan83}. In particular, the QR factorization in section
  5.2, Housholder transformations and Givens rotations in 5.1 there.
\end{intro}

\subsection{Definition and existence}
\begin{Definition}{qr-decomposition}
  The \define{QR factorization} of a matrix $\mata\in\C^{m\times n}$
  with $m\ge n$ is the product
  \begin{gather}
    \mata = \matq\matr,
  \end{gather}
  such that $\matq \in\C^{m\times n}$ is unitary and
  $\matr\in \C^{n\times n}$ is upper triangular.
\end{Definition}

\begin{Lemma}{qr-columns}
  Let $\mata = \matq\matr$. Then, the column vectors of $\mata$ and
  $\matq$ admit the relation
  \begin{gather}
    \va_k = \sum_{i=1}^k r_{ik} \vq_i.
  \end{gather}
  If $r_{ii}\neq 0$ for $i=1,\dots,k$, this relation is uniquely
  invertible. In particular,
  \begin{gather}
    \spann{\vq_1,\dots,\vq_k}
    = 
    \spann{\va_1,\dots,\va_k}.
  \end{gather}
\end{Lemma}


\begin{Theorem}{qr-existence}
  Every matrix $\mata\in\C^{m\times n}$ with $m\ge n$ of full rank
  admits a QR factorization. It is unique under the condition that for
  all $i$ there holds $r_{ii} > 0$.
\end{Theorem}

\subsection{Householder transformations}

\subsection{Givens rotation}

%%% Local Variables:
%%% mode: latex
%%% TeX-master: "main"
%%% End:


\section{Matrix norms and spectral radius}

\begin{Definition*}{matrix-norm}{Matrix norm}
  The space of matrices $\Cnn$ is a vector space of dimension $n^2$ and thus any vector space norm is a norm on this space. For an actual matrix norm, we require compatibility with matrix multiplication, namely
  \begin{gather}
    \norm{\mata\matb} \le \norm{\mata}\norm{\matb},
  \end{gather}
  for any matrices where this makes sense. A matrix norm is called \define{operator norm} or \define{induced norm}, if $\mata\colon V\to W$ and
  \begin{gather}
    \norm\mata = \norm{\mata}_{V\to W} = \sup_{\vv\in V}\frac{\norm{\mata\vv}_W}{\norm{\vv}_V}.
  \end{gather}
\end{Definition*}

\begin{Definition}{spectral-radius}
  The \define{spectral radius} of a matrix $\mata\in\Cnn$ is the
  maximal absolute value of its eigenvalues, that is
  \begin{gather}
    \rho(\mata) = \max_{\lambda\in\sigma(\mata)} \abs{\lambda}.
  \end{gather}
\end{Definition}

\begin{Lemma*}{spectral-radius}{Properties of the spectral radius}
  For any matrix norm $\norm\cdot$ and for any matrix $\mata\in\Cnn$ there holds
  \begin{gather}
    \rho(\mata) = \lim_{k\to\infty} \norm{\mata^k}^{1/k}.
  \end{gather}

  The sequence $\vx^{(k)} = A^k\vx$ converges to zero for all
  $\vx\in\C^n$ if and only if $\rho(\mata)<1$.

  For any $\epsilon>0$
  there is a matrix norm $\norm\cdot$ such that
  \begin{gather}
    \norm{\mata} \le (1+\epsilon) \rho(\mata) \qquad \forall \mata\in\Cnn.
  \end{gather}
\end{Lemma*}

\begin{Definition}{matrix-condition}
  The \define{condition number} of a matrix $\mata\in\Cnn$ for a given
  matrix norm is
  \begin{gather*}
    \cond\mata = \norm{\mata}\norm{\mata^{-1}}.
  \end{gather*}
  In particular, we define the \define{spectral condition number}
 $\cond_2\mata = \norm{\mata}_2\norm{\mata^{-1}}_2$.
\end{Definition}

\section{Chebyshev polynomials}

\begin{Definition}{chebyshev-polynomials}
  The \define{Chebyshev polynomials} $\pchebyshev_k$ are defined by the
  three-term recurrence relation
  \begin{gather}
    \pchebyshev_{k}(x) = 2x \pchebyshev_{k-1}(x) - \pchebyshev_{k-2}(x),
  \end{gather}
  with $T_0 \equiv 1$ and $T_1(x) = x$.
  They are orthogonal with respect to the inner product
  \begin{gather}
    \scal(p,q) = \int_{-1}^1 \tfrac1{\sqrt{1-x^2}} \,p(x)q(x)\dx.
  \end{gather}
\end{Definition}

\begin{Lemma}{chebyshev-representation}
  Chebyshev polynomials admit the trigonometric representation
  \begin{gather}
    \pchebyshev_k(x) = \cos(k \operatorname{arccos} x).
  \end{gather}
  
  Furthermore, there holds
    \begin{gather}
    \pchebyshev_k = \frac12
    \left(
      \left(x-\sqrt{x^2-1}\right)^k
      +
      \left(x+\sqrt{x^2-1}\right)^k
    \right),\qquad\abs{x}\ge 1.
  \end{gather}
\end{Lemma}

\begin{Lemma}{chebyshev-abscissae}
  The Chebyshev polynomial $\pchebyshev_n$ has $n$ roots in the
  interval $(-1,1)$ at
  \begin{gather}
    x_k = \cos\left(\pi\frac{(k-\nicefrac12)}{n}\right),
    \qquad k=1,\dots,n.
  \end{gather}
  It alternatingly assumes the values $\pm1$ at the \define{Chebyshev abscissae}
  \begin{gather}
    x_k = \cos\left(\pi\frac kn\right),
  \end{gather}
  wiyh $\pchebyshev_n(1) = 1$ and $\pchebyshev_n(-1) = (-1)^n$.
\end{Lemma}

\begin{Theorem}{chebyshev-minimal-1}
  For every polynomial $p\in \P_n$ with leading coefficient 1 there is $x\in[-1,1]$ such that $\abs{p(x)} \ge \nicefrac1{2^{n-1}}$. There holds
  \begin{gather}
   \frac1{2^{n-1}} \pchebyshev_n(x)
   = \operatorname*{arg min}_{\substack{p\in\P_n\\p = x^n+\cdots}}
   \max_{x\in[-1,1]}\abs{p(x)}.
  \end{gather}
\end{Theorem}

\begin{proof}
  See \cite[Satz 7.19]{DeuflhardHohmann08}.  The three-term recursion
  for Chebyshev polynomials implies that the leading coefficient of
  $\pchebyshev_n$ is $2^{n-1}$.

  Assume that there is another polynomial $p\in \P_n$ with leading
  coefficient $2^{n-1}$ such that
  \begin{gather}
    \max_{x\in[-1,1]} \abs{p(x)} < 1.
  \end{gather}
  Then, $q_n = \pchebyshev_n-p \in \P_{n-1}$. Furthermore, for the
  Chebyshev abscissae $\tilde x_j = \cos(\nicefrac{j\pi}{n})$ with
  $j=0,\dots,n$ there holds
  \begin{xalignat}4
    \pchebyshev_n(\tilde x_j) &= 1,
    & p(\tilde x_j) &< 1
    & q_n(\tilde x_j) &> 0,
    & j&\text{ gerade}\\
    \pchebyshev_n(\tilde x_j) &= -1,
    & p(\tilde x_j) &> -1
    & q_n(\tilde x_j) &< 0,
    & j&\text{ ungerade}.
  \end{xalignat}
  Therefore, 
  $q_n$ changes sign at at least $n$ points and thus has at least $n$ roots.
  From
  $q_n\in\P_{n-1}$ we deduce $q_n=0$ and thus $p=\pchebyshev_n$ as a contradiction.
  After scaling by $2^{n-1}$ we obtain
  \begin{gather}
    \operatorname*{min}_{\substack{p\in\P_n\\p = x^n+\cdots}}
   \max_{x\in[-1,1]}\abs{p(x)} \ge 1,
 \end{gather}
 with equality for the scaled Chebyshev polynomial.
\end{proof}

\begin{Lemma}{chebyshev-growth}
  Let
  \begin{gather}
    K = \bigl\{ p\in \P_n \;\big|\; \max_{x\in[-1,1]} \abs{p(x)} =1 \bigr\}.
  \end{gather}
  Then,
  \begin{gather}
    \abs{\pchebyshev_n(x)} \ge p(x) \qquad \forall p\in K, \quad \forall \abs{x} > 1.
  \end{gather}
\end{Lemma}

\begin{proof}
  We conduct the proof for $x>1$ where $\pchebyshev_n(x) > 0$.
  Let $\tilde p\in K$ such that
  $\abs{\tilde p(y)} \ge \pchebyshev_n(y)$ for some $y>1$. Let
  $\gamma = \pchebyshev_n(y)/\tilde p(y)$ and
  $p(x) = \tilde p(x)\gamma$, such that
  $q(x) = p(x) - \pchebyshev_n(x) \in \P_n$ has a root in $y$.
  Furthermore,
  \begin{gather}
    \max_{x\in[-1,1]} \abs{p(x)} =\gamma < 1
  \end{gather}

  Thus, $q(x)$ has alternating sign in the $n+1$ Chebyshev abscissae and due to continuity $n$ roots in
  $(-1,1)$. Hence, it has $n+1$ roots and therefore
  $q \equiv 0$. We conclude $p \equiv \pchebyshev_n$ and since
  $\norm{p}_{\infty;[0,1]} = 1$ there holds $\tilde p = p$.
\end{proof}

\begin{Corollary}{chebyshev-minimal-2}
  Let $[a,b]$ be an interval with $0 < a$. Then, the polynomial
  \begin{gather}
    \widehat \pchebyshev_n(x)
    = \frac{\pchebyshev_n\left(\frac{a+b-2x}{b-a}\right)}%
    {\pchebyshev_n\left(\frac{a+b}{b-a}\right)}
  \end{gather}
  solves the minimization problem
  \begin{gather}
    \widehat \pchebyshev_n(x)
    = \operatorname*{argmin}_{\substack{p\in\P_n\\p(0) = 1}}
    \max_{x\in[a,b]}{\abs{p(x)}}.
  \end{gather}
  There holds
  \begin{align}
    \label{eq:chebyshev-cg1}
    \max_{x\in[a,b]}{\abs{\widehat \pchebyshev_n(x)}}
    &= \left(\pchebyshev_n\left(\frac{a+b}{b-a}\right)\right)^{-1}\\
    & \le 2 \left(\frac{\sqrt b-\sqrt a}{\sqrt b + \sqrt a}\right)^n.
  \end{align}
\end{Corollary}

\begin{proof}
  The Chebysshev polynomial $\pchebyshev_n$ is the one with minimal maximum on $[-1,1]$ and maximal growth outside this interval. Thus, if we transform it to the interval $[a,b]$ by mapping
  \begin{gather}
    x \mapsto \frac{a+b-2x}{b-a},
  \end{gather}
  it is the polynomial with maximal absolute value 1 inside $[a,b]$
  with maximal value at 0. Dividing by this value, it solves the
  stated minimization problem and \eqref{eq:chebyshev-cg1} holds.

  To further estimate this value note that
  \begin{align}
    \pchebyshev_n(x)
    &= \frac12
      \left(x-\sqrt{x^2-1}\right)^n
      +
      \left(x+\sqrt{x^2-1}\right)^n\\
    &\ge \left(x+\sqrt{x^2-1}\right)^n.
  \end{align}
  Entering $x = \frac{a+b}{b-a}$ yields
  \begin{align}
    \frac{a+b+\sqrt{(a+b)^2-(b-a)^2}}{b-a}
    &= \frac{(\sqrt a + \sqrt b)^2}{b-a}\\
    &= \frac{\sqrt b + \sqrt a}{\sqrt b-\sqrt a}.
  \end{align}
\end{proof}

\bibliographystyle{alpha}
\bibliography{all}
\printindex

%%% Local Variables:
%%% mode: latex
%%% TeX-master: "main"
%%% End:
