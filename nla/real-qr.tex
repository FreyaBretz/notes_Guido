\subsection{The real symmetric eigenvalue problem}

\begin{Lemma}{qr-tridiagonal}
  Let $\matt\in\Rnn$ be a real, symmetric, tridiagonal matrix and
  $\matq\matr=\matt$ its QR factorization. Then, $\tilde\matt=\matr\matq$ is also
  symmetric and tridiagonal. Furthermore, $\matr$ is zero except for
  its main and the first two upper diagonals.

  The same holds for the shifted version with $\sigma\in\R$,
  \begin{gather}
    \matq\matr = \matt-\sigma\id,\qquad \tilde\matt = \matr\matq+\sigma\id.
  \end{gather}
\end{Lemma}

\begin{proof}
  See homework.
\end{proof}

\begin{Lemma*}{perfect-shift}{Perfect shift}
  Let $\matt\in\Rnn$ be an unreduced, symmetric, tridiagonal matrix,
  $\sigma\in\sigma(\matt)$, and $\matr\matq=\matt-\sigma\id$ the
  shifted QR factorization. Then, $r_{nn}=0$. Furthermore, the last
  column of $\tilde\matt = \matr\matq+\sigma\id$ is equal to
  $\sigma\ve_n$.
\end{Lemma*}

%  \begin{todo}
%\begin{proof}
%  If $\matt$ is unreduced, the first $n-1$ columns of
%  $\matt-\sigma\id$ are linearly independent for any $\sigma$.    
%\end{proof}
%  \end{todo}

\begin{Remark*}{real-symmetric-qr}{QR-Iteration for real, symmetric matrices}
  In this case, many things simplify
  \begin{enumerate}
  \item Hessenberg form is tridiagonal
  \item The Schur normal form is
    \begin{gather}
      \mata = \matq^T\matd\matq
    \end{gather}
    with real, diagonal matrix $\matd$
  \item QR factorization uses $\bigo(n)$ operations and $\matr$
    consists only of the main diagonal and one upper diagonal.
  \end{enumerate}
  Accumulating the matrix $\matq$ still needs $\bigo(n^2)$ operations
\end{Remark*}

\begin{Algorithm*}{qr-explicit-shift}{QR iteration with explicit shift}
  \begin{algorithmic}[1]
    \Require $\mata\in\Rnn$ symmetric
    \State $\matt_0 = \matq_0^*\mata\matq_0$\Comment{tridiagonal}
    \For {$k=1,\ldots$ until convergence}
    \State $\matq_k\matr_k = \matt_{k-1} - \sigma_k\id$\Comment{QR factorization}
    \State $\matt_{k} = \matr_k\matq_k + \sigma_k\id$
    \EndFor
  \end{algorithmic}
\end{Algorithm*}

\begin{Lemma}{wilkinson-shift}
  Let
  \begin{gather}
    \matt =
    \begin{pmatrix}
      a_1&b_1\\
      b_1&\ddots&\ddots\\
      &\ddots&a_{n-1}&b_{n-1}\\
      &&b_{n-1}&a_n
    \end{pmatrix}.
  \end{gather}
  Then, the \putindex{Wilkinson shift} $\sigma$ can be computed as
  \begin{gather}
    \sigma = a_n + d - \operatorname{sign}(d) \sqrt{d^2+b_{n-1}^2},
    \qquad d=\frac{a_{n-1}-a_n}2.
  \end{gather}
\end{Lemma}

\begin{Algorithm*}{implicit-symmetric-shift}{Symmetric QR step with implicit shift}
  \begin{algorithmic}[1]
    \Require $\matt\in\Rnn$ symmetric, unreduced, tridiagonal; $\sigma\in\R$
    \State Compute $\matg_{12} = \matg_{12}[t_{11}-\sigma,t_{21}]$\Comment{Givens rotation}
    \State $\matt \gets \matg_{12}^T\matt\matg_{12}$
    \For {$k=2,\dots,n-1$} \Comment{Bulge chasing}
    \State $\matg_{k,k+1} = \matg_{k,k+1}[t_{k,k-1},t_{k+1,k-2}]$
    \State $\matt \gets \matg_{k,k+1}^T\matt\matg_{k,k+1}$
    \EndFor
  \end{algorithmic}
\end{Algorithm*}

%  \begin{todo}
%\begin{example}
%    Graphical representation of bulge chasing
%\end{example}
%  \end{todo}

\begin{Theorem}{implicit-symmetric-shift}
  Given $\matt\in\Rnn$ symmetric, unreduced, and tridiagonal. Let
  \begin{gather}
    \matt^{(e)} = \matq^*\matt\matq,
    \qquad
    \matt^{(i)} = \matz^*\matt\matz,
  \end{gather}
  where $\matt_e$ is computed by the QR step with explicit shift and
  $\matz=\matg_1\matg_2\dots\matg_{n-1}$ is the matrix of the QR step
  with implicit shift. Then, there holds
  \begin{xalignat}2
    \vz_1&=\vq_1,\\
    \vz_i&=\pm \vq_i,& i&=2,\dots,n,\\
    \abs{t_{i,i-1}^{(i)}}&=\abs{t_{i,i-1}^{(e)}},& i&=2,\dots,n.
  \end{xalignat}
\end{Theorem}

\begin{proof}
  Without loss of generality, we can assume that the QR factorization
  in the explicit shift step is computed by Givens rotations.  Then,
  the matrices $\matq$ and $\matz$ are both defined as a product of
  Givens rotations $\matg_{12}\matg_{23}\dots\matg_{k-1,k}$, where the
  first column is defined by the first rotation only. And the first
  rotation matrix is the same for both algorithms, such that
  $\vz_1 = \vq_1$. For the remaining results, we can use the
  \putindex{Implicit Q Theorem}.
\end{proof}

\subsection{The eigenvalue problem for nonsymmetric real matrices}

\begin{Theorem*}{real-schur-form}{The real Schur form}
  For every matrix $\mata\in \Rnn$ there is an orthogonal matrix
  $\matq\in\Rnn$ and a matrix $\matr\in\Rnn$ such that
  \begin{gather}
    \mata = \matq\matr\matq^*,
    \qquad
    \matr =
    \begin{pmatrix}
      R_{11} &* & *&*\\
      &R_{22}&*&*\\
      &&\ddots&*\\
      &&& R_{jj}
    \end{pmatrix},
  \end{gather}
  where the diagonal blocks are either of dimension one containing the
  real eigenvalues or of dimension 2 for complex conjugate eigenvalue
  pairs. The latter correspond to scaled rotation matrices with the
  according eigenvalue pair.
\end{Theorem*}

\begin{intro}{double-shift}
  When applying the QR step with Wilkinson shift, the shift parameter
  might be complex, thus leading to a bad approximation and
  consequently to slow convergence. Therefore, we have to circumvent
  this situation and find a working method in real arithmetic. 
%  Using double shifts, the QR-iteration can be made to converge to the
%  real Schur form using double shifts in real arithmetic. This method
%  is also known as the \define{Francis QR step}~\cite[Algorithm
%  7.5-1]{GolubVanLoan83}.
\end{intro}

\begin{Algorithm*}{double-shift-step}{Explicit double-shift QR step}
  \begin{algorithmic}[1]
    \Require $\matH\in\Rnn$ Hessenberg form, $\sigma_1,\sigma_2\in\C$ shifts
    \State $\matq_1\matr_1 \gets \matH - \sigma_1\id$ \Comment{QR factorization}
    \State $\matH_1 \gets \matr_1\matq_1 + \sigma_1\id$
    \State $\matq_2\matr_2 \gets \matH_1 - \sigma_2\id$ \Comment{QR factorization}
    \State $\matH_2 \gets \matr_2\matq_2 + \sigma_2\id$
  \end{algorithmic}
\end{Algorithm*}

\begin{Lemma}{double-shift-matrix}
  Let $\sigma_1,\sigma_2\in\C$ be the eigenvalues of the $2\times2$-matrix
  \begin{gather}
    \matg =
    \begin{pmatrix}
      h_{n-1,n-1}&h_{n-1,n}\\h_{n,n-1}&h_{n,n}
    \end{pmatrix}.
  \end{gather}
  Then, the unitary matrices $\matq_1,\matq_2$ of the double shift
  algorithm with shifts $\sigma_1,\sigma_2$ can be chosen such that
  $\matz = \matq_1\matq_2$ and thus $\matH_2 = \matz^T\matH\matz$ are
  real matrices in exact arithmetic.
\end{Lemma}

\begin{proof}
  First, we realize (homework) that
  \begin{gather}
    \matq_1\matq_2\matr_2\matr_1 = \matm = (\matH-\sigma_1\id)(\matH-\sigma_2\id).
  \end{gather}
  Hence,
  \begin{gather}
    \matm = \matH^2-s\matH+t\id,
  \end{gather}
  where
  \begin{gather}
    \begin{aligned}
      s &= \sigma_1+\sigma_2 = \operatorname{tr}(\matg)&\in&\R,\\
      t &= \sigma_1\sigma_2 = \det(\matg)&\in&\R.
    \end{aligned}
  \end{gather}
  Thus, $\matm\in\Rnn$. Since there is a real QR factorization of
  $\matm$, we can choose $\matz = \matq_1\matq_2\in\Rnn$. Thus, we
  conclude
  \begin{gather}
    \matH_2 = \matq_2^*\matH_1\matq_2 = (\matq_1\matq_2)^*\matH(\matq_1\matq_2) = \matz^T\matH\matz.
  \end{gather}
\end{proof}

\begin{remark}
  The explicit double step has several drawbacks. First, the algorithm
  must choose $\matq_1$ and $\matq_2$ such that their product is
  real. But then, roundoff errors will cause imaginary contributions
  in the result of the double step.

  We could also explicitly compute $\matm$ and then $\matz$ by real QR
  factorization. But here, we need a matrix vector product with
  $\bigo(n^3)$ operations.
\end{remark}

\begin{Theorem}{implicit-double-shift}{Implicit double-shift}
  The following QR step is essentialy equivalent to the explicit double-shift:
  \begin{enumerate}
  \item Compute the first column of $\vm_1$ of $\matm$.
  \item Compute a Householder matrix $\matq_0$ such that $\matq_0\vm_1$ is a multiple of $\ve_1$.
  \item Compute Householder matrices $\matq_1,\dots,\matq_{n-2}$ such
    that for $\matp = \matq_0\dots\matq_{n-2}$ there holds
    \begin{enumerate}
    \item $\matp^T\matH\matp$ is a Hessenberg matrix
    \item The first columns of $\matp$ and of $\matz = \matq_1\matq_2$
      of the explicit shift algorithm coincide.
    \end{enumerate}
  \end{enumerate}
\end{Theorem}

\subsection{Singular Value Decomposition (SVD)}

\begin{Definition}{svd}
  The \define{singular value decomposition} (\define{SVD}) of a matrix $\mata\in\C^{m\times n}$ is a facorization
  \begin{gather}
    \mata = \matu\matsigma\matv^*
  \end{gather}
  with unitary matrices $\matu\in\C^{m\times m}$ and $\matv\in\Cnn$ as
  well as a real, diagonal matrix $\matsigma$ with diagonal entries
  \begin{gather}
    \sigma_1 \ge \sigma_2 \ge \dots \ge \sigma_p \ge 0,
  \end{gather}
  where $p = \min\{m,n\}$.
\end{Definition}

\begin{Theorem}{svd}
  Every matrix $\mata\in\C^{m\times n}$ admits a singular value
  decomposition. Every real matrix admits a singular value
  decomposition with orthogonal matrices $\matu$, $\matv$.
\end{Theorem}

\begin{Corollary}{svd-rank}
  Let $\mata = \matu\matsigma\matv^*$ be the SVD of $\mata$ with
  \begin{gather}
    \sigma_1 \ge \dots \ge \sigma_r > \sigma_{r+1} = \dots = \sigma_p = 0.
  \end{gather}
  Then, $\rank \mata = r$
\end{Corollary}

\begin{Corollary}{svd-inverse}
  If $\mata\in\Cnn$ is invertible, then
  \begin{gather}
    \mata^{-1} = \matv\matsigma^{-1}\matu^*,
  \end{gather}
  where
  \begin{gather}
    \matsigma^{-1} = \diag\left(\frac1{\sigma_1},\dots,\frac1{\sigma_n}\right).
  \end{gather}
\end{Corollary}

\begin{Remark}{svd-geometry}
  Let
  \begin{gather}
    E = \bigl\{ \vy\in \R^m \big| \vy=\mata\vx, \norm{\vx}_2 = 1 \bigr\},
  \end{gather}
  be the ellipsoid obtained by mapping the unit sphere though
  $\mata$. Then, the column vectors of $\matu$ and the singular values
  $\sigma_i$ are the directions and lengths of the semi-axes of this
  ellipsoid, respectively.
\end{Remark}

\begin{Lemma}{svd-ata}
  The singular values of $\mata$ are the square roots of the
  eigenvalues of $\mata^*\mata$ and of $\mata\mata^*$, respectively. For $m\ge n$ there holds
  \begin{align}
    \matv^*(\mata^*\mata)\matv &= \diag(\sigma_1^2,\dots,\sigma_n^2)
    &&\in \R^{n\times n}\\
    \matu^*(\mata\mata^*)\matu &= \diag(\sigma_1^2,\dots,\sigma_n^2, 0,\dots,0)
    &&\in \R^{m\times m}
  \end{align}
\end{Lemma}

%%% Local Variables:
%%% mode: latex
%%% TeX-master: "main"
%%% End:
