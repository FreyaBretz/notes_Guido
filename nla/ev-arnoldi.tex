\begin{Lemma}{ev-arnoldi-a-posteriori}
  Let $(\lambda_i^{(m)},\vy_i^{(m)})$ be an eigenpair of the projected
  matrix $\matH_m$ of Arnoldi's method. Then, the Ritz vector
  $\vu_i^{(m)} = \matv_m \vy_i^{(m)}$ has a residual represented as
  \begin{gather}
    \left(A-\lambda_i^{(m)}\identity\right)\vu_i^{(m)}
    = h_{m+1,m}\ve_m^*\vy_i^{(m)}\vv_{m+1}.
  \end{gather}
  In particular, its norm is
  \begin{gather}
    \norm*{\left(A-\lambda_i^{(m)}\identity\right)\vu_i^{(m)}}_2
    = h_{m+1,m} \abs{\ve_m^*\vy_i^{(m)}}.
  \end{gather}
\end{Lemma}

\begin{proof}
  See~\cite[Proposition 6.8]{Saad11}.
\end{proof}

\begin{Corollary}{ev-arnoldi-breakdown}
  If the Arnoldi algorithm breaks down, the generated Ritz vectors are
  eigenvectors.
\end{Corollary}

%%% Local Variables:
%%% mode: latex
%%% TeX-master: "main"
%%% End:
