\begin{Lemma}{cg-tridiagonal}
  Let $\alpha_k$ and $\beta_k$ be the sequences of coefficients in the conjugate gradient method
  \slideref{Algorithm}{cg}. Then, the entries of the projected matrix
  \begin{gather}
    \matH_m =
    \begin{pmatrix}
            \gamma_1 & \delta_2\\
      \delta_2 & \gamma_2 & \ddots\\
      &\ddots&\ddots&\delta_m\\
      &&\delta_{m}&\gamma_m
    \end{pmatrix}
  \end{gather}
  are computed by the formulas
  \begin{gather}
    \delta_{k+1} = \frac{\sqrt{\beta_{k-1}}}{\alpha_{k-1}},
    \qquad
    \gamma_{k+1} =
    \begin{cases}
      \tfrac1{\alpha_0}&k=0,\\
      \tfrac1{\alpha_k} + \tfrac{\beta_{k-1}}{\alpha_{k-1}} &k>0.\\
    \end{cases}
  \end{gather}
\end{Lemma}

\begin{proof}
  See~\cite[Section 6.7.3]{Saad00}.
\end{proof}

\begin{remark}
  The previous lemma allows us to compute eigenvalues on the fly while
  solving a linear system with the conjugate gradient method. We could
  also run the Lanczos process \slideref{Algorithm}{lanczos} itself.

  In both cases, if $m$ becomes very large, orthogonality may be lost
  due to round-off errors.
\end{remark}

\begin{Lemma}{lanczos-angle}
  Let $\mata\in\Rnn$ be symmetric, positive definite with eigenvalues
  \begin{gather}
    \lambda_1 \ge \lambda_2 \ge \dots \ge \lambda_n.
  \end{gather}
  Let $P_i$ be the projector to the subspace spanned by the
  eigenvector(s) for $\lambda_i$. Let the initial vector in Lanczos
  method be chosen such that $P_i \vv_1\neq 0$. Then, the angle
  $\theta_i^{(m)}$ between the eigenvector(s) $\vu_i$ and the Krylov space
  $\krylov_m(\mata,\vv_1)$ admits the estimate
  \begin{gather}
    \tan\theta_i^{(m)} = \min_{\substack{p\in\P_{m-1}\\p(\lambda_i)=1}} \norm{p(\mata)\vy_i} \tan \theta_i^{(1)}.
  \end{gather}
  Here,
  \begin{gather}
    \vy_i = \frac{(\id-P_i)\vv_1}{\norm{(\id-P_i)\vv_1}},
  \end{gather}
  or $\vy_i=0$ if the norm in the denominator is zero.
\end{Lemma}

\begin{proof}
  See~\cite[Lemma 6.1]{Saad11}
\end{proof}

\begin{Theorem}{lanczos-angle}
  With the notation and assumptions of \slideref{Lemma}{lanczos-angle}, there holds for $i\le m < n$
  \begin{gather}
    \tan\theta_i^{(m)} \le \frac{\kappa_i}{\pchebyshev_{m-i}(1+2\gamma_i)}\tan\theta_i^{(1)},
  \end{gather}
  where $\pchebyshev_{m-i}$ is the Chebyshev polynomial of degree $m-i$ and
  \begin{gather}
    \kappa_i=\prod_{\lambda_k < \lambda_i} \frac{\lambda_k-\lambda_n}{\lambda_k-\lambda_i},
    \qquad \gamma_i = \frac{\lambda_{i}-\lambda_{i+1}}{\lambda_{i+1}-\lambda_{n}}.
  \end{gather}
  The same result holds for $\vu_{n-i}$ by replacing
  \begin{gather}
    \kappa_i=\prod_{\lambda_k > \lambda_i} \frac{\lambda_1-\lambda_k}{\lambda_i-\lambda_k},
    \qquad \gamma_i = \frac{\lambda_{i-1}-\lambda_{i}}{\lambda_{1}-\lambda_{i-1}}.
  \end{gather}  
\end{Theorem}

\begin{proof}
  See also~\cite[Theorem 6.3]{Saad11}.
  First, we expand the vector $\vy_i$ in the basis of eigenvectors,
  \begin{gather}
    \vy_i = \sum_{k\neq i} \alpha_k \vu_k.
  \end{gather}
  Since $\vy_i$ is normalized, $\sum \abs{\alpha_k}^2 = 1$. Thus,
  \begin{gather}
    \norm{p(\mata)\vy_i}^2 = \sum_{k\neq i} \abs{p(\lambda_k)\alpha_k}^2
    \le \max_{k\neq i} \abs{p(\lambda_k)}^2 \sum_{k\neq i} \abs{\alpha_k}^2
    = \max_{k\neq i} \abs{p(\lambda_k)}^2.
  \end{gather}

  Now, we would like to mimick the proof of
  \slideref{Corollary}{cg-condition-number}, but we can only do this
  for the two extremal eigenvalues $\lambda_1$ and $\lambda_n$. Hence,
  we continue as follows:
  \begin{gather}
    \norm{p(\mata)\vy_1}^2 \le \min_{\substack{p\in\P_{m-1}\\p(\lambda_1)=1}}
    \max_{\lambda\in[\lambda_n,\lambda_2]} \abs{p(\lambda)}^2.
  \end{gather}
  Using the shifted Chebyshev polynomials of
  \slideref{Corollary}{chebyshev-minimal-2} for the interval
  $[\lambda_n,\lambda_2]$ and observing that normalization takes place
  at $\lambda_1$ instead of zero, we indeed obtain
  \begin{gather}
    \norm{p(\mata)\vy_1}
    \le \frac1{\pchebyshev_{m-1}\left(\tfrac{\lambda_2+\lambda_n}{\lambda_2-\lambda_n}+\lambda_1\right)}
    = \frac1{\pchebyshev_{m-1}(1+2\gamma_1)}.
  \end{gather}
  A similar estimate is obtained for $\vu_n$ by exchanging $\lambda_n$
  for $\lambda_1$ and $[\lambda_{n-1},\lambda_1]$ for
  $[\lambda_n,\lambda_2]$, respectively.

  For eigenvalues $\lambda_2,\dots,\lambda_m$ this approach does not work, since there is no result for polynomials minimizing on both sides of a point. Therefore, we construct a polynomial
  \begin{gather}
    p(\lambda) = z(\lambda) q(\lambda),
  \end{gather}
  where $z(\lambda)$ is zero in all eigenvalues greater than
  $\lambda_i$ and $q(\lambda)$ minimizes on all eigenvalues less than
  $\lambda_i$. Choose
  \begin{gather}
    z(\lambda) = \prod_{\lambda_k>\lambda_i} \frac{\lambda_k-\lambda}{\lambda_k-\lambda_i},
  \end{gather}
  which is of degree $i-1$ and there holds $z(\lambda_i) = 1$. The
  remainder $q(\lambda)$ is the solution to the minimization problem
  \begin{gather}
    \min_{\substack{q\in\P_{m-i}\\q(\lambda_i) = 1}} \max_{\lambda\in[\lambda_n,\lambda_{i+1}]} \abs{q(\lambda)},
  \end{gather}
  which is again solved by the shifted Chebyshev polynomial. Again, we
  can apply the same construction from the other end of the spectrum
  to obtein estimates for $\vu_{n-1},\dots,\vu_{n-m}$.
\end{proof}

\begin{remark}
  Note that if $m<\nicefrac{n}2$, there are eigenvectors for which
  \slideref{Theorem}{lanczos-angle} does not provide an
  estimate. Moreover, since $\pchebyshev_0\equiv 1$ and $\kappa_i>1$
  for $i>1$, the estimate for $\theta_m^{(m)}$ is worse than
  $\theta_m^{(1)}$, which is clearly suboptimal, since the angle does
  not get larger if we increase the dimension of the subspace. Thus,
  the estimate is most useful for the eigenvalues at the ends of the
  spectrum.

  Changing the angle of view, the theorem provides estimates for $2m$
  eigenvectors with an $m$-dimensional subspace. Therefore, we should
  not be surprised that some of the estimates are useless.

  The theorem holds for multiple eigenvalues, and inspecting the
  structure more closely, the index $i$ refers to the $i$-th
  \textbf{distinct} eigenvalue.
\end{remark}

\begin{Theorem}{lanczos-eigenvalue-estimate}
  The discrete eigenvalues $\lambda_i^{(m)}$ after $m$ steps of the Lanczos method satisfy the estimate
  \begin{gather}
    0\le \lambda_i-\lambda_i^{(m)}
    \le (\lambda_1-\lambda_n) \left(
      \frac{\kappa_i^{(m)}}{\pchebyshev(1+2\gamma_i)}
      \tan\theta_i^{(1)}\right)^2,
    \end{gather}
    where $\gamma_i$ is defined as before and
    \begin{gather}
    \kappa_i^{(m)}=\prod_{k=1}^{i-1} \frac{\lambda_k^{(m)}-\lambda_n}{\lambda_k^{(m)}-\lambda_i}.
  \end{gather}
  Again, a similar estimate can be obtained for $\lambda_{n-i}^{(m)} - \lambda_{n-i}$.
\end{Theorem}

\begin{todo}
  Move this to the beginning.
\end{todo}

\begin{Algorithm*}{ev-lanczos}{Lanczos algorithm for eigenvalues}
  \begin{algorithmic}[1]
    \Require $\vr\in\C^n$ and $\delta_0=\norm{\vr}_2$
    \For {$m=1,\dots$ until convergence}
    \State $\vv^{(m)} \gets \frac{\vr}{\delta_{m-1}}$
    \State $\vr \gets \mata\vv^{(m)}$
    \State $\vr \gets \vr-\delta_{m-1}\vv^{(m-1)}$
    \State $\gamma_m = \scal(\vr,\vv^{(m)})$
    \State $\vr \gets \vr-\gamma_m\vv^{(m)}$
    \State Reorthogonalization if necessary
    \State $\delta_m \gets \norm{\vr}_2$
    \EndFor
    \State $\matH_m = \maty\matd_m\maty^*$ \Comment{Compute Ritz values}
    \State $\matx=\matv_m\maty$ \Comment{Compute Ritz vectors}
  \end{algorithmic}
\end{Algorithm*}

\begin{remark}
  Reorthogonalization will be necessary once the Krylov space
  approximates an invariant subspace, which in turn is a sign for
  convergence and thus desirable. Then, $\vr$ will be almost linearly
  dependent and its norm after orthogonalization will be small,
  indicating cancellation.

  Note that this differs from the conjugate gradient method, where the
  Lanczos algorithm was modified as to avoid cancellation. But there
  is a more fundamental difference between the two algorithms. When we
  solve a linear system, we only compute a single vector, which is
  represented by the current state of the iteration. For the
  eigenvalue problem, we must keep the whole set of vectors $\matv_j$
  and we must make sure they are orthogonal with sufficient accuracy
  to compute the eigenvector approximation.
\end{remark}

\begin{remark}
  Like the vector iteration, this algorithm will put a focus on large
  eigenvalues. Thus, we can also apply the shift and invert technology
  in order to approximate other eigenvalues. To this end, line 3 of
  the algorithm is replaced by
  \begin{algorithmic}
    \State Solve $(\mata-\mu\id)\vr = \vv_j$.
  \end{algorithmic}
\end{remark}



%%% Local Variables:
%%% mode: latex
%%% TeX-master: "main"
%%% End:
