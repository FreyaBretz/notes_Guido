\begin{Definition}{chebyshev-polynomials}
  The \define{Chebyshev polynomials} $\pchebyshev_n$ are defined by the
  three-term recurrence relation
  \begin{gather}
    \pchebyshev_{n}(x) = 2x \pchebyshev_{n-1}(x) - \pchebyshev_{n-2}(x),
    \quad n=2,\ldots
  \end{gather}
  with $T_0 \equiv 1$ and $T_1(x) = x$.
  They are orthogonal with respect to the inner product
  \begin{gather}
    \scal(p,q) = \int_{-1}^1 \tfrac1{\sqrt{1-x^2}} \,p(x)q(x)\dx.
  \end{gather}
\end{Definition}

\begin{Lemma}{chebyshev-representation}
  Chebyshev polynomials admit the trigonometric representation
  \begin{gather}
    \pchebyshev_n(x) = \cos(n \operatorname{arccos} x), \qquad\abs{x} \le 1.
  \end{gather}
  
  Furthermore, there holds
    \begin{gather}
    \pchebyshev_n = \frac12
    \left(
      \left(x-\sqrt{x^2-1}\right)^n
      +
      \left(x+\sqrt{x^2-1}\right)^n
    \right),\qquad\abs{x}\ge 1.
  \end{gather}
\end{Lemma}

\begin{Lemma}{chebyshev-abscissae}
  The Chebyshev polynomial $\pchebyshev_n$ has $n$ roots in the
  interval $(-1,1)$ at
  \begin{gather}
    x_k = \cos\left(\pi\frac{(k-\nicefrac12)}{n}\right),
    \qquad k=1,\dots,n.
  \end{gather}
  It alternatingly assumes the values $\pm1$ at the \define{Chebyshev abscissae}
  \begin{gather}
    x_k = \cos\left(\pi\frac kn\right), \qquad k=0,\dots,n,
  \end{gather}
  with $\pchebyshev_n(1) = 1$ and $\pchebyshev_n(-1) = (-1)^n$.
\end{Lemma}


\begin{Theorem}{chebyshev-growth}
  Let $K$ be the set of polynomials of degree $n$ with values in
  $[-1,1]$ on the interval $[-1,1]$, namely
  \begin{gather}
    K = \bigl\{ p\in \P_n \;\big|\; \max_{x\in[-1,1]} \abs{p(x)} =1 \bigr\}.
  \end{gather}
  Then, The Chebyshev polynomial $\pchebyshev_n$ is the polynomial in
  $K$ which grows fastest outside the interval $[-1,1]$, hence
  \begin{gather}
    \abs{\pchebyshev_n(x)} \ge p(x) \qquad \forall p\in K,
    \quad \forall \abs{x} > 1.
  \end{gather}
\end{Theorem}

\begin{proof}
  We conduct the proof for $x>1$ where $\pchebyshev_n(x) > 0$.
  Let $\tilde p\in K$ such that
  $\tilde p(y) \ge \pchebyshev_n(y)$ for some $y>1$. Let
  $\gamma = \pchebyshev_n(y)/\tilde p(y)$ and
  $p(x) = \tilde p(x)\gamma$, such that
  $q(x) = p(x) - \pchebyshev_n(x) \in \P_n$ has a root in $y$.
  Furthermore,
  \begin{gather}
    \max_{x\in[-1,1]} \abs{p(x)} = \gamma\max_{x\in[-1,1]} \abs{\tilde p(x)} = \gamma < 1
  \end{gather}

  Thus, $q(x)$ has alternating sign in the $n+1$ Chebyshev abscissae and due to continuity $n$ roots in
  $(-1,1)$. Hence, it has $n+1$ roots and therefore
  $q \equiv 0$. We conclude $p \equiv \pchebyshev_n$ and since
  $\norm{p}_{\infty;[0,1]} = 1$ there holds $\tilde p = p$.
\end{proof}

\begin{Corollary}{chebyshev-minimal-2}
  Let $[a,b]$ be an interval with $0 < a$. Then, the polynomial
  \begin{gather}
    \widehat \pchebyshev_n(x)
    = \frac{\pchebyshev_n\left(\frac{a+b-2x}{b-a}\right)}%
    {\pchebyshev_n\left(\frac{a+b}{b-a}\right)}
  \end{gather}
  solves the minimization problem
  \begin{gather}
    \widehat \pchebyshev_n(x)
    = \operatorname*{argmin}_{\substack{p\in\P_n\\p(0) = 1}}
    \max_{x\in[a,b]}{\abs{p(x)}}.
  \end{gather}
  There holds
  \begin{gather}
    \label{eq:chebyshev-cg1}
    \max_{x\in[a,b]}{\abs{\widehat \pchebyshev_n(x)}}
    = \frac1{\pchebyshev_n\left(\frac{a+b}{b-a}\right)}
    \le 2 \left(\frac{\sqrt b-\sqrt a}{\sqrt b + \sqrt a}\right)^n.
  \end{gather}
\end{Corollary}

\begin{proof}
  The Chebyshev polynomial $\pchebyshev_n$ is the one with $\norm{\pchebyshev_n}_{\infty;[-1,1]}$ and maximal growth outside this interval. Thus,
  if we transform it to the interval $[a,b]$ by mapping
  \begin{gather}
    x \mapsto \frac{a+b-2x}{b-a},
  \end{gather}
  it is the polynomial with maximal absolute value 1 inside $[a,b]$
  and maximal value at 0. Dividing by this value, it solves the
  stated minimization problem and \eqref{eq:chebyshev-cg1} holds.

  To further estimate this value note that
  \begin{align}
    \pchebyshev_n(x)
    &= \frac12
      \left(\left(x-\sqrt{x^2-1}\right)^n
      +
      \left(x+\sqrt{x^2-1}\right)^n\right)\\
    &\ge \frac12\left(x+\sqrt{x^2-1}\right)^n.
  \end{align}
  Entering $x = \frac{a+b}{b-a}$ yields
  \begin{align}
    \frac{a+b+\sqrt{(a+b)^2-(b-a)^2}}{b-a}
    &= \frac{(\sqrt a + \sqrt b)^2}{b-a}\\
    &= \frac{\sqrt b + \sqrt a}{\sqrt b-\sqrt a}.
  \end{align}
\end{proof}

%%% Local Variables:
%%% mode: latex
%%% TeX-master: "main"
%%% End:
