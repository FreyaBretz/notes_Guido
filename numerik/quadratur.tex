\subsection{Summierte Quadratur}
\begin{Definition}{quadratur}
  Eine \define{Quadraturformel} $Q_{[a,b]}(f)$ ist eine Approximation
  des Integrals
  \begin{gather}
    Q_{[a,b]}(f) \approx \int_a^b f(x)\dx
  \end{gather}
  in der Form
  \begin{gather}
    Q_{[a,b]}(f) = \sum_{i=0}^n \omega_i f(x_i).
  \end{gather}
  Die Stützstellen $x_i$ bezeichnen wir auch als
  \define{Quadraturpunkte}, die Zahlen $\omega_i$ als
  \define{Quadraturgewichte}.

  Lässt sich die Quadraturormel bezüglich einer Zerlegung
  $\mathcal I_h$ des Intervalls $[a,b]$ in der Form
  \begin{gather}
    Q_{[a,b]}(f) = \sum_{i=1}^n Q_{I_i} (f)
  \end{gather}
  schreiben, so sprechen wir von \textbf{summierter},
  \textbf{iterierter} oder \textbf{stückweiser Quadratur}.
\end{Definition}

\begin{Definition}{lokale-fehlerordnung}
  Gilt bei einer summierten Quadraturformel die Abschätzung
  \begin{gather}
    \left|\int_{I_i} f(x)\dx - Q_{I_i}(f)\right|
    =\bigo\left(h_i^{k+1}\right)
  \end{gather}
  für jedes Teilintervall $I_i$ und Funktionen $f\in C^{k+1}[a,b]$, so
  sprechen wir von der \textbf{lokalen
    Fehlerordnung}\defindex{Fehlerordnung}\defindex{lokale Fehlerordnung} $k+1$.
\end{Definition}

\begin{Satz}{summierte-quadratur}
  Sei $\mathcal I_h$ eine Zerlegung von $[a,b]$ der Feinheit $h$ und
  $c_q$ sei so gewählt, dass
  \begin{gather}
    c_q \min_{I_i\in \mathcal I_h} h_i \ge h.
  \end{gather}
  Sind dann die Formeln $Q_{I_i}$ von lokaler Fehlerordnung $k+1$ für
  $f\in C^{k+1}[a,b]$, so gilt für die summierte Quadratur $Q_{[a,b]}$
  die Abschätzung
  \begin{gather}
    \left|\int_a^b f(x)\dx - Q_{[a,b]}(f)\right|
    = \mathcal O\left(h^{k}\right).
  \end{gather}
\end{Satz}

\begin{proof}
  Das kleinste Intervall hat die Länge $h/c_q$. Damit ist die Anzahl
  der Intervalle beschränkt durch $n_{\max}=c_q (b-a)/h$. Aus der
  lokalen Fehlerordnung ergibt sich die Existenz einer Konstanten $c$,
  so dass
  \begin{gather}
    \left|\int_{I_i} f(x)\dx - Q_{I_i}(f)\right| \le c h_i^{k+1}.
  \end{gather}
  Damit schätzen wir ab
  \begin{align}
    \left|\int_a^b f(x)\dx - Q_{[a,b]}(f)\right|
    &= \sum_{I_i\in\mathcal I_h}  \left|\int_{I_i} f(x)\dx - Q_{I_i}(f)\right|\\
    &\le \sum_{I_i\in\mathcal I_h} c h^{k+1}\\
    & \le n_{\max} c h^{k+1} = \bigo\left(h^{k}\right).
  \end{align}
\end{proof}

\begin{Definition}{grad-exaktheit}
  Eine Quadraturformel $Q_I$ heißt \define{exakt vom Grad $k$} und $k$
  heißt der \define{Grad der Exaktheit} von $Q_I$, wenn sie exakt für
  alle Polynome vom Grad bis zu $k$ ist, also
  \begin{gather}
    \int_I p(x)\dx - Q_{I}(p) = 0 \qquad \forall p\in \P_k.
  \end{gather}
\end{Definition}

\begin{Lemma}{exakt-ordnung}
  Seid die Quadraturformel $Q_I$ exakt vom Grad $k$ und
  $\abs{I} \le h$. Dann gilt für $f\in C^{k+1}(I)$
  \begin{gather}
    \left|\int_{I} f(x)\dx - Q_{I}(f)\right| = \bigo\bigl(h^{k+2}\bigr)
  \end{gather}
\end{Lemma}

% \begin{Definition}{quasi-uniform}
%   Eine Zerlegung $\I_h$ des Intervalls $[a,b]$ in $n$ Teilintervalle
%   heißt \define{uniform}, wenn $h_i = h_k$ für alle
%   $i,k=1,\dots,n$. Insbesondere ist dann $h_i = (b-a)/n$.

%   Eine Familie von Zerlegungen heißt \define{quasi-uniform}, wenn als schwächere
%   Bedingung gilt:
% \end{Definition}


%%% Local Variables:
%%% mode: latex
%%% TeX-master: "main"
%%% End:
