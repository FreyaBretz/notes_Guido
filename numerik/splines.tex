Dieser Abschnitt folgt recht eng der Darstellung in \cite[Abschnitt 2.3]{Rannacher17}.

\subsection{Interpolation auf Teilintervallen}

\begin{Notation}{indices}
  In diesem Abschnitt bezeichne für die monotone Folge
  \begin{gather}
    a = x_0 < x_1 \dots < x_n = b
  \end{gather}
  stets
  \begin{gather}
    \mathcal I_h = \bigl\{ I_i = [x_{i-1},x_i] \big|
    \; i=1,\dots,n\bigr\}
  \end{gather}
  eine \define{Zerlegung} des Intervalls $I=[a,b]$, also
  \begin{gather}
    [a,b] = \bigcup_{i=1}^n I_h.
  \end{gather}
  Die Länge der Teilintervalle bezeichnen wir mit
  $h_i = \abs{I_i} = x_{i} - x_{i-1}$, mit $h=\max h_i$ die
  \define{Feinheit} der Unterteilung.
\end{Notation}

\begin{Notation}{reference-interval}
  Wir bezeichnen $\hat I = [-1,1]$ als Referenzintervall. Jedes
  Intervall $I_i$ einer Zerlegung $\mathcal I_h$ ergibt sich als Bild
  von $\hat I$ unter der affinen Abbildung
  \begin{gather}
    \begin{split}
      \Phi_i\colon \hat I &\to I_i\\
      \hat x &\mapsto \tfrac{x_{i}+x_{i-1}}{2} + \tfrac{h_i}{2} \hat x.
    \end{split}
  \end{gather}
\end{Notation}

\begin{Definition*}{piecewise-interpolation}{Stückweise Interpolation}
  Sei $\mathcal I_h$ eine Zerlegung von $[a,b]$. Auf dem
  Referenzintervall $\hat I$ sei eine Interpolationsaufgabe durch die
  Stützstellen $\hat x_0,\dots, \hat x_k$ definiert. Dann lautet die
  Aufgabe der stückweisen Interpolation auf $\mathcal I_h$: finde eine
  Funktion $s$ auf $[a,b]$, so dass für jedes $i=1,\dots,n$ die
  Einschränkung $s_{|I_i} \in \P_k$ der Interpolationsaufgabe mit den
  Stützstellen
  \begin{gather}
    x_{ij} = \Phi_i(\hat x_j),\qquad j=1,\dots,k
  \end{gather}
  genügt.
\end{Definition*}

\begin{Lemma}{piecewise-solvable}
  Die stückweise Interpolationsaufgabe hat eine eindeutige Lösung,
  wenn die Interpolationsaufgabe auf dem Referenzintervall eine solche
  besitzt.
\end{Lemma}

\begin{Lemma*}{scaling-interpolation}{Skalierungsargument}
  Für die Lösung $\hat p\in \P_k$ der Interpolationsaufgabe auf dem
  Referenzintervall gelte mit einer Konstanten $C$ unabhängig von
  $\hat f\in C^{k+1}(\hat I)$ die Fehlerabschätzung
  \begin{gather}
    \norm{\hat f- \hat p}_{\infty;\hat I} \le C \norm{\hat
      f^{(k+1)}}_{\infty;\hat I}.
  \end{gather}
  Dann ist der Fehler der stückweisen Interpolation beschränkt ist durch
  \begin{gather}
    \norm{f-s}_{\infty;[a,b]}
    \le \frac{C}{2^{k+1}} h^{k+1} \norm{f^{(k+1)}}_{\infty;[a,b]}.
  \end{gather}
\end{Lemma*}

\begin{Bemerkung}{scaling-interpolation-local}
  Genauere Betrachtung der Analyse ergibt die schärfere Abschätzung
  \begin{gather}
    \norm{f-s}_{\infty;[a,b]}
    \le \frac{C}{2^{k+1}} \max_{i=1,\dots,n}
    \Bigl(h_i^{k+1}\norm{f^{(k+1)}}_{\infty;I_i}\Bigr).
  \end{gather}  
\end{Bemerkung}

\subsection{Splines}

\begin{Definition}{spline-raum}
  Für stückweise Polynome auf dem Intervall $[a,b]$ mit einer
  Zerlegung $\mathcal I_h$ definieren wir die
  \textbf{Spline-Räume}\index{Spline-Raum}
  \begin{gather}
    S^{(k,m)}_h \ \bigl\{ s\in C^m[a,b]
    \big| s_{|I_i} \in \P_k, i=1,\dots,n\bigr\}
  \end{gather}
  mit $m<k$.
\end{Definition}

\begin{Lemma}{spline-raum}
  Die Dimension von $S^{(k,m)}_h$ ist
  \begin{gather}
    \operatorname{dim}S^{(k,m)}_h = (k-m)n + m+1
  \end{gather}
\end{Lemma}

\begin{proof}
  Betrachten wir die $n$ Wiederholungen des Raums $\P_k$, eine für
  jedes Intervall $I_i$, so ergibt sich $(k+1)n$.  Die Bedingung
  $s\in C^m[a,b]$ bedeutet, dass die Werte und die ersten $m$
  Ableitungen der Funktionen in $S^{(k,m)}$ in jedem inneren Punkt
  $x_i$ für die beiden Intervalle $I_i$ und $I_{i+1}$
  übereinstimmen. Daraus ergeben sich $(n-1)(m+1)$ lineare
  Beschränkungen, so dass die Dimension $(k+1)n - (n-1)(m+1)$ ist.
\end{proof}

\begin{Definition}{cubic-spline}
  Die Interpolationsaufgabe mit kubischen \define{Splines} lautet:
  finde eine Funktion $s\in S_h^{(3,2)}$, so dass
  \begin{gather}
    s(x_i) = f_i,\qquad i=0,\dots,n.
  \end{gather}
\end{Definition}

\begin{Definition}{cubic-spline-bc}
  Da die Anzahl der Interpolationsbedingungen um 2 geringer ist als
  die Dimension des Raumes $S_h^{(3,2)}$ definieren wir folgende,
  alternative Randbedingungen:
  \begin{description}
  \item[Natürlich]
    \begin{gather}
      s''(a) = s''(b) = 0
    \end{gather}
  \item[Periodisch]
    \begin{gather}
      s'(a) = s'(b) \quad \wedge \quad s''(a) = s''(b)
    \end{gather}
  \item[Vollständig approximierend]
    \begin{gather}
      s'(a) = f'(a) \quad \wedge \quad s'(b) = f'(b)
    \end{gather}
  \end{description}
\end{Definition}

\begin{Satz}{cubic-spline}
  Die stückweise kubische Spline-Interpolierende $s\in S_h^{(3,2)}$
  mit natürlicher Randbedingung existiert und ist eindeutig bestimmt.
\end{Satz}

\begin{proof}
  Wie meistens beginnen wir mit der Eindeutigkeit. Seinen $s_1$ und
  $s_2$ zwei Interpolierende der Werte $f_i$ in den Punkten $x_i$,
  $i=0,\dots,n$ und $s=s_2-s_1$. Dann gilt
  \begin{gather}
    \label{eq:splines:n}
    s \in N_h = \bigl\{ w\in C^2[a,b]
    \;\big|\; w(x_i) = 0, \quad i=0,\dots,n \bigr\}.
  \end{gather}
  Zusätzlich gilt $s_{| I_i}\in \P_3$ für alle Intervalle. Wir beobachten, dass für beliebiges $w\in N_h$ gilt
  \begin{align}
    \int_{I_i} s''(x) w''(x)\dx
    &= s''w'\Bigr|^{x_i}_{x_{i-1}} - \int_{I_i} s^{(3)}(x) w'(x)\dx\\
    &= s''w'\Bigr|^{x_i}_{x_{i-1}} - s^{(3)}w \Bigr|^{x_i}_{x_{i-1}}
      + \int_{I_i} s^{(4)}(x) w(x)\dx\\
    &= s''w'\Bigr|^{x_i}_{x_{i-1}}.
  \end{align}
  Summieren wir über alle Intervalle, so ergibt sich
  \begin{gather}
    \int_a^b s''(x) w''(x)\dx = \sum_{i=1}^n s''w'\Bigr|^{x_i}_{x_{i-1}}
    = s''(b) w'(b) - s''(a)w'(a).
  \end{gather}
  Wegen der natürlichen Randbedingung ist dies aber null. Insbesondere
  können wir $w=s$ einsetzen und erhalten
  \begin{gather}
    \int_a^b \abs{s''(x)}^2 \dx = 0
  \end{gather}
  und $s$ muss ein lineares Polynom sein. Aus $s(a) = s(b) = 0$ folgt
  damit $s\equiv 0$ im Widerspruch zur Annahme, dass es zwei Lösungen
  gebe.

  Nach \slideref{Lemma}{spline-raum} hat $S_h^{(3,2)}$ die Dimension
  $n+3$. Andererseits haben wir $n+1$ Interpolationsbedingungen und 2
  Randbedingungen, so dass aus der Eindeutigkeit die Existenz folgt.
\end{proof}

\begin{Lemma}{spline-optimality}
  Unter allen Funktionen $f\in C^2[a,b]$ mit vorgegebenen
  Funktionswerten $f(x_i) = y_i$, $i=0,\dots,n$ ist der natürliche
  Spline $s\in S_h^{(3,2)}$, der diese Punkte interpoliert, diejenige
  mit der kleinsten mittleren zweiten Ableitung, es gilt also
  \begin{gather}
    \int_a^b \abs{s''(x)}^d\dx \le \int_a^b \abs{f(x)}^2 \dx
    \qquad \forall f\in C^2[a,b].
  \end{gather}
\end{Lemma}

\begin{proof}
  Siehe \cite[Satz 2.9]{Rannacher17}
\end{proof}

\begin{Lemma}{splines-konkret}
  Seien die Momente
  \begin{gather}
    M_i = s''(x_i),\qquad i=0,\dots,n
  \end{gather}
  bekannt. Dann berechnen sich die Koeffizienten der Polynome auf den
  Teilintervallen $I_i$ , $i=1,\dots,n$, dargestellt durch
  \begin{gather}
    s_{|I_i}(x) = a_{i0} + a_{i1} (x-x_i) + a_{i2}(x-x_i)^2 + a_{i3}(x-x_i)^3,
  \end{gather}
  aus den Formeln
  \begin{xalignat}2
    a_{i0} &= f_i,
    & a_{i1} &= \tfrac{f_i-f_{i-1}}{h_i}
    + \tfrac{h_i(2M_{i} + M_{i-1})}{6},\\
    a_{i2} &= \tfrac{M_i}{2}
    & a_{i3} &= \tfrac{M_{i} - M_{i-1}}{6h_i}.
  \end{xalignat}
\end{Lemma}

\begin{proof}
  Siehe \cite[Abschnitt 2.4.2]{Stoer83}. Wir bemerken: $s''$ ist eine
  stückweise lineare Funktion, die die werte $M_i$ interpoliert. Daher
  gilt
  \begin{gather}
    s''(x) = M_{i-1} \frac{x_{i}-x}{h_i} + M_{i}\frac{x-x_{i-1}}{h_i},
    \qquad x\in I_i.
  \end{gather}
  Daraus erhalten wir durch Integration
  \begin{gather}
    \label{eq:splines:2}
    \begin{split}
    s'(x) &= -M_{i-1} \frac{(x_i-x)^2}{2h_i} + M_{i} \frac{(x-x_{i-1})^2}{2h_i} + A_i\\
    s(x) &= M_{i-1} \frac{(x_i-x)^3}{6h_i} + M_{i} \frac{(x-x_{i-1})^3}{6h_i} + A_i(x-x_{i-1}) + B_i      
    \end{split}
  \end{gather}
  mit Integrationskonstanten $A_i$ und $B_i$. Wegen der
  Interpolationsbedingungen in $x_{i-1}$ und $x_{i}$ muss gelten
  \begin{gather}
    B_i = y_{i-1} - M_{i-1} \frac{h_i^2}{6},
    \qquad A_i = \frac{f_{i}-f_{i-1}}{h_i} - \frac{h_i}{6} (M_i-M_{i-1}).
  \end{gather}
  Aus dieser Darstellung und der Beziehung $s^{(j)}(x_i) = j!a_{ij}$
  erhalten wir die gewünschten Koeffizienten.
\end{proof}

\begin{Lemma}{splines-momente}
  Die Momente $M_i$ genügen dem linearen Gleichungssystem
  \begin{gather}
    \begin{pmatrix}
      2 & \lambda_0 \\
      \mu_1 & 2 & \lambda_1\\
      & \ddots & \ddots & \ddots \\
      && \mu_{n-1} & 2 & \lambda_{n-1}\\
      &&&\mu_n & 2
    \end{pmatrix}
    \begin{pmatrix}
      M_0\\\\\vdots\\\\M_n
    \end{pmatrix}
    =
    \begin{pmatrix}
      d_0\\\\\vdots\\\\d_n
    \end{pmatrix}
  \end{gather}
  wobei für $i=1,\dots,n-1$
  \begin{gather}
    \lambda_i = \tfrac{h_{i+1}}{h_{i}+h_{i+1}},
    \qquad \mu_i = 1-\lambda_i = \tfrac{h_{i}}{h_{i}+h_{i+1}},\\
    d_i = \tfrac{6}{h_{i}+h_{i+1}}
    \left[\tfrac{f_{i+1}-f_i}{h_{i+1}} - \tfrac{f_{i}-f_{i-1}}{h_{i}}\right]
  \end{gather}
  Für natürliche Splines sind $\lambda_0 = \mu_n =0$ und $d_0 = d_n = 0$.
  Für vollständig approximierende Splines ist $\lambda_0 = \mu_n = 1$ und
  \begin{gather}
    d_0 = \frac{6}{h_1}\left(\frac{f_1-f_0}{h_1}-f_0'\right),
    \qquad
    d_n = \frac{6}{h_n}\left(f'_n - \frac{f_n-f_{n-1}}{h_n}\right).
  \end{gather}
\end{Lemma}

\begin{proof}
  Siehe \cite[Abschnitt 2.4.2]{Stoer83}. Die Stetigkeit von $s(x)$ und
  $s''(x)$ in den inneren Punkten $x_i$ ergibt sich im vorhergehenden
  Beweis aus der Interpolation der $f_i$ und $M_i$. Zusätzlich müssen
  wir die Stetigkeit von $s'(x)$ fordern. Dazu benutzen wir die in
  Gleichung~\eqref{eq:splines:2} hergeleitete Form: für $x\in I_i$ gilt
  \begin{gather}
    s'(x) = -M_{i-1} \frac{(x_i-x)^2}{2h_i} + M_{i} \frac{(x-x_{i-1})^2}{2h_i}
    + \frac{f_{i}-f_{i-1}}{h_i} - \frac{h_i}{6} (M_i-M_{i-1}).
  \end{gather}
  Damit gilt am Punkt $x_i$
  \begin{align}
    s'(x_i) &= \frac{f_{i}-f_{i-1}}{h_i} - \frac{h_i}{6} (M_i-M_{i-1})
              + M_i \frac{h_i}{2}\\
            &= \frac{f_{i}-f_{i-1}}{h_i} + \frac{h_i}{3}M_i + \frac{h_i}{6}M_{i-1}\\
    s'(x_i) &= \frac{f_{i+1}-f_{i}}{h_{i+1}} - \frac{h_{i+1}}{6} (M_{i+1}-M_{i})
              - M_i \frac{h_{i+1}}{2}\\
    &= \frac{f_{i+1}-f_{i}}{h_{i+1}} - \frac{h_{i+1}}{3}M_i - \frac{h_{i+1}}{6}M_{i+1}
  \end{align}
  Aus der Gleichheit ergibt sich damit für $i=1,\dots,n-1$
  \begin{gather}
    \frac{h_i}{6}M_{i-1} + \frac{h_i+h_{i+1}}{3}M_i + \frac{h_{i+1}}{6}M_{i+1}
    =  \frac{f_{i+1}-f_{i}}{h_{i+1}} - \frac{f_{i}-f_{i-1}}{h_i}.
  \end{gather}
  Multiplizieren dieser Gleichungen mit $6/(h_i+h_{i+1})$ ergibt die
  Gestalt der Matrix.  Die natürliche Randbedingung ergibt $M_0 = 0$
  und $M_n = 0$, was die Einträge in der ersten und letzten Zeile ergibt.
\end{proof}

\begin{Lemma}{spline-invertierbar}
  Die Matrix
  \begin{gather}
    A =  \begin{pmatrix}
      2 & \lambda_0 \\
      \mu_1 & 2 & \lambda_1\\
      & \ddots & \ddots & \ddots \\
      && \mu_{n-1} & 2 & \lambda_{n-1}\\
      &&&\mu_n & 2
    \end{pmatrix}
  \end{gather}
  aus \slideref{Lemma}{splines-momente} hat die folgende Eigenschaft:
  für jeden Vektor $x\in \R^{n+1}$ und $y=Ax$ gilt
  \begin{gather}
    \norm{x}_\infty \le \norm{y}_\infty.
  \end{gather}
  Insbesondere ist $A$ invertierbar.
\end{Lemma}

\begin{proof}
  Sei $k$ ein Index, so dass $\abs{x_k} = \norm{x}_\infty$. Dann gilt
  \begin{gather}
    y_k \mu_k x_{k-1} + 2 x_{k} + \lambda_k x_{k+1}.
  \end{gather}
  Aus der Definition folgt $\abs{\lambda_k}< 1$ und $\abs{\mu_k} <
  1$. damit gilt
  \begin{align*}
    \norm{y}_{\infty} \ge \abs{y_k}
    &\ge 2 \abs{x_k} - \mu_k \abs{x_{k-1}} - \lambda_k \abs{x_{k+1}}\\
    &\ge \abs{x_k}(2-\mu_k-\lambda_k)\\
    & \ge \abs{x_k} = \norm{x}_\infty.
  \end{align*}
  Wäre nun $A$ singulär. Dann gäbe es $x\neq 0$ mit $Ax = 0$. Nach der
  Normabschätzung gilt dann aber $\norm{x}_\infty = 0$ im Widerspruch
  zur Annahme.
\end{proof}

\begin{Satz}{spline-approximation}
  Sei $f\in C^4[a,b]$ und sei $\mathcal I_h$ eine Zerlegung der
  Feinheit $h$, für die es zusätzlich eine Konstante $c>0$ gibt mit
  \begin{gather}
    \min_{i} h_i \ge c h.
  \end{gather}
  Dann gilt für den vollständig approximierenden Spline $s$ zu den
  Funktionswerten $f(x_i)$ die Abschätzung
  \begin{gather}
    \norm{f^{(\nu)}-s^{(\nu)}}_{\infty;[a,b]}
    \le c_\nu c h^{4-\nu}\norm{f^{(4)}}_{\infty;[a,b]}
  \end{gather}
  mit Konstanten $c_\nu$ unabhängig von $\mathcal I_h$ und $f$.
\end{Satz}

\begin{proof}
  Sei $g = (f''(x_0),\dots,f''(x_n))^T$ der Vektor der zweiten
  Ableitungen von $f$ in den Punkten $x_i$. Der Schlüssel ist die
  Abschätzung
  \begin{gather}
    \norm{M-g}_\infty \le \tfrac34 \norm{f^{(4)}}_\infty h^2.
  \end{gather}
  Dazu untersuchen wir den Vektor $r = A(M-g) = d-Ag$. Nach
  \slideref{Lemma}{spline-invertierbar} gilt
  \begin{gather}
    \label{eq:splines:3a}
    \norm{M-g}_\infty \le \norm{r}_\infty.
  \end{gather}
  Wir betrachten den Punkt $x_0$ und nutzen die Taylor-Interpolation
  \begin{align}
    f(x_1) &= f(x_0) + h_1 f'(x_0) + \frac{h_1^2}{2} f''(x_0)
    + \frac{h_1^3}{6} f^{(3)}(x_0) + \frac{h_1^4}{24} f^{(4)}(\xi_0),\\
    f''(x_1) &= f''(x_0) + h_1 f^{(3)}(x_0)
                 + \frac{h_1^2}{2} f^{(4)}(\xi_1)
  \end{align}
  wobei $\xi_0,\xi_1\in I_0$ gilt. Daraus folgt
  \begin{align}
    r_0 =& d_0 - 2f''(x_0) - f''(x_1)\\
    =& \frac{h_1}6\left(\frac{f_1-f_0}{h_1} - f'_0\right)
       - 2f''(x_0) - f''(x_1)\\
    =& \frac{6}{h_1}\left[
      f'(x_0) + \frac{h_1}{2} f''(x_0)
    + \frac{h_1^2}{6} f^{(3)}(x_0) + \frac{h_1^3}{24} f^{(4)}(\xi_0)
      - f'(x_0)\right]\\
      &-2 f''(x_0) - \left[
        f''(x_0) + h_1 f^{(3)}(x_0) + \frac{h_1^2}{2} f^{(4)}(\xi_1)
        \right]\\
    =& \frac{h_1^2}{4} f^{(4)}(\xi_0) - \frac{h_1^2}{2} f^{(4)}(\xi_1).
  \end{align}
  Damit gilt
  \begin{gather}
    \abs{r_0} \le \tfrac34 \norm{f^{(4)}}_\infty h^2.
  \end{gather}
  Dasselbe gilt für $r_n = d_n - f''(x_{n-1}) - 2 f''(x_n)$. Für die
  anderen Punkte ist mit demselben Argument und mehr Rechenaufwand
  \begin{align}
    r_i &= d_i - \mu_i f''(x_{i-1}) - 2 f''(x_i) - \lambda_i f''(x_{i+1})
    \\\notag
    &= \frac1{h_i+h_{i+1}} \left[
      \frac{h_{i+1}^3}{4} f^{(4)}(\xi_1)
      +\frac{h_i^3}{4} f^{(4)}(\xi_2)
      -\frac{h_{1+1}^3}{2} f^{(4)}(\xi_3)
      -\frac{h_i^3}{2} f^{(4)}(\xi_4)
      \right].
  \end{align}
  Daher ist
  \begin{gather}
    \abs{r_i} \le \tfrac34 \norm{f^{(4)}}_{\infty;[a,b]} h^2,
    \qquad i=1,\dots,n-1.
  \end{gather}
  Aus~\eqref{eq:splines:3a} schließen wir
  \begin{gather}
    \label{eq:splines:3}
    \norm{M-g}_{\infty} \le \norm{r}_\infty
    \le \frac34 h^2\norm{f^{(4)}}_{\infty;[a,b]}.
  \end{gather}
  
  Nun zeigen wir die Behauptung des Satzes für $\nu=3$. Sei
  $e(x) = s(x) - f(x)$. Für $x\in I_i$ ist
  \begin{multline}
    \label{eq:splines:4}
    e^{(3)}(x)
    = \frac{M_i-M_{i-1}}{h_i} - f^{(3)}(x) \\
    = \frac{M_i-f''(x_i)}{h_i} - \frac{M_{i-1}-f''(x_{i-1})}{h_i}
    \\
    + \frac{f''(x_i) - f''(x) -\bigl(f''(x_{i-1}) - f''(x)\bigr)}{h_i}
    - f^{(3)}(x).
  \end{multline}
  Die Werte in den Stützpunkten schätzen wir nun durch
  Taylor-Entwicklung um $x$ ab:
  \begin{gather}
    \begin{split}
    f''(x_i) &= f''(x) + f^{(3)}(x)(x_i-x)
    + \frac{f^{(4)}(\xi_1)}{2} (x_i-x)^2
    \\
    f''(x_{i-1}) &= f''(x) + f^{(3)}(x)(x_{i-1}-x)
    + \frac{f^{(4)}(\xi_1)}{2} (x_{i-1}-x)^2.
    \end{split}
  \end{gather}
  Setzen wir dies und~\eqref{eq:splines:3} in~\eqref{eq:splines:4}
  ein, so erhalten wir
  \begin{align}
    \notag
    \abs{s^{(3)}(x) - f(x)^{(3)}}
    &\le \frac32 \frac{h^2}{h_i} \norm{f^{(4)}}_{\infty;[a,b]}
      + \frac{h_i^2}{2h_i} \norm{f^{(4)}}_{\infty;[a,b]}
    \\
    \label{eq:splines:5}
    &\le 2 c h \norm{f^{(4)}}_{\infty;[a,b]}
  \end{align}
  Nun $\nu=2$. Sei $\tilde x\in\{x_{i-1},x_i\}$ der nächste Stützpunkt zu $x$, so dass $\abs{x-\tilde x} \l \nicefrac h2$. Es gilt für die zweiten Ableitungen
  \begin{gather}
    e''(x) = e''(\tilde x) + \int_{\tilde x}^x e^{(3)}(t)\dt,
  \end{gather}
  so dass wir mit~\eqref{eq:splines:3} und~\eqref{eq:splines:5} folgern
  \begin{align}
    \notag
    \abs{s''(x) - f''(x)} &\le \frac34 h^2 \norm{f^{(4)}}_{\infty;[a,b]}
    + ch^2 \norm{f^{(4)}}_{\infty;[a,b]}\\
    \label{eq:splines:6}
    &\le \frac74 ch^2 \norm{f^{(4)}}_{\infty;[a,b]}
    .
  \end{align}
  Aus den Interpolationsbedingungen folgt $e(x_i) = 0$ für
  $i=0,\dots,n$.  Damit gibt es nach dem Satz von Rolle in jedem
  Intervall $I_i$ ein $\xi_i$ mit $e'(\xi_i) = 0$. Somit gilt für $x\in I_i$
  \begin{gather}
    e'(x) = \int_{\xi_i}^x e''(t)\dt
  \end{gather}
  und daher mit~\eqref{eq:splines:6}
  \begin{gather}
    \abs{s'(x) - f'(x)} \le \frac74 ch^3 \norm{f^{(4)}}_{\infty;[a,b]}.
  \end{gather}
  Für $\nu=0$ können wir wieder $\tilde x$ wie oben wählen und erhalten aus
  \begin{gather}
    e(x) = \int_{\tilde x}^x e'(t)\dt
  \end{gather}
  die Abschätzung
  \begin{gather}
    \abs{s'(x) - f'(x)} \le \frac78 ch^4 \norm{f^{(4)}}_{\infty;[a,b]}.
  \end{gather}  
\end{proof}

\begin{remark}
  Werte wie $\nicefrac74$ oder $\nicefrac78$ in der obigen
  Abschätzung suggerieren, dass sie sehr scharf ist. In der Tat ist
  das aber nur so, wenn weder $f^{4}(x)$, noch $h_i$ stark
  variieren. Wir haben mehrfach $\norm{f^{k+1}}_{\sup;I_i}$ durch
  $\norm{f^{k+1}}_{\sup;[a,b]}$ sowie $h_i$ durch $h/c$
  ersetzt. Jedesmal hat sih der Fehler erhöht.

  Daraus ergibt sich, dass die wesentliche Aussage der Abschätzung ist: es gilt
  \begin{gather}
    \norm{f^{(\nu)}-s^{(\nu)}}_{\infty;[a,b]}
    = \mathcal O(h^{4-\nu}),
  \end{gather}
  wobei die Konstante von der 4. Ableitung der Funktion und
  der Gleichmäßigkeit des Gitters abhängt.
\end{remark}

%%% Local Variables:
%%% mode: latex
%%% TeX-master: "main"
%%% End:
