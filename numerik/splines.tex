Dieser Abschnitt folgt recht eng der Darstellung in \cite[Abschnitt 2.3]{Rannacher17}.

\subsection{Interpolation auf Teilintervallen}

\begin{Notation}{indices}
  In diesem Abschnitt bezeichne für die monotone Folge
  \begin{gather}
    a = x_0 < x_1 \dots < x_n = b
  \end{gather}
  stets
  \begin{gather}
    \mathcal I_h = \bigl\{ I_i = [x_{i-1},x_i] \big|
    \; i=1,\dots,n\bigr\}
  \end{gather}
  eine \define{Zerlegung} des Intervalls $I=[a,b]$, also
  \begin{gather}
    [a,b] = \bigcup_{i=1}^n I_h.
  \end{gather}
  Die Länge der Teilintervalle bezeichnen wir mit
  $h_i = \abs{I_i} = x_{i} - x_{i-1}$, mit $h=\max h_i$ die
  \define{Feinheit} der Unterteilung.
\end{Notation}

\begin{Notation}{reference-interval}
  Wir bezeichnen $\hat I = [-1,1]$ als Referenzintervall. Jedes
  Intervall $I_i$ einer Zerlegung $\mathcal I_h$ ergibt sich als Bild
  von $\hat I$ unter der affinen Abbildung
  \begin{gather}
    \begin{split}
      \Phi_i\colon \hat I &\to I_i\\
      \hat x &\mapsto \tfrac{x_{i}+x_{i-1}}{2} + \tfrac{h_i}{2} \hat x.
    \end{split}
  \end{gather}
\end{Notation}

\begin{Definition*}{piecewise-interpolation}{Stückweise Interpolation}
  Sei $\mathcal I_h$ eine Zerlegung von $[a,b]$. Auf dem
  Referenzintervall $\hat I$ sei eine Interpolationsaufgabe durch die
  Stützstellen $\hat x_0,\dots, \hat x_k$ definiert. Dann lautet die
  Aufgabe der stückweisen Interpolation auf $\mathcal I_h$: finde eine
  Funktion $s$ auf $[a,b]$, so dass für jedes $i=1,\dots,n$ die
  Einschränkung $s_{|I_i} \in \P_k$ der Interpolationsaufgabe mit den
  Stützstellen
  \begin{gather}
    x_{ij} = \Phi_i(\hat x_j),\qquad j=1,\dots,k
  \end{gather}
  genügt.
\end{Definition*}

\begin{Lemma}{piecewise-solvable}
  Die stückweise Interpolationsaufgabe hat eine eindeutige Lösung,
  wenn die Interpolationsaufgabe auf dem Referenzintervall eine solche
  besitzt.
\end{Lemma}

\begin{Lemma*}{scaling-interpolation}{Skalierungsargument}
  Für die Lösung $\hat p\in \P_k$ der Interpolationsaufgabe auf dem
  Referenzintervall gelte mit einer Konstanten $C$ unabhängig von
  $\hat f\in C^{k+1}(\hat I)$ die Fehlerabschätzung
  \begin{gather}
    \norm{\hat f- \hat p}_{\infty;\hat I} \le C \norm{\hat
      f^{(k+1)}}_{\infty;\hat I}.
  \end{gather}
  Dann ist der Fehler der stückweisen Interpolation beschränkt ist durch
  \begin{gather}
    \norm{f-s}_{\infty;[a,b]}
    \le C h^{k+1} \norm{f^{(k+1)}}_{\infty;[a,b]}.
  \end{gather}
\end{Lemma*}

\subsection{Splines}

\begin{Definition}{cubic-spline}
  Die Interpolationsaufgabe mit kubischen \define{Splines} lautet: finde eine Funktion $s\in S_h^{(3,2)}$
\end{Definition}

%%% Local Variables:
%%% mode: latex
%%% TeX-master: "main"
%%% End:
