\section{Interpolation mit Splines}

\subsection{Interpolation auf Teilintervallen}

\begin{Notation}{indices}
  $\P_k$, $n$ Intervalle indiziert mit $i$
\end{Notation}

\begin{Lemma*}{scaling-interpolation}{Skalierungsargument}
  Seit $\hat I$ ein fest gewähltes Intervall mit
  Interpolations-Stützpunkten $\hat x_0,\dots,\hat x_n$. Sei das
  Intervall $[a,b]$ unterteilt in Teilintervalle
  $I_k = \Phi_k(\hat I)$ der Länge $h_k \le h$. Eine Funktion
  $f\in C^{n+1[a,b]}$ werde in den Stützpunkten
  $x_{ki} = \Phi_k(\hat x_i)$ durch eine Funktion $v\in S^{n,0}$
  interpoliert.  Dann gibt es eine Konstante $C$, die nur von den
  Punkten $\hat x_0,\dots,\hat x_n$ abhängt, so dass der
  Interpolationsfehler beschränkt ist durch
  \begin{gather}
    \norm{f-v}_{\infty;[a,b]}
    \le C \frac{h^{n+1}}{(n+1)!} \norm{f^{(n+1)}}_{\infty;[a,b]}.
  \end{gather}
\end{Lemma*}


\subsection{Splines}
%%% Local Variables:
%%% mode: latex
%%% TeX-master: "main"
%%% End:
