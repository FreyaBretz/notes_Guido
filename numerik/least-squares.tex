\begin{intro}
  Die Methode der kleinsten Fehlerquadrate führt auf die Minimierungsaufgabe
  \begin{gather}
    \norm{Ax-b}_2 = \min.
  \end{gather}
\end{intro}

\begin{Satz}{normalengleichungen}
  Sei $A\in \R^{m\times n}$ mit $m\ge n$ und $b\in \R^m$. Dann ist
  $x\in\R^n$ genau dann eine Lösung des linearen Ausgleichsproblems
  \begin{gather}
    \norm{Ax-b}_2 = \min,
  \end{gather}
  wenn $x$ Lösung der \define{Normalengleichungen}
  \begin{gather}
    A^TA x = A^Tb
  \end{gather}
  ist. Insbesondere ist die Minimierungsaufgabe eindeutig lösbar, wenn
  $A$ vollen Rang hat.
\end{Satz}

%%% Local Variables:
%%% mode: latex
%%% TeX-master: "main"
%%% End:
