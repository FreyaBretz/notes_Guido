\begin{intro}
  Die Methode der kleinsten Fehlerquadrate führt auf die Minimierungsaufgabe
  \begin{gather}
    \norm{Ax-b}_2 = \min.
  \end{gather}
\end{intro}

\begin{Satz}{normalengleichungen}
  Sei $A\in \R^{m\times n}$ mit $m\ge n$ und $b\in \R^m$. Dann ist
  $x\in\R^n$ genau dann eine Lösung des linearen Ausgleichsproblems
  \begin{gather}
    \norm{Ax-b}_2 = \min,
  \end{gather}
  wenn $x$ Lösung der \define{Normalengleichungen}
  \begin{gather}
    A^TA x = A^Tb
  \end{gather}
  ist. Insbesondere ist die Minimierungsaufgabe eindeutig lösbar, wenn
  $A$ vollen Rang hat.
\end{Satz}

\begin{remark}
  Wir können die Normalengleichungen lösen, indem wir die symmetrische Matrix $C = A^TA\in \R^{n\times n}$ berechnen und dann eines der Verfahren der vorigen Abschnitte auf diese Matrix anwenden.
  
  Das Lemma nach der nächsten Definition legt nahe, dass das keine gute Idee ist, da sich die Konditionszahl durch das Matrixprodukt quadriert und damit die Lösungsgenauigkeit leidet.  
\end{remark}

\begin{Definition}{condition-rectangular}
  Die Konditionszahl einer rechteckigen Matrix maximalen Rangs bezüglich der Operatornorm zur Vektornorm $\norm{\cdot}$ ist
  \begin{gather}
    \cond(A) = \frac{\sup\limits_{\norm{x}=1}\norm{Ax}}{\inf\limits_{\norm{x}=1}\norm{Ax}}.
    \end{gather}
  Die Definition ist konsistent zur Definition für invertierbare, quadratische Matrizen.
\end{Definition}

\begin{Lemma}{condition-squared}
  Für eine Matrix $A\in\R^{m\times n}$ maximalen Rangs mit $m\ge n$ gilt
  \begin{gather}
    \cond_2(A^TA) = \cond_2(A)^2.
  \end{gather}
\end{Lemma}

\begin{proof}
  \begin{align}
    \cond_2(A)^2  &= \frac{\sup\norm{Ax}_2^2}{\inf\norm{Ax}_2^2}\\
                  &= \frac{\sup(x^TA^TAx)}{\inf(x^TA^TAx)}\\
                  &= \frac{\lambda_{\max}(A^TA)}{\lambda_{\in}(A^TA)}\\
                  &= \cond_2(A^TA).
  \end{align}
\end{proof}


\begin{Lemma}{qr-rectangular}
  Zu jeder Matrix $A\in \R^{m\times n}$ maximalen Rangs mit $m\le n$
  gibt es eine QR-Zerlegung
  \begin{gather}
  A = QR
  \end{gather}
  mit einer oberen Dreiecksmatrix $R\in\R^{n\times n}$ und einer Matrix $Q\in\R^{m\times n}$, deren Spalten ein Orthonormalsystem bilden.
  Unter der Zusatzbedingung $r_{ii}>0$ ist
  diese Zerlegung eindeutig.
\end{Lemma}

\begin{Satz}
  Sei $QR=A$ eine QR-Zerlegung. Dann kann die Lösung der Normalengleichungenvberechnet werden durch die Lösung des Systems
  \begin{gather}
    Rx = Q^Tb.
  \end{gather}
\end{Satz}

\begin{proof}
  Einsetzen der QR-Zerlegung in die Normalengleichungen ergibt
  \begin{gather}
  R^TQ^TQR x = R^T R x = R^TQ^T b.
  \end{gather}
  Da $R^T$ invertierbar ist, können wir die Inverse von links anwenden und erhalten das Resultat.
\end{proof}

% \begin{Definition}{pseudo-inverse}
%   Die \define{Pseudoinverse} $A^\dagger\in \R^{n\times m}$ einer Matrix $A\in\R^{m\times n}$ ist definiert dadurch, dass $x=a^\dagger b$ die Lösung des Minimierungsproblems 
% \end{Definition}

%%% Local Variables:
%%% mode: latex
%%% TeX-master: "main"
%%% End:
