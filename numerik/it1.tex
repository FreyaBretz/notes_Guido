\begin{Definition}{iteration}
  Ein \define{Iterationsverfahren} berechnet schrittweise
  Approximationen an die Lösung $x$ einer Aufgabe aus einem Startwert
  $x^{(0)}$ mit der Verfahrensvorschrift der Form
  \begin{gather}
    x^{(k+1)} = G(x^{(k)}), \qquad k=0,1,2,\dots.
  \end{gather}
  Das Verfahren heißt \define{konvergent}, wenn gilt $x^{(k)} \to x$.
\end{Definition}

\begin{Definition}{iteration-ordnung}
  Ein Iterationsverfahren
  \index{Konvergenzordnung!Iterationsverfahren} ist konvergent
  mindestens von Ordnung $p>1$ zum Grenzwert $x$, wenn es eine
  Konstante $c>0$ gibt, so dass
  \begin{gather}
    \norm*{x^{(k+1)} - x} \le  c \norm*{x^{(k)} - x}^p
  \end{gather}
  gilt. Es ist linear konvergent, wenn
  \begin{gather}
    \norm*{x^{(k+1)} - x} \le  c \norm*{x^{(k)} - x}
  \end{gather}
  mit einer Konstanten $c<1$. Wir sprechen von superlinearer Konvergenz, wenn
  \begin{gather}
    \norm*{x^{(k+1)} - x} = \smallo\left(\norm*{x^{(k)} - x}\right)
  \end{gather}
\end{Definition}

\begin{Definition}{kontraktion}
  Sei $M\subset\R^n$. Eine Abbildung $f\colon M\to M$ ist eine \define{Kontraktion} auf $M$, wenn es eine Konstante $\rho < 1$ gibt, so dass
  \begin{gather}
    \norm{f(x) - f(y)} \le \rho \norm{x-y} \qquad\forall x,y\in M.
  \end{gather}
\end{Definition}

\begin{Satz*}{bfs}{Banachscher Fixpunktsatz}
  Sei $f$ eine Kontraktion auf der abgeschlossenen Menge $M\subset\R^n$. Dann gibt es genau einen \define{Fixpunkt} $x\in M$, also
  \begin{gather}
    x = f(x).
  \end{gather}
\end{Satz*}

\begin{Satz}{optimize-solve}
  Sei $g\colon \R^n\to \R$. Dann gilt für eine Minimalstelle $x^*$ von $g$,
  also
  \begin{gather}
    x^* = \operatorname*{argmin}_{x\in\R^n} g(x),
  \end{gather}
  notwendig
  \begin{gather}
    \nabla g(x^*) = 0.
  \end{gather}
  Das Minimierungsproblem lässt sich also auf das Finden einer
  Nullstelle von $f\colon \R^n\to \R^n$ mit $f(x) = \nabla g(x)$
  reduzieren. Umgekehrt lässt sich die Aufgabe, eine Nullstelle einer
  Funktion $f\colon \R^n \to \R^n$ zu finden, durch die Minimierung
  des Funktionals $g(x) = \norm{f(x)}$ darstellen.
\end{Satz}

%%% Local Variables:
%%% mode: latex
%%% TeX-master: "main"
%%% End:
