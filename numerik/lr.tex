\begin{Notation}{quadratische-matrizen}
  Da wir uns in diesem Abschnitt mit der Lösung quadraischer
  Gleichungssysteme beschäftigen, gelte für alle Matrizen, soweit
  nicht anders vermerkt, dass ihre Dimension $n\times n$ sei.
\end{Notation}

\subsection{Dreiecksmatrizen und Frobeniusmatrizen}

\begin{Definition}{dreiecksmatrix}
  Für eine \define{untere Dreiecksmatrix} $L \in \R^{n\times n}$ gilt
  \begin{gather}
    \ell_{ij} = 0,\qquad j>i.
  \end{gather}
  Für eine \define{obere Dreiecksmatrix} $R \in \R^{n\times n}$ gilt
  \begin{gather}
    r_{ij} = 0,\qquad j<i.
  \end{gather}
\end{Definition}

\begin{Satz}{dreieck-gruppe}
  Die Mengen der invertierbaren oberen und unteren Dreiecksmatrizen
  bilden jeweils eine multiplikative Gruppe. Die Determinante einer
  Dreiecksmatrix ist das Produkt ihrer Diagonalelemente.
\end{Satz}

\begin{proof}
  Hausaufgabe
\end{proof}

\begin{Korollar}{dreieck-inverse}
  Eine Dreiecksmatrix ist invertierbar genau dann, wenn alle ihre
  Diagonalelemente von null verschieden sind.
\end{Korollar}

\begin{Algorithmus}{vorwaerts-rueckwaerts}
  Die Lösung der linearen Gleichungssysteme
  \begin{gather}
    Lx = b \qquad Rx = b
  \end{gather}
  mit einer unteren Dreiecksmatrix $L$ und einer oberen Dreiecksmatrix
  $R$ lässt sich sukzessive durch Vorwärts- bzw.\ Rückwärtseinsetzen
  berechnen.
  \begin{minipage}[t]{.45\linewidth}
    \lstinputlisting[basicstyle=\footnotesize]{code/forsub.py}    
  \end{minipage}
  \begin{minipage}[t]{.45\linewidth}
    \lstinputlisting[basicstyle=\footnotesize]{code/backsub.py}    
  \end{minipage}
\end{Algorithmus}

\begin{Definition}{frobenius-matrix}
  Eine Matrix der Gestalt
  \begin{gather}
    G_k=\begin{bmatrix}
      1 & & & & & \\
      &\ddots & & & & \\
      &   & 1& & &\\
      &   & g_{k+1,k}&1 & &\\
      &   & \vdots& &\ddots &\\
      &   & g_{nk}& & &1
    \end{bmatrix}
  \end{gather}
  mit von null verschiedenen Subdiagonaleinträgen nur in Spalte $k$
  heißt \define{Frobenius-Matrix}.
\end{Definition}

\begin{Lemma}{frobenius-matrix}
  Für Frobenius-Matrizen gilt
  \begin{gather}
    G_k^{-1} = 2\identity-G_k.
  \end{gather}
  Das Ergebnis des Produktes $G_kA$ einer Frobeniusmatrix mit einer
  beliebigen Matrix ergibt sich aus $A$ dadurch, dass auf die $j$-te
  Zeile das $g_{jk}$-fache der $k$-ten Zeile addiert wird.

  Sei $k_1<\dots<k_m$ eine aufsteigende Folge von Indizes. Dann ist
  \begin{gather}
    G_{k_1}\cdots G_{k_m} = \sum_{i=1}^m G_i - (m-1) \identity.
  \end{gather}
  Insbesondere gilt
  \begin{gather}
    G_1\dots G_n =
    \begin{bmatrix}
      1\\
      g_{21} & 1 \\
      \vdots & \ddots & \ddots \\
      g_{n1}  & \dots & g_{n,n-1} & 1
    \end{bmatrix}
  \end{gather}
\end{Lemma}

\subsection{Konstruktion der LR-Zerlegung}

\begin{Lemma}{elimination-1}
  Bei der Gauß-Elimination lässt sich die Elimination der
  Subdiagonalelemente der $k$-ten Spalte als Matrix-Produkt
  \begin{gather}
    A^{(k+1)} = L^{-1}_k A^{(k)},
    \qquad b^{(k+1)} = L^{-1}_k b^{(k)},
    \qquad k=1,\dots,n-1
  \end{gather}
  mit $A^{(1)}=A$, $b^{(1)}=b$ und den Frobenius-Matrizen
  \begin{gather}
    L_k =\begin{bmatrix}
      1 & & & & & \\
      &\ddots & & & & \\
      &   & 1& & &\\
      &   & \ell_{k+1,k}&1 & &\\
      &   & \vdots& &\ddots &\\
      &   & \ell_{nk}& & &1
    \end{bmatrix},
    \qquad
    \ell_{ik} = \frac{a_{ik}^{(k)}}{a_{kk}^{(k)}}
  \end{gather}
  schreiben.
\end{Lemma}

\begin{Satz}{elimination-2}
  Nach $n-1$ Schritten der Gauß-Elminiation erhält man das transformierte lineare Gleichungssystem
  \begin{gather}
    R x = y,\qquad R = L^{-1}A, \qquad y=L^{-1}b,
    \qquad L = L_1\cdots L_{n-1},
  \end{gather}
  und die LR-Zerlegung
  \begin{gather}
    A = LR
  \end{gather}
  mit einer oberen Dreiecksmatrix $R$ und einer unteren Dreiecksmatrix
  $L$, deren Diagonale aus Einsen besteht.
\end{Satz}

\begin{Lemma}{aufwand-LR}
  Der Aufwand der LR-Zerlegung einer $n\times n$-Matrix ist
  \begin{gather}
    \tfrac13 n^3 + \bigo(n^2).
  \end{gather}
\end{Lemma}

\begin{Satz}{existenz-LR}
  Ist die Matrix $A$ invertierbar, dann ist im $k$-ten Schritt der
  Gauß-elimination wenigstens eins der Elemente $a^{(k)}_{jk}$ mit
  $j\ge k$ von null verschieden. für den Fall, dass
  $a^{(k)}_{kk} = 0$, kann damit die Elimination nach Vertauschen der
  Zeilen $j$ und $k$ fortgesetzt werden.
\end{Satz}

\begin{Definition}{spalten-pivot}
  Führt man im $k$-ten Schritt der Gauß-Elimination eine
  Zeilenvertauschung durch, so dass
  \begin{gather}
    \abs{a^{(k)}_{kk}} = \max_{j\ge k} \abs{a^{(k)}_{jk}},
  \end{gather}
  so spricht man von Gauß-Elimination mit
  \define{Spalten-Pivotierung}. Vertauscht man sogar die verbleibenden
  Zeilen und spalten, so dass
  \begin{gather}
    \abs{a^{(k)}_{kk}} = \max_{i,j\ge k} \abs{a^{(k)}_{ij}},
  \end{gather}
  handelt es sich um \define{vollständige Pivotierung}.
\end{Definition}

\begin{Lemma}{spalten-pivot}
  Führt man die Gauß-Elimination mit Spalten-Pivotierung durch, so gilt
  für die Matrix $L$:
  \begin{gather}
    \abs{\ell_{ij}} \le 1,\qquad1\le i,j\le n.
  \end{gather}
\end{Lemma}

\subsection{Fehleranalyse}


%%% Local Variables:
%%% mode: latex
%%% TeX-master: "main"
%%% End:
