\usepackage[notref,notcite]{showkeys}
% \usepackage{hyperref}
\lstset{language=Python}
\usetikzlibrary{svg.path}
\excludecomment{solution}
\tikzset{velox/.style={color=black,draw,fill=red,thick,%
    shape=diamond,aspect=.4,
    inner sep=1.3pt,transform shape}}
\tikzset{veloy/.style={color=black,draw,fill=red,thick,%
    shape=diamond,aspect=2.5,
    inner sep=1.3pt,transform shape}}
\tikzset{veloxy/.style={color=black,draw,fill=red,thick,%
    shape=star,star points=4,star point ratio=2.2,
    inner sep=1.3pt,transform shape}}
\tikzset{pressure/.style={color=black,draw,fill=cyan,thick,%
    shape=circle,inner sep=2pt,transform shape}}
\tikzset{velo/.style={transform shape,double=red,arrows={-Stealth[open,fill=red]}}}

%% Macros for drawing degrees of freedom for different shapes/elements.
%% Arguments are always:
%%   #1: Starting point
%%   #2: End point
%%   #3: polynomial degree
%%   #4: node settings

\tikzset{pics/edgenormal/.style args={#1/#2/#3/#4}{%
    code={%
      \draw #1 -- #2
      node foreach \x [evaluate=\x as \xval] in {1,...,#3} [#4,sloped,pos=\xval/(#3+1)] {};
      }
}}


%% Macros for drawing degrees of freedom for different shapes/elements.
%% Arguments are always:
%%   #1: polynomial degree
%%   #2: node settings

\tikzset{pics/tripile/.style args={#1/#2}{%
    code={%
      \coordinate (top) at (0,#1);
      \foreach \i in{0,...,#1}
      \foreach \j in{0,...,\i}
      {
        \tikzmath{
          \y = .3*(2/3*#1-\i)*cos(30);
          \x = .3*(\i/2-\j);
        }
        \node[#2] at (\x,\y) {};
      }
    }
}}

\tikzset{pics/tensor/.style args={#1/#2/#3}{%
    code={%
      \coordinate (top) at (0,#1);
      \foreach \i in{0,...,#1}
      \foreach \j in{0,...,#2}
      {
        \tikzmath{
          \y = 2*(\i+1)/(#1+2);
          \x = 2*(\j+1)/(#2+2);
        }
        \node[#3] at (\x,\y) {};
      }
    }
}}

\tikzset{pics/pfem/.style args={#1/#2}{%
    code={%
      \tikzmath{ \ytop=2*cos(30); }
      \coordinate (top) at (0,\ytop);

      \foreach \i in{0,...,#1}
      \foreach \j in{0,...,\i}
      {
        \tikzmath{
          \y = \ytop-\ytop*\i/#1;
          \x = 2*(\i/2-\j)/#1+1;
        }
        \node[#2] at (\x,\y) {};
      }
    }
}}

\tikzset{pics/qfem/.style args={#1/#2}{%
    code={%
      \foreach \i in{0,...,#1}
      \foreach \j in{0,...,#1}
      {
        \tikzmath{
          \y = 2-2*\i/#1;
          \x = 2-2*\j/#1;
        }
        \node[#2] at (\x,\y) {};
      }
    }
}}

%%% Local Variables:
%%% mode: latex
%%% TeX-master: "all"
%%% End:


\def\constref#1{C_{\text{\ref{#1}}}}
\title{Einführung in die Numerik}
\author{Guido Kanschat}
\date{\today}

\newcommand{\rd}{\operatorname{rd}}
\newcommand{\eps}{\texttt{eps}}
\begin{document}
\maketitle
\tableofcontents
\chapter{Orthogonale Polynome}
\section{Polynomräume}

% \begin{Satz}{nullstellen}
%   Ein reelles Polynom vom Grad $n$ hat höchstens $n$ Nullstellen oder es ist das Nullpolynom.
% \end{Satz}

% \begin{proof}
%   Für $n=1$ handelt es sich um ein lineares Polynom und die Aussage
%   des Satzes ist unmittelbar klar. Sei nun $p$ ein Polynom strikt vom
%   Grad $n>1$ mit Nullstelle $x_0$. Dann gibt es nach dem euklidischen
%   Algorithmus zur Division mit Rest ein Polynom $q$ vom Grad $n-1$ und
%   eine Konstante $c$, so dass
%   \begin{gather}
%     p(x) = (x-x_0)q(x)+c.
%   \end{gather}
%   Daraus folgt $p(x_0) = c$, so dass folgt $c=0$. Wir können dieses
%   Verfahren für alle weiteren Nullstellen $x_1,\dots,x_m$ wiederholen
%   und erhalten
%   \begin{gather}
%     p(x) =  r(x) \prod_{k=0}^m (x-x_i),
%   \end{gather}
%   wobei $r(x)$ ein Polynom vom Grad $n-m$ sein muss, da $p$ vom Grad
%   $n$ ist. Insbesondere muss gelten $m\le n$.
% \end{proof}

% \begin{Korollar}{polynome-identisch}
%   Zwei reelle Polynome vom Grad $n$ sind identisch, wenn sie in
%   mindestens $n+1$ Punkten übereinstimmen. 
% \end{Korollar}


\begin{Lemma}{monome-linear-unabhaengig}
  Die Menge der Monome $\{x^0, x^1,\dots,x^n\}$ ist linear unabhängig.
\end{Lemma}

\begin{proof}
  Sei $p$ ein Polynom vom Grad $n$, also
  \begin{gather}
     p(x) = a_nx^n+a_{n-1}x^{n-1}+\dots+a_1x+a_0
   \end{gather}
   $p$ ist also gerade eine Linearkombination der Monome.  Zu zeigen
   ist, dass aus der Eigenschaft $p \equiv 0$ folgt, dass alle
   Koeffizienten verschwinden, also
  \begin{gather}
    p(x) \equiv 0
    \quad\Rightarrow\quad a_n = \dots = a_0 = 0.
  \end{gather}
  Zu diesem Zweck berechnen wir die $n$-te Ableitung von $p$ und
  erhalten, da mit $p$ auch alle seine Ableitungen identisch
  verschwinden,
  \begin{gather}
    n! a_n = 0.
  \end{gather}
  Daraus schließen wir $a_n = 0$. Nun gilt für die $(n-1)$-te Ableitung
  \begin{gather}
    n! a_n x + (n-1)! a_{n-1} = (n-1)! a_{n-1} = 0.
  \end{gather}
  Auf diese Weise schließen wir rekursiv bis $a_0$, dass alle Koeffizienten verschwinden. Damit ist das Lemma bewiesen.
\end{proof}

\begin{Satz}{polynomraum}
  Die Polynome vom maximalen Grad $n$ bilden einen Vektorraum der
  Dimension $n+1$.  Wir bezeichnen ihn mit $\P_n$.
\end{Satz}

\begin{proof}
  Es ist leicht nachzurechnen, dass sowohl die Summe, als auch reelle
  Vielfache von Polynomen wieder Polynome sind. Insbesondere erhöhen
  beide Operationen den Grad nicht. Damit ist $\P_n$ ein
  Vektorraum. Er wird per definitionem von den Monomen vom Grad bis zu
  $n$ erzeugt. Da diese nach
  \slideref{Lemma}{monome-linear-unabhaengig} linear unabhängig sind,
  bilden sie eine Basis und die Dimension von $\P_n$ ist $n+1$.
\end{proof}

\begin{Quiz}{Polynomräume}
  Gegeben beliebige Werte $x_j\in\R$ mit $j=1,\dots,n$. Die Menge der
  Polynome $p_i$ definiert durch
  \begin{align*}
    p_0(x) &= 1\\
    p_i(x) &= \prod_{j=1}^i (x-x_j),\qquad i=1,\dots,n
  \end{align*}
  \begin{enumerate}[A]
  \item ist linear unabhängig
  \item ist linear abhängig
  \item ist ein Erzeugendensystem für $\P_n$
  \item ist eine Basis von $\P_n$
  \end{enumerate}
\end{Quiz}
\section{Skalarprodukt und Orthogonalität}
\begin{Definition}{skalarprodukt}
  Sei $V$ ein reeller Vektorraum. Eine Abbildung
  $a\colon V \times V \to \R$ heißt \define{Bilinearform}, wenn für
  $u,v,w\in V$ und $\lambda,\mu\in \R$ gilt
  \begin{align}
    a(\lambda u + \mu v,w) &= \lambda a(u,w) + \mu a(v,w)\\
    a(w,\lambda u + \mu v) &= \lambda (w,u) + \mu a(w,v).
  \end{align}
  Eine Bilinearform heißt \define{symmetrisch}, wenn für $u,v\in V$ gilt
  \begin{gather}
    a(u,v) = a(v,u).
  \end{gather}
  Sie heißt \define{positiv semi-definit}, wenn $a(u,u) \ge 0$ für alle
  $u\in V$ und \define{positiv definit}, wenn zusätzlich
  \begin{gather}
    a(u,u) = 0 \quad \Longrightarrow \quad u=0.
  \end{gather}
  Eine symmetrische, positiv definite Bilinearform heißt
  \define{Skalarprodukt}, in der Regel notiert als $\scal(\cdot,\cdot)$.
\end{Definition}

%%%%%%%%%%%%%%%%%%%%%%%%%%%%%%%%%%%%%%%%%%%%%%%%%%%%%%%%%%%%%%%%%%%%%%
\begin{Lemma*}{bcs}{Bunjakowski-Cauchy-Schwarzsche Ungleichung}
  Sei $\scal(\cdot,\cdot)$ ein Skalarprodukt auf $V$.  Für zwei beliebige Elemente $u,v\in V$ gilt
  \begin{gather}
    \abs{\scal(u,v)} \le \sqrt{\scal(u,u)} \, \sqrt{\scal(v,v)}.
  \end{gather}
  Gleichheit gilt genau dann, wenn $u$ und $v$ kollinear sind, also
  $v=\alpha u$ mit einem skalaren Faktor $\alpha$.
\end{Lemma*}

\begin{proof}
  Zunächst zeigen wir nur die Ungleichung: Für $v=0\in V$ ist sie
  offensichtlich.
  
  Seien nun $v,u \in V$ keine Nullvektoren. Für beliebige $\mu, \lambda \in \R$
  gilt wegen der Bilinearität 
  \begin{gather}
   0 \le \scal(\lambda u + \mu v,\lambda u +  \mu v)
    = \lambda^{2} \scal(u,u)+2 \mu \lambda \scal(u,v) +\mu^{2} \scal(v,v)
  \end{gather}
  Setze $\lambda := \scal(v,v) \neq 0$
  \begin{gather}
   0 \le \scal(v,v)^{2} \scal(u,u) + 2\mu \scal(v,v)\scal(u,v) +\mu^{2}\scal(v,v)
  \end{gather}
  Dividiere durch$\scal(v,v)$
  \begin{gather}
   0 \le \scal(v,v) \scal(u,u) + 2\mu \scal(u,v) +\mu^{2}
  \end{gather}
  Setze nun $\mu := -\scal(u,v)$
  \begin{gather}
    0 \le \scal(v,v) \scal(u,u) -2\scal(u,v)^{2}+\scal(u,v)^{2}
  \end{gather}
  Daraus folgt
  \begin{gather}
    \scal(u,v)^{2} \le \scal(u,u) \scal(v,v)
  \end{gather}
  und mit der Monotonie der Quadratfunktion die Ungleichung.

  Nun bleibt die Äquivalenz für die Gleichheit zu zeigen.
  Für $v=0$ ist dies wieder trivial erfüllt. Seien zunächst $u,v$ linear abhängig, also zum Beispiel $u=av$.
  Dann gilt mit der Abkürzung $f(v) = \sqrt{\scal(v,v)}$
  \begin{gather}
    \abs{\scal(u,v)} = \abs{\scal(av,v)}
    = \abs{a} \cdot f(v) \cdot f(v)
    = f(av) \cdot f(v) =f(u) \cdot f(v).
  \end{gather}

  Gelte nun umgekehrt $\scal(u,v) = \sqrt{\scal(u,u)}\sqrt{\scal(v,v)}$.
  Es folgt
  \begin{gather}
     \scal(v,v) \scal(u,u) -2\scal(u,v)^{2}+\scal(u,v)^{2} = 0.
  \end{gather}
  Setze $\mu = \scal(u,v)\neq 0 $ und
  $\lambda = \scal(v,v)\neq 0$. Dann erhält man
  \begin{gather}
    \lambda \scal(u,u) - 2 \mu \scal(u,v) + \mu^2 = 0.
  \end{gather}
  Multipliplikation mit $\scal(v,v)$ ergibt
  \begin{gather}
   \lambda^2 \scal(u,u)+2\mu \scal(u,v)\scal(v,v) +\mu^{2}\scal(v,v) = 0 = \scal(\lambda u-\mu v,\lambda u-\mu v).
  \end{gather}
  
  Wegen der Definitheit folgt nun
  $\lambda u + \mu v = 0$ und da $\mu$ und $\lambda$ ungleich Null sind gilt,
  dass $ u,v$ linear abhängig sind
\end{proof}

%%%%%%%%%%%%%%%%%%%%%%%%%%%%%%%%%%%%%%%%%%%%%%%%%%%%%%%%%%%%%%%%%%%%%%
\begin{Lemma}{hilbertnorm}
  Sei $V$ ein reeller Vektorraum mit Skalarprodukt
  $\scal(\cdot,\cdot)$. Dann ist durch
  \begin{gather}
    \norm{u} = \sqrt{\scal(u,u)}
  \end{gather}
  auf $V$ eine Norm definiert. Ein reeller Vektorraum $V$ mit
  Skalarprodukt und zugehöriger Norm heißt \define{euklidischer
    Vektorraum}.
\end{Lemma}

\begin{proof}
  Das Skalarprodukt ist nicht negativ, daher ist die Abbildung $\norm{\cdot}\colon V \to \R$ wohldefiniert.
  Wir müssen nun die Normeigenschaften nachrechnen. Sei dazu $u \in V$. Es gilt
  \begin{enumerate}
  \item Nichtnegativität und Definitheit folgen sofort aus den entsprechenden Eigenschaften des Skalarprodukts.
  \item Homogenität
  \begin{gather}
    \norm{\lambda u} = \sqrt{\scal(\lambda u,\lambda u)}
    =\sqrt{\lambda^{2}\scal(u,u)}
    = \abs{\lambda}\sqrt{\scal(u,u)}
    =\abs{\lambda}\norm{u}
  \end{gather}
  \item Deiecksungleichung
  \begin{gather}
    \begin{aligned}
      \norm{u+v}^{2}
      &= \scal(u+v,u+v)\\
      &= \scal(u,u)+ 2 \scal(u,v) + \scal(v,v)\\
      \label{eq:orthopoly:1}
      &\le \scal(u,u)+ 2 \norm{u} \, \norm{v} + \scal(v,v)\\
      &=\norm{u}^{2}+ 2 \norm{u} \, \norm{v}+ \norm{v}^{2}\\
      & =(\norm{u}+\norm{v})^{2}\\
    \end{aligned}
  \end{gather}
  Daraus folgt durch Wurzelziehen auf beiden Seiten $\norm{u+v} \le \norm{u}+\norm{v}$.
  Für die Abschätzung in Zeile~\eqref{eq:orthopoly:1} haben wir die
  Bunyakovsky-Cauchy-Schwarz-Ungleichung aus \slideref{Lemma}{bcs} verwendet.
  \end{enumerate}
\end{proof}

\begin{Lemma*}{l2-norm}{$L^2$-Skalarprodukt}
  Auf dem Raum $V=\P_n$ der reellen Polynome vom Grad bis zu $n$ ist durch
  \begin{gather}
    \scal(p,q) = \int_{-1}^1 p(x)q(x)\dx
  \end{gather}
  ein Skalarprodukt definiert. Dieses wird $L^2$ Skalarprodukt genannt.
\end{Lemma*}

\begin{proof}
  Hier gilt es zu prüfen, ob die Abbildung auch die vier Eigenschaften eines
  Skalarprodukts erfüllt.\\
  Seien  $p,q,g \in \P_n$ in diesem Beweis.\\
  Da wir schon von einem Skalarprodukt ausgehen, empfiehlt es sich
  zuerst die Symmetrie zu zeigen.
  \begin{gather}
    \scal(p,q) =  \int_{-1}^1 p(x)q(x)\dx = \int_{-1}^1 q(x)p(x)\dx
    =\scal(q,p)
  \end{gather}
  Wenn wir nun zeigen, dass es eine Bilinearform ist müssen wir nur noch eine
  Identität zeigen, da wir schon wissen, dass die Symmetrieeigenschaft
  erfüllt ist.
  \begin{gather}
    \begin{aligned}
    \scal(\lambda p + \mu q, g)
    &= \int_{-1}^1 (\lambda p(x)+ \mu q(x))g(x)\dx\\
   & = \int_{-1}^1 \lambda p(x)g(x)+ \mu q(x)g(x)\dx \\
   &= \int_{-1}^1 \lambda  p(x)g(x)\dx + \int_{-1}^1 \mu q(x)g(x)\dx \\
   &= \lambda \int_{-1}^1 p(x)g(x)\dx + \mu  \int_{-1}^1 q(x)g(x)\dx \\
   &= \lambda \scal(p,g) + \mu \scal(q,g)
    \end{aligned}
  \end{gather}
  Da wir die Symmetrie vorher gezeigt haben, gilt Linearität auch
  im zweiten Argument.\\
  
  Als letztes zeigen wir, dass die Abbildung positiv definit ist.
  \begin{gather}
    0 = \scal(p,p) = \int_{-1}^1 p(x)p(x)\dx =\int_{-1}^1 p(x)^{2}\dx
  \end{gather}
  Aus den Integraleigenschaften folgt
  \begin {gather}
    0 = p(x)^{2} \quad \forall x
  \end{gather}
  Dies kann nur der Fall sein, wenn $p \equiv 0$ ist.\\
  Somit haben wir nachgerechnet, dass es sich um Skalarprodukt handelt.
  \end{proof}

\begin{Definition}{l2-norm}
  Nach dem \slideref{Lemma}{hilbertnorm} können wir mit diesem Skalarprodukt eine Norm auf $\P_n$
  definierten. Diese Norm wird als die $L^2$ Norm bezeichnet.
  \begin{gather}
    \norm{f}_{L^2} = \sqrt{\scal(f,f)_{L^2}} = \int_{-1}^1 f(x)^2 dx
  \end{gather}
  \end{Definition}

\begin{Definition}{orthogonal}
  Zwei Vektoren $u,v\in V$ heißen \define{orthogonal}, wenn
  \begin{gather}
    \scal(u,v) = 0.
  \end{gather}
  Ein Vektor $u\in V$ ist orthogonal zum Untervektorraum $W\subset V$, wenn
  \begin{gather}
    \scal(u,v) = 0\quad\forall v\in W.
  \end{gather}
\end{Definition}

\begin{Notation}{euklidischer-vr}
  Von nun an bezeichnet $V$ immer einen endlichdimensionalen, reellen,
  euklidischen Vektorraum.
\end{Notation}

\begin{Lemma*}{pythagoras}{Pythagoras}
  Seien zwei Vektoren $u\in V$ und $v\in V$ orthogonal zueinander. Dann gilt
  \begin{gather}
    \norm{u+v}^{2} = \norm{u}^{2} + \norm{v}^{2}
  \end{gather}
\end{Lemma*}

\begin{proof}
  Seien $u,v \in V$. Es gilt $ 0 = \scal(u,v)$
   \begin{gather}
    \norm{u+v}^{2} = \scal(u+v,u+v)
    %=\scal(u+v,u)+\scal(u+v,v)
    %=\scal(u,u)+\scal(v,u)+\scal(u,v)+\scal(v,v)
    =\norm{u}^{2} + \norm{v}^{2} +2\scal(u,v) = \norm{u}^{2} + \norm{v}^{2}
  \end{gather}
\end{proof}

\section{Bestapproximation und orthogonale Projektion}
\begin{Definition}{bestapproximation}
  Sei $A\subset V$ ein affiner Unterraum eines euklidischen
  Vektorraums. Dann ist die Bestapproximation $u_b\in A$ eines Vektors
  $u\in V$ in $A$ definiert durch die Beziehung
  \begin{gather}
    \norm{u-u_b} = \min_{v\in A} \norm{u-v}.
  \end{gather}
\end{Definition}

\begin{Satz}{bestapproximation}
  Sei $w \in V$ und $W \subset V$.
  Sei $A=w+W$ ein nichtleerer, affiner Unterraum von $V$. Dann
  existiert die Bestapproximation nach
  \slideref{Definition}{bestapproximation} und ist eindeutig
  bestimmt. Es gilt die notwendige und hinreichende Bedingung
  \begin{gather}
    \scal(u-u_b,v) = 0 \quad \forall v\in W.
  \end{gather}
  Das heißt $ u_b$ ist Bestapproximation genau dann wenn $u-u_b$
  orthogonal zu $W$ bzgl. des Skalarprodukts $\scal(\cdot,\cdot)$ ist.
\end{Satz}

\begin{proof}
  Der Beweis gliedert sich in drei Teile. Zuert wird die Äquivalenz
  gezeigt danach zeigen wir die Eindeutigkeit und zum Schluss
  erst die Existenz.\\ \\
 $\glqq \Rightarrow \glqq$
  Sei $ u_b \in A$ die Bestapproximation des Vektors $ u \in V$\\
  Wir defnieren nun eine Funktion:
  \begin{gather}
    F_v(x):= \norm{u-u_b-xv}^{2}, x \in \R,  v\in A
  \end{gather}

  Diese Funktion besitzt ein Minimum bei x=0. Folglich gilt
  \begin{gather}
    \left. \frac{d}{dx} F(x) \right|_{x=0}
    =\left. \frac{d}{dx}\norm{u-u_b-xv}^{2} \right|_{x=0}=0
  \end{gather}
    
  Dies kann weiter umgeformt werden zu
  $\scal(u-u_b-xv,v)|_{x=0}=0 \ \forall v\in A$ und folglich zu
  \begin{gather}
   \scal(u-u_b,v)=0 \ \forall v\in A
  \end{gather}

  $\grqq \Leftarrow \glqq$
  Nun erfüllt $u_b\in A$ die Bedingung.\\
  Dann gilt mit einem beliebigen $v\in A$:
  \begin{gather}
   \norm{u-u_b}^{2}=\scal(u-u_b,u-u_b)\\
   = \scal(u-u_b,u-v)+\scal(u-u_b,v-u_b)\\
   \le \norm{u-u_b}\cdot\norm{u-v}\\
  \end{gather}
  Daraus folgt $\norm{u-u_b} \le \inf_{v\in A}{\norm{u-v}}$\\
  Damit erfüllt $u_b$ eben die Definiton der Bestapproximation\\ \\
  Nun zur Eindeutigkeit:\\
  Seien $u_b$ und $u_d \in$ A zwei Bestapproximationen.
  Dann gilt notwendigerweise
  \begin{gather}
   \scal(u-u_b,v) = 0 = \scal(u-u_d,v) \quad \forall v\in A
  \end{gather}
  Dies wird umgeformt zu
  \begin{gather}
  \scal(u-u_d,v)-\scal(u-u_b,v)=0 \quad  \forall v\in A \\
  \scal(u_b-u_d,v) = 0 \quad \forall v \in A
  \end{gather}
  Wähle nun $v:=u_b-u_d \in A$. Dies ergibt
  $\norm{u_b-u_d}^{2} =0$ und somit folgt $u_b = u_d$\\
  Die Existenz:\\
  Der endliche dimensionale Teilraum A$\subseteq$V besitzt eine Basis
  $(b_1,\dots, b_n)$ mit $n:=dim V$. Die gesuchte Approximation
  $u_b\in A$ lässt sich
  durch die Basis in folgender Form darstellen
  \begin{gather}
   u_b = \sum_{k=1}^n a_k b_k
  \end{gather}
  Dies wird in die notwendige Orthogonalitätsbedingung
  \slideref{Satz}{bestapproximation} eingesetzt.
  \begin{gather}
   \scal(u-\sum_{k=1}^n a_k b_k,v)=\scal(u,v)-\sum_{k=1}^n a_k\scal(b_k,v)=0
   \quad \forall v\in A
   \end{gather}
 Dies ist bei der Wahl von $v:=b_i \quad i=1,\dots,n$ äquivalent zu dem
 linearen $n$x$n$ Gleichungssystem.
 \begin{gather}
   \sum_{k=1}^n\scal(b_k,b_i) a_k= \scal(u,b_i) \quad i=1,\dots,n
 \end{gather}
 Definiere nun $A,x,b$ wie folgt
 \begin{gather}
  A:=(\scal(b_k,b_i))_{i,k=1}^n \quad x:=(a_k)_{k=1}^n\quad b:=(\scal(u,b_i))_{i=1}^n
 \end{gather}
 Dadurch lässt sich das LGS in der Form $Ax=b$ schreiben.
 Betrachte nun folgendes
 \begin{gather}
  x^{T}Ax =\sum_{i,k=1}^n a_i a_k\scal(b_k,b_i)=\scal(u_b,u_b)\ge 0
 \end{gather}
 $A$ ist folglich positiv definit. Das Gleichungssystem $Ax=b$ ist also für
 jede rechte Seite $b$, das heißt für jedes $u \in V$ eindeutig lösbar.
 Folglich bestimmt die Orthogonalitätsbedingung eindeutig ein Element
 $u_b \in A$, welches dann die Bestapproximation von $u$ ist.
 
\end{proof}

\begin{Definition}{komplement-projektion}
  Sei $W \subset V$ ein Untervektorraum. Dann gilt
  $V = W \oplus W^\perp$, wobei das \define{orthogonale Komplement}
  $W^\perp$ eindeutig definiert ist durch
  \begin{gather}
    W^\perp = \bigl\{ v\in V \big| \scal(v,w) = 0 \quad\forall w\in W\bigr\}.
  \end{gather}
  Die Lösung der Bestapproximationsaufgabe bezeichnen wir mit
  \begin{gather}
    \Pi_W u = u_b\in W
  \end{gather}
  und nennen es die \define{orthogonale Projektion} von $u\in V$ auf $W$.
\end{Definition}

\begin{Lemma}{komp-projekt-wohldefiniert}
  Das orthogonale Komplement und die orthogonale Projektion sind wohldefiniert.
\end{Lemma}

\begin{proof}
  \slideref{Satz}{bestapproximation}.
\end{proof}

\begin{Beispiel}{polynom-bestapproximation}
  Die Aufgabe der Gaußschen Ausgleichsrechnung lautet: finde zu einer
  gegebenen Funktion $f$ das Polynom vom Grad höchstens $n$, das auf
  dem Intervall $[-1,1]$ den mittleren quadratischen Abstand
  minimiert, also $p\in \P_n$ mit
  \begin{gather}
    \int_{-1}^1 \bigl(f(x)-p(x)\bigr)^2 \dx
    = \min_{q\in \P_n} \int_{-1}^1 \bigl(f(x)-q(x)\bigr)^2 \dx.
  \end{gather}
  Die Lösung erfüllt
  \begin{gather}
    \int_{-1}^1 p(x)q(x) \dx = \int_{-1}^1 f(x)q(x) \dx
    \qquad\forall q\in \P_n.
  \end{gather}
\end{Beispiel}

\begin{remark}
  Mit unserem Wissen über die $L^2$ Norm aus \slideref{Definition}{l2-norm} erkennen wir, dass sich
  die Gaußsche Ausgleichsrechnung auch über die Norm formulieren lässt.
  \begin{gather}
    \norm{f-p}_{L^2}^2 = \min_{q\in \P_n} \norm{f-q}_{L^2}^2
  \end{gather}
\end{remark}

\section{Orthogonale Basen}

\begin{Lemma}{gram-system}
  Wählt man eine Basis $\{\phi_i\}$ für $W$, so transformiert wird die
  Orthogonalitätsbedingung in \slideref{Satz}{bestapproximation} zum
  linearen Gleichungssystem
  \begin{gather}
    \matg\vx = \vb.
  \end{gather}
  Hier sind $\vx$ der Koeffizientenvektor der Lösung $u_b$, $\matg$ die
  \define{Gramsche Matrix} und $\vb$ die rechte Seite gegeben durch
\begin{gather}
  g_{ij} = \scal(\phi_i,\phi_j), \qquad
  b_i = \scal(u,\phi_i).
\end{gather}
\end{Lemma}

\begin{remark}
  Das Gleichungssystem hängt nur von der Wahl einer Basis in $W$ ab,
  nicht in $V$.
\end{remark}

\begin{Definition}{ortho-system}
  Eine Menge von Vektoren $\{\phi_1,\dots,\phi_n\}\subset V$ bildet
  ein \define{Orthogonalsystem}, wenn
  \begin{gather*}
    \scal(\phi_i,\phi_j) = 0
    \qquad \forall 1\le i < j \le n.
  \end{gather*}
  Sie ist ein \define{Orthonormalsystem}, wenn zusätzlich
  $\norm{\phi_i} = 1$ für alle Elemente gilt. Ein Orthonormalsystem, das eine Basis bildet, heißt \define{Orthonormalbasis} (\define{ONB}).
\end{Definition}

\begin{Lemma}{ortho-lu}
  Jedes Orthogonalsystem ist linear unabhängig.
\end{Lemma}

\begin{Lemma*}{parseval}{Parsevalsche Gleichung}
  Sei $\{\phi_i\}$ für $i=1,\dots,n$ eine ONB von $V$. dann gilt für
  jedes $v\in V$ mit der Basisdarstellung
  \begin{gather}
    v = \sum_{i=1}^n x_i \phi_i
  \end{gather}
  die Identität
  \begin{gather}
    \norm{v}^2 = \sum_{i=1}^n x_i^2.
  \end{gather}
\end{Lemma*}
\begin{Lemma}{least-squares-orthogonal}
  Bezüglich einer ONB ist die Gramsche Matrix die
  Einheitsmatrix. Damit berechnen sich die Einträge des
  Koeffizientenvektors $\vx$ in \slideref{Lemma}{gram-system} durch
  die einfache Formel
  \begin{gather}
    x_i = b_i = \scal(u,\phi_i).
  \end{gather}
\end{Lemma}

\begin{Theorem*}{gram-schmidt}{Gram-Schmidt-Verfahren}
  Jede linear unabhängige Menge von Vektoren
  $\{v_1,\dots,v_n\}\subset V$ wird mit dem folgenden Verfahren in ein
  Orthonormalsystem $\{\phi_1,\dots,\phi_n\}\subset V$ umgeformt:
  \begin{gather}
    \begin{aligned}
      \phi_1 &= \tfrac1{\norm{v_1}} \,v_1\\
      w_j &= v_j - \sum_{i=1}^{j-1} \scal(v_j,\phi_i)\,\phi_i
      & \quad \phi_j &= \tfrac1{\norm{w_j}}\, w_j
      &\quad j=2,\dots,n
    \end{aligned}
  \end{gather}
  Für alle $1\le k \le n$ gilt
  \begin{gather}
    \operatorname{span}\{\phi_1,\dots,\phi_k\}
    =
    \operatorname{span}\{v_1,\dots,v_k\}
  \end{gather}
\end{Theorem*}

\begin{proof}
  Per Induktion über $n$ zeigen wir Orthogonalität und Normierung.\\

  $Indukionsanfang$ Sei $n=1$.\\
  Wird nur ein Vektor aus dem Raum gewählt, so erfüllt dieser
  die Orthogonalitätsbedingung, da er der einzige Vektor im System ist.
  Wird dieser Vektor zusätzlich normiert erhält man ein Orthonormalsystem.\\
  
  $Induktionsschritt$ Das Verfahren gelte
  für $\{v_1,\dots,v_{n-1}\}$ Vektoren aus V. \\
  $n-1 \rightarrow n$\\
  Sei $(\phi_1,\dots,\phi_{n-1})$ ein Orthonormalsystem\\
  Annahme $w_n$ nicht wohldefiniert. Dann gilt
  \begin{gather}
    w_n = v_n -\sum_{i=1}^{n-1}\scal(v_n,\phi_i)\,\phi_i = 0
  \end{gather}
  In diesem Fall sind $(v_1,\dots,v_n)$ linear abhängig.
  Das ist ein Widerspruch zur Voraussetzung,
  dass $(v_1,\dots,v_n)$ linear unabhängig sind.\\
  $w_n$ wird nun normiert über $\frac{1}{\norm{w_n}} \cdot w_n =\phi_n $.
  Nun zur Orthogonalität:
  \begin{gather}
    \scal(\phi_n,\phi_j)=\scal(v_n,v_j)-
    \sum_{i=1}^{n-1}\scal(v_n,\phi_i)
    \,\underbrace{\scal(\phi_i,\phi_j)}_{=\delta_{ij}}  = 0
    \quad j=1,\dots,n-1
  \end{gather}
\end{proof}

\begin{Algorithmus*}{gram-schmidt}{Gram-Schmidt}
  \lstinputlisting{code/gram-schmidt.py}
\end{Algorithmus*}

\begin{remark}
  Um den Code zu verstehen ist es ratsam ihn zu Beginn einmal durchzugehen und sich Stellen zu
  merke an denen Zuweisungen getätigt werden. Ebenso sollte der Code mit einem einfachen Beispiel
  probiert werden, um die Parallelen zum Verfahren besser zu erkennen.\\  
   1 Es wird eine Funktion mit dem Namen $gram schmidt$. Dieser Funktion wird eine Matrix $v$
      übergeben. In dieser Matrix stehen die Vektoren $v_1$ bis $v_n$ in Spalten. \\ 
   2 Es wird $n$ die Länge der Zeilen(Anzahl der Vektoren) zugewiesen und
      $m$ wird die Länge der Spalten (Anzahl der Einträge im Vektor) zugewiesen. \\
   3 Beginn des GS Verfahrens. Es geht von Vektor 1 bis Vektor $n$ \\
   4 Initialsieren eines Vektors $delta$ der Länge $m$ mit Nullen als Einträge\\
   5 Es wird eine weitere Schleife begonnen in der ein Index i über alle bisher orthogonalisierten
      Vektoren läuft. Dies entspricht der Summe aus dem Verfahren. \\
   6 $r$ ist das Skalarprodukt aus dem Vektor $v_j$ und einem bereits orthogonalisierten Vektor.
      Die Vektoren befinden sich in der Matrix $v$ und über diesen Befehl wird darauf zugegriffen.\\
   7 delta = delta $+$ Skalarprodukt $*$ dem orthogonalen Vektor\\  
   8 Hier wird die zweite for-Schleife wieder verlassen! Es reicht tatsächlich das Einrücken.
      Die Summe wird vom Vektor $v_j$ abgezogen, somit $v_j$ orthogonalisiert und wieder in der
      Matrix $v$ an der richtigen Stelle zugewiesen.  \\
   9  Die Norm von $v_j$ wird berechnet.\\  
   10 Wie im Verfahren wird in  dieser Zeile $v_j$ normiert und in der Matrix $v$ an der Stelle des
       früheren $v_j$ zugewiesen.
\end{remark}

\begin{Beispiel}{gram-schmidt}
  Wir wählen für Polynome das $L^2$-Skalarprodukt aus
  \slideref{Lemma}{l2-norm} und die Basis $\{1,x,\dots,x^{n-1}\}$
  für $\P_{n-1}$. Wir verwenden die Iplementation in
  \slideref{Algorithmus}{gram-schmidt} und messen den Erfolg nach der
  Größe der Nebendiagonaleinträge der Gramschen Matrix.
  \begin{center}
    \begin{tabular}{c|c}
      $n$ & $\max_{i\neq j} \abs{g_{ij}}$ \\
      \hline
      5 & $8.9\cdot 10^{-16}$ \\
      10 & $9.1\cdot 10^{-12}$ \\
      15 & $1.2\cdot 10^{-7}$ \\
      20 & $0.23$
    \end{tabular}
  \end{center}
\end{Beispiel}

\begin{Algorithmus*}{mgs}{Modifizierter Gram-Schmidt}
  \lstinputlisting{code/modified-gram-schmidt.py}  
\end{Algorithmus*}

\begin{remark}
  In diesem Programm wurde der Zwischenschritt über das delta ausgelassen, was mögliche Rundungsfehler
  verringert.\\
  4 Hier wird direkt $r$ mit Nullen initialisiert.\\
  7 Der Vektor $v_j$ wird hier orthogonalisiert, ohne dass die Summe aus dem Verfahren in delta
    zwischengespeichert wird und direkt wieder an entsprechender Stelle in der
    Matrix $v$ zugewiesen.\\
\end{remark}

\begin{Beispiel}{gs-mgs}
  In dieser Tabelle wiederholen wir die Zahlen
  $\max_{i\neq j} \abs{g_{ij}}$ aus \slideref{Beispiel}{gram-schmidt}
  und stellen sie den entsprechenden Ergebnissen des modifizierten
  Verfahrens in \slideref{Algorithmus}{mgs} gegenüber.
  \begin{center}
    \begin{tabular}{c|cc}
      $n$ &  Gram-Schmidt & modifiziert\\
      \hline
      5 & $8.9\cdot 10^{-16}$ & $1.3\cdot 10^{-16}$ \\
      10 & $9.1\cdot 10^{-12}$ & $2.9\cdot 10^{-12}$ \\
      15 & $1.2\cdot 10^{-7}$ & $2.7\cdot 10^{-9}$ \\
      20 & $0.23$ & $3.9\cdot 10^{-5}$
    \end{tabular}
  \end{center}
\end{Beispiel}

\begin{remark}
  Wir sehen, dass die Wahl der Implementation eines Rechenverfahrens
  bei mathematischer Äquivalenz durchaus erheblichen Einfluss auf das
  Ergebnis haben kann. Dieses Phänomen werden wir in
  \Cref{sec:stability} näher untersuchen. Zunächst diskutieren wir
  aber eine weitere Variante der Erzeugung orthogonaler Basen in
  Polynomräumen.
\end{remark}

\section{Drei-Term-Rekursion}

\begin{Satz*}{dreiterm}{Dreiterm-Rekursion}
  Zu jedem Skalarprodukt $\scal(\cdot,\cdot)$ auf dem Raum der
  stetigen Funktionen gibt es genau eine Folge von orthogonalen
  Polynomen $p_k\in \P_k$ mit führendem Koeffizienten eins. Sie
  genügen der Dreiterm-Rekursionsformel
  \begin{gather}
    p_k(x) = (x-a_k)p_{k-1}(x) - b_k p_{k-2}(x),
    \qquad k=1,2,\ldots
  \end{gather}
  mit Startwerten $p_{-1} \equiv 0$ und $p_0 \equiv 1$. Die
  Koeffizienten sind
  \begin{gather}
    a_k = \frac{\scal(x p_{k-1},p_{k-1})}{\scal(p_{k-1},p_{k-1})}
    \qquad\text{und}\qquad
    b_k = \frac{\scal(p_{k-1},p_{k-1})}{\scal(p_{k-2},p_{k-2})}.
  \end{gather}
\end{Satz*}

\begin{proof}
  Siehe \cite[Satz 6.2]{DeuflhardHohmann08}
\end{proof}

\begin{Bemerkung}{dreiterm-normierung}
  Der Beweis ergibt, eigentlich die ``Eindeutigkeit einer Orthogonalfolge bis auf Normierung''. Tatsächlich werden in der Literatur immer wieder veschiedene Normierungen benutzt. Beispiele sind:
  \begin{enumerate}
  \item Führender Koeffizient eins, $p_k = x^k + \dots$
  \item $\norm{p_k} = 1$
  \item $p_k(1) = 1$
  \end{enumerate}
\end{Bemerkung}

\begin{Definition}{legendre-polynome}
  Die \define{Legendre-Polynome} $L_k$ sind definiert durch
  die Dreiterm-Rekursion
  \begin{gather}
    L_{k} = \tfrac{2k-1}{k}x L_{k-1}(x) - \tfrac{k-1}{k} L_{k-2}(x).
  \end{gather}
  Sie sind orthogonal bezüglich des $L^2$-Skalarprodukts in
  \slideref{Lemma}{l2-norm}.
\end{Definition}

\begin{Beispiel}{least-squares-legendre}
  Das Problem der Gaußschen Ausgleichsrechnung war: Zu einer gegebenen
  Funktion $f$ finde $p\in \P_n$, so dass
  \begin{gather}
    \norm{f-q}_{L^2}^2
    = \min_{q\in\P_n} \norm{f-q}_{L^2}^2.
  \end{gather}
  Mit Hilfe der Legendre-Polynome können wir nun die Lösung explizit angeben als
  \begin{gather}
    p(x) = \sum_{i=0}^n \alpha_i L_i(x)
    \qquad\text{mit}\qquad
    \alpha_i = \frac1{\norm{L_i}^2}\int_{-1}^1 f L_i(x)\dx.
  \end{gather}
\end{Beispiel}

\begin{Definition}{chebyshev-polynome}
  Die \define{Tschebyscheff-Polynome} $T_k$ sind definiert durch
  die Dreiterm-Rekursion
  \begin{gather}
    T_{k} = 2x T_{k-1}(x) - T_{k-2}(x).
  \end{gather}
  Sie sind orthogonal bezüglich des Skalarprodukts
  \begin{gather}
    \scal(p,q) = \int_{-1}^1 \tfrac1{\sqrt{1-x^2}} \,p(x)q(x)\dx.
  \end{gather}
\end{Definition}

%%% Local Variables:
%%% mode: latex
%%% TeX-master: "main"
%%% End:


\chapter{Konditionierung und Stabilität}
\label{sec:stability}
\begin{Theorem*}{awa-stability}{Stability}
  Let $f(t,u)$ and $g(t,u)$ be two continuous functions on a
  cylinder $D = I \times \Omega$ where the interval $I$ contains
  $t_0$ and $\Omega$ is a convex set in $\R^d$.  Furthermore, let
  $f$ admit a Lipschitz condition with constant $L$ on $D$. Let $u$
  and $v$ be solutions to the IVP
  \begin{xalignat}{2}
    \label{eq:awa:20}
    u'&=f(t,u) \quad\forall t\in I,& u(t_0)&= u_0,\\
    \label{eq:awa:21}
    v'&=g(t,v) \quad\forall t\in I,& v(t_0)&= v_0.
  \end{xalignat}
  Then, there holds
  \begin{gather}
    \label{eq:awa:22}
    \abs{u(t)-v(t)} \le e^{L|t-t_0|}
    \left[ \abs{u_0-v_0}
      + \int_{t_0}^{t} \max_{x\in\Omega}
      \abs{f(s,x)-g(s,x)}\ds
    \right].
  \end{gather}
\end{Theorem*}

%%% Local Variables:
%%% mode: latex
%%% TeX-master: "../notes"
%%% End:


\chapter{Interpolation und Quadratur}

\begin{intro}
  Ziel dieses Kapitels ist die Herleitung von Methoden zur
  Approximation des Integrals einer Funktion über ein Intervall
  $[a,b]$. Diese Aufgabe wird in zwei Teile geteilt:
  \begin{enumerate}
  \item Wir unterteilen das Intervall in Subintervalle und summieren
    die Teilintegrale
    \begin{gather}
      \int_a^b f \dx = \sum_{i=1}^n \int_{x_{i-1}}^{x_i} f \dx,
      \qquad
      a = x_0 < x_1 < \dots < x_n = b.
    \end{gather}
  \item Auf jedem Teilintervall finden wir Approximationen für das
    lokale Integral.
  \end{enumerate}
  Da wir Polynome exakt integrieren können, nutzen wir wieder die
  Approximation von Funktionen durch Polynome, um uns diesem Problem
  zu nähern.
\end{intro}

\subsection{Definition und Konditionsabschätzung}

\begin{Definition}{lagrange-interpolation}
  Die \define{Interpolation}saufgabe nach Lagrange lautet: seien $n+1$
  paarweise verschiedene \define{Stützstellen} $x_0,\dots,x_n$ mit
  zugehörigen Funktionswerten $f_i$ gegeben. Finde ein Polynom
  $p\in \P_n$ mit der Eigenschaft
  \begin{gather}
    p(x_i) = f_i.
  \end{gather}
  Alternativ ist die Interpolationsaufgabe aufzufassen als eine Abbildung
  \begin{gather}
    \begin{split}
      I_n\colon C[a,b] &\to \P_n\\
      p(x_i) &= f(x_i),
    \end{split}
  \end{gather}
  wobei das Interval $[a,b]$ alle Stützpunkte enthält. 
  Wir nennen diese Abbildung den
  \define{Lagrange-Interpolationsoperator} oder kurz
  \define{Lagrange-Interpolation}.
\end{Definition}

\begin{Satz}{lagrange-interpolation}
  Die Interpolationsaufgabe nach Lagrange hat eine eindeutige Lösung,
  bezeichnet als (Lagrange-)\define{Interpolierende} der Funktion $f$
  \begin{gather}
    p(x;f;x_0,\dots,x_n)
  \end{gather}
\end{Satz}
\begin{proof}
  Der Beweis ist eine direkte Konsequenz des folgenden Lemmas.
\end{proof}

\begin{Lemma}{lagrange-basis}
  Seien die Punkte $x_0,\dots,x_n$ paarweise verschieden. Dann gilt
  für die \define{Lagrange-Polynome}
  \begin{gather}
    \plagrange_i(x) = \plagrange_{i;n}(x) = \plagrange_{i;x_0,\dots,x_n}(x)
    = \prod_{\substack{j=0\\j\neq i}}^n \frac{x-x_j}{x_i-x_j}
  \end{gather}
  die Eigenschaft
  \begin{gather}
    \plagrange_i(x_j) = \delta_{ij},\qquad 0 \le i,j \le n.
  \end{gather}
  Die Lagrange-Polynome sind \putindex{orthonormal} bezüglich des
  Skalarprodukts
  \begin{gather}
    \scal(p,q) = \sum_{i=0}^n p(x_i)q(x_i).
  \end{gather}
  Daher sind sie linear unabhängig und formen eine Basis von $\P_n$.
\end{Lemma}

\begin{Korollar}{lagrange-interpolation}
  Die Lösung der Interpolationsaufgabe nach Lagrange eerlauben die Darstellung
  \begin{gather}
    p(x;f;x_0,\dots,x_n) = \sum_{i=0}^n f_i \plagrange_{i;x_0,\dots,x_n}(x).
  \end{gather}
  
  Die Lagrange-Interpolation eingeschränkt auf den Raum $\P_n$ ist die
  Identität
\end{Korollar}

\begin{remark}
  Die Lagrangesche Interpolationsaufgabe kann auch als Gaußsche
  Ausgleichsrechnung mit dem obigen Skalarprodukt aufgefasst werden.
\end{remark}

\begin{Lemma}{linear-bounded}
  Sei $f\colon X \to Y$ eine lineare Abbildung zwischen Vektorräumen
  $X$ und $Y$. Dann sind folgende Aussagen äquivalent:
  \begin{enumerate}
  \item In einem beliebigen Punkt $x\in X$ gilt für das gestörte Problem
    $y+\delta y = f(x+\delta x)$ die Abschätzung
    \begin{gather}
      \norm{\delta y} \le \kappa^{\text{abs}} \norm{\delta x}
      \qquad\forall \delta x \in X.
    \end{gather}
  \item Für $y = f(x)$ gilt die Abschätzung
    \begin{gather}
      \norm{y} \le \kappa^{\text{abs}} \norm{x}
      \qquad\forall x \in X.
    \end{gather}
  \end{enumerate}
\end{Lemma}

\begin{remark}
  Es genügt also, die Konditionierung um die null zu untersuchen, was
  die Analyse vereinfacht.

  Nun gilt für eine lineare Abbildung $f(0) = 0$. In diesem Falle ist
  also die Konditionszahl für den relativen Fehler aus
  \slideref{Definition}{datenfehler}
  bzw. \slideref{Lemma}{diff-fehler} nicht sinnvoll definiert. Wir
  benutzen daher die Konditionszahlen für den absoluten Fehler. 
\end{remark}

\begin{Satz*}{lagrange-kondition}{Konditionszahl der Lagrange-Interpolation}
  Die Konditionszahl des absoluten Fehlers in der Supremumsorm der
  Lagrange-Interpolation zu den Punkten $a = x_0 < \dots < x_n = b$
  ist die \define{Lebesgue-Konstante}
  \begin{gather}
    \Lambda_{x_0,\dots,x_n} = \max_{x\in [a,b]}
    \sum_{i=0}^n \abs{\plagrange_{i;x_0,\dots,x_n}(x)}.
  \end{gather}
  Es gilt also
  \begin{gather}
    \max _{x\in[a,b]} \abs{I_n f(x)}
    \le \Lambda_{x_0,\dots,x_n} \max _{x\in[a,b]} \abs{f(x)}.
  \end{gather}
  Diese Abschätzung ist scharf.
\end{Satz*}

\begin{proof}
  Siehe \cite[Satz 7.3]{DeuflhardHohmann08}.
\end{proof}

\begin{Beispiel}{lagrange-kondition-equi}
  Für äquidistante Stützstellen erhält man exemplarisch die Konditionszahlen in der zweiten Spalte. Später entwickeln wir einen optimalen Satz von Stützstellen. Die Konditionszahlen dazu sind in der rechten Spalte.
  \begin{center}
    \begin{tabular}{r|rr}
      & \multicolumn{2}{c}{ $\Lambda_{0,\dots,n}$}\\
      $n$ & äquidistant & optimal\\\hline
      5 & 3.1 & 2.1\\
      10 & 30 & 2.5 \\
      15 & 512 & 2.7 \\
      20 & 10986 & 2.9
    \end{tabular}
  \end{center}
  Quelle: \cite{DeuflhardHohmann08}
\end{Beispiel}

\subsection{Rekursive Interpolation}

\begin{Lemma*}{Aitken}{Aitken}
  Für das Interpolationspolynom
  \begin{gather}
    p_{0,\dots,n}(x) = p(x;f;x_0,\dots,x_n)
  \end{gather}
  zu paarweise verschiedenen Stützstellen $x_0,\dots,x_n$ gilt die
  Rekursionsformel
  \begin{gather}
    p_{0,\dots,n}(x)
    = \frac{(x-x_0) p_{1,\dots,n}(x) - (x-x_n) p_{0,\dots,n-1}(x)}{x_n-x_0}.
  \end{gather}
\end{Lemma*}

\begin{proof}
  Der Beweis benutzt wieder Induktion. Für eine einzige Stützstelle ist das Interpolationspolynom konstant, $p_i(x) = f_i$ und daher $p_i\in P_0$.
  Sei nun $\phi(x)$ der Bruch auf der rechten Seite. Durch Induktion sehen wir sofort, dass $\phi\in \P_n$. Ferner gilt für $i=1,\dots,n-1$
  \begin{gather}
    \begin{split}
      \phi(x_i)
      &= \frac{(x_i-x_0) p_{1,\dots,n}(x_i) - (x_i-x_n)p_{0,\dots,n-1}(x_i)}{x_n-x_0}\\
      &= \frac{(x_i-x_0) f_i - (x_i-x_n) f_i}{x_n-x_0}\\
      &= f_i.
    \end{split}
  \end{gather}
  Ebenso verschwindet für $x_0$ und $x_n$ je ein Term und es gilt
  dieselbe Aussage.
\end{proof}

\begin{Algorithmus*}{Neville}{Neville}
  Sei für eine Stelle $x$ an der das Interpolationspolynom berechnet
  werden soll $p_{ik} = p_{i-k,\dots,i}(x)$ für $i\ge k$. Dann lässt
  sich $p_{0,\dots,n}(x) = p_{nn}$ rekursiv berechnen durch
  \begin{enumerate}
  \item Für $k=0$ setze
    \begin{gather}
      p_{i0} = f_i \qquad i=0,\dots,n.
    \end{gather}
  \item Für $k=1,\dots,n$ berechne
    \begin{gather}
      p_{ik} = p_{i,k-1} + \frac{x-x_i}{x_i-x_{i-k}}
      \bigl( p_{i,k-1} - p_{i-1,k-1} \bigr)
      \qquad i=k,\dots,n.
    \end{gather}
  \end{enumerate}
\end{Algorithmus*}

\begin{Definition}{newton-basis}
  Als \define{Newton-Basis} der Lagrange-Interpolation bezeichnen wir
  die Polynome
  \begin{gather}
    \omega_i(x)
    = \omega_{0,\dots,i}(x)
    = \prod_{j=0}^{i-1} (x-x_j),
    \qquad i=0,\dots,n
  \end{gather}
  wobei das leere Produkt für $i=0$ den Wert 1 annehme.
\end{Definition}

\begin{Lemma}{newton-basis}
  Sei $Q_k\in \P_n$ ein Polynom dargestellt bezüglich der Newton-Basis
  durch
  \begin{gather}
    Q_k(x) = \sum_{i=0}^k a_i \omega_i(x),\qquad k=0,\dots,n.
  \end{gather}
  Dann gilt
  \begin{align}
    Q_k(x) &= Q_{k-1}(x) + a_k \omega_k(x),
  \end{align}
  und $a_k$ ist der Koeffizient vor $x^k$ in der Monomdarstellung von
  $Q_k(x)$.
\end{Lemma}

\begin{Definition}{div-diff-1}
  Als \define{dividierte Differenzen} zur
  Lagrange-Interpolationsaufgabe bezeichnen wir die rekursiv
  definierten Werte
  \begin{align}
    [x_i]f
    &= f_i \\
    [x_i,\dots,x_{i+k}]f
    &= \frac{[x_{i+1},\dots,x_{i+k}]f - [x_i,\dots,x_{i+k-1}]f}{x_{i+k}-x_i}
  \end{align}
\end{Definition}

\begin{Satz}{newton-lagrange}
  Für das Lagrange-Interpolationspolynom $p_{i,\dots,i+k}(x)$ zu den
  paarweise verschiedenen Stützpunkten $x_i,\dots,x_{i+k}$ gilt
  \begin{gather}
    p_{i,\dots,i+k}(x)
    = \sum_{j=i}^{i+k} [x_i,\dots,x_{i+k}]f\; \frac{\omega_j(x)}{\omega_i(x)}.
  \end{gather}
\end{Satz}

\begin{proof}
  In der Newton-Darstellung gilt
  \begin{gather}
    p_{i,\dots,i+k}(x) = p_{i,\dots,i+k-1}(x)
    + \alpha \frac{\omega_{i+k}(x)}{\omega_i(x)}.
  \end{gather}
  Zu zeigen ist also $\alpha = [x_i,\dots,x_{i+k}]$, was nach
  \slideref{Lemma}{newton-basis} der Koeffizient vor $x^k$ ist. Nach
  Induktionsannahme ist
  \begin{gather}
    \begin{split}
      p_{i,\dots,i+k-1}(x) &= [x_i,\dots,x_{i+k-1}]f \,x^{k-1} +\bigo(x^{k-2})\\
      p_{i+1,\dots,i+k}(x) &= [x_{i+1},\dots,x_{i+k}]f \,x^{k-1} +\bigo(x^{k-2})
    \end{split}
  \end{gather}
  Nach dem Lemma von Aitken gilt
  \begin{gather}
    p_{i,\dots,i+k} = \frac{(x-x_i)p_{i+1,\dots,i+k}
      - (x-x_{i+k})p_{i,\dots,i+k-1}}{x_{i+k} - x_i}.
  \end{gather}
  Dessen höchster Koeffizient ist aber gerade die dividierte Differenz.
\end{proof}

\begin{remark}
  Der Bruch im vorherigen Satz ist nicht problematisch, da
  \begin{gather}
    \frac{\omega_j(x)}{\omega_i(x)} = \prod_{\ell=i}^{j-1} (x-x_\ell).
  \end{gather}
\end{remark}

\begin{Satz}{Lagrange-restglied}
  Sei $f \in C^{n+1}[a,b]$ und $p\in \P_n$ die
  Lagrange-Interpolierende zu den Stützstellen
  $a=x_0\neq\dots\neq x_n=b$. Dann gibt es zu jedem $x\in \R$ einen Punkt
  $\xi$ im kleinsten Intervall $I$, das die Punkte $x$, $a$ und $b$
  enthält, so dass
  \begin{gather}
    f(x)- p(x) = \frac{f^{(n+1)}(\xi)}{(n+1)!} \omega_{0,\dots,n}(x).
  \end{gather}
  Wir bezeichnen diese Aussage als \define{Fehlerdarstellung}, die
  rechte Seite auch als \define{Restglied}.
\end{Satz}

\begin{proof}
  Der Beweis folgt \cite[Satz 2.1.4.1]{Stoer83}.  Zunächst bemerken
  wir, dass für alle Stützstellen $x_i$ gilt, dass
  $f(x_i) - p(x_i) = 0$. Dort ist also nichts zu beweisen.
  Sei nun
  \begin{gather}
    \label{eq:interpolation:1}
    F(y) = f(y)-p(y) - \alpha \omega_n(y)
  \end{gather}
  und $\alpha$ soll so gewählt werden, dass $F(x) = 0$. Damit hat $F(y)$ im
  Intervall $I$ insgesamt die $n+2$ Nullstellen $x,x_0,\dots,x_n$.
  Wiederholte Anwendung des Satzes von Rolle ergibt, dass $F'(y)$
  insgesamt $n+1$ Nullstellen hat und das $F^{(n+1)}(y)$ eine
  Nullstelle $\xi$ besitzt. Da $p\in \P_n$ gilt
  \begin{gather}
    0 = F^{(n+1)}(\xi) = f^{(n+1)}(\xi) - \alpha (n+1)!
  \end{gather}
  und damit
  \begin{gather}
    \alpha = \frac{f^{(n+1)}(\xi)}{(n+1)!}.
  \end{gather}
\end{proof}

\begin{Korollar}{Lagrange-restglied}
  Sei $f \in C^{n+1}[a,b]$ und alle Stützstellen $x_i$ im Intervall
  $[a,b]$. Dann gibt es $\xi\in[a,b]$, so dass
  \begin{gather}
    [x_0,\dots,x_n]f = \frac{f^{(n)}}{n!}(\xi).
  \end{gather}
\end{Korollar}

\begin{proof}
  Formel~\eqref{eq:interpolation:1} gibt gerade an, dass $\alpha$ der
  Koeffizient vor dem nächsten Newton-Basispolynom ist, wenn man den
  Punkt $x$ der Menge der Stützstellen hinzufügt.
\end{proof}

\begin{Korollar}{Lagrange-fehler-1}
  Es gelten die Voraussetzungen von
  \slideref{Satz}{Lagrange-restglied}. Dann gibt es eine Konstante
  $C$, die nur von der Wahl der Stützstellen abhängt, so dass
  \begin{gather}
    \max_{x\in[a,b]} \abs{f(x)-p_{0,\dots,n}(x)}
    \le \frac{C \abs{b-a}^{n+1}}{(n+1)!} \max_{x\in[a,b]} \abs{f^{(n+1)}(x)} 
  \end{gather}
\end{Korollar}

\begin{remark}
  Die Fehlerabschätzung in \slideref{Korollar}{Lagrange-fehler-1}
  können wir auch kürzer schreiben als
  \begin{gather}
    \norm{f-p_{0,\dots,n}}_\infty \le \frac{C \abs{b-a}^{n+1}}{(n+1)!}
    \norm{f^{(n+1)}}_{\infty}.
  \end{gather}
  Die rechte Seite besteht dabei aus dem Produkt aus einem Teil, der
  nur von den Daten abhängt, $\norm{f^{(n+1)}}_{\infty}$ und einem
  Anteil, der durch das Verfahren bestimmt ist.

  Wir sehen, dass Interpolation auf einem Intervall um so genauer ist,
  je kürzer das Intervall ist.
\end{remark}

\begin{intro}
  Der Rest dieses Abschnitts befasst sich mit der Frage, wie die
  Stützstellen $x_0,\dots,x_n$ gewählt werden können, damit die
  Konstante $C$ in der Fehlerabschätzung optimal ist.  Aus der
  Fehlerdarstellung in \slideref{Satz}{Lagrange-restglied} folgt, dass
  wir dazu ein Polynom finden müssen, dessen führender Koeffizient 1
  ist, und das minimalen Betrag auf dem Intervall $[a,b]$ hat.
  Tatsächlich erlauben uns die Tschebyscheff-Polynome, diese
  Optimalität zu erreichen.
\end{intro}

\begin{Lemma}{chebyshev-properties}
  Die Tschebyscheff-Polynome, die der Rekursionsformel in
  \slideref{Definition}{chebyshev-polynome} genügen, haben die
  Darstellung
  \begin{gather}
    T_k = \cos(k \operatorname{arccos} x)
  \end{gather}
  Insbesondere gilt
  \begin{alignat}3
    T_k(1)&=1\\
    T_k(-1) &= (-1)^k\\
    \abs{T_k(x)} &\le 1, &\quad x&\in [-1,1]\\
    T_k(x) &=(-1)^j,
                   &\quad x&=\cos\left(\frac{j}{k}\pi\right),
                   &\quad j&=0,\dots,k\\
    T_k(x) &= 0,
             &\quad x&=\cos\left(\frac{2j-1}{2k}\pi\right),
             &\quad j&=1,\dots,k
  \end{alignat}
\end{Lemma}

\begin{proof}
  Hausaufgabe
\end{proof}

\begin{Satz}{chebyshev-minimal-1}
  Jedes Polynom $p\in \P_n$ mit führendem Koeffizienten 1 nimmt im
  Intervall $[-1,1]$ einen Wert $\abs{p(x)} \ge \nicefrac1{2^{n-1}}$
  an und es gilt
  \begin{gather}
   \frac1{2^{n-1}} T_n(x)
   = \operatorname*{arg min}_{\substack{p\in\P_n\\p = x^n+\cdots}}
   \max_{x\in[-1,1]}\abs{p(x)}.
  \end{gather}
\end{Satz}

\begin{proof}
  Siehe auch \cite[Satz 7.19]{DeuflhardHohmann08}. Aus der
  Rekursionsformel folgt sofort, dass der höchste Koeffizient von
  $T_n$ den Wert $2^{n-1}$ annimmt. Sei nun als Widerspruchsannahme
  $p\in \P_n$ ein weiteres Polynom mit höchstem Koeffizienten
  $2^{n-1}$, so dass
  \begin{gather}
    \max_{x\in[-1,1]} \abs{p(x)} < 1.
  \end{gather}
  Dann ist $q_n = T_n-p \in \P_{n-1}$ und für die
  \define{Tschebyscheff-Abszisse}n
  $\tilde x_j = \cos(\nicefrac{j\pi}{n})$ mit $j=0,\dots,n$ gilt
  \begin{xalignat}4
    T_n(\tilde x_j) &= 1,
    & p(\tilde x_j) &< 1
    & q_n(\tilde x_j) &> 0,
    & j&\text{ gerade}\\
    T_n(\tilde x_j) &= -1,
    & p(\tilde x_j) &> -1
    & q_n(\tilde x_j) &< 0,
    & j&\text{ ungerade}.
  \end{xalignat}
  $q_n$ wechselt also an mindestens $n$ Stellen das Vorzeichen und hat
  damit als stetige Funktion mindestens ebensoviele Nullstellen. Aus
  $q_n\in\P_{n-1}$ folgt damit im Widerspruch $q_n=0$ und $p=T_n$.
  Damit gilt nach Skalierung um den Faktor $2^{n-1}$
  \begin{gather}
    \operatorname*{min}_{\substack{p\in\P_n\\p = x^n+\cdots}}
   \max_{x\in[-1,1]}\abs{p(x)} \ge 1,
  \end{gather}
  und Gleichheit für das skalierte Tschebyscheff-Polynom.
\end{proof}

\begin{Korollar}{Lagrange-chebychev}
  Wählt man als Stützstellen die Werte
  \begin{gather}
    x_i = \frac{a+b}2 + \frac{b-a}2 \cos\left(\frac{2i+1}{2n+2}\pi\right),
    \qquad i=0,\dots,n,
  \end{gather}
  So gilt für den Fehler der Lagrange-Interpolation
  \begin{gather}
    \norm{f-p_{0,\dots,n}}_\infty \le \frac{\abs{b-a}^{n+1}}{2^{2n}(n+1)!}
    \norm{f^{(n+1)}}_{\infty}.
  \end{gather}
\end{Korollar}

\begin{proof}
  Zunächst transformieren wir die Aufgabe vom Intervall $[a,b]$ auf
  das Intervall $[-1,1]$ durch die Abbildung
  \begin{gather}
    x = \Phi(\xi) = \frac{a+b}2 + \frac{b-a}2 \xi.
  \end{gather}
  Es gilt $\Phi(-1) = a$, $\Phi(1)=b$ und di Punkte $x_i$ sind die
  Bilder der Tschebyscheff-Abszissen $\xi_i$ zu $T_{n+1}$. Es gilt
  $\Phi'(\xi) = (b-a)/2$ und für die Funktion $F(\xi) = f(\Phi(\xi))$
  gilt
  \begin{gather}
    \frac{d^k}{d\xi^k} F(\xi) = \left(\frac{b-a}2\right)^k
    \frac{d^k}{dx^k} f(x).
  \end{gather}
  Auf $[-1,1]$ folgern wir aus \slideref{Satz}{Lagrange-restglied} und
  \slideref{Satz}{chebyshev-minimal-1}, dass für die Interpolation gilt
  \begin{gather}
    \max_{\xi\in[-1,1]} \abs{F(\xi) - P(\xi)}
    \le 2^{1-n} \max_{\xi\in[-1,1]}\abs{F^{(n+1)}(\xi)},
  \end{gather}
  woraus folgt
  \begin{gather}
    \max_{x\in[a,b]} \abs{f(x) - p(x)}
    \le 2^{1-n} \left(\frac{b-a}2\right)^{n+1} \max_{x\in[a,b]}\abs{f^{(n+1)}(x)}.
  \end{gather}  
\end{proof}

\subsection{Hermite-Interpolation}

\begin{Definition}{hermite-interpolation}
  Die \define{Hermite-Interpolation} benutzt neben Funktionswerten
  auch Ableitungswerte zur Interpolation. Das Interpolationspolynom
  $p\in \P_n$ genügt in $m$ paarweise verschiedenen Punkten den
  Bedingungen
  \begin{gather}
    \frac{d^j p}{dx^j}(x_i) = f_i^{j},
    \qquad i = 0,\dots, m, \quad j=0,\dots,n_i-1,
  \end{gather}
  und es gilt
  \begin{gather}
    \sum_{i} n_i = n+1.
  \end{gather}
  Die definierenden Funktionale\footnote{Als Funktional bezeichnet man
    eine Abbildung aus einem Vektorraum in den zugehörigen Körper} der Gestalt
  $\nicefrac{d^j}{dx^j} p(x_i)$ werden auch als \define{Knotenwerte}
  oder \define{Knotenfunktionale} bezeichnet.
\end{Definition}

\begin{Satz}{hermite-interpolation}
  \slideref{Definition}{hermite-interpolation} bestimmt das
  Interpolationspolynom eindeutig.
\end{Satz}

\begin{proof}
  Analog zur Lagrange-Interpolation identifizieren wir wieder eine
  Basis $\{H_{ij}(x)\}$, diesmal doppelt indiziert, die bezüglich der
  Interpolationsbedingungen orthogonal ist. Damit stellen wir das
  Interpolationspolynom dar als
  \begin{gather}
    p(x) = \sum_{i=0}^m \sum_{j=0}^{n_i-1} f_i^j H_{ij}(x).
  \end{gather}
  Zunächst führen wir die Hilfspolynome
  \begin{gather}
    q_{ij}(x) = \frac{(x-x_i)^j}{j!}\prod_{k\neq i}
    \left(\frac{x-x_k}{x_i-x_k}\right)^{n_k}
  \end{gather}
  ein. Es gilt
  \begin{gather}
    \begin{aligned}
      \frac{d^j q_{i,n_i-1}}{d x^j} (x_k) &=0,
      \quad &k\neq i,&\quad& j&=0,\dots,n_{k}-1,\\
      \frac{d^j q_{i,n_i-1}}{d x^j} (x_i) &=0,
      \qquad &&& j&=0,\dots,n_{i}-2,\\   
      \frac{d^{n_{i}-1} q_{i,n_i-1}}{d x^{n_{i}-1}} (x_i) &=1.
      \qquad &&&& 
    \end{aligned}
  \end{gather}
  Damit können wir rekursiv definieren
  \begin{gather}
    \begin{aligned}
      H_{i,n_i-1}(x) &= q_{i,n_i-1}(x)
      & i&= 0,\dots,m\\
      H_{ij}(x) &= q_{ij}(x) - \sum_{k=j+1}^{n_i-1} q_{ij}^{(k)}(x_i) H_{ik}(x),
    \end{aligned}
  \end{gather}
  wobei die letzte Zeile die Anwendung des Gram-Schmidt-Verfahrens
  ist. Per constructionem gilt für diese Basis
  \begin{gather}
    \frac{d^\ell}{dx^\ell} H_{ij}(x_k) = \delta_{ik}\delta_{j\ell}.
  \end{gather}
\end{proof}

\begin{Notation}{interpolation-ascending}
  Bei der Polynominterpolation ist die Anordnung der
  Interpolationspunkte beliebig. Das ist auch weiterhin der Fall. Für
  die Darstellung der Resultate und Beweise ist es aber oft hilfreich
  anzunehmen, dass sie in aufsteigender Folge angeordnet sind. Wir
  nehmen daher ab jetzt an, dass
  \begin{gather}
    a = x_0 \le x_1 \le \dots \le x_n = b.
  \end{gather}
  Dabei sollen $k$-fach wiederholte Stützstellen bedeuten, dass dort
  nicht nur der der Funktionswert, sondern auch die ersten $k-1$
  Ableitungen interpoliert werden. Damit haben wir für die
  Interpolation in $\P_n$ immer eine Folge von $n+1$ Stützstellen.
\end{Notation}

\begin{Beispiel}{taylor-polynom}
  Sind alle Stützstellen $x_0 = \dots = x_n$ identisch, so erhalten
  wir duch Interpolation einer Funktion $f\in C^n[a,b]$ das
  Taylor-Polynom vom Grad $n$
  \begin{gather}
    p(x;f;x_0,\dots,x_n) = \sum_{k=0}^n \frac{(x-x_0)^k}{k!} f^{(k)}(x_0).
  \end{gather}
\end{Beispiel}

\begin{Beispiel}{hermite-kubisch}
  Die kubische Hermite-Interpolation auf dem Intervall $[a,b]$ ist
  definiert durch die Knotenwerte
  \begin{gather}
    p(a), p'(a), p(b), p'(b).
  \end{gather}
\end{Beispiel}

\begin{Satz}{Hermite-interpolation}
  Das Hermite-Interpolationspolynom genügt der Darstellung
  \begin{gather}
    p_{0,\dots,n}(x) = \sum_{j=0}^n [x_0,\dots,x_j]f\;\omega_j(x)
  \end{gather}
  mit den verallgemeinerten dividierten Differenzen definiert durch
  die Rekursion
  \begin{gather}
    [x_i,\dots,x_{i+k}]f =
    \begin{cases}
      \frac{f^{(k)}(x_i)}{k!} &x_i=x_{i+k}\\
      \frac{[x_{i+1},\dots,x_{i+k}]f - [x_i,\dots,x_{i+k-1}]f}{x_{i+k}-x_i}
      &x_i\neq x_{i+k}.
    \end{cases}
  \end{gather}
\end{Satz}

\begin{proof}
  Der Beweis folgt im wesentlichen dem analogen
  \slideref{Satz}{newton-lagrange}. Wir müssen dort nur die Argumente
  anpassen, die auf paarweise verschiedenen Stützstellen beruhen.
  
  Zunächst benutzen wir \slideref{Korollar}{Lagrange-restglied},
  wonach für $x_i < x_{i+1}< \dots < x_{i+k}$ gilt: es gibt ein
  $\xi\in[x_i,x_{i+k}]$ mit
  \begin{gather}
    [x_i,\dots,x_{i+k}]f = \frac{f^{(k)}(\xi)}{k!}.
  \end{gather}
  Da diese Eigenschaft unabhängig vom Abstand der Stützstellen gilt,
  können wir den Limes $x_j \to x_i$ für $j=1,\dots,k$ bilden, und es
  gilt für $x_i=x_{i+k}$
  \begin{gather}
    [x_i,\dots,x_{i+k}]f \to \frac{f^{(k)}(x_i)}{k!},
  \end{gather}
  sowie
  \begin{gather}
    \omega_{i,\dots,i+k}(x) \to (x-x_i)^{k}.
  \end{gather}
  Für das zugehörige Interpolationspolynom gilt dann
  \begin{gather}
    \frac{d^j}{dx^j} p_{i,\dots,i+k}(x_i) = f^{j}(x_i)
    \qquad j=0,\dots,k-1.
  \end{gather}
  Damit haben wir im Neville-Schema den Induktionsanfang geschafft. Es
  bleibt zu zeigen, dass die Rekursionsformel von Aitken auch
  weiterhin für $x_i \neq x_{i+k}$ gilt. Das ist unmittelbar
  einsichtig, wenn $x_i \neq x_{i+1}$ und $x_{i+k-1} \neq x_{i+k}$, da
  dann beide Polynome in der Rekursion alle Zwischenpunkte $x_j$
  interpolieren.

  Sei nun zunächst
  $x_i=x_{i+1}= x_{i+r} < x_{r+1} \le \dots < x_{i+k}$. Es ist zu
  zeigen, dass das Polynom
  \begin{gather}
    q(x) = \frac{(x-x_i) p_{i+1,\dots,i+k}(x)
      - (x-x_k) p_{i,\dots,i+k-1}(x)}{x_{i+k}-x_i}
  \end{gather}
  alle Knotenfunktionale interpoliert. Für $x_j\neq x_i$ folgt dies
  wie bei der Lagrange-Interpolation aus der Induktionsannahme. Doch
  auch für $x_i$ gilt dies, da der erste Term in der Summe
  verschwindet und $p_{i,\dots,i+k-1}$ bereits alle geforderten
  Ableitungen interpoliert.
\end{proof}

\begin{Satz}{Hermite-restglied}
    Sei $f \in C^{n+1}[a,b]$ und $p\in \P_n$ die
  Hermite-Interpolierende zu den Stützstellen
  $a=x_0\le\dots\le x_n=b$. Dann gibt es zu jedem $x\in \R$ einen Punkt
  $\xi$ im kleinsten Intervall $I$, das die Punkte $x$, $a$ und $b$
  enthält, so dass
  \begin{gather}
    f(x)- p(x) = \frac{f^{(n+1)}(\xi)}{(n+1)!} \omega_{0,\dots,n}(x).
  \end{gather}
\end{Satz}

\begin{proof}
  Der Beweis folgt exakt denselben Argumenten wie der von
  \slideref{Satz}{Lagrange-restglied}.
\end{proof}

\begin{Korollar}{Taylor-restglied}
  Für das \define{Taylor-Polynom} zu $f\in C^{n+1}(a,b)$ in einem
  Punkt $x_0\in(a,b)$,
  \begin{gather}
    p(x) = \sum_{i=0}^n \frac{f^{i}(x_0)}{i!} (x-x_0)^i
  \end{gather}
  gilt die folgende Fehlerdarstellung: es gibt ein $\xi\in[x_0,x]$ so dass gilt
  \begin{gather}
    f(x) - p(x) = \frac{f^{n+1}(\xi)}{(n+1)!} (x-x_0)^{n+1}.
  \end{gather}
\end{Korollar}

%%% Local Variables:
%%% mode: latex
%%% TeX-master: "main"
%%% End:


\bibliographystyle{apalike}
\bibliography{all}
\printindex

%%% Local Variables:
%%% mode: latex
%%% TeX-master: "main"
%%% End:
