\begin{intro}
  The interior penalty method introduced so far is $V_h$-elliptic and
  consistent, but it is not bounded on $H^1(\domain)$. This was a
  reason, why we could not use standard techniques for the proof of
  the convergence result and after applying consistency had to
  estimate each term separately.

  In this section, we will introduce a reformulation of the interior
  penalty method, which is equivalent to the original method on $V_h$,
  but is also bounded in $H^1(\domain)$. As an unpleasant side effect,
  it turns out that this method is inconsistent, and we have to
  estimate the consistency error.
  
  The main technique applied here is the use of lifting operators,
  such that the traces of derivatives on faces can be replaced by
  volume terms. Note that the lifting operators, while very useful for
  the analysis of the method, are not actually used in the
  implementation of the interior penalty method.
\end{intro}

\begin{Definition}{dg-lifting}
  Define the auxiliary space 
  \begin{gather}
    \label{eq:ip-lifting:1}
    \Sigma_h = \bigl\{ \tau\in L^2(\domain;\R^d) \big\vert
    \,\forall \cell\in \mesh_h: \tau_{|\cell} \in \Sigma_T \bigr\},
  \end{gather}
  where $\Sigma_T$ is a (possibly mapped) polynomial space chosen such that
  $\nabla V_T \subset \Sigma_T$. Then, we define the \define{lifting
    operator}
  \begin{gather}
    \label{eq:ip-lifting:2}
    \lifting\colon V+V_h \to \Sigma_h
  \end{gather}
  by
  \begin{gather}
    \label{eq:ip-lifting:3}
    \form(\lifting v,\tau)_{\mesh_h}
    = 2\forme(\mvl{\tau},\mvl{v\n})_{\faces_h^i}
    + \forme(\tau\cdot\n,v)_{\faces_h^\d}.
  \end{gather}
\end{Definition}

\begin{Lemma}{ip-lifting-bounded}
  The lifting operator is a bounded operator from $L^2(\faces_h)$ to
  $\Sigma_h$, such that
  \begin{gather}
    \label{eq:ip-lifting:4}
    \norm{\lifting v}_{L^2(\domain)}
    \le c \norm*{\tfrac1{\sqrt{h}}\jmp{v}}_{\faces_h^i}
    + \norm*{\tfrac1{\sqrt{h}} v}_{\faces_h^\d}.
  \end{gather}
  In particular, it is bounded on $H^1(\domain)$.
\end{Lemma}

\begin{proof}
  It is clear, that the operator is bounded on $L^2(\faces_h)$, since
  its definition involves face integrals weighted with polynomial
  functions. The dependence on the mesh size is due to the standard
  scaling argument.
\end{proof}

\begin{Definition}{ip-lifting}
  The \define{interior penalty method} with lifting operators uses the
  bilinear form
  \begin{multline}
    \label{eq:ip-lifting:5}
    a_h(u,v) = \form(\nabla u,\nabla v)_{\mesh_h}
    - \form(\lifting u, \nabla v)_{\mesh_h}
    - \form(\nabla u, \lifting v)_{\mesh_h}
    \\
    + \forme(\ipp_h\jmp{u},\jmp{v})_{\faces_h^-}
    + \forme(\ipp_h u,v)_{\faces_h^\d}
    .
  \end{multline}
  and the linear form~\eqref{eq:ip:3} of the original interior penalty
  method. Its residual operator is
  \begin{gather}
    \label{eq:ip-lifting:7}
    \Res(u,v) = a_h(u,v) - \form(f,v).
  \end{gather}
\end{Definition}

\begin{Lemma}{ip-equivalence}
  The interior penalty method in flux form (\slideref{Definition}{ip})
  and in lifting form (\slideref{Definition}{ip}) coincide on the
  discrete space $V_h$ if $\Sigma_h$ is chosen such that $\nabla V_h
  \subset \Sigma_h$.
  
  The lifting form is bounded on $H^1(\domain)+V_h$, but not consistent. The flux form is consistent for continuos solutions $u\in H^{\nicefrac32}(\domain)$ but not bounded on $H^1(\domain)+V_h$.
\end{Lemma}

\begin{proof}
  Since $\nabla V_h \subset \Sigma_h$, $\nabla u_h$ and $\nabla v_h$
  are valid test functions in the definition~\eqref{eq:ip-lifting:3}
  of the lifting operator, and  the equality
  \begin{gather*}
    \form(\lifting{u_h},\nabla v_h)_{\mesh_h}
    = 2\forme(\mvl{u_h\n},\mvl{\nabla v_h})_{\faces_h^i}
    + \forme(u_h,\d_n v_h)_{\faces_h^\d}.
  \end{gather*}
  
  The remaining statements of the lemma summarize and contextualize our analysis of the method in flux form and the application of \slideref{Lemma}{ip-lifting-bounded}.
\end{proof}

\begin{Definition}{ip-residual}
  Let $V\subset H^1(\domain)$ and let $u,u^*\in V$ solve the primal
  and dual problems
  \begin{gather}
    a(u,v) = f(v),
    \qquad
    a(v,u^*) = \psi(v),
    \qquad
    \forall v\in V,
  \end{gather}
  with a bounded, $V$-elliptic bilinear form $a(.,.)$. For a discrete
  bilinear form $a_h(.,.)$ defined on $V+V_h$, we define the primal
  and dual \define{residual operator}s
  \begin{gather}
    \label{eq:ip-lifting:8}
    \begin{split}
      \Res(u,v) &= a_h(u,v) - f(v), \\
      \Res^*(u^*,v) &= a_h(v,u^*) - \psi(v).
    \end{split}
  \end{gather}
\end{Definition}

% This IS Strang's second lemma!
\begin{Lemma*}{ip-lifting-strang}{Strang's 2nd Lemma}
  Let $a_h(.,.)$ be a bounded bilinear form on $V+V_h$ and elliptic on
  $V_h$ with norm $\norm{.}_{V_h}$ and constant $\ellipa$. Then, the error
  $u-u_h$ admits the estimate
  \begin{gather}
    \label{eq:ip-lifting:9}
    \norm{u-u_h}_{V_h} \le \frac1{\ellipa}
    \norm{\Res(u,.)}_{V_h^*}
    + \left(1+\frac{\norm{a_h}}{\ellipa}\right)
    \inf_{w_h\in V_h}\norm{u-w_h}
  \end{gather}
\end{Lemma}*

\begin{proof}
  First, by the definition of the residual, we have the error equation
  \begin{gather}
    a_h(u-u_h, v_h) = \Res(u,v_h),\qquad\forall v_h\in V_h.
  \end{gather}
  Inserting $w_h-w_h$ for an arbitrary element $w_h\in V_h$, we obtain
  \begin{gather*}
    a_h(w_h-u_h, v_h) = \Res(u,v_h) - a_h(u-w_h, v_h),\qquad\forall v_h\in V_h.
  \end{gather*}
  Using $v_h = w_h-u_h$ and ellipticity, we obtain
  \begin{align*}
    \ellipa \norm{w_h-u_h}_{V_h}^2
    &\le a_h(w_h-u_h,w_h-u_h)\\
    & = \Res(u,w_h-u_h) - a_h(u-w_h, w_h-u_h)\\
    & \le \bigl(\norm{\Res(u,.)}_{V_h^*} + \norm{a_h} \norm{u-w_h}_{V_h}\bigr)
      \norm{w_h-u_h}_{V_h}.
  \end{align*}
  Hence, by triangle inequality
  \begin{gather*}
    \norm{u-u_h}_{V_h} \le \frac1{\ellipa} \norm{\Res(u,.)}_{V_h^*}
    + \left(1+\frac{\norm{a_h}}{\ellipa}\right)
    \inf_{w_h\in V_h}\norm{u-w_h}_{V_h}
  \end{gather*}
\end{proof}

% From Girault/Kanschat/Riviere

\begin{Lemma}{ip-lifting-residual-1}
  Let $u\in V$ be the solution to the Poisson equation
  with right hand side $f\in L^2(\domain)$. Assume $u\in H^s(\domain)$
  with $s>3/2$. Then, we have for $v\in V+V_h$:
  \begin{gather}
    \label{eq:hdivdg:11}
    \form(f,v) = \form(\nabla u,\nabla v)_{\mesh_h}
    - 2\forme(\nabla u,\mvl{v \n})_{\faces_h^i}
    -\forme(\d_n u,v)_{\faces_h^\d}.
  \end{gather}
\end{Lemma}

\begin{proof}
  We set out from the strong form of the Poisson equation and
  integrate by parts.
  \begin{gather*}
    \form(f,v) = \form(-\Delta u, v)
    = \form(\nabla u,\nabla v)_{\mesh_h}
    - \sum_{\cell\in\mesh_h} \forme(\d_n u,v)_{\d\cell}
    .
  \end{gather*}
  Under the regularity assumptions of the lemma, all of these
  integrals make sense at least as duality pairings. In particular,
  $\d_n u\in L^2(\d\cell)$, and thus we can split $\d\cell$ into
  individual faces. Therefore,
  \begin{gather*}
    \sum_{\cell\in\mesh_h} \forme(\d_n u,v)_{\d\cell}
    = 2\forme(\nabla u,\mvl{v \n})_{\faces_h^i}
    + \forme(\d_n u,v)_{\faces_h^\d}.
  \end{gather*}
  The proof concludes by collecting the results.
\end{proof}

\begin{Lemma}{ip-lifting-residual-2}
  Let $k\ge 1$ and let $V_h$ such that $\P_{k-1} \subset \Sigma_T$. Then, if
  $u\in H^{k+1}(\domain)$ and $v\in V+V_h$, there holds
  \begin{gather}
    \label{eq:ip-lifting:10}
    \begin{split}
    \abs{\Res(u,v)}
    &\le c h^{k} \snorm{u}_{k+1}
    \bigl(
    \norm{\sqrt{\ipp_h}\jmp{v}}_{\faces_h^i}
    +
    \norm{\sqrt{\ipp_h}v}_{\faces_h^\d}\bigr)
    \\
    &\le c h^{k} \snorm{u}_{k+1} \norm{v}_{1,h}.
    \end{split}
  \end{gather}
  % Furthermore, for $v_h\in V_h$, there holds
  % \begin{gather}
  %   \label{eq:ip-lifting:11}
  %   \Res(u,v_h) = 0.
  % \end{gather}
\end{Lemma}

\begin{proof}
  First, we observe that by the regularity assumption, $\jmp{u} = 0$
  and thus, $\lifting u=0$. Hence,
  \begin{gather*}
    a_h(u,v) = \form(\nabla u, \nabla v)_{\mesh_h}
    - \form(\nabla u,\lifting v)_{\mesh_h}.
  \end{gather*}
  By \slideref{Lemma}{ip-lifting-residual-1} and regularity of $u$,
  \begin{align*}
    \Res(u,v)
    &= 2\forme(\nabla u,\mvl{v\n})_{\faces_h^i}
      + \forme(\d_n u,v)_{\faces_h^\d}
      - \form(\nabla u,\lifting v)_{\mesh_h}
    \\
    &= 2\forme(\mvl{\nabla u},\mvl{v\n})_{\faces_h^i}
      + \forme(\d_n u,v)_{\faces_h^\d}
      - \form(\nabla u,\lifting v)_{\mesh_h}
    \\
    &= 2\forme(\mvl{\nabla u},\mvl{v\n})_{\faces_h^i}
      + \forme(\d_n u,v)_{\faces_h^\d}
      - \form(\Pi_{\Sigma_h} \nabla u,\lifting v)_{\mesh_h},
  \end{align*}
  where $\Pi_{\Sigma_h}$ is the $L^2$-projection. Now, we can apply
  the definition of the lifting term to obtain
  \begin{multline*}
    \Res(u,v) = 2\forme(\tfrac1{\ipp_h}\mvl{\nabla u -
      \Pi_{\Sigma_h}\nabla u}, \ipp_h\mvl{v\n})_{\faces_h^i}
    \\
    + \forme(\tfrac1{\ipp_h} (\nabla u -\Pi_{\Sigma_h} \nabla
    u)\cdot\n, \ipp_h v)_{\faces_h^\d}.
  \end{multline*}
  Application of standard approximation and trace estimates yields the
  result observing that $\ipp_h = \ipp_0/h$.
\end{proof}

\begin{Theorem}{ip-lifting-h1}
  Let $k\ge 1$ and $V_h$ such that $\P_k \subset V_\cell$. Let
  $u\in H^{k+1}(\domain)$ be the solution to the continuous Poisson
  problem. Let $a_h(.,.)$ be the interior penalty method with lifting
  operators such that $\nabla V_h\subset\Sigma_h$. Then, there holds
  \begin{gather}
    \norm{u-u_h}_{1,h} \le c h^{k} \snorm{u}_{k+1}.
  \end{gather}
\end{Theorem}

\begin{proof}
  Application of \slideref{Lemma}{ip-lifting-strang},
  \slideref{Lemma}{ip-lifting-residual-2}, and standard interpolation
  results.
\end{proof}

\begin{Theorem}{ip-lifting-l2}
  Let the assumptions of \slideref{Theorem}{ip-lifting-h1} hold and in
  addition assume that the problem
  \begin{gather*}
    a(v,u^*) = \psi(v),\qquad\forall v\in V,
  \end{gather*}
  admits the \putindex{elliptic regularity} estimate
  \begin{gather}
    \label{eq:ip-lifting:12}
    \norm{u^*}_{H^2(\domain)} \le c \norm{\psi}_{L^2(\domain)}.
  \end{gather}
  Then, there holds
  \begin{gather}
    \label{eq:ip-lifting:13}
    \norm{u-u_h}_{L^2(\domain)} \le c h^{k+1} \snorm{u}_{H^{k+1}(\domain)}.
  \end{gather}
\end{Theorem}

\begin{proof}
  The proof uses the duality argument by Aubin and Nitsche, which sets
  out solving the auxiliary problem
  \begin{gather*}
    a(v,u^*) = \form(u-u_h,v),\qquad\forall v\in V.
  \end{gather*}
  Using the definition of the dual residual, we obtain the equation
  \begin{gather*}
    \form(u-u_h, v) = a_h(v,u^*) - \Res^*(u^*,v),\qquad\forall v\in V+V_h.
  \end{gather*}
  Testing with $v=u-u_h$ yields
  \begin{gather*}
    \norm{u-u_h}^2 = a_h(u-u_h,u^*) - \Res^*(u^*,u-u_h).
  \end{gather*}
  Additionally, we us the error equation
  \begin{gather*}
    a_h(u-u_h, v_h) = \Res(u,v_h),
  \end{gather*}
  tested with $v_h = I_h u^*$, to obtain
  \begin{multline*}
    \norm{u-u_h}^2 = a_h(u-u_h, u^*-I_h u^*) - \Res^*(u^*,u-u_h)
    + \Res(u,I_h u^*).
  \end{multline*}
  Using the regularity of $u^*$, the first term on the right
  admits the estimate
  \begin{gather*}
    \abs{a_h(u-u_h, u^*-I_h u^*)}
    \le \norm{u-u_h}_{1,h}\norm{u^*-I_hu^*}_{1,h}
    \le c h \norm{u-u_h}_{1,h}.
  \end{gather*}
  For the second term, we use \slideref{Lemma}{ip-lifting-residual-2}
  to obtain
  \begin{gather*}
    \abs{\Res^*(u^*,u-u_h)} \le c h \snorm{u^*}_2 \norm{u-u_h}_{1,h}.
  \end{gather*}
  Finally, using $\jmp{u^*} = 0$, the same lemma yields
  \begin{align*}
    \abs{\Res(u, I_h u^*)}
    &\le c h^k \snorm{u}_{k+1}
      \bigl(\norm{\sqrt{\ipp_h}\jmp{I_h u^*}}_{\faces_h^i}
      + \norm{\sqrt{\ipp_h}I_h u^*}_{\faces_h^\d}\bigr)
    \\
    & = c h^k \snorm{u}_{k+1}
      \bigl(\norm{\sqrt{\ipp_h}\jmp{u^*-I_h u^*}}_{\faces_h^i}
      + \norm{\sqrt{\ipp_h}(u^*-I_h u^*)}_{\faces_h^\d}\bigr)
    \\
    & \le c h^k \snorm{u}_{k+1} h \snorm{u^*}_2
  \end{align*}
  Using the energy estimate in \slideref{Theorem}{ip-lifting-h1} we
  can conclude the prove.
\end{proof}

%%% Local Variables:
%%% mode: latex
%%% TeX-master: "main"
%%% End:
