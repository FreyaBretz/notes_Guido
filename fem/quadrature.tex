\begin{intro}
  We begin the investigation of variational crimes by studying the
  effect of using numerical quadrature instead of exact integration on
  mesh cells. In particular, we investigate approximations of the form
  \begin{gather}
    \int_\cell f(\vx)\dvx
    \approx \sum_{k=1}^{n_q} \omega_k f(\vx_k) =: Q_\cell(f).
  \end{gather}
  First, we observe that $Q_\cell$ is not a bounded operator on
  $L^1(\domain)$, such that $Q_\cell(\nabla u\cdot \nabla v)$ is
  undefined for functions in $H^1(\domain)$. The surprising result of
  this section is, that quadrature is still admissible for the
  implementation of a finite element method.

  We will first set a theoretical framework and then investigate
  quadrature rules in detail. The presentation follows~\cite[Chapter
  4]{Ciarlet78}.
\end{intro}

\begin{Lemma*}{strang-1}{Strang's first lemma}
  Let $a(.,.)$ be a bounded and elliptic bilinear form on the Hilbert
  space $V$. Let $V_n\subset V$ and let $a_h(.,.)$ be a bilinear form,
  bounded and elliptic on $V_n$ with constants $M_n$ and
  $\alpha_n$. Let $f, f_n\in V^*$. If $u \in V$ and
  $u_n\in V_n \subset V$ are solutions to
  \begin{gather*}
    \begin{aligned}
      a(u,v) &= f(v) & \qquad\forall v&\in V,\\
      a_n(u_n,v_n) &= f_n(v_n) & \qquad\forall v_n&\in V_n,
    \end{aligned}
  \end{gather*}
  respectively, there holds
  \begin{multline}
    \norm{u-u_n}_V \\
    \le \inf_{v_n\in V_n}\biggl[\left(1+\frac{M}{\alpha_n}\right)
    \norm{u-v_n}_V
    + \frac1{\alpha_n}
    \norm{a_n(v_n,.)-a(v_n,.)}_{V_h^*}\biggr]\\
         + \frac1{\alpha_n}\norm{f_n-f}_{V_h^*}.
  \end{multline}
\end{Lemma*}

\begin{proof}
  Using $V_n$-ellipticity of $a_n(.,.)$ yields for arbitrary $v_n\in V_n$
  \begin{align*}
    \alpha_n \norm{u_n-v_n}^2
    \le &a_n(u_n-v_n,u_n-v_n)\\
    =& f_n(u_n-v_n) + a(u-v_n, u_n-v_n)
       - f(u_n-v_n)
    \\ &+ a(v_n,u_n-v_n) - a_n(v_n, u_n-v_n)
  \end{align*}
  We estimate separately
  \begin{align*}
    \frac{a(u-v_n, u_n-v_n)}{\norm{u_n-v_n}}
    &\le M \norm{u_n-v_n},\\
    \frac{\abs{f_n(u_n-v_n) - f(u_n-v_n)}}{\norm{u_n-v_n}}
    &\le \sup_{w_n\in V_n} \frac{\abs{f_n(w_n)-f(w_n)}}{\norm{w_n}},
    \\
    \frac{\abs{a(v_n,u_n-v_n) - a_n(v_n, u_n-v_n)}}{\norm{u_n-v_n}}
    &\le \sup_{w_n\in V_n} \frac{\abs{a_n(v_n, w_n)-a(v_n,w_n)}}{\norm{w_n}}.
  \end{align*}
  Combining all terms, we obtain
  \begin{multline}
    \norm{u-u_n}
    \le \norm{u-v_n} + \norm{u_n-v_n} \\
    \le \norm{u-v_n} + \frac1{\alpha_n}\left(
         M \norm{u-v_n}
         + \norm{a_n(v_n,.)-a(v_n,.)}_{V_h^*}
         + \norm{f_n-f}_{V_h^*}\right).
  \end{multline}
\end{proof}

\begin{remark}
  Strang's lemma states, that the error can be split into the
  \putindex{approximation error} of the space $V_h$ and the
  \putindex{consistency error} of the approximations $a_h(.,.)$ und
  $f_h(.)$. Both errors are scaled with the stability factor
  $\nicefrac1{\alpha _n}$ of the discrete problem. So far, this is
  consistent with error estimates for instance for Runge-Kutta
  methods. There is an important difference though: the consistency
  error is not evaluated for the exact solution, but only for its
  discrete approximation.
\end{remark}

\begin{remark}
  We will apply Strang's lemma to a family of meshes indexed by mesh
  size $h$ and assess the infimum by an interpolation operator. It is
  clear, that we will only obtain optimal convergence rates compared
  to the interpolation estimate, if there exists $\alpha_0 >0$ such
  that $\alpha_n \ge \alpha_0$ uniformly with respect to $n$. While it
  is not a prerequisite of Strang's lemma, it is our goal for all
  discretizations.
\end{remark}

\begin{remark}
  Quadrature would be infeasible, if we had to devise a quadrature
  rule for every mesh cell $\cell$. Instead, we tabulate quadrature
  formulas for the reference cell $\refcell$ by choosing quadrature
  points $\refvx_k$ and weights $\omega_k$ and write
  \begin{gather}
    Q_{\refcell}(\reference f)
    = \sum_{k=1}^{n_q} \omega_k \reference f(\refvx_k).
  \end{gather}
  We compute integrals over $\cell$ though mapping,
  \begin{gather}
    Q_T(f) = \sum_{k=1}^{n_q} \det\nabla\Phi_\cell(\refvx_k) \omega_k \reference f(\refvx_k).
  \end{gather}
  Thus, $Q_T$ is defined by quadrature points
  $\vx_k = \Phi(\refvx_k)$ and quadrature weights
  $\det\nabla\Phi_\cell(\refvx_k) \omega_k$.

  Quadrature rules on the reference cell $\refcell$ are obtained by
  interpolation after choosing quadrature points by employing the
  properties of orthogonal polynomials.  The construction of such
  quadrature rules for simplices is somewhat complicated, and we refer
  to tables in the cited literature.

  An important consequence of the use of quadrature points as roots of
  orthogonal polynomials is the fact that all weights are positive.
\end{remark}

\begin{Definition}{tensor-product-quadrature}
  Given a one-dimensional quadrature rule $Q_I$ on the interval
  $I=[0,1]$ with
  \begin{gather}
    Q_I(\reference f) = \sum_{k=1}^{n_q} \omega_k \reference f(\refx_k).
  \end{gather}
  Then, a quadrature rule on $\refcell = [0,1]^d$ is defined by
  \begin{gather}
    Q_{\refcell}(\reference f)
    = \sum_{k_1}^{n_q}\dots\sum_{k_d}^{n_q} \omega_{k_1}\cdots\omega_{k_d}
    \reference f(x_{k_1},\dots,x_{k_d}).
  \end{gather}
\end{Definition}

\begin{Lemma}{tensor-product-gauss}
  If $Q_I$ is an $n$-point Gauß formula, the tensor product quadrature
  in the preceding definition is exact on $\Q_{2n-1}$.
\end{Lemma}

\begin{remark}
  When we look at Poisson's equation on simplicial meshes, the integrals
  \begin{gather}
    a_\cell(\phi_j,\phi_i)
    = \int_\cell \nabla\phi_j\cdot\nabla\phi_i\dvx
    = \int_{\refcell} \nabla\Phi^{-T}\refgrad\refphi_j
    \cdot\nabla\Phi^{-T}\refphi_i \det\nabla\Phi\dvxref,
  \end{gather}
  can be computed exactly, since $\nabla\Phi$ is a constant
  matrix. Already on arbitrary quadrialterals, $\nabla\Phi^{-T}$ is a
  rational function, wich is harder to integrate. But, in order to
  justify the use and investigate the properties of approximations by
  numerical quadrature, we are considering a model problem with
  varying coefficients.
\end{remark}

\begin{Assumption}{bilinear-form-quadrature}
  Let in this section the bilinear form be defined as
  \begin{gather}
    a(u,v) = \int_\domain \nabla u A \nabla v^T \dx,
  \end{gather}
  where $A= A(\vx)$ is a matrix with
  \begin{gather}
    \xi^T A(\vx) \xi \ge \alpha \abs{\xi^2}
    \qquad\forall \xi\in\R^d,\quad \forall \vx\in\domain.
  \end{gather}
  The function $u\in V=H^1_0(\domain)$ is the solution to
  \begin{gather}
    a(u,v) = \form(f,v) \qquad\forall v\in V.
  \end{gather}
  We assume that the coefficients $a_{ij}$ and the right hand side $f$
  are functions which are well-defined in all quadrature points on all
  meshes of a family $\{\mesh_h\}$.
\end{Assumption}

\begin{Definition}{bilinear-quadrature}
  The bilinear form $a_h(.,.)$ obtained by numerical quadrature
  $Q_{\refcell}$ on the mesh $\mesh_h$ is defined as
  \begin{gather}
    a_h(u,v) = \sum_{\cell\in\mesh_h} Q_\cell (\nabla u A \nabla v^T),
  \end{gather}
  where $Q_\cell$ is obtained from $Q_{\refcell}$ by the mapping
  $\Phi_\cell$. The right hand side is
  \begin{gather}
    f_h(v) = \sum_{\cell\in\mesh_h} Q_\cell (fv).
  \end{gather}
\end{Definition}

\begin{Lemma}{quadrature-stability}
  Let a finite element method be defined by a mesh $\mesh_h$ and a
  shape function space $\shapespace$ on the reference cell $\refcell$.
  Let $Q_{\refcell}$ be a quadrature formula, such that all quadrature
  weights are positive. Furthermore, assume that one of the following
  holds:
  \begin{enumerate}
  \item The function values in the quadrature points $\refvx_k$ are
    unisolvent on
    \begin{gather}
      \mathscr G_1
      = \bigl\{ \d_i p\big| p\in \shapespace, i=1,\dots,d\bigr\},
    \end{gather}
  \item The quadrature rule is exact on
    \begin{gather}
      \mathscr G_2
      = \bigl\{ \nabla p\cdot \nabla q \big| p,q\in \shapespace \bigr\}.
    \end{gather}
  \end{enumerate}
  Then, there is a constant $\alpha_0>0$ depending on shape regularity
  and the quadrature rule, but independent of $h$, such that
  \begin{gather}
    a_h(u_h,u_h) \ge \alpha_0 \snorm{u_h}_{1;\domain}^2
    \qquad\forall u_h\in V_h.
  \end{gather}
\end{Lemma}

\begin{remark}
  If $\shapespace = \P_k$, then $\mathscr G_1 = \P_{k-1}$ and
  $\mathscr G_2 = \P_{2k-2}$. For $\Q_k$, these relations are more
  complicated, and the simplest we can say is
  $\mathscr G_1 \subset \Q_k$ and $\mathscr G_2 \subset \Q_{2k}$
\end{remark}

\begin{proof}[Proof of \slideref{Lemma}{quadrature-stability}]
  Assume first condition 1. From the poisitivity of quadrature
  weights, we conclude that
  \begin{gather}
    Q_{\refcell}(\nabla p\cdot\nabla p) = 0
    = \sum_{k=1}^{n_q} \sum_{i=1}^d \omega_k (\d_ip(\refvx_k))^2 = 0
  \end{gather}
  implies $\d_ip(\refvx_k) = 0$. Unisolvence implies
  $\d_i p \equiv 0$. Therefore,
  $\sqrt{Q_{\refcell}(\nabla p\cdot\nabla p)}$ defines a norm on the
  space $\shapespace/\R$. Since this space is finite dimensional and
  $\snorm{.}_{1;\refcell}$ is a second norm, we conclude from norm
  equivalence the existence of a constant $c>0$ such that
  \begin{gather}
    c \snorm{p}_{1;\refcell}^2 \le Q_{\refcell}(\nabla p\cdot\nabla p)
    \qquad\forall p\in \shapespace.
  \end{gather}
  On the other hand, condition 2 yields the same estimate with $c=1$.

  It is easily verified, that $\snorm{p}_{1;\refcell}^2$ and
  $Q_{\refcell}(\nabla p\cdot\nabla p)$ scale equally under the
  mapping $\Phi_\cell$, such that the same estimate holds on $\cell$
  with a different constant depending on shape regularity, again
  denoted by $c$. Finally,
  \begin{align*}
    \alpha_0 c \snorm{u_h}_{1;\domain}^2
    &= \alpha_0 c \sum_{\cell\in\mesh_h} \snorm{u_h}_{1;\cell}^2 \\
    &\le \alpha_0 \sum_{\cell\in\mesh_h}Q_{\refcell}(\nabla u_h\cdot\nabla u_h)\\
    & \le \sum_{\cell\in\mesh_h} Q_{\refcell}(\nabla u_hA\nabla u_h^T)
      = a_h(u_h,u_h).
  \end{align*}
\end{proof}

\begin{intro}
  Now that we have established conditions for stability of the
  discrete problem, we can address an estimate of the consistency
  error. In order to avoid additional overhead in an already technical
  argument, we restrict the analysis to the spaces
  $\shapespace_\cell = \P_k$ on simplices. Instead of proving a
  general result for arbitrary quadrature formulas, we first take ask,
  what the order of the quadrature error should be. Indeed, we have
  the interpolation estimate $\norm{u-I_h u}_{1} = \mathcal
  O(h^k)$.
  Therefore, we investigate quadrature rules introducing consistency
  errors of the same order.
\end{intro}

\begin{Theorem}{quadrature-error-bilinear}
  Assume in addition to
  \slideref{Assumption}{bilinear-form-quadrature} for $k\ge1$ that
  for all cells $\shapespace_\cell = \P_k$ and $a_{ij}\in
  W^{k,\infty}(\domain)$.
  Let the quadrature rule $Q_{\refcell}$ be exact for polynomials in
  $\P_{2k-2}$.  Then, there exists a constant $c$ such that for all
  meshes of a quasi-uniform family $\{\mesh_h\}$ of meshes there holds
  \begin{gather}
    a(v_h, w_h) - a_h(v_h, w_h) \le c h^k \norm{A}_{k,\infty;\domain}
    \norm{v_h}_{k;h} \snorm{w_h}_{1;h}.
  \end{gather}
\end{Theorem}

\begin{Lemma}{product-sobolev-norm}
  Let $u\in W^{k,p}(\domain)$ and $v\in W^{k,\infty}(\domain)$. Then,
  $uv\in W^{k,p}(\domain)$ and
  \begin{gather}
    \snorm{uv}_{k,p;\domain}
    \le c \sum_{i=0}^{k} \snorm{u}_{i,q;\domain}\,
    \snorm{v}_{k-i,\infty;\domain},
  \end{gather}
  with a constant $c$ depending on $k$ and the space dimension, but
  not on $\domain$.
\end{Lemma}

\begin{proof}
  This is basically the product rule. For a multi-index $\alpha$ with
  $\abs{\alpha} = k$, we have
  \begin{gather*}
    \d^{\alpha}(uv) = \sum_{i=0}^k
    \sum_{\substack{\abs{\beta}=i\\\beta_j\le \alpha_j}} \d^{\beta}u\,\d^{\alpha-\beta} v.
  \end{gather*}
  Then, Hörder inequality is applied to the sums and the integrals.
\end{proof}

\begin{proof}[Proof of \slideref{Theorem}{quadrature-error-bilinear}]
  We first argue on the reference cell and define the quadrature error
  \begin{gather}
    E_{\refcell}(f) = \int_{\refcell} f \dvxref - Q_{\refcell}(f).
  \end{gather}
  Our goal is the estimation of $E_{\refcell}(apq)$ for functions
  $a\in W^{k,\infty}(\refcell)$ and polynomials $p,q\in \P_{k-1}$,
  such that we obtain an estimate for the constituents of
  $E_{\refcell}(\refgrad\refu\reference A \refgrad\refv)$.

  We combine $\phi = ap\in W^{k,\infty}(\refcell)$ Since
  $W^{k,\infty}(\refcell)\hookrightarrow C(\refcell)$, we obtain
  \begin{gather}
    \abs{E_{\refcell}(\phi q)} \le c \norm{\phi q}_{\infty}
      \le c\norm{\phi}_{0,\infty} \norm{q}_{0,\infty}
      \le c \norm{\phi}_{k,\infty} \norm{q}_{L^2(\refcell)}.
  \end{gather}
  For the last inequality, we used the natural bound of the norm in
  $W^{0,\infty}$ by that of $W^{k,\infty}$ and the norm equivalence on
  $\P_{k-1}$. We conclude that $J_q(\phi) = E_{\refcell}(\phi q)$ is a
  bounded linear functional on $W^{k,\infty}$. Furthermore, it
  vanishes for $\phi\in\P_{k-1}$ by the assumption on the quadrature
  rule. Thus by the Bramble-Hilbert lemma (\slideref{Lemma}{bramble-hilbert}),
  \begin{gather}
    \abs{E_{\refcell}(\phi q)}
    \le c \snorm{\phi}_{k,\infty}\norm{w}_{L^2(\refcell)}.
  \end{gather}
  Now, we apply \slideref{Lemma}{product-sobolev-norm}, to obtain
  \begin{gather}
    \begin{split}
    \snorm{\phi}_{k,\infty}
    &= \snorm{ap}_{k,\infty}\\
    &\le c \sum_{i=0}^{k-1} \snorm{a}_{k-i,\infty} \snorm{p}_{i,\infty}\\
    &\le c \sum_{i=0}^{k-1} \snorm{a}_{k-i,\infty} \snorm{p}_{i},
    \end{split}
  \end{gather}
  where we used that $\snorm{p}_k = 0$ and norm equivalence on
  $\P_{k-1}$.  Collecting everything and reintroducing $\reference a$,
  $\reference p$, and $\reference q$ for $a$, $p$, and $q$,
  respectively, we obtain
  \begin{gather}
    \abs{E_{\refcell}(\reference a\reference p\reference q)}
    \le c \left(\sum_{i=1}^{k-1} \snorm{\reference a}_{k-1,\infty;\refcell}
      \snorm{\reference p}_{i;\refcell}\right)
    \norm{\reference q}_{L^2(\refcell)}.
  \end{gather}
  The scaling lemma yields
  \begin{xalignat}2
    \snorm{\reference a}_{k-1,\infty;\refcell}
    &\le c h_\cell^{k-i} \snorm{a}_{k-i,\infty;\cell}
    & 0 & \le i \le k-1,
    \\
    \snorm{\reference p}_{i;\refcell}
    &\le c h_\cell^{i-\nicefrac{d}{2}} \snorm{p}_{i;\cell}
    & 0 & \le i \le k-1,
    \\
    \norm{\reference q}_{L^2(\refcell)}
    &\le c h_\cell^{-\nicefrac{d}{2}} \norm{q}_{L^2(\cell)}.
  \end{xalignat}
  Therefore,
  \begin{gather}
    \begin{split}
    \abs{E_\cell(a p q)}
    &\le c h_\cell^{\nicefrac{d}{2}}
    \abs{E_{\refcell}(\reference a\reference p\reference q)}
    \\ & \le c h_\cell^{k} \left(
      \sum_{i=1}^{k-1}\snorm{a}_{k-i,\infty;\cell}\snorm{p}_{i;\cell}
    \right)
    \norm{q}_{L^2(\cell)}
    \\ & \le c h_\cell^{k} \norm{a}_{k-i,\infty;\cell}\norm{p}_{i;\cell}
    \norm{q}_{L^2(\cell)}.
    \end{split}
  \end{gather}
  Entering the derivatives of $u_h$ and $v_h$ for $p$ and $q$,
  respectively, and summing up over all cells yields the result.
\end{proof}

\begin{remark}
  Note that the assumption of quasi-uniformity is convenient for
  notation, but excessive. Indeed, shape regularity is sufficcient,
  but in that case, the estimate must be localized, that is,
  \begin{gather}
    a(v_h, w_h) - a_h(v_h, w_h) \le c \sum_{\cell\in\mesh_h}
    h_\cell^k \norm{A}_{k,\infty;\cell}
    \norm{v_h}_{k;\cell} \snorm{w_h}_{1;\cell}.
  \end{gather}
\end{remark}

\begin{Theorem}{quadrature-error-rhs}
  Let the finite element use shape function spaces
  $\shapespace_\cell = \P_k$ and let the quadrature rule
  $Q_{\refcell}$ be exact for polynomials in $\P_{2k-2}$. Then, there
  exists a constant $c$ such that for all cells of a quasi-uniform
  family $\{\mesh_h\}$ of meshes there holds
  \begin{gather}
    \abs{E_\cell(fp)} \le c h_\cell^k \norm{f}_{k;\cell}\norm{p}_{1;\cell}
    \qquad\forall f\in H^2(\domain),\quad \forall p\in\shapespace.
  \end{gather}
\end{Theorem}


%%% Local Variables: 
%%% mode: latex
%%% TeX-master: "main"
%%% End: 
