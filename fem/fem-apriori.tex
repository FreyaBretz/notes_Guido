\begin{intro}
  In this section, we develop error estimates of the following type
  \begin{verse}
    If the size of mesh cells converges to zero, then the difference
    between the true solution and the finite element solution
    converges to zero with a certain order.
  \end{verse}
  They are thus a prediction, that the solutions actually converge,
  and they measure the asymptotic convergence rate. They do contain
  unknown constants, such that they are no prediction of the error on
  a given mesh.

  The theory in this chapter is about bilinear forms which are bounded
  and elliptic on a subspace $V \subset H^1(\domain)$ determined by
  boundary conditions. We will choose $V = H^1_0(\domain)$, even if
  more general boundary conditions are possible. It is a good approach
  to think of the Dirichlet problem for the Laplacian, even if we
  allow for somewhat more general equations.
\end{intro}

\subsection{Approximation of Sobolev spaces by finite elements}

\begin{Lemma*}{poincare}{Poincaré inequality}
  Let $\domain$ be a bounded Lipschitz domain. For any function
  $u\in H^1(\domain)$ define
  \begin{gather}
    \overline u = \frac1{\abs{\domain}} \int_\domain u(\vx) \dvx,
  \end{gather}
  where $\abs{\domain}$ denotes the measure of $\domain$. There exists
  a constant $c$ depending on the domain only, such that each of the
  following inequalities hold:
  \begin{align}
    \norm{u-\overline u}_{L^2(\domain)} &\le c \norm{\nabla u}_{L^2(\domain)}\\
    \norm{u}_{L^2(\domain)}^2 &\le c \Bigl(\norm{\nabla u}_{L^2(\domain)}^2 + \overline u^2\Bigr)
  \end{align}
\end{Lemma*}

\begin{proof}
  The proof exceeds the tools we have developed in this class. The
  proof in \cite[Section 7.8]{GilbargTrudinger98} seems elementary and
  direct, but is technical and requires star-shaped domains. The proof
  in \cite[Section 5.8.1]{Evans98} is more elegant, but it uses
  compact embedding and is indirect, such that the constant cannot be
  determined.
\end{proof}

\begin{Lemma*}{bramble-hilbert}{Bramble-Hilbert}
  Let $\cell\subset \R^d$ be a domain with Lipschitz boundary and let
  $s(.)$ be a bounded sublinear functional on $H^{k+1}(\cell)$ with
  the property
  \begin{gather}
    s(p) = 0 \qquad\forall p\in \P_k.
  \end{gather}
  Then, there exists a constant $c$ only dependent on $\cell$ such that
  \begin{gather}
    \abs{s(v)} \le c \snorm{v}_{k+1,\cell}.
  \end{gather}
\end{Lemma*}

\begin{proof}
  Since $s(\cdot)$ is sublinear and vanishes on $\P_k$, we have for
  $v\in H^{k+1}(\cell)$:
  \begin{gather}
    \label{eq:bramble-hilbert-1}
    \abs{s(v)} \le \abs{s(v+p)} + \abs{s(p)} = \abs{s(v+p)}
    \qquad\forall p\in\P_k.
  \end{gather}
  We will construct a polynomial, such that
  \begin{gather}
    \label{eq:bramble-hilbert-2}
    \overline{\d^\alpha(v+p)}
    = \frac1{\abs{\cell}}\int_\cell \d^\alpha (v+p) \dx = 0
    \qquad\forall \abs{\alpha} \le k,
  \end{gather}
  that is, the sum $v+p$ and all its derivatives up to order $k$ are
  mean-value free. Thus, by recursive application of Poincaré
  inequality, we get for $\abs{\alpha}\le k$
  \begin{alignat*}2
    \norm{v+p}_{L^2(\cell)}^2
    &\le c \left[ \norm{\nabla(v+p)}_{L^2(\cell)}^2 + \overline{v+p}^2\right]
      && \le c \snorm{v+p}_{1;\cell}^2\\
    \norm{\d^\alpha(v+p)}_{L^2(\cell)}^2,
    &\le c \left[ \norm{\nabla\d^\alpha(v+p)}_{L^2(\cell)}^2
      + \overline{\d^\alpha(v+p)}^2\right]
      && \le c \snorm{v+p}_{\abs{\alpha}+1;\cell}^2,
  \end{alignat*}
  such that $\norm{v+p}_{k+1;\cell} \le \snorm{v+p}_{k+1;\cell}$.
  Furthermore, since $p\in\P_k$
  \begin{gather*}
    \norm{\d^\alpha(v+p)}_{L^2(\cell)} = \norm{\d^\alpha v}_{L^2(\cell)}
    \qquad\forall \abs{\alpha} = k+1.
  \end{gather*}
  Combining with~\eqref{eq:bramble-hilbert-1}, we obtain
  \begin{gather*}
    \abs{s(v)} \le c \snorm{v+p}_{k+1;\cell} \le c \snorm{v}_{k+1;\cell}.
  \end{gather*}
  It remains to construct the polynomial $p$ with the desired
  properties. To this end, we note that for two multi-indices $\alpha$
  and $\beta$ holds that $\d^\alpha \vx^\beta =0$ as soon as
  $\alpha_i>\beta_i$ for some index $i$. Let
  \begin{gather*}
    p(\vx) = \sum_{\abs{\beta}\le k} a_\beta \vx^\beta.
  \end{gather*}
  Then, for any $\abs{\alpha} = k$ we get
  \begin{gather*}
    \d^\alpha p(\vx) = \alpha! \,\delta_{\alpha\beta},
  \end{gather*}
  where $\alpha! = \alpha_1!\alpha_2!\dots\alpha_d!$. Thus, we can use
  condition~\eqref{eq:bramble-hilbert-2} to fix the coefficients
  $a_\beta$ to
  \begin{gather*}
    a_\beta = \frac1{\beta!\,\abs{\cell}} \int_\cell \d^\beta v\dx
    \qquad \abs{\beta} = k.
  \end{gather*}
  Thus, we have decomposed $p = \tilde p_k + p_{k-1}$, where
  $\tilde p_k$ is known and $p_{k-1}\in \P_{k-1}$. Thus, we can repeat
  the process determining the coefficients of highest order in
  $p_{k-1}$,
  \begin{gather*}
    a_\beta = \frac1{\beta!\,\abs{\cell}} \int_\cell \d^\beta (v-\tilde p_k)\dx
    \qquad \abs{\beta} = k-1.
  \end{gather*}
  Recursion down to $k=0$ yields the polynomial $p$ with the desired property.
\end{proof}

\begin{Corollary}{b-h-projector}
  Let $\Pi\colon H^{k+1}(\cell) \to \P_k$ be a continuous, linear
  projector. For any $m\le k$ there exists a constant $c$ such that
  \begin{gather}
    \norm{u-\Pi u}_{m,\cell} \le c \snorm{u}_{k+1,\cell}.
  \end{gather}
\end{Corollary}

\begin{Definition}{mesh-family}
  Let $\{\mesh_h\}$ for $h>0$ be a family of meshes parametrized by
  the parameter
  \begin{gather}
    h = \max_{\cell\in\mesh_h} h_\cell,
  \end{gather}
  where $h_\cell$ is the characteristic length from the discussion of
  mappings, for instance the diameter of $\cell$. Such a family is
  called \define{shape regular}, if the constants $M_\cell$,
  $m_\cell$, $d_\cell$, and $D_\cell$ in the scaling lemma
  can be chosen independent of the cell $\cell\in\mesh_h$ and of
  $h>0$. The family is called \define{quasi-uniform}, if in addition
  there is a positive constant independent of $h$ such that
  \begin{gather}
    h \le c \min_{\cell\in\mesh_h} h_\cell.
  \end{gather}
\end{Definition}

\begin{Definition}{nodal-interpolation}
  Let $V$ be a function space on $\domain$, and let
  $V_h = V_{\mesh_h}$ be a finite element space on the mesh $\mesh_h$
  on $\domain$ with node functionals $\nodal_i$ and dual basis $p_i$,
  where $i=1,\dots,n_h$. We define the \define{nodal interpolation}
  operator by
  \begin{gather}
    \begin{split}
      I_h\colon V&\to V_h\\
      v &\mapsto \sum \nodal_i(v)p_i.
    \end{split}
  \end{gather}
\end{Definition}

\begin{Lemma}{nodal-interpolation}
  The nodal interpolation operator $I_h$ is a projector. It is
  continuous on $H^2(\domain)$ if the dimension is $d=2,3$ and the
  node functionals are defined as Lagrange interpolation.
\end{Lemma}

\begin{Definition}{broken-sobolev-norm}
  On a mesh $\mesh_h$, we define the \define{broken Sobolev norm} and
  seminorm by
  \begin{gather}
    \begin{split}
      \norm{u}_{k;h}^2 &= \sum_{\cell\in\mesh_h} \norm{u}_{k;\cell}^2\\
      \snorm{u}_{k;h}^2 &= \sum_{\cell\in\mesh_h} \snorm{u}_{k;\cell}^2
    \end{split}
  \end{gather}
\end{Definition}

\begin{Theorem}{fe-interpolation}
  Let $\{\mesh_h\}$ be a shape regular family of meshes.
  Let the finite element spaces $V_h = V_{\mesh_h}$. Let the nodal
  interpolation operator $I_h$ be surjective onto $\P_k$ on every cell
  $\cell\in\mesh_h$ and continuous on $H^{k+1}(\domain)$. Then, there
  is a constant $c$ such that for any $u\in H^{k+1}(\domain)$ and
  $m\le k+1$ there holds
  \begin{gather}
    \label{eq:fe-interpolation}
    \norm{u-I_h u}_{m;h} \le c h^{k+1-m} \snorm{u}_{k+1;h}.
  \end{gather}
\end{Theorem}

\begin{proof}
  We have by definition
  \begin{gather*}
    \snorm{u-I_h u}_{m;h} = \sum_{\cell\in\mesh_h} \snorm{u-I_h u}_{m;\cell}.
  \end{gather*}
  Using the scaling lemma, we get
  \begin{gather*}
    \snorm{u-I_h u}_{m;\cell}
    \le c h_\cell^{\nicefrac d2-m} \snorm{\refu - \widehat{I_hu}}.
  \end{gather*}
  On the reference cell, we use the Bramble-Hilbert lemma, more
  precisely, \slideref{Corollary}{b-h-projector} to obtain
  \begin{gather*}
    \snorm{\refu - \widehat{I_hu}} \le c \snorm{\refu}_{k+1;\refcell}.
  \end{gather*}
  Scaling back yields
  \begin{gather*}
    \snorm{\refu}_{k+1;\refcell}
    \le c h_\cell^{k+1-\nicefrac d2} \snorm{u}_{k+1;\cell}.
  \end{gather*}
  Combining, we obtain
  \begin{gather*}
    \snorm{u-I_h u}_{m;\cell} \le c h_\cell^{k+1-\nicefrac d2 - m +
      \nicefrac d2}\snorm{u}_{k+1;\cell}.
  \end{gather*}
  Summing up and pulling the maximum of $h_\cell^{k+1-m}$ out of the
  sum yields the result for $h_\cell \le 1$.
\end{proof}

\begin{remark}
  Strictly speaking, we have only proven the result for $h_T \le
  1$.
  But then, if $\diam\domain = 1$, this condition is always
  true. Therefore, by rescaling the domain before computing, the
  estimate holds in general.
\end{remark}

\begin{Corollary}{fe-approximation}
  Let $a(.,.)$ be a bounded and elliptic bilinear form on
  $V = H^1_0(\domain)$ and let the finite element space $V_h$ be
  defined on a shape-regular family of meshes $\{\mesh_h\}$, such that
  the interpolation estimate~\eqref{eq:fe-interpolation} holds. If
  furthermore the solution $u\in H^{k+1}(\domain)$, the error of the
  finite element solution $u_h\in V_h \subset V$ admits the estimate
  \begin{gather}
    \norm{u-u_h}_{1;h} \le c h^{k} \snorm{u}_{k+1;h}.
  \end{gather}  
\end{Corollary}

\begin{intro}
  For 2nd order elliptic problems, we have now derived estimates of
  the $H^1$-norm of the error under the assumption that the solution
  exhibits further regularity. Now, let us drop this assumptionfor an
  assumption on the boundary condition only, to obtain a qualitative
  convergence result.
\end{intro}

\begin{Theorem}{fe-convergence}
  Let $a(.,.)$ be a bounded and elliptic bilinear form on
  $V = H^1_0(\domain)$ and let the finite element space $V_h$ be
  defined on a shape-regular family of meshes $\{\mesh_h\}$, such that
  the interpolation estimate~\eqref{eq:fe-interpolation} holds. Let
  $u\in V$ and $u_h\in V_h$ be solutions to the exact and the finite
  element versions of a 2nd order boundary value problem. Then,
  \begin{gather}
    \lim_{h\searrow 0} \norm{u-u_h}_{1;\domain} = 0.
  \end{gather}
\end{Theorem}

\subsection{Estimates of stronger norms}

\begin{intro}
  So far, we have seen error estimates in a ``natural norm'' defined
  as a norm such that the Lax-Milgram lemma holds for a given bilinear
  form $a(.,.)$. In this subsection, we now consider the question of
  estimates in stronger norms, such that the bilinear form is not
  elliptic with respect to this norm.
\end{intro}

\begin{Definition}{stronger-norm}
  Let $\norm{\cdot}_X$ and $\norm{\cdot}_Y$ be norms on a vector space
  $V$. We call $\norm{\cdot}_X$ a \define{stronger norm} than
  $\norm{\cdot}_Y$, if there is a constant $c$ such that for all
  $v\in V$:
  \begin{gather}
    \norm{v}_Y \le c \norm{v}_X.
  \end{gather}
  In this case, $\norm{\cdot}_Y$ is called the \define{weaker
    norm}. If a converse inequality holds, the norms are called
  \define{equivalent}.
\end{Definition}

\begin{example}
  For a bounded domain $\domain$, the norms $\norm{\cdot}_{k+1}$ and
  $\norm{\cdot}_{k}$ are both defined on $V = H^1_0(\domain)$. By the
  Sobolev embedding theorem, there is a constant $c$ such that for any
  $v\in V$
  \begin{gather*}
    \norm{v}_{k} \le c \norm{v}_{k+1}
  \end{gather*}
\end{example}

\begin{Lemma*}{inverse-estimate}{Inverse estimate}
  Let $\cell$ be a mesh cell of size $h_\cell$. Then, there is a
  constant only depending on $k$ and the constants of the scaling
  lemma, such that for every $u\in \P_k$ there holds
  \begin{gather}
    \norm{u}_{1;\cell} \le c h_T^{-1} \norm{u}_{0;\cell}.
  \end{gather}
\end{Lemma*}

\begin{Theorem}{h2-error}
  Let $a(.,.)$ be a bounded and elliptic bilinear form on
  $V\subset H^1(\domain)$ and let $\{\mesh_h\}$ be a family of
  quasi-uniform meshes with finite element spaces $V_h \subset V$
  containing the space $\P_k$ with $k\ge 2$ on each mesh cell. If
  furthermore the solution $u\in H^{k+1}(\domain)$, the error between
  the exact and the finite element solution to a uniquely solvable
  elliptic boundary value problem, admits the estimate
  \begin{gather}
    \norm{u-u_h}_{2;h} \le c h^{k-1} \snorm{u}_{k+1;h}.
  \end{gather}  
\end{Theorem}

\begin{proof}
  We cannot apply Céa's lemma directly, since the bilinear form is not
  elliptic with respect to the broken $H^2$-inner product on the space
  $H^1(\domain)$. On the other hand, the error is not polynomial, such
  that we cannot apply the inverse estimate to it. Instead, we use
  triangle inequality
  \begin{gather*}
    \norm{u-u_h}_{2;h} \le \norm{u-I_h u}_{2;h} + \norm{I_h u-u_h}_{2;h},
  \end{gather*}
  and observe, that we already have the desired estimate for the first
  term. For the second, we estimate by inverse estimate on each cell
  \begin{gather*}
    \min_{\cell\in\mesh_h} h_T \norm{I_h u-u_h}_{2;h}
    \norm{I_hu-u_h}_{1} \le c \norm{I_h u-u_h}_{1}^2,
  \end{gather*}
  and
  \begin{align*}
    \norm{I_h u-u_h}_{1}^2
    & \le \frac c\gamma a(I_h u-u_h, I_h u-u_h)
    \\& = \frac c\gamma a(I_h u-u, I_h u-u_h)
    \\& \le \frac{cM}{\gamma}\norm{I_h u-u}_{1} \norm{I_h u-u_h}_{1}
    \\& \le C h^{k} \snorm{u}_{k+1;h} \norm{I_h u-u_h}_{1},
  \end{align*}
  where we have used Galerkin orthogonality and the interpolation
  estimate. Combining the two and using the quasi-uniformity, we
  obtain the result of the theorem.
\end{proof}

\subsection{Estimates of weaker norms and linear functionals}

\begin{intro}
  Deriving optimal error estimates in the $L^2$-norm cannot be
  achieved by the same technique as used for the $H^2$-norm, as the
  following simple argument shows: using triangle inequality, we obtain
  \begin{gather*}
    \norm{u-u_h}_{L^2} \le \norm{u-I_h u}_{L^2} + \norm{I_h u-u_h}_{L^2},
  \end{gather*}
  we obtain that the first term is of order $h^{k+1}$. Thus, for an
  optimal error estimate, we require that the second term be of order
  $h^{k+1}$ as well. We need a replacement of the inverse estimate,
  which gains an order of $h$ instead of loosing it. This is Poincaré
  inequality, but it requires an almost mean-value free function. And
  since $I_h u$ is not the interpolant of $u_h$, we cannot guarantee a
  small mean value of the difference on each mesh cell.

  To the rescue comes a ``duality argument'' known as Aubin-Nitsche
  trick, which we introduce now.
\end{intro}

\begin{Definition}{dual-problem}
  Let the weak form of a boundary value problem on the domain
  $\domain$ be defined as: find $u\in V$ such that
  \begin{gather*}
    a(u,v) = f(v) \qquad\forall v\in V.
  \end{gather*}
  Then, the \define{dual problem}, also called \define{adjoint
    problem}, with right hand side $g\in V^*$ is: find $u^* \in V$ such
  that
  \begin{gather*}
    a(v,u^*) = g(v) \qquad\forall v\in V.
  \end{gather*}
\end{Definition}

\begin{Lemma}{poisson-dual}
  The adjoint problem of the Dirichlet boundary value problem for
  Poisson's equation is equal to the dual problem, that is, for
  $u\in H^1_0(\domain)$, the two statements
  \begin{xalignat*}2
    a(u,v) &= f(v) &\forall v&\in V,\\
    a(v,u) &= f(v) &\forall v&\in V,
  \end{xalignat*}
  are equivalent.
\end{Lemma}

\begin{Assumption*}{elliptic-regularity}{Elliptic regularity}
  Let $a(.,.)$ be a bounded and elliptic bilinear form on
  $H^1_0(\domain)$. We say that a boundary value problem has
  \define{elliptic regularity}, if for any $g\in L^2(\domain)$ the
  solution $u$ is in $H^2(\domain)$. In other words, there is a
  constant $c$ independent of $g$, such that
  \begin{gather}
    \norm{u}_{2;\domain} \le c \norm{g}_{0;\domain}.
  \end{gather}
\end{Assumption*}

\begin{example}
  By \slideref{Remark}{classical-convex}, a second order PDE with
  coefficients $a_{ij}\in C^{0,1}(\overline\domain)$ and
  $b_i, c\in L^\infty(\domain)$ has elliptic regularity. The same
  holds by integration by parts for ints adjoint.

  The same boundary value problem does not have elliptic regularity,
  if the domain has a nonconvex corner, since we have corner
  singularity functions which are not in $H^2(\domain)$, and we can
  construct a right hand side $g\in L^2(\domain)$, which produces such
  singularities.
\end{example}

\begin{Theorem}{fe-l2}
  Let $a(.,.)$ be a bounded and elliptic bilinear form on
  $V = H^1_0(\domain)$ and let the finite element space $V_h$ be
  defined on a shape-regular family of meshes $\{\mesh_h\}$, such that
  the interpolation estimate~\eqref{eq:fe-interpolation} holds. If
  furthermore the solution $u\in H^{k+1}(\domain)$ and the dual
  problem has \putindex{elliptic regularity}, the error of the finite
  element solution $u_h\in V_h \subset V$ admits the estimate
  \begin{gather}
    \norm{u-u_h}_{0} \le c h^{k+1} \snorm{u}_{k+1;h}.
  \end{gather}
\end{Theorem}

\begin{Corollary}{fe-functional}
  Under the assumptions of \slideref{Theorem}{fe-l2}, let $J(.)$ be a
  bounded linear functional on $L^2(\domain)$. Then,
  \begin{gather}
    \abs{J(u)-J(u_h)}_{0} \le c h^{k+1} \snorm{u}_{k+1;h}.
  \end{gather}
\end{Corollary}

\subsection{Green's function and maximum norm estimates}

\begin{intro}
  In this section, we will introduce the basic concepts needed for
  pointwise error estimation. They heavily rely on Green's function,
  which is useful for a general understanding of the solution
  structure as well.
\end{intro}

\begin{Definition}{greens-function}
  For a differential equation $Lu = f$ in $\domain$ with boundary
  conditions $u=0$ on the boundary $\d\domain$, we define
  \define{Green's function} $G(\vy,\vx)$ associated to the point
  $\vy\in\domain$ as solution to the problem
  \begin{gather}
    \begin{aligned}
      L G(\vy,\vx) &= \delta(\vx-\vy)&\qquad\forall \vx&\in\domain,\\
      G(\vy,\vx) &= 0&\qquad\forall \vx&\in\d\domain.
    \end{aligned}
  \end{gather}
\end{Definition}

\begin{Theorem}{greens-function-rd}
  Green's function for Poisson's equation on the whole space
  $\domain=\R^d$ is
  \begin{gather}
    G(\vy,\vx) =
    \begin{cases}
      - \frac1{2\pi} \log \abs{\vx-\vy} & d=2\\
      \frac1{d(d-2)\abs{B_1(0)}} \frac1{\abs{\vx-\vy}^{d-2}} & d\ge3.
    \end{cases}
  \end{gather}
\end{Theorem}

\begin{proof}
  See~\cite[Section 2.2]{Evans98}.
\end{proof}

\begin{Lemma}{greens-function-domain}
  Let $\domain\subset \R^d$ be a bounded domain. Then, Green's
  function associated to a point $\vy\in\domain$ for a linear
  differential operator $L$ on $\domain$ is obtained as the sum
  \begin{gather}
    G(\vy,\vx) = G_\infty(\vy,\vx) - G_0(\vy,\vx),
  \end{gather}
  where $G_\infty(\vy,\vx)$ is Green's function for the whole space
  and $G_0(\vy,\vx)$ solves $LG_0(\vy,\vx)=0$ with boundary values
  $G_\infty(\vy,\vx)$. If the domain is convex, then,
  $G_0(\vy,.)\in H^2(\domain)$ for any interior point $\vy$.
\end{Lemma}

\begin{Theorem}{greens-function-representation}
  Let $f\in C(\overline\domain)$ and let $\domain$ be such that the
  solution to
  \begin{gather}
    \begin{aligned}
      Lu &= f &\qquad\text{in }&\domain,\\
      u &= 0 &\text{on }&\d\domain,
    \end{aligned}
  \end{gather}
  is in $C^2(\overline\domain)$. Then,
  \begin{gather}
    u(\vx) = \int_\domain f(\vy)G(\vy,\vx)\dvy.
  \end{gather}
\end{Theorem}

\begin{proof}
  See \cite[Section 2.2]{Evans98}
\end{proof}

\begin{Theorem}{linfty-error}
  Let $a(.,.)$ be a bounded and elliptic bilinear form on
  $V = H^1_0(\domain)$ on a bounded, convex domain $\domain\subset \R^2$. Let the
  finite element space $V_h$ be defined on a quasi-uniform family of
  meshes $\{\mesh_h\}$ with local spaces $\P_1$ or $\Q_1$. If
  furthermore the solution $u\in W^{2,\infty}(\domain)$, the error of
  the finite element solution $u_h\in V_h \subset V$ admits the
  estimate
  \begin{gather}
    \norm{u-u_h}_{\infty} \le c h^{2} (1+\abs{\log h}) \snorm{u}_{2,\infty}.
  \end{gather}
\end{Theorem}

\begin{proof}
  The proof relies on solving a dual problem for the error in a single
  point. The solution is Green's function for the adjoint equation,
  which is very irregular at the point of interest. Then, complicated
  analysis is needed to generate useful approximation estimates in
  spite of the singularity. Details can be found
  in~\cite{SchatzWahlbin77,RannacherScott82}.
\end{proof}


\begin{remark}
  This estimate extends to $\domain\in\R^d$ for $d\ge 3$ and to higher
  polynomial degrees \emph{without} the logarithmic factor. The
  regularity assumptions are quite high then.
\end{remark}

%%% Local Variables: 
%%% mode: latex
%%% TeX-master: "main"
%%% End: 
