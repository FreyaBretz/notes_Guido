\subsection{Linear second order PDE}

\begin{Notation}{coordinates}
  Dimension of ``physical space'' will be denoted by $d$.  We denote
  coordinates in $\R^d$ as
  \begin{gather*}
    \vx = (x_1,\dots,x_d)^T  .
  \end{gather*}
  In the special cases $d=2,3$ we also write
  \begin{gather*}
  \vx =
  \begin{pmatrix}
    x\\y
  \end{pmatrix},
  \qquad\qquad
  \vx = \begin{pmatrix}
    x\\y\\z
  \end{pmatrix},
  \end{gather*}
  respectively.
  The Euclidean norm on $\R^d$ is denoted as
  \begin{gather*}
     \abs{\vx} = \sqrt{\sum_{i=1}^d x_i^2}.
  \end{gather*}
\end{Notation}

\begin{Notation}{partial-derivative}
  Partial derivatives of a function $u\in C^1(\R^d)$ are denoted by
  \begin{gather*}
    \frac{\d u(\vx)}{\d x_i} = \tfrac{\d}{\d x_i} u(\vx)
    = \d_{x_i} u(\vx) = \d_i u(\vx).
  \end{gather*}

  The \define{gradient} of $u \in C^1$ is the row vector
  \begin{gather*}
    \nabla u = (\d_1u,\dots,\d_du)
  \end{gather*}

  The \define{Laplacian} of a function $u\in C^2(\R^d)$ is
  \begin{gather*}
    \Delta u = \d_1^2 u + \dots + \d_d^2 u = \sum_{i=1}^d \d_i^2 u
  \end{gather*}
\end{Notation}

\begin{Notation}{elim-coord}
  When we write equations, we typically omit the independent variable
  $\vx$. Therefore,
  \begin{gather*}
    \Delta u \equiv \Delta u(\vx).
  \end{gather*}
\end{Notation}

\begin{Definition}{lin-pde-2order}
  A linear PDE of second order in divergence form for a function
  $u\in C^2(\R^d)$ is an equation of the form
  \begin{gather}
    -\sum_{i,j=1}^d \d_i \bigl(a_{ij}(\vx) \d_j u\bigr)
    + \sum_{i=1}^d \bigl(b_i(\vx) \d_i u\bigr) + c(\vx) u = f(x)
  \end{gather}
\end{Definition}

\begin{Definition}{poisson-eqn}
  An important model problem for the equations we are going to study
  is \define{Poisson's equation}
  \begin{gather}
    \label{eq:Poisson}
    -\Delta u = f.
  \end{gather}
\end{Definition}

\begin{intro}
  Already with ordinary differential equations we experience that we
  typically do not search for solutions of the equation itself, but
  that we ``anchor'' the solution by solving an initial value problem,
  fixing the solution at one point on the time axis.

  It does not make sense to speak about an initial point in
  $\R^d$. Instead, it turns out that it is appropriate to consider
  solutions on certain subsets of $\R^d$ and impose conditions at the
  boundary.
\end{intro}

\begin{Definition}{domain}
  A \define{domain} in $\R^d$ is a connected, open set of $\R^d$. We
  typically use the notation $\domain\subset\R^d$.

  The \define{boundary} of a domain $\domain$ is denoted by
  $\d\domain$. To any point $\vx\in\d\domain$, we associate the outer
  unit \define{normal vector} $\vn \equiv \vn(\vx)$.

  The symbol $\d_n u \equiv (\nabla u) \vn$ denotes the \define{normal
    derivative} of a function $u\in C^1(\overline{\domain})$ at a point
  $\vx\in\d\domain$.
\end{Definition}

\begin{Definition}{boundary-conditions}
  We distinguish three types of boundary conditions for Poisson's
  equation, namely for a point $\vx\in\d\domain$ with a given function $g$
  \begin{enumerate}
  \item Dirichlet:
    \begin{gather*}
      u(\vx) = g(\vx)
    \end{gather*}
  \item Neumann:
    \begin{gather*}
      \d_n u(\vx) = g(\vx)
    \end{gather*}
  \item Robin: for some positive function $\alpha$ on $d\domain$
    \begin{gather*}
      \d_n u(\vx) + \alpha(\vx) u(\vx) = g(\vx)
    \end{gather*}
  \end{enumerate}
  While only one of these boundary conditions can hold in a single
  point $\vx$, different boundary conditions can be active on
  different subsets of $\d\domain$. We denote such subsets as
  $\Gamma_D$, $\Gamma_N$, and $\Gamma_R$.
\end{Definition}

\begin{Definition}{dirichlet-problem-differential}
  The \define{Dirichlet problem} for \putindex{Poisson's equation} (in
  differential form) is: find
  $u\in C^2(\domain)\cap C(\overline{\domain})$, such that
  \begin{subequations}
    \begin{xalignat}2
      -\Delta u(\vx) &= f(\vx) & x&\in \domain, \\
      u(\vx) &= g(\vx) & x&\in \d\domain.
    \end{xalignat}
  \end{subequations}
  Here, the functions $f$ on $\domain$ and $g$ on $\d\domain$ are data
  of the problem.

  The Dirichlet problem is called \define{homogeneous}, if $g\equiv 0$.
\end{Definition}

\begin{Theorem*}{Dirichlet-principle}{Dirichlet principle}
  If a function $u\in C^2(\domain)\cap C(\overline{\domain})$ solves
  the \putindex{Dirichlet problem}, then it minimizes the
  \define{Dirichlet energy}
  \begin{gather}
    \label{eq:Dirichlet-energy}
    E(v) = \int_{\domain} \tfrac12 \abs{\nabla v}^2 \dvx - \int_{\domain} f v \dvx,
  \end{gather}
  among all functions $v$ from the set
  \begin{gather}
    V_g = \bigl\{ v\in C^2(\domain)\cap C(\overline{\domain})
    \big| v_{|\d\domain} = g \bigr\}.
  \end{gather}
\end{Theorem*}

\begin{proof}
    Using variation of $E$, we will show that
    \begin{gather*}
        \frac{\diffd}{\diffd \varepsilon} E(u+\varepsilon v)
        \Big|_{\varepsilon=0} = 0
    \end{gather*}
    for all $v \in V_0$ since this implies $u + \varepsilon v = g$
    on $\d\domain$. By evaluating the square we have
    \begin{gather*}
        \frac{\diffd}{\diffd \varepsilon} E(u+\varepsilon v)
          = \int_\domain \nabla u \nabla v + \varepsilon \abs{\nabla v}^2 - fv \dx
    \end{gather*}
    Since we are intersted in $E(u)$, we now consider $\varepsilon=0$. We
    get that $u$ minimises $E(u+\varepsilon v)$ at $\varepsilon = 0$ implies
    \begin{gather*}
        \int_\domain \nabla u \nabla v \dx = \int_\domain fv \dx,
          \qquad\forall v\in V_0.
    \end{gather*}
    By Green's formula
    \begin{gather*}
        \int_\domain \nabla u \nabla v \dx = \int_\domain - \Delta u v \dx
          + \int_{\d\domain} \d_n u v \ds
    \end{gather*}
    we obtain that if $u$ minimises $E(-)$, then
    \begin{gather*}
         \int_\domain \nabla u \nabla v \dx = \int_\domain fv \dx,
          \qquad\forall v\in V_0,
    \end{gather*}
    since $v \in V_0$ vanishes on $\d\domain$. In summary, we have
      proven so far that if $u$ solves Poisson's Equation, then it
        is a stationary point of $E(-)$. It remains to show that $E(u) \le E(v)$.
          Using $\int_\domain fv \dx = \int_\domain \nabla u \nabla v$ yields
    \begin{gather*}
        E(v) - E(u) = \frac 12 \int_\domain |\nabla v|^2 - 2 \nabla v \nabla u
          + |\nabla u|^2 \dx = \frac 12 |\nabla v - \nabla u|^2 \dx \ge 0.
    \end{gather*}
    To show that the minimiser exists, we show that $E(-)$ is bounded from
      below and hence the exists an infinum. Thus, there also exists a series
        $\{u^{(n)}\}_{n \in \mathbb{N}}$ converging to this infinum, i.e.
    \begin{gather*}
        \lim_{n \to \infty} E(u^{(n)}) = \inf_{v \in V_0} E(v).
    \end{gather*}
    Second, we show that $\{u^{(n)}\}_n$ is a Cauchy sequence.
    
    For the first part we use Friedrich's inequality
    \begin{gather*}
        \norm{v}_{L^2(\domain)} \le \lambda(\Omega)
          \norm{\nabla v}_{L^2(\domain)} \qquad v \in V_0.
    \end{gather*}
    The proof of this result will be given later. Using Hölder's inequality
    we obtain
    \begin{gather*}
        E(v) = \frac 12 \norm{\nabla v}^2 _{L^2(\domain} - \int_\domain fv \dx
          \ge \frac 12 \norm{\nabla v}^2 _{L^2(\domain}
            - \norm{f}_{L^2(\domain} \norm{v}_{L^2(\domain}
    \end{gather*}
    Applying Friedrich's inquality yields that the above expression is
      greater or equal than
    \begin{gather*}
        \frac 12 \norm{\nabla v}^2 _{L^2(\domain}
          - \norm{\nabla v}_{L^2(\domain} \frac 1{\lambda (\domain)}
            \norm{f}_{L^2(\domain}.
    \end{gather*}
    Finally, we apply Young's inequality $ab \le \nicefrac 12 (a^2 + b^2)$ to obtain
    \begin{gather*}
        \frac 12 \norm{\nabla v}^2 _{L^2(\domain}
          - \norm{\nabla v}_{L^2(\domain}^2 - \frac 1{2\lambda (\domain)} \norm{f}_{L^2(\domain}^2
    \end{gather*}
    which yields $E(v) \ge - \frac 1{2\lambda (\domain)^2} \norm{f}_{L^2(\domain}^2$
    as a lower bound independent of $v$. To prove the second part,
    we use the parallelogram identity $\abs{v+w}^2 + \abs{v-w}^2 = 2\abs{v}^2 + 2 \abs{w}^2$.
    Let $m, n$ be natural numbers, then
    \begin{align*}
        \snorm{u^{(n)} - u^{(m)}}^2 _1 =& 2 \snorm{u^{(n)}}^2 _1
        + 2 \snorm{u^{(m)}}^2_1- 4 \snorm{\nicefrac 12 (u^{(n)} + u^{(m)}}^2_1 \\
        =& 4 E(u^{(n)}) + 4\int f u^{(n)} \dx + 4 E(u^{(m)}) + 4\int f u^{(m)} \dx \\
        &- 8 E(\nicefrac 12 (u^{(n)} + u^{(m)}) - 8 \int \nicefrac 12 f(u^{(n)} + u^{(m)}) \\
        =& 4 E(u^{(n)}) + 4 E(u^{(m)}) - 8 E(\nicefrac 12 f(u^{(n)} + u^{(m)}))
    \end{align*}
    Taking the limit $m,n\to \infty$ yields $4 E(u^{(n)}) + 4 E(u^{(m)})
    \to 8 \inf_{v \in V_0} E(v)$. Lastly, $-E(\nicefrac 12 f(u^{(n)} + u^{(m)}))$ can
    be bounded by $inf_{v \in V_0} E(v)$. It follows that $\limsup_{m,n\to\infty}
    \snorm{u^{(n)}-u^{(m)}}^2_1 \le 0$ and consequently as desired
    \begin{gather*}
        \lim_{m,n\to\infty} \snorm{u^{(n)}-u^{(m)}}^2_1 = 0.
    \end{gather*}
\end{proof}

\begin{notes}{Dirichlet-proof}
  Dirichlet's principle proved essential for the development of a
  rigorous solution theory for Poisson's equation.  Its proof will be
  deferred to the next theorem.
\end{notes}

\subsection{Variational principle and weak formulation}
\begin{Theorem}{Dirichlet-variational-principle}
  A function $u\in V_g$ minimizes the Dirichlet energy, if and only if
  there holds
  \begin{gather}
    \int_{\domain} \nabla u\cdot\nabla v \dx
    = \int_{\domain} fv\dx, \qquad\forall v\in V_0.
  \end{gather}
  Moreover, any solution to the Dirichlet problem in
  \slideref{Definition}{dirichlet-problem-differential} solves this
  equation.
\end{Theorem}

\begin{Corollary}{Dirichlet-uniqueness}
  If a minimizer of the Dirichlet energy exists, it is necessarily unique.
\end{Corollary}

\begin{Lemma}{reduction-to-zero-bc}
  A function $u\in V_g$ minimizes the Dirichlet energy admits the
  representation $u = u_g + u_0$, where $u_g\in V_g$ is arbitrary and
  $u_0\in V_0$ solves
  \begin{gather}
    \int_{\domain} \nabla u\cdot\nabla v \dx
    = \int_{\domain} fv\dx
    - \int_{\domain} \nabla u_g\cdot\nabla v \dx,
    \qquad\forall v\in V_0.
  \end{gather}
  The function $u_0$ depends on the choice of $u_g$, but not the minimizer $u$.
\end{Lemma}

\begin{Notation}{l2}
  The inner product of $L^2(\domain)$ is denoted by
  \begin{gather*}
    \form(u,v) \equiv \form(u,v)_{\domain}
    \equiv \form(u,v)_{L^2(\domain)}
    = \int_{\domain} u v \dvx.
  \end{gather*}
  Its norm is
  \begin{gather*}
    \norm{u} \equiv \norm{u}_{\domain} \equiv \norm{u}_{L^2(\domain)}
    \equiv \norm{u}_{L^2} = \sqrt{\form(u,v)_{L^2(\domain)}}.
  \end{gather*}
\end{Notation}
\begin{Lemma*}{Friedrichs-continuous}{Friedrichs inequality}
  For any function in $v\in V_0$ there holds
  \begin{gather}
      \norm{v}_{\domain}
      \le \diam(\domain) \norm{\nabla v}_{\domain}.
  \end{gather}
\end{Lemma*}

\begin{Lemma}{h1-norm}
  The definitions
  \begin{gather}
    \begin{split}
      \abs{v}_1 &= \norm{\nabla v}_{L^2(\domain)},\\
      \norm{v}_1 &= \sqrt{\norm{v}^2_{L^2(\domain)}
        + \abs{v}^2_1},
    \end{split}
  \end{gather}
  both define a norm on $V_0$.
\end{Lemma}

\begin{Problem}{Friedrichs}
  Prove the Friedrichs inequality.
\end{Problem}

\begin{Lemma}{Dirichlet-energy-boundedness}
  The Dirichlet energy with homogeneous boundary conditions is bounded
  from below and thus has an infimum. In particular, there exists a
  \define{minimizing sequence} $\{u^n\}$ such that as $n\to\infty$,
  \begin{gather}
    E(u^n) \to \inf_{v\in V_0} E(v).
  \end{gather}
\end{Lemma}

\begin{Lemma}{minimizing-sequence}
  The minimizing sequence for the Dirichlet energy is a
  \putindex{Cauchy sequence}.
\end{Lemma}

\begin{Definition}{h10}
  The completion of $V_0$ under the norm $\norm{v}_1$ is the
  \define{Sobolev space} $H^1_0(\domain)$.
\end{Definition}

\begin{Lemma*}{Friedrichs-h1}{Friedrichs inequality}
  For any function in $v\in H^1_0$ there holds
  \begin{gather}
      \norm{v}_{\domain}
      \le \diam(\domain) \norm{\nabla v}_{\domain}.
  \end{gather}
\end{Lemma*}

\begin{proof}
  Let $v\in H^1_0(\domain)$.  We make use of the fact, that by
  definition of $H^1_0(\domain)$, there is a sequence $v_n \to v$ with
  $v_n \in V_0$. By~\slideref{Lemma}{Friedrichs-continuous},
  Friedrichs' inequality holds for $v_n$ uniformly in $n$. We conclude
  \begin{align*}
    \norm{v}_\domain &\le \norm{v - v_n}_\domain + \norm{v_n}_\domain \\
    & \le \norm{v - v_n}_\domain + \diam\domain \norm{\nabla v_n}_\domain \\
    & \le \norm{v - v_n}_\domain + \diam\domain
      \bigl(\norm{\nabla v_n - \nabla v}_\domain + \norm{\nabla v}_\domain\bigr)
  \end{align*}
  As $n\to\infty$, the norms of the differences converge to zero, such
  that the desired result holds in the limit.
\end{proof}

\begin{Definition}{weak-formulation}
  The \putindex{Dirichlet problem} for Poisson's equation in weak form
  reads: find $u\in H^1_g(\domain)$ such that
  \begin{gather}
    \int_{\domain} \nabla u\cdot\nabla v \dx
    = \int_{\domain} fv\dx, \qquad\forall v\in H^1_0(\domain).
  \end{gather}
\end{Definition}

\begin{Theorem}{weak-unique-solution-1}
  The weak formulation in \slideref{Definition}{weak-formulation} has
  a unique solution.
\end{Theorem}

\subsection{Boundary conditions in weak form}

\begin{Lemma}{neumann-weak}
  Let $u\in V=H^1(\domain)$ be a solution to the weak formulation
  \begin{gather}
    \int_{\domain} \nabla u\cdot\nabla v \dx
    = \int_{\domain} fv\dx, \qquad\forall v\in V(\domain).
  \end{gather}
  If $u\in C^2(\domain) \cap C^1(\overline{\domain})$ and $\domain$
  has $C^1$-boundary, then $u$ solves the boundary value problem
  \begin{gather}
    \begin{aligned}
      -\Delta u &= f &\qquad \text{in } &\domain\\
      \d_n u &= 0 &\text{on } &\d\domain.
    \end{aligned}
  \end{gather}
\end{Lemma}

\begin{Definition}{natural-bc}
  A boundary condition inherent in the weak formulation and not
  explicitly stated is called \define{natural boundary condition}. If
  boundary values are obtained by constraining the function space it
  is called \define{essential boundary condition}.

  We also call a boundary condition in strong form, if it is a
  constraint on the function space, and in weak form, if it is part of
  the weak formulation.
\end{Definition}

\begin{remark}
  Dirichlet and homogeneous Neumann boundary conditions are examples
  for essential and natural boundary conditions.
\end{remark}

\begin{Lemma}{mixed-bc-weak}
  The boundary value problem
  \begin{gather}
    \begin{aligned}
      -\Delta u &= f &\qquad \text{in } &\domain\\
      u &= 0 &\text{on } &\Gamma_D \subset \d\domain\\
      \d_n u + \alpha u &= g &\text{on } &\Gamma_R \subset \d\domain,
    \end{aligned}
  \end{gather}
  has the weak form: find $u\in V$ such that
  \begin{gather}
    \int_{\domain} \nabla u\cdot\nabla v \dx
    + \int_{\Gamma_R}\alpha u v \ds
    = \int_{\domain} f v\dx
    + \int_{\Gamma_R} g v \ds, \qquad\forall v\in V(\domain).    
  \end{gather}
\end{Lemma}

%%% Local Variables: 
%%% mode: latex
%%% TeX-master: "main"
%%% End: 
