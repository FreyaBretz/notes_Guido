\begin{Notation}{up-to-a-constant}
  In asymptotic estimates, we will use the notation
  \begin{gather*}
    a \lesssim b \qquad\text{and}\qquad b \gtrsim a,
  \end{gather*}
  if there is a constant c independent of the relevant quantities, such that
  \begin{gather*}
    a \le c \,b\qquad\text{and}\qquad c\,b \gtrsim a.
  \end{gather*}
  We use
  \begin{gather*}
    a \simeq b,
  \end{gather*}
  if both hold. The latter is more precise than $a = \mathcal O(b)$,
  since it implies the same asymptotic order.
\end{Notation}

\begin{example}
  The interpolation estimate~\eqref{eq:fe-interpolation} could be written as
  \begin{align*}
    \norm{u-I_h u}_{m;h} \lesssim h^{k+1-m} \snorm{u}_{k+1;h},\\
    \intertext{instead of}
    \norm{u-I_h u}_{m;h} \le c h^{k+1-m} \snorm{u}_{k+1;h}.
  \end{align*}
  The statement of the equivalence of two norms becomes
  \begin{gather*}
    \norm{\cdot}_X \simeq \norm{\cdot}_Y,
  \end{gather*}
  while 
  \begin{gather*}
    \norm{\cdot}_X \lesssim \norm{\cdot}_Y
  \end{gather*}
  states that the $Y$-norm is not weaker than the $X$-norm.
\end{example}

\begin{Definition}{mass-stiffness}
  Let $\mesh_h$ be a mesh on the domain $\domain$ and $V_h$ a finite
  element space with basis $\{\phi_i\}$. Then, the matrix associated
  with the $L^2$-inner product,
  \begin{gather*}
    m_{ij} = \int_\Omega \phi_j(x) \phi_i(x) \dx,
  \end{gather*}
  is called \define{mass matrix}. The matrix associated with the
  bilinear form of the equation is called \define{stiffness matrix},
  \begin{gather*}
    a_{ij} = a(\phi_j,\phi_i).
  \end{gather*}
\end{Definition}

\begin{Lemma}{condition-number-mass-matrix}
  Let $\{\phi_i\}$ be the basis of a finite element shape function
  space on a quasi-uniform family of meshes $\{\mesh_h\}$. Let
  $\matm_h$ be the mass matrix. Then,
  \begin{gather*}
    \Lambda(\matm) \simeq h^{d} \simeq  \lambda(\matm)
  \end{gather*}
  Therefore, the condition number is (asymptotically in $h$)
  \begin{gather*}
    \kappa(\matm) \simeq 1.
  \end{gather*}
\end{Lemma}

\begin{proof}
  % It is easy to verify, that $m_{ii}> 0$, and that there are not more
  % entries in each row as edges of the triangulation meet in one vertex
  % are different from zero. Furthermore, that the size of those entries
  % is of order $h^d$, where $d$ is the space dimension. From these two
  % facts we immediately obtain
  % \begin{gather*}
  %   \Lambda(\matm) = \mathcal O(h^{d}).
  % \end{gather*}
  % The argument for $\lambda(\matm)$ is more subtle. 
  For any mesh cell
  $\cell\in\mesh_h$, let $\vecx_T$ be the entries of the vector $\vecx$ which
  belong to node values of the cell $\cell$. Let $\matm_\cell$ be the cell
  mass matrix obtained by restricting the $L^2$-inner product to
  $\cell$. Then,
  \begin{gather*}
    \vecx^T \matm_h \vecx
    = \sum_{T\in\T_h} \vecx^T_T \matm_T \vecx_T
    \begin{cases}
    \ge \min\limits_{T\in\T_h} \frac{\vecx^T_T \matm_T \vecx_T}{\abs{\vecx_T}} \sum_{T\in\T_h} \abs{\vecx_T}^2 \ge \lambda(\matm_T) \abs{\vecx}^2,\\
    \le \max\limits_{T\in\T_h} \frac{\vecx^T_T \matm_T \vecx_T}{\abs{\vecx_T}} \sum_{T\in\T_h} \abs{\vecx_T}^2 \le c \Lambda(\matm_T) \abs{\vecx}^2,
    \end{cases}
  \end{gather*}
  where the constant in the upper bound is due to degrees of freedom
  shared by different elements. Dividing by $\abs{\vecx}^2$, we
  obtain
  \begin{gather}
    \lambda(\matm_\cell) \le \frac{\vecx^T \matm_h \vecx}{\abs{\vecx}^2} \le c \Lambda(\matm_\cell).
  \end{gather}
  In order to estimate the eigenvalues of $\matm_T$, we note that for
  a unisolvent element, the norms $\abs{\vecx_T}$ and $\norm{u}_{0,T}$ are
  equivalent on the reference cell, and the $L^2$-norm scales with
  $h^d$ when transforming to the real cell $T$. Thus, we have
  $\lambda(\matm_h) = \mathcal O(h^{d}) = \Lambda(\matm_h)$.
\end{proof}

\begin{Corollary}{refined-condition-number}
  Let $\mesh_h$ be a shape-regular mesh with cell sizes ranging
  between the minimum $h$ and the maximum $H$. Then, we have
  \begin{align*}
    \Lambda(\matm) &\simeq H^{d} \\
    \lambda(\matm) &\simeq h^{d} \\
    \kappa(\matm) &\simeq \left(\frac Hh\right)^{d}
  \end{align*}
\end{Corollary}

\begin{remark}
  We have only computed the dependence of the condition number on the
  mesh size $h$, not on the polynomial degree of the shape function
  space. Indeed, the used equivalence between the $L^2$-norm of a
  polynomial on the reference cell and the Euclidean norm of its
  coefficient vector depends not only on the degree of the shape
  function space, but on its basis.
\end{remark}

\begin{Theorem}{condition-number-stiffness-matrix}
  Let $\{\phi_i\}$ be the nodal basis of a finite element shape
  function space on a quasi-uniform family of meshes
  $\{\mesh_h\}$. Let $\mata_h$ be the stiffness matrix of the
  Laplacian with Dirichlet boundary values.  Then,
  \begin{gather*}
    \Lambda(\mata) \simeq h^{d-2}, \qquad \lambda(\mata)  \simeq h^{d}
  \end{gather*}
  Therefore, the condition number is
  \begin{gather*}
    \kappa(\matm) \simeq h^{-2}.
  \end{gather*}
\end{Theorem}

\begin{proof}
  First, we investigate the largest eigenvalue. By the scaling
  argument, there holds $a_{ij} \simeq h^{d-2}$. Since by the
  Gershgorin theorem all eigenvalues are bounded by the sum of the
  elements in a row, and this number is bounded uniformly on shape
  regular meshes, we have $\Lambda(\mata_h) \lesssim h^{d-2}$. Now, we
  choose a particular example, namely a vector $\vecx$ with a single one and
  all other entries zero. The corresponding finite element function
  $u_h$ has maximum one and support of radius $h$. Therefore,
  \begin{gather*}
    \vecx^T\mata_h\vecx=
    \int_\domain \abs{\nabla u_h}^2 \dvx
    \simeq h^{d-2}.
  \end{gather*}
  Thus, $\Lambda(\mata_h) \gtrsim h^{d-2}$. In order to estimate the
  smallest eigenvalue, we use the Rayleigh quotient and the finite
  element function $u_h$ corresponding to the coefficient vector
  $\vecx$ to obtain
  \begin{multline}
    \lambda(\mata_h)
    = \min_{\vecx\in\R^n}\frac{\vecx^T\mata_h\vecx}{\vecx^T\vecx}
    = \min_{\vecx\in\R^n}\frac{\vecx^T\mata_h\vecx}{\form(u_h,u_h)_{L^2}}
    \frac{\form(u_h,u_h)_L^2}{\vecx^T\vecx}
    \\
    = \min_{\vecx\in\R^n}\left(\frac{a(u_h,u_h)}{\form(u_h,u_h)}
    \frac{\vecx^T\matm_h\vecx}{\vecx^T\vecx}\right)
    \ge \min_{u_h\in V_h}\frac{a(u_h,u_h)}{\form(u_h,u_h)_L^2}
    \min_{\vecx\in\R^n}\frac{\vecx^T\matm_h\vecx}{\vecx^T\vecx}.
  \end{multline}
  Thus, the smallest eigenvalue of $\mata_h$ is bounded by the product
  of the smallest eigenvalue of the mass matrix and a value $\mu$ given by
  \begin{gather}
    \min_{u_h\in V_h}\frac{a(u_h,u_h)}{\form(u_h,u_h)_L^2} \ge
    \min_{u\in V}\frac{a(u,u)}{\form(u,u)_L^2} =: \mu.
  \end{gather}
  Here, we need a result from the spectral analysis of the
  Laplacian. When we define the variational eigenvalue problem
  \begin{gather}
    a(u,v) = \lambda \form(u,v)\qquad\forall v\in V = H^1_0(\domain),
  \end{gather}
  then, there is a positive minimal solution $\mu$, the first
  eigenvalue of the Laplacian, and an associated eigenfunction $u$,
  such that the equation holds for all $v$. Therefore, the eigenvalue
  is bounded with $\lambda(\mata_h) \gtrsim h^{-d}$. Furthermore, it
  can be shown that this function is smooth and does not
  oscillate. This actually follows from the regularity result in
  \slideref{Theorem}{gt-8-13}. Therefore, the solution $u_h\in V_h$ to
  \begin{gather}
    a(u_h, v_h) = a(u,v_h) \qquad\forall v_h\in V_h,
  \end{gather}
  converges to $u$ in $H^1$ and $L^2$. With our standard estimates, we obtain
  \begin{gather}
    \form(u, u_h)
    = \form(u_h,u_h) + \form(u-u_h,u_h)
    = \form(u_h,u_h) + r_h(u),
  \end{gather}
  where $r_h(u)\to 0$ as $h\to 0$. Hence,
  \begin{gather}
    a(u_h,u_h) = a(u,u_h) = \mu \form(u,u_h) \to \mu \form(u_h,u_h).
  \end{gather}
  We conclude that the lower bound for the eigenvalue is sharp and the
  theorem holds.
\end{proof}

%%% Local Variables:
%%% mode: latex
%%% TeX-master: "main"
%%% End:
