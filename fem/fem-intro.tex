\begin{Definition}{facets}
  Let $\cell\subset \R^d$ be a polyhedron. We call the lower
  dimensional polyhedra constituting its boundary \define{facet}s. A
  facet of dimension zero is called \define{vertex}, of dimension one
  \define{edge}, and a facet of codimension one is called a
  \define{face}.
\end{Definition}

\begin{Definition}{mesh}
  A \define{mesh} $\mesh$ is a nonoverlapping subdivision of the
  domain $\domain$ into polyhedral \define{cell}s denoted by $\cell$,
  for instance simplices, quadrilaterals, or hexahedra. The
  faces of a cell are denoted by $\face$, the
  vertices by $\vertex$. Cells are typically considered open sets.

  A mesh $\mesh$ is called regular, if each face
  $\face \subset \d\cell$ of the cell $\cell\in\mesh$ is either a
  face of another cell $\cell\prime$, that is,
  $\overline{\face} = \overline{\cell} \cap \overline{\cell\prime}$,
  or a subset of $\d\domain$.
\end{Definition}

\begin{remark}
  For this introduction, we will assume that indeed $\domain$ is the
  union of mesh cells, which means, that its boundary consists of a
  finite union of planar faces. The more general case of a mesh
  approximating the domain will be deferred to later discussion.
\end{remark}

\begin{Definition}{finite-element}
  With a mesh cell $\cell$, we associate a finite dimensional
  \define{shape function} space $\shapespace(\cell)$ of dimension
  $n_\cell$. The term \define{node functional} denotes linear
  functionals on this space.

  A set of node functionals $\{\nodal_\cell^i\}_{i=1,\dots,n_\cell}$ is called
  \define{unisolvent} on $\shapespace(\cell)$ if for any vector
  $\vu = (u_1,\dots,u_{n_\cell})^T$ there exists a unique
  $u\in \shapespace(\cell)$ such that
  \begin{gather}
    \nodal_\cell^i(u) = \vu_i,\quad i=1,\dots,n.
  \end{gather}

  A \define{finite element} is a set of shape function spaces
  $\shapespace(\cell)$ for all $\cell\in\mesh$ together with
  unisolvent set of node functionals.
\end{Definition}

\begin{Notation}{dofs}
  If the node functionals $\nodal^i$ are unisolvent on
  $\shapespace(\cell)$, then, there is a basis $\{p_k\}$ of $\shapespace(\cell)$
  such that
  \begin{gather}
    \nodal^i(p_k) = \delta_{ik}.
  \end{gather}
  We refer to $\{p_k\}$ as \define{shape function basis} and use the
  term \define{degrees of freedom} for both the node functionals and
  the basis functions.
\end{Notation}

\begin{Definition}{node-topology}
  Node functionals can be associated with the cell $\cell$ or with one
  of its lower dimensional boundary facets. We call this association
  the \define{topology} of the finite element.
\end{Definition}

\begin{Definition}{fe-space}
  The \define{finite element space} on the mesh $\mesh$, denoted by
  $V_\mesh$ is a subset of the concatenation of all shape function
  spaces,
  \begin{gather}
    V_\mesh \subset \bigl\{ f\in L^2(\domain) \big|
    f_{\cell} \in \shapespace(\cell) \bigr\}.
  \end{gather}
  The \define{degrees of freedom} of $V_\mesh$ are the union of all
  node functionals, where we identify node functionals associated to
  boundary facets among all cells sharing this facet. The resulting
  dimension is
  \begin{gather}
    n = \dim V_\mesh \le \sum n_\cell.
  \end{gather}
\end{Definition}

\begin{Notation}{global-local}
  When we enumerate the degrees of freedom of $V_\mesh$, we obtain a
  global numbering of degrees of freedom $\nodal^i$ with
  $i=1,\dots,n$. For each mesh cell, we have a local numbering
  $\nodal_\cell^j$ with $j=1,\dots,n$. By construction of the finite
  element space, there is a unique $i$, such that
  $\nodal_\cell^j(f) = \nodal^i(f)$ for all cells $\cell$ and local
  indices $j$. The converse is not true due to the identification
  process.
\end{Notation}

\begin{Definition}{local-global}
  We refer to the mapping between $\nodal^i$ and $\nodal_\cell^j$ as
  the mapping between global and local indices
  \begin{gather}
    \iota: (\cell, j) \mapsto i.
  \end{gather}
  It induces a
  ``natural'' basis $\{v_i\}$ of $V_\mesh$ by
  \begin{gather}
    v_{i|\cell} = p_{\cell,j},
  \end{gather}
  where $\{p_{\cell,j}\}$ is the shape function basis on $\cell$. For
  each $\nodal^i$, we define $\mesh(\nodal^i)$ as the set of cells
  $\cell$ sharing the node functional $\nodal^i$, and
  \begin{gather}
    \domain\left(\nodal^i\right) = \bigcup_{\cell\in \mesh(\nodal^i)} \cell.
  \end{gather}
\end{Definition}

\begin{Lemma}{fe-support}
  The support of the basis function $v_i\in V_\mesh$ is
  \begin{gather*}
    \operatorname{supp}(v_i) \subset \domain\left(\nodal^i\right).
  \end{gather*}
\end{Lemma}

\begin{Lemma}{mesh-continuity}
  Let $\mesh$ be a subdivision of $\domain$, and let $u$ be a function
  on $\domain$, such that $u_{|\cell} \in C^1(\cell)$. Then,
  \begin{gather}
    u\in H^1(\domain)
    \quad \Longleftrightarrow\quad
    u\in C(\overline\domain).
  \end{gather}
\end{Lemma}

\begin{Lemma}{nodal-continuity}
  We have $V_\mesh\subset C(\overline{\domain})$ if and only if for
  every facet $F$ of dimension $d_F < d$ there holds that
  \begin{enumerate}
  \item the traces of the spaces $\shapespace(\cell)$ on $F$ coincide
    for all cells $\cell$ having $F$ as a facet,
  \item The node functionals associated to the facet are unisolvent on
    this trace space.
  \end{enumerate}
\end{Lemma}

%%%%%%%%%%%%%%%%%%%%%%%%%%%%%%%%%%%%%%%%%%%%%%%%%%%%%%%%%%%%%%%%%%%%%%
%%%%%%%%%%%%%%%%%%%%%%%%%%%%%%%%%%%%%%%%%%%%%%%%%%%%%%%%%%%%%%%%%%%%%%
\subsection{Shape function spaces on simplices}
%%%%%%%%%%%%%%%%%%%%%%%%%%%%%%%%%%%%%%%%%%%%%%%%%%%%%%%%%%%%%%%%%%%%%%
%%%%%%%%%%%%%%%%%%%%%%%%%%%%%%%%%%%%%%%%%%%%%%%%%%%%%%%%%%%%%%%%%%%%%%

\begin{Definition}{barycentric-coordinates}
  A simplex $\cell\in \R^d$ with vertices $\vertex_0,\dots,\vertex_d$
  is described by a set of $d+1$ \define{barycentric coordinates}
  $\vlambda = (\lambda_0,\dots,\lambda_d)^T$ such that
  \begin{xalignat}2
    0\le\lambda_i &\le 1& i&=0,\dots,d;\\
    \lambda_i(\vertex_j) &= \delta_{ij}& i,j&=0,\dots,d\\
    \sum \lambda_i(\vx) &= 1,
  \end{xalignat}
  and there holds
  \begin{gather}
    T = \Bigl\{x\in\R^d \Big| x = \sum \vertex_k\lambda_k \Bigr\}.
  \end{gather}
\end{Definition}

\begin{Lemma}{barycentric-affine}
  There is a matrix $B_T\in \R^{d+1\times d}$ and a vector
  $b_T\in\R^{d+1}$, such that
  \begin{gather}
    \vlambda = B_T\vx + b_T.
  \end{gather}
\end{Lemma}

\begin{Corollary}{barycentric-interpolation}
  The barycentric coordinates $\lambda_0,\dots,\lambda_d$ are the
  linear Lagrange interpolating functions for the points
  $\vertex_0,\dots,\vertex_d$. In particular, $\lambda_k \equiv 0$ on
  the facet not containing $\vertex_k$.
\end{Corollary}

\begin{example}
  We can use barycentric coordinates to define interpolating polynomials on
  simplicial meshes easily, as in
  Table~\ref{tab:barycentric-shapes}.
  \begin{table}[tp]
    \centering
    \begin{tabular}{|c|l|}
      \hline Degrees of freedom
      & Shape functions \\\hline
      \adjustbox{valign=center,margin=3pt}{\includegraphics[width=2cm]{mixed/fig/p1-p.tikz}}
      &
        {\begin{minipage}[b]{6cm}
          \begin{gather*}
            \phi_i = \lambda_i,
            \quad i=0,1,2
          \end{gather*}
        \end{minipage}}
      \\\hline
      \adjustbox{valign=center,margin=3pt}{\includegraphics[width=2cm]{mixed/fig/p2-p.tikz}}
      &
        {\begin{minipage}[b]{6cm}
          \begin{xalignat*}2
            \phi_{ii} &= 2\lambda_i^2 - \lambda_i,
            &i&=0,1,2\\
            \phi_{ij} &= 4\lambda_i\lambda_j
            &j&\neq i
          \end{xalignat*}
        \end{minipage}}
        \\\hline
      \adjustbox{valign=center,margin=3pt}{\includegraphics[width=2cm]{mixed/fig/p3-p.tikz}}
      &
        {\begin{minipage}[b]{6cm}
          \begin{xalignat*}2
          \phi_{iii} &= \tfrac12 \lambda_i(3\lambda_i-1)(3\lambda_i-2)
          &i&=0,1,2\\
          \phi_{ij} &= \tfrac92\lambda_i\lambda_j(3\lambda_j-1)
          &j&\neq i\\
          \phi_0 &= 27\lambda_0\lambda_1\lambda_2
        \end{xalignat*}
        \end{minipage}}
        \\\hline
    \end{tabular}
    \caption{Degrees of freedom and shape functions of simplicial elements
      in terms of barycentric coordinates}
    \label{tab:barycentric-shapes}
  \end{table}
\end{example}

\begin{remark}
  The functions $\lambda_i(x)$ are the shape functions of the linear
  $P_1$ element on $T$. They allow us to define basis functions on the
  cell $T$ without use of a reference element $\widehat T$.

  Note that $\lambda_i\equiv 0$ on the face opposite to the
  vertex $x_i$.
\end{remark}

%%%%%%%%%%%%%%%%%%%%%%%%%%%%%%%%%%%%%%%%%%%%%%%%%%%%%%%%%%%%%%%%%%%%%%
%%%%%%%%%%%%%%%%%%%%%%%%%%%%%%%%%%%%%%%%%%%%%%%%%%%%%%%%%%%%%%%%%%%%%%
\subsection{The Galerkin equations and Céa's lemma}
%%%%%%%%%%%%%%%%%%%%%%%%%%%%%%%%%%%%%%%%%%%%%%%%%%%%%%%%%%%%%%%%%%%%%%
%%%%%%%%%%%%%%%%%%%%%%%%%%%%%%%%%%%%%%%%%%%%%%%%%%%%%%%%%%%%%%%%%%%%%%

\begin{Definition*}{galerkin-approximation}{Galerkin approximation}
  Let $u\in V$ be determined by the weak formulation
  \begin{gather*}
    a(u,v) = f(v) \qquad\forall v\in V,
  \end{gather*}
  where $V$ is a suitable function space including boundary
  conditions. The \define{Galerkin approximation}, also called
  \define{conforming approximation} of this problem reads as follows:
  choose a subspace $V_n\subset V$ of dimension $n$ and find
  $u_n\in V_n$, such that
  \begin{gather*}
    a(u_n,v_n) = f(v_n) \qquad\forall v_n\in V_n.
  \end{gather*}
  We will refer to this equation as the \define{discrete problem}.
\end{Definition*}

\begin{Corollary*}{galerkin-equations}{Galerkin equations}
  After choosing a basis $\{v_i\}$ for $V_n$, the Galerkin equations are
  equivalent to a linear system
  \begin{gather}
    \mata \vu = \vf,
  \end{gather}
  with $\mata\in\R^{n\times n}$ and $\vf\in \R^n$ defined by
  \begin{gather}
    a_{ij} = a(v_j, v_i), \qquad f_i = f(v_i).
  \end{gather}
\end{Corollary*}

\begin{Lemma}{discrete-lax-milgram}
  If the lemma of Lax-Milgram holds for $a(.,.)$ on $V$, it holds on
  $V_n\subset V$. In particular, solvability of the Galerkin equations
  is implied.
\end{Lemma}

\begin{Lemma*}{cea}{Céa}
  Let $a(.,.)$ be a bounded and elliptic bilinear form on the Hilbert
  space $V$. Let $u \in V$ and $u_n\in V_n$ be the solution to the
  weak formulation and its Galerkin approximation, respectively. Then,
  there holds
  \begin{gather}
    \norm{u-u_h}_V \le \frac{M}{\alpha}
    \inf_{v_n\in V_n}\norm{u-v_h}_V.
  \end{gather}
\end{Lemma*}

\begin{Lemma}{fe-matrix}
  For a finite element discretization of Poisson's equation with the
  space $V_\mesh$, the Galerkin equations can be computed using the
  following formulas:
  \begin{alignat*}3
    a_{ij} &= \int\limits_\domain \nabla v_j \cdot \nabla v_i \dx
    &&= \int\limits_{\domain(\nodal^i)} \nabla v_j \cdot \nabla v_i \dx
    &&= \sum_{\cell\in\mesh(\nodal^i)}\int\limits_\cell \nabla v_j \cdot \nabla v_i \dx\\
    f_{i} &= \int\limits_\domain f v_i \dx
    &&= \int\limits_{\domain(\nodal^i)} f v_i \dx
    &&= \sum_{\cell\in\mesh(\nodal^i)}\int\limits_\cell f v_i \dx
  \end{alignat*}
\end{Lemma}

\begin{Algorithm*}{matrix-assembling}{Assembling the matrix}
  \begin{enumerate}
  \item Start with a matrix $\mata = 0 \in \R^{n\times n}$
  \item Loop over all cells $\cell\in\mesh$
  \item On each cell $\cell$, compute a cell matrix
    $\mata_\cell \in \R^{n_\cell\times n_\cell}$ by integrating
    \begin{gather}
      a_{\cell,ij} = \int_\cell \nabla p_{\cell,j}\cdot\nabla p_{\cell,i}\,dx,
    \end{gather}
    where $\{p_{\cell,i}\}$ is the shape function basis.
  \item Assemble the cell matrices into the global matrix by
    \begin{gather}
      a_{\iota(i),\iota(j)} = a_{\iota(i),\iota(j)} + a_{\cell,ij}
      \qquad i,j = 1,\dots,n_\cell.
    \end{gather}
  \end{enumerate}
\end{Algorithm*}

%%%%%%%%%%%%%%%%%%%%%%%%%%%%%%%%%%%%%%%%%%%%%%%%%%%%%%%%%%%%%%%%%%%%%%
%%%%%%%%%%%%%%%%%%%%%%%%%%%%%%%%%%%%%%%%%%%%%%%%%%%%%%%%%%%%%%%%%%%%%%
\subsection{Mapped finite elements}
%%%%%%%%%%%%%%%%%%%%%%%%%%%%%%%%%%%%%%%%%%%%%%%%%%%%%%%%%%%%%%%%%%%%%%
%%%%%%%%%%%%%%%%%%%%%%%%%%%%%%%%%%%%%%%%%%%%%%%%%%%%%%%%%%%%%%%%%%%%%%

\begin{Definition}{mapped-mesh}
  A mapped mesh $\mesh$ is a set of cells $\cell$, which are defined
  by a single \define{reference cell} $\refcell$ and individual
  smooth mappings
  \begin{gather}
    \begin{split}
      \Phi_\cell \colon \refcell &\to \R^d\\
      \Phi_\cell(\refcell) &= \cell.
    \end{split}
  \end{gather}
  The definition extends to small sets of reference cells, for
  instance for triangles and quadrilaterals.
\end{Definition}

\begin{Example}{mapping-linear}
  Let the reference triangle $\refcell$ be defined by
  \begin{gather}
    \refcell = \left\{
      \begin{pmatrix}
        \refx\\\refy
      \end{pmatrix}
      \middle|
      \refx,\refy >0, \refx+\refy < 1
    \right\}.
  \end{gather}
  Then, every cell $\cell$ spanned by the vertices $\vertex_0$,
  $\vertex_1$, and $\vertex_2$ is obtained by mapping $\refcell$ by
  the \putindex{affine mapping}
  \begin{gather}
    \Phi_\cell(\refvx) =
    \begin{pmatrix}
      X_1-X_0 & X_2 - X_0 \\ Y_1-Y_0 & Y_2 - Y_0
    \end{pmatrix}
    \begin{pmatrix}
      \refx \\ \refy
    \end{pmatrix}
    +
    \begin{pmatrix}
      X_0 \\ Y_0
    \end{pmatrix} =: \matb_\cell \refvx + \vb_\cell
  \end{gather}
\end{Example}

\begin{Example}{mapping-bilinear}
  The reference cell for a quadrilateral is the reference square
  $\refcell = (0,1)^2$. Every quadrilateral $\cell$ spanned by the
  vertices $\vertex_0$ to $\vertex_3$ is then obtained by the
  \putindex{bilinear mapping}
  \begin{gather}
    \Phi_\cell(\refvx)
    = \vertex_0 (1-\refx)(1-\refy)
    + \vertex_1 \refx(1-\refy)
    + \vertex_2 (1-\refx)\refy
    + \vertex_3 \refx\refy
  \end{gather}
\end{Example}

\begin{Definition}{mapped-fe}
  Mapped shape functions $\{p_i\}$ on a mesh cell $\cell$ are defined by a
  set of shape functions $\{\refp_i\}$ on the reference cell
  $\refcell$ through \define{pull-back}
  \begin{gather}
    \begin{split}
      p_i(\vx) &= \refp_i\left(\Phi^{-1}(\vx)\right) = \refp_i(\refvx),\\
      \nabla p_i(\vx) &= \nabla\Phi^{-T}(\refvx)\refgrad\refp_i(\refvx)
    \end{split}
  \end{gather}
\end{Definition}

\begin{Lemma}{mapped-norms-affine}
  Let $\refcell$ be the reference triangle and let $\cell$ be a
  triangular mesh cell with mapping
  $\vx = \Phi_\cell(\refvx) = \matb \refvx + \vb$. Let there hold
  $u(\vx) = \refu(\refvx)$. Then, $u\in H^k(\cell)$ if and only if
  $\refu\in H^k(\refcell)$ and we have with some constant $c$ the
  estimates
  \begin{gather}
    \begin{split}
      \snorm{\refu}_{k,\refcell}
      &\le c \norm{\matb}^k (\det \matb)^{-\nicefrac12}
      \snorm{u}_{k,\cell},\\
      \snorm{u}_{k,\cell}
      &\le c \norm{\matb^{-1}}^k (\det \matb)^{\nicefrac12}
      \snorm{\refu}_{k,\refcell}.
    \end{split}
  \end{gather}
\end{Lemma}

\begin{Lemma}{shape-regular-transformation}
  For a cell $\cell$, let $R$ be the radius of the circumscribed
  circle and $\rho$ the radius of the inscribed circle. Then,
  \begin{gather}
    \norm{\matb} \le c R, \qquad \norm{\matb^{-1}} \le c \rho^{-1}.
  \end{gather}
\end{Lemma}

\begin{Assumption}{mapping-decomposition}
  For more general mappings $\Phi\colon \refcell\to \cell$, we
  make the assumption, that they can be decomposed into three factors,
  \begin{gather}
    \Phi = \Phi_W \circ \Phi_S \circ \Phi_O,
  \end{gather}
  where $\Phi_O$ is a combination of translation and rotation,
  $\Phi_S$ is a scaling with a characteristic length $h_T$, and
  $\Phi_W$ is a warping function not changing the characteristic length.
\end{Assumption}

\begin{Lemma}{scaling-1}
  Assume there are constants
  $0 < M_\cell, m_\cell, d_\cell, D_\cell$, such that
  \begin{gather}
    \begin{split}
      \norm{\nabla\Phi_W(\refvx)} \le M_\cell,
      \\
      \norm{\nabla\Phi_W^{-1}(\refvx)} \le m_\cell^{-1} ,
      \\
      d^2_\cell \le \det \nabla\Phi_W(\refx)) \le D^2_\cell.      
    \end{split}
  \end{gather}
  for all $\refvx\in\refcell$. Then, for $k=0,1$
  \begin{gather}
    \begin{split}
      \snorm{\refu}_{k,\refcell}
      &\le \frac{M_\cell}{d_\cell}  h_\cell^{k-\nicefrac12}
      \snorm{u}_{k,\cell},\\
      \snorm{u}_{k,\cell}
      &\le \frac{D_\cell}{m_\cell} h_\cell^{\nicefrac12-k}
      \snorm{\refu}_{k,\refcell}.
    \end{split}
  \end{gather}
  This extend to higher derivative under assumptions on higher
  derivatives of $\Phi_\cell$.
\end{Lemma}

%%%%%%%%%%%%%%%%%%%%%%%%%%%%%%%%%%%%%%%%%%%%%%%%%%%%%%%%%%%%%%%%%%%%%%
%%%%%%%%%%%%%%%%%%%%%%%%%%%%%%%%%%%%%%%%%%%%%%%%%%%%%%%%%%%%%%%%%%%%%%
\subsection{Shape functions on tensor product cells}
%%%%%%%%%%%%%%%%%%%%%%%%%%%%%%%%%%%%%%%%%%%%%%%%%%%%%%%%%%%%%%%%%%%%%%
%%%%%%%%%%%%%%%%%%%%%%%%%%%%%%%%%%%%%%%%%%%%%%%%%%%%%%%%%%%%%%%%%%%%%%

%%% Local Variables: 
%%% mode: latex
%%% TeX-master: "main"
%%% End: 
