\begin{Definition}{inner-product}
  Let $V$ be a vector space over $\R$. An \define{inner product} on $V$ is a mapping
  $\scal(.,.): V\times V \to \R$ with the properties
  \begin{xalignat}2
    \label{eq:inner-product:1}
    \scal(\alpha x+y,z) &= \alpha \scal(x,z) + \scal(y,z)
    && \forall x,y,z \in V; \alpha \in \mathbb K\\
    \label{eq:inner-product:2}
    \scal(x,y) &= \scal(y,x) && \forall x,y \in V \\
    \label{eq:inner-product:3}
    \scal(x,x) & \ge 0 \quad\forall x\in V &&\text{and} \\
    \label{eq:inner-product:4}
    \scal(x,x) & =0 \Leftrightarrow x=0,
  \end{xalignat}
  usually referred to as (bi-)linearity, symmetry, and
  positive definiteness. We note that linearity in the second argument follows
  immediately by symmetry.
\end{Definition}

\begin{Theorem*}{bcs-inequality}{Bunyakovsky-Cauchy-Schwarz inequality}
  For every \putindex{inner product} there holds the inequality
  \begin{gather}
    \label{eq:hilbert:6}
    \scal(v,w) \le \sqrt{\scal(v,v)} \sqrt{\scal(w,w)}.
  \end{gather}
  Equality holds if and only if $v$ and $w$ are collinear.
\end{Theorem*}

\begin{proof}
  The case $w = 0$ is trivial. Without loss of generality we can therefore
  assume that $w \not = 0$. Define $\lambda \in \R$ as $\lambda =
  \frac{\scal(v,w)}{\scal(w,w)}$. By (\ref{eq:inner-product:3}) we have
  \begin{gather*}
  0 \le \scal(v - \lambda w, v - \lambda w) \label{eq:bcs:1}
  \end{gather*}
  and by ~(\ref{eq:inner-product:2}) the right-hand side extends to
  \begin{gather*}
  \scal(v, v) - \scal(v, \lambda w) - \scal(\lambda w, v) +
    \scal(\lambda w, \lambda w) \\
  = \scal(v, v) - \overline{\lambda} \scal(v, w) - \overline{\lambda} \scal(v, w)
    + \lambda \overline{\lambda}\scal(w, w).
  \end{gather*}
  Evaluating $\lambda$ yields the inequality
  \begin{gather*}
  0 \le \scal(v, v) - \frac{\overline{\scal(v, w)} \scal(v, w)}{\scal(w, w)}
    - \frac{\overline{\scal(v, w)} \scal(v, w)}{\scal(w, w)}
    + \frac{\overline{\scal(v, w)} \scal(v, w)\scal(w, w)}{\scal(w, w)^2}.
  \end{gather*}
  The result follows from multiplication with $\scal(w, w)$ and arranging
  the summands.
  
  For the second part let $v, \, w$ be colinear, i.e. there is a $\lambda \in
  \mathbb K$ such that $v = \lambda w$. Then deducing the equality is trivial.
  Now let equality hold for (\ref{eq:hilbert:6}). We immediately get that the equality
  must also hold for
  \begin{gather*}
  0 = \scal(v - \lambda w, v - \lambda w).
  \end{gather*}
  However, by ~(\ref{eq:inner-product:4}) this implies
  \begin{gather*}
  0 = v - \lambda w.
  \end{gather*}
  Thus, $v$ and $w$ are colinear.
\end{proof}

\begin{Lemma}{inner-product-norm}
  Every inner product defines a norm by
  \begin{gather}
    \label{eq:hilbert:7}
    \norm{v} = \sqrt{\scal(v,v)}.
  \end{gather}
\end{Lemma}

\begin{proof}
  Definiteness and homogeneity follow from the properties of the inner
  product. It remains to show the triangle inequality
  \begin{gather*}
    \norm{u+v} \le \norm{u}+\norm{v}.
  \end{gather*}
  Squaring the left hand side yields with the
  Bunyakovsky-Cauchy-Schwarz inequality
  \begin{multline*}
    \norm{u+v}^2
    = \scal(u+v,u+v)
    = \norm{u}^2 + 2 \scal(u,v) + \norm{v}^2
    \\
    \le \norm{u}^2 + 2 \norm u \norm v + \norm{v}^2
    = \bigl(\norm u + \norm v\bigr)^2.
  \end{multline*}
\end{proof}

\begin{Definition}{complete}
  A space $V$ with is \define{complete} with respect to a norm, if all
  \putindex{Cauchy sequence}s with elements in $V$ have their limit in
  $V$. A subspace $W\subset V$ is \define{closed} if it is complete in
  the topology of $V$.

  The \define{completion} of a space $V$ with respect to a norm
  consists of the space $V$ and the limits of all Cauchy sequences in
  $V$. We denote the completion of a space $V$ by
  \begin{gather}
    \label{eq:hilbert:8}
    \overline{V} = \overline{V}^{\norm{\cdot}_V}.
  \end{gather}
\end{Definition}

\begin{Definition}{Banach-hilbert}
  A \define{normed vector space} is a vector space $V$ with a norm
  $\norm\cdot$. We may also write $\norm{\cdot}_V$ to highlight the
  connection.

  A normed vector space $V$ which is complete with respect to its norm is
  called a \define{Banach space}.
  
  A vector space $V$ equipped with an inner product $\scal(.,.)$ is called
  an \define{inner product space} or \define{pre-Hilbert space}. A
  \define{Hilbert space} is a pre-Hilbert space which is also \putindex{complete}.
\end{Definition}

\begin{Definition}{orthogonal}
  Let $V$ be an inner product space over a field $\mathbb K$. Two
  vectors $x,y\in V$ are called \define{orthogonal} if $\scal(x,y) = 0$. We
  write $x\perp y$. Let $W$ be a subspace of $V$. We say that a vector $v$
  is orthogonal to the subspace $W$, if it is orthogonal to every vector in
  $W$.

  A set of nonzero mutually orthogonal vectors
  $\{x_i\} \subset V$ is called \define{orthogonal set}. If
  additionally $\norm{x_i} = 1$ for all vectors, it is called an
  \define{orthonormal set}. These notions transfer directly from
  finite to countable sets.
\end{Definition}

\begin{Definition}{orthogonal-complement}
  Let $W\subset V$ be a subspace of a Hilbert space $V$. We define its
  \define{orthogonal complement} $\ortho W\subset V$ by
  \begin{gather}
    \label{eq:infsup:7}
    \ortho W = \bigl\{v\in V \big| \scal(v,w)_{V} = 0
    \;\forall\,w\in W\bigr\}.
  \end{gather}
\end{Definition}

\begin{Lemma}{orthogonal-closed}
  The orthogonal complement $\ortho W$ of a subspace $W\subset V$
  is closed in the sense of ~\slideref{Definition}{complete}.
\end{Lemma}

\begin{proof}
  By the \putindex{Bunyakovsky-Cauchy-Schwarz inequality}, the inner
  product is continuous on $V\times V$. Therefore, the mapping
  \begin{align*}
    \phi_w\colon V &\to \R,\\
    v&\mapsto \scal(v,w),
  \end{align*}
  is continuous. For any $w\in W$, the kernel of $\phi_w$ is closed as
  the pre-image of the closed set $\{0\}$. Since
  \begin{gather*}
    \ortho W = \bigcap_{w\in W} \ker{\phi_w},
  \end{gather*}
  it is closed as the intersection of closed sets.
\end{proof}

\begin{Theorem}{orthogonal-complement}
  Let $W$ be a subspace of a Hilbert space $V$ and $W^\perp$ its
  orthogonal complement. Then, $W^\perp = \overline{W}^\perp$. Further,
  $V = W \oplus W^\perp$ if and only if $W$ is closed.
\end{Theorem}

\begin{proof}
  Clearly, $\overline{W}^\perp \subset W^\perp$ since
  $W\subset\overline{W}$. Let now $u\in W^\perp$. Then, $\phi =
  \scal(u,\cdot)$ is a continuous linear functional on $V$. Therefore,
  if a sequence $w_n \subset W$ converges to $w\in \overline{W}$, we
  have
  \begin{gather*}
    \scal(u,w) = \lim_{n\to\infty} \scal(u,w_n) = 0,
  \end{gather*}
  since $u \in W^\perp$. Hence, $u\in \overline{W}^\perp$ and
  $W^\perp = \overline{W}^\perp$.

  Now, the ``only if'' follows by the fact, that if $W$ is not
  closed, there is an element $w\in \overline{W}$ but not in $W$ such that
  $\scal(w,u)=0$ for all $u\in W^\perp$. Thus, $w\not\in W^\perp$ and
  consequently $w\not\in W^\perp \oplus W$.

  Let now $W$ be closed. We show that for all $v \in V$ there is a unique
  decomposition
  \begin{gather}
    \label{eq:infsup:8}
    v = w + u,\qquad \text{with} \qquad w\in W, \;u\in W^\perp.
  \end{gather}
  This is equivalent to $V = W \oplus W^\perp$. Uniqueness follows,
  since
  \begin{gather*}
    v = w_1+u_1 = w_2+u_2
  \end{gather*}
  implies that for any $y\in V$
  \begin{gather*}
    0 = \scal(w_1-w_2+u_1-u_2,y) = \scal(w_1-w_2,y) + \scal(u_1-u_2,y).
  \end{gather*}
  Choosing $y=u_1-u_2$ and $w_1-w_2$ in turns, we see that one of the
  inner products vanishes for orthogonality and the other implies that
  the difference is zero.

  If $v\in W$, we choose $w=v$ and $u=0$. For $v\not\in W$, we prove
  existence by considering that due to the closedness of $W$ there holds
  \begin{gather*}
    d=\inf_{w' \in W} \norm{v-w'} >0.
  \end{gather*}
  Let $w_n$ be a minimizing sequence. Using the parallelogram identity
  \begin{gather*}
    \norm{a+b}^2+\norm{a-b}^2 = 2\norm{a}^2+2\norm{b}^2,
  \end{gather*}
  we prove that $\{w_n\}$ is a Cauchy sequence by
  \begin{align*}
    \norm{w_m-w_n}^2 &= \norm{(v-w_n)-(v-w_m)}^2\\
    &= 2\norm{v-w_n}^2+2\norm{v-w_m}^2-\norm{2v-w_m-w_n}^2\\
    &= 2\norm{v-w_n}^2+2\norm{v-w_m}^2-4\norm*{v-\frac{w_m+w_n}2}^2\\
    &\le 2\norm{v-w_n}^2+2\norm{v-w_m}^2-4d^2,
  \end{align*}
  since $(w_m+w_n)/2\in W$ and $d$ is the infimum. Now we use the
  minimizing property to obtain
  \begin{gather*}
    \lim_{m,n\to\infty}\norm{w_m-w_n}^2 = 2d^2+2d^2 -4d^2=0.
  \end{gather*}
  Since $V$ is given as a Hilbert space and as such complete, $w=\lim w_n$
  exists and by the closedness of $W$, we have $w\in W$. Let $u=v-w$.
  By continuity of the norm, we have $\norm{u}=d$. It remains to show
  that $u\in W^\perp$. To this end, we introduce the variation
  $w+\epsilon \tilde w$ with $\tilde w \in W$ to obtain
  \begin{align*}
    d^2 &\le \norm{v-w-\epsilon \tilde w}^2\\
    &= \norm{u}^2-2\epsilon\scal(u,\tilde w)+\epsilon^2 \norm{\tilde w},
  \end{align*}
  implying for any $\epsilon>0$
  \begin{gather*}
    0\le-2\epsilon\scal(u,\tilde w)+\epsilon^2 \norm{\tilde w},
  \end{gather*}
  which requires $\scal(u,\tilde w) = 0$. Since $\tilde w \in W$ was chosen
  arbitrarily, we have $u \in W^\perp$.
\end{proof}

% \begin{Corollary}{ortho-density}
%   A subspace $W$ of a Hilbert space $V$ is dense in $V$ if and only if
%   $\ortho W = \{0\}$.
% \end{Corollary}

% \begin{proof}
%   The ``only if'' is an immediate application of
%   \slideref{Theorem}{orthogonal-complement}. For the opposite
%   direction, assume $\overline W \neq V$. Choose $v\in V$ such that
%   $v\not\in \overline W$. By
%   \slideref{Theorem}{orthogonal-complement}, there are unique elements
%   $w\in W$ and $u\in \ortho W$, such that $v=w+u$. In particular,
%   $u\neq 0$.
% \end{proof}

\begin{Definition}{ortho-projection}
  Let $W$ be a closed subspace of the Hilbert space $V$ and $\ortho W$
  be its orthogonal complement. Then, the
  \define{orthogonal projection} operators
  \begin{gather}
    \label{eq:lafa:9}
    \begin{split}
      \Pi_W &\colon V\to W\\
      \Pi_{\ortho W} &\colon V\to \ortho W\\
    \end{split}
  \end{gather}
  are defined by the unique decomposition
  \begin{gather}
    \label{eq:lafa:10}
    v = \Pi_W v + \Pi_{\ortho W} v.
  \end{gather}
\end{Definition}

\begin{Definition}{dual-space}
  A \define{linear functional} on a vector space $V$ is a
  \putindex{linear mapping} from $V$ to $\mathbb K$.

  The \define{dual space} $V^*$ of a vector space $V$, also called the
  \define{normed dual}, is the space of all bounded linear functionals
  on $V$ equipped with the norm
  \begin{gather}
    \norm{\phi}_{V^*} = \sup_{v\in V} \frac{\phi(v)}{\norm{v}_V}.
  \end{gather}
\end{Definition}

\begin{Theorem*}{Riesz-representation}{Riesz representation theorem}
  Let $V$ be a Hilbert space. Then, $V$ is isometrically isomorphic to
  $V^*$. In particular, there is an isomorphism
  \begin{gather}
    \label{eq:hilbert:1}
    \begin{split}
      \varrho\colon V & \to V^*, \\
      y & \mapsto f,
    \end{split}
  \end{gather}
  such that
  \begin{gather}
    \label{eq:hilbert:2}
    \begin{split}
    \scal(x,y) &= f(x) \qquad \forall x\in V,\\
    \norm{y}_V &= \norm{f}_{V^*}.
    \end{split}
  \end{gather}
  We refer to $\varrho$ as \define{Riesz isomorphism}.
\end{Theorem*}

\begin{proof}
  The proof is constructive and makes use of the orthogonal
  complement.

  First, it is clear that for any $y\in V$ a linear functional
  $f\in V^*$ is defined by $f(\cdot) = \scal(\cdot,y)$. Furthermore,
  $\varrho$ is injective, since
  \begin{gather*}
    \scal(x,y) = 0 \qquad\forall x\in V
  \end{gather*}
  implies $y\in \ortho V = \{0\}$. By the
  \putindex{Bunyakovsky-Cauchy-Schwarz inequality}, we have
  \begin{gather*}
    \norm{f}_{V^*} = \sup_{x\in V}\frac{\abs{f(x)}}{\norm{x}_V}
    = \sup_{x\in V}\frac{\abs{\scal(x,y)}}{\norm{x}_V}
      \le \norm{y}_V,
  \end{gather*}
  with equality for $x=y$.  It remains to show that $\varrho$ is
  surjective. To this end, let $f\in V^*$ be arbitrary and let
  $N = \ker f$. If $N=V$, we choose $y=0$. If not, choose
  $\ortho y \in \ortho N$ and let
  \begin{gather}
    \label{eq:lafa:13}
    y = \frac{f(\ortho y)}{\norm*{\ortho y}^2} \ortho y \in
    \ortho N,
  \end{gather}
  such that $f(y) = \abs*{f(\ortho y)}^2/\norm*{\ortho y}^2 \neq 0$.
  Let now $x\in V$ be chosen arbitrarily. Then, there holds
  \begin{gather*}
    x = \left(x-\frac{f(x)}{f(y)} y\right)
    + \frac{f(x)}{f(y)} y,
    \\
  \end{gather*}
  where $\frac{f(x)}{f(y)}$ denotes a scalar.
  Since
  \begin{gather*}
      f \left(x-\frac{f(x)}{f(y)} y\right) 
      = \left(f(x)-f(x) \frac{f(y)}{f(y)}\right)
      = 0,
  \end{gather*}
  this decomposition amounts to $x = x^0+\ortho x$ with $x^0\in N$ and
  $\ortho x \in \ortho N$. It is unique according to
  \slideref{Definition}{ortho-projection}. Thus, we have that $\ortho x$ is a
  multiple of $y$, say $\ortho x=\alpha y$ with $\alpha=\frac{f(x)}{f(y)}$ and thus
  \begin{gather*}
    \begin{aligned}
    f(x) &= f(x^0) + f(\ortho x)
    &=& \alpha f(y)
    &=& \alpha \frac{\abs*{f(\ortho y)}^2}{\norm*{\ortho y}^2}
    \\
    \scal(x,y) &= \scal(x^0,y)  + \scal(\ortho x,y)
    &=& \alpha \norm{y}_V^2
    &=& \alpha \frac{\abs*{f(\ortho y)}^2}{\norm*{\ortho y}^4}
    \norm*{\ortho y}^2
    \end{aligned}
  \end{gather*}
  Hence, the two terms are equal and $\varrho$ is surjective.
\end{proof}

\begin{Definition}{bilinear-form}
  A \define{bilinear form} $a(.,.)$ on a Hilbert space $V$ is a
  mapping $a\colon V\times V \to \R$, which is linear in both
  arguments. The bilinear form is \textbf{bounded}, if there is a
  constant $M$ such that
  \begin{gather}
    a(u,v) \le M \norm{u}_V \norm{v}_V, \qquad \forall u,v\in V.
  \end{gather}
  It is called \define{coercive} or \define{elliptic}, if there is a
  constant $\alpha$ such that
  \begin{gather}
    a(u,u) \ge \alpha \norm{u}_V^2 \qquad\forall u\in V.
  \end{gather}
\end{Definition}

\begin{Lemma}{pde-bilinear}
  Let $a_{ij} \in C^1(\domain)$, $b_i, c\in C^0(\domain)$. A solution
  to the Dirichlet problem
  \begin{gather}
    \label{eq:hilbert:div-eq}
    \begin{aligned}
      -\sum_{i,j=1}^d \d_i \bigl(a_{ij} \d_j u\bigr)
      + \sum_{i=1}^d \bigl(b_i \d_i u\bigr) + c u &= f
      & \qquad\text{in }&\domain\\
      u &= 0
      & \qquad\text{on }&\d\domain
    \end{aligned}
  \end{gather}
  solves the weak problem: find $u\in V_0$ such that for all
  $v\in V_0$
  \begin{gather}
    \label{eq:hilbert:weak}
    a(u,v) \equiv \form(\mathbf A \nabla u,\nabla v)
    + \form(\mathbf b\cdot\nabla u,v)
    +\form(cu,v) = \form(f,v),
  \end{gather}
  where $\mathbf A(\vx) = \bigl(a_{ij}(\vx)\bigr)$ is the matrix of
  coefficients of the second order term and
  $\mathbf b(\vx) = \bigl(b_{i}(\vx)\bigr)$ is the vector of
  coefficients of the first order term.

  If additionally $u\in C^2(\domain)$ holds, then the solution to the
  weak problem solves the Dirichlet problem in differential form.
\end{Lemma}

\begin{Lemma*}{lax-milgram}{Lax-Milgram}
  Let $a(.,.)$ be a bounded, coercive bilinear form on a Hilbert space
  $V$ and let $f \in V^*$. Then, there is a unique element $u\in V$
  such that
  \begin{gather}
    \label{eq:hilbert:lax-milgram}
    a(u,v) = f(v) \qquad\forall v\in V.
  \end{gather}
  Furthermore, there holds
  \begin{gather}
    \label{eq:hilbert:lax-milgram-estimate}
    \norm{u}_V \le \frac1\alpha \norm{f}_{V^*}.
  \end{gather}
\end{Lemma*}

\begin{proof}
  To prove Lax-Milgram we first consider uniqueness and then the
  existence of a solution.

  Assume that there are solutions $u_1, u_2\in V$ of \eqref{eq:hilbert:lax-milgram},
  i.\,e. there holds $a(u_1,v)=f(v)$ and $a(u_2,v)=f(v)$ for all $v\in V$.
  Thus, $a(u_1-u_2,v)=0$ for all $v\in V$.
  Now choose $v=u_1-u_2\in V$.
  Since $a(.,.)$ is coercive with $\alpha>0$ there holds
  \begin{gather*}
    0 = a(u_1-u_2,u_1-u_2)\geq \alpha \norm{u_1-u_2}_{V}^2 
  \end{gather*}
  which implies $u_1-u_2=0$. Hence $u_1=u_2$.

  Let us now consider the existence of a solution.
  We will define a linear functional to apply Riesz representation theorem
  and Banach fixed point theorem.
  For all $y\in V$ define
  \begin{gather*}
    \scal(y,\cdot) - \omega\left[a(y,\cdot) -f(\cdot)\right] \in V^*
  \end{gather*}
  with $\omega>0$.
  Due to Riesz representation theorem there exists an isomorphism $\rho:V^*\rightarrow V$
  that maps a given $\scal(y,\cdot) - \omega \left[ a(y,\cdot)-f(\cdot) \right]$
  to $z\in V$ such that
  \begin{gather*}
    \scal(v,z) = \scal(y,v) - \omega\left[a(y,v) - f(v) \right] \qquad \forall v\in V.
  \end{gather*}
  Now we define the operator $T_\omega:V\rightarrow V$ that maps $y\mapsto z$ and
  define $A:V\rightarrow V$ such that
  \begin{gather*}
    \scal(Au,v) = a(u,v) \qquad \forall v\in V 
  \end{gather*}
  with $\norm{Au}_V \leq M \norm{u}_V$ for $M>0$.
  This leads for all $v\in V$ to
  \begin{gather*}
    \scal(T_\omega y,v)
    = \scal(y,v) - \omega\left[a(y,v) \right]
    = \scal(y,v) - \omega \scal(Ay,v).
  \end{gather*}
  Thus, we can conclude that $T_\omega y=y-\omega Ay$.
  Applying the norm and using the fact that it is induced by the inner product of $V$
  we get
  \begin{gather*}
    \begin{aligned}
      \norm{T_\omega y - T_\omega x}_V^2 &= \norm{y-x}_V^2 - 2 \omega \scal(A(y-x),y-x) + \omega^2 \norm{A(y-x)}^2 \\
      &\leq \norm{y-x}^2 - 2\omega a(y-x,y-x) + \omega^2 M^2 \norm{y-x}^2 \\
      &\leq (1-2\omega\alpha + \omega^2 M^2) \norm{y-x}^2.
    \end{aligned}
  \end{gather*}
  As we want to apply Banach fixed point theorem, we need $T_\omega$ to be a contraction.
  Therefore, we need $1-2\omega\alpha + \omega^2 M^2 < 1$.
  Hence, if we choose $\omega\in\left(0,\frac{2\alpha}{M^2}\right)$, then $T_\omega$ is a
  contraction and there exists a $u\in V$ such that $T_\omega u = u$
  and $\scal(T_\omega u,u) = \scal(u,u) - \omega\left[ a(u,v) - f(v) \right]$
  which implies $a(u,v)=f(v)$.
  Thus, there exists a solution $u$.

  Now the stability estimate \eqref{eq:hilbert:lax-milgram-estimate} is left to prove.
  Using the coercivity of our bilinear form yields
  \begin{gather*}
    \alpha \norm{u}_V \leq \frac{a(u,u)}{\norm{u}_V} = \frac{f(u)}{\norm{u}_V} \leq \sup_{u\in V} \frac{\snorm{f(u)}}{\norm{u}_V} = \norm{f}_{V^*}.
  \end{gather*}
\end{proof}

\begin{Lemma}{weak-well-posed}
  Let $a_{ij}, c\in L^\infty(\domain)$, $b_i \in C^1(\overline{\domain})$ such
  that there holds for a positive constant $\alpha$
  \begin{gather}
    \label{eq:hilbert:elliptic}
    \begin{split}
    \alpha\abs{\xi}^2 &\le \xi^T \mathbf A(\vx) \xi,
    \qquad \forall \xi \in \R^d,
    \\
    0 &\le c + \div \mathbf b.
    \end{split}
  \end{gather}
  Then, the associated bilinear form is coercive and bounded on
  $H^1_0(\domain)$, and thus the weak formulation has a unique
  solution.
\end{Lemma}

\begin{Definition}{elliptic}
  A differential equation of second order in divergence
  form~\eqref{eq:hilbert:div-eq} such that~\ref{eq:hilbert:elliptic}
  holds is called \define{elliptic}. The lower bound $\alpha$ is the
  ellipticity constant.
\end{Definition}

%%% Local Variables: 
%%% mode: latex
%%% TeX-master: "main"
%%% End: 
