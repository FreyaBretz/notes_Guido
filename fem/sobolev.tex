\begin{Notation}{multi-index}
  For a multi-index $\alpha = (\alpha_1,\dots,\alpha_d)$ with
  nonnegative integer $\alpha_i$ and a function with sufficient
  differentiability, we define the derivative
  \begin{gather*}
    \d^\alpha f = \d_1^{\alpha_1}\cdots\d_d^{\alpha_d} f.
  \end{gather*}
  The order of $\d^\alpha$ is
  \begin{gather*}
    \abs{\alpha} = \sum \alpha_i.
  \end{gather*}
\end{Notation}

\begin{Definition}{distributional-derivative}
  If for a given function $u$ there exists a function $w$ such that
  \begin{gather}
    \label{eq:hk:1}
     \int_\Omega w \phi \dx
     =
     -\int_\Omega u \partial_i \phi \dx,
     \qquad\forall \phi\in \coo(\domain),
  \end{gather}
  then we define $\partial_i u := w$ as the \define{distributional
    derivative} (partial) of $u$ with respect to $x_i$. Here,
  $\coo(\domain)$ is the space of all functions in $C^\infty(\domain)$
  with compact support in $\domain$.

  Similarly through integration by parts, we define distributional
  directional derivatives, distributional gradients, $\d^\alpha u$,
  etc.

  We call a distributional derivative \define{weak derivative} in
  $L^p$ if it is a function in this space.
\end{Definition}

\begin{remark}
  Formula~\eqref{eq:hk:1} is the usual integration by
  parts. Therefore, whenever $u\in\co^1$ in a neighborhood of $x$, the
  distributional derivative and the usual derivative coincide.
\end{remark}

\begin{example}
  Let $\Omega=\R$ and $u(x) = |x|$. Intuitively,
  it is clear that the distributional derivative, if it exists, must
  be the \define{Heaviside function}
  \begin{gather}
    \label{eq:hk:2}
    w(x) =
    \begin{cases}
      -1 & x<0 \\ 1 & x>0.
    \end{cases}
  \end{gather}
  The proof that this is actually the distributional derivative is
  left to the reader.
\end{example}

\begin{example}
  For the derivative of the \putindex{Heaviside function}
  in~\eqref{eq:hk:2}, we first observe that it must be zero whenever
  $x\neq 0$, since the function is continuously differentiable
  there. Now, we take a test function $\phi\in\co^\infty$ with support
  in the interval $(-\epsilon,\epsilon)$ for some positive
  $\epsilon$. Let $w'(x)$ be the derivative of $w$. Then, by
  integration by parts
  \begin{gather*}
    \int_{-\epsilon}^\epsilon w(x) \phi'(x)\dx
    = -\int_{-\epsilon}^0 w(x)' \phi(x)\dx
    -\int_0^\epsilon w(x)' \phi(x)\dx
    + 2 \phi(0) = 2\phi(0),
  \end{gather*}
  since $w'(x) = 0$ under both integrals. Thus, $w'(x)$ is an object
  which is zero everywhere except at zero, but its integral against a
  test function $\phi$ is nonzero. This contradicts our notion, that
  integrable functions can be changed on a set of measure zero without
  changing the integral. Indeed, $w'$ is not a function in the usual
  sense, and we write $w'(x) = 2 \delta(x)$, where $\delta(x)$ is the
  \define{Dirac $\delta$-distribution}, which is defined by the two
  conditions
  \begin{gather*}
    \begin{alignedat}{2}
      \delta(x) &= 0, & \forall x & \neq 0
      \\
      \int_\R \delta(x) \phi(x)\dx &= \phi(0), \quad & \forall \phi
      &\in \co^0(\R).
    \end{alignedat}
  \end{gather*}
  We stress that $\delta$ is not an integrable function, or a function
  at all.
\end{example}

\begin{Definition}{Wkp}
  The \define{Sobolev space} $W^{k,p}(\domain)$ is the space
  \begin{gather}
    W^{k,p}(\domain) = \bigl\{
    u\in L^p(\domain) \big|
    \d^\alpha u\in L^p(\domain) \forall \abs{\alpha} \le k
    \bigr\},
  \end{gather}
  where the derivatives are understood in weak sense. Its norm is
  defined by
  \begin{gather}
    \norm{v}_{k,p}^p = \norm{v}_{k,p;\domain}^p
    = \sum_{\abs{\alpha} \le k} \norm{\d^\alpha v}_{L^p(\domain)}^p.
  \end{gather}
  The following seminorm will be useful:
  \begin{gather}
    \snorm{v}_{k,p}^p = \snorm{v}_{k,p;\domain}^p
    = \sum_{\abs{\alpha} = k} \norm{\d^\alpha v}_{L^p(\domain)}^p.
  \end{gather}
\end{Definition}

\begin{Notation}{zero-norm}
  We will use the notation
  \begin{gather*}
    \norm{v}_0 = \norm{v}_{0;\domain} = \norm{v}_{L^2(\domain)}.
  \end{gather*}
  Accordingly, $W^{0,p}(\domain) = L^p(\domain)$.
\end{Notation}

\begin{Corollary}{wkp-embedding}
  There holds
  \begin{gather}
    W^{k,p}(\domain) \subset W^{k-1,p}(\domain) \subset \dots \subset
%    W^{1,p}(\domain) \subset
    W^{0,p}(\domain) = L^{p}(\domain)
  \end{gather}
\end{Corollary}

\begin{Definition}{hkp}
  The \define{Sobolev space} $H^{k,p}(\domain)$ is the completion of
  $C^\infty(\domain)$ with respect to the norm $\norm{\cdot}_{k,p}^p$.

  In the case $p=2$, we write $H^k(\domain) = H^{k,2}(\domain)$.
\end{Definition}

\begin{Theorem*}{meyers-serrin}{Meyers-Serrin}
  \begin{gather*}
    H^{k,p}(\domain)\cong W^{k,p}(\domain)
  \end{gather*}
\end{Theorem*}

\begin{example}
  Functions, which are in $W^{k,p}(\domain)$ or not.
  \begin{enumerate}
  \item The function $x/\abs{x}$ is in $H^1(B_1(0))$ if $d=3$, but not
    if $d=2$.
  \end{enumerate}
\end{example}

\begin{Definition}{boundary-smoothness}
  A bounded domain $\domain\subset\R^d$ is said to have $C^k$-boundary
  or to be a $C^k$-domain, if there is a finite covering $\{U_i\}$ of
  its boundary $\d\domain$, such that for each $U_i$ there is
  a mapping $\Phi_i \in C^k(U_i)$ with the following properties:
  \begin{gather}
    \begin{split}
      \Phi_i(\d\domain \cap U_i)
      &\subset \left\{ \vx\in\R^d \mid x_1 = 0 \right\},\\
      \Phi_i(\domain \cap U_i)
      &\subset \left\{ \vx\in\R^d \mid x_1 > 0 \right\}.      
    \end{split}
  \end{gather}
  The domain is called Lipschitz, if such a construction exists with
  Lipschitz-continuous mappings.
\end{Definition}

\begin{Definition}{continuous-embedding}
  We say that a normed vector space $U \subset V$ is
  \define{continuously embedded} in another space $V$, in symbolic
  language
  \begin{gather}
    U \hookrightarrow V,
  \end{gather}
  if the inclusion mapping $U \ni x \mapsto x\in V$ is continuous, that is, there is a constant $c$ such that
  \begin{gather}
    \norm{x}_V \le c \norm{x}_U.
  \end{gather}
  If the spaces $U$ and $V$ consist of equivalence classes, the
  inclusion may involve choosing representatives on the left or on the
  right.
\end{Definition}

\begin{Theorem}{sobolev-embedding}
  Let $\domain\subset\R^d$ be a bounded Lipschitz domain. For the space
  $W^{k,p}(\domain)$ define the number
  \begin{gather}
    s = k-\tfrac dp.
  \end{gather} Assume $k_1 \le k_2$ and $p_1,p_2\in [1,\infty)$.
  Then, if $s_1 \ge s_2$, we have the continuous embedding
  \begin{gather}
    W^{k_1,p_1}(\domain) \hookrightarrow  W^{k_2,p_2}(\domain).
  \end{gather}
\end{Theorem}

\begin{Lemma}{trace-continuous}
  Let $\domain$ be a bounded Lipschitz domain in $\R^d$. Then, there
  exists a constant $c$ only depending on $\domain$, such that every
  function $u\in H^1(\domain) \cap C^1(\overline{\domain})$ admits the
  estimate
  \begin{gather}
    \norm{u}_{L^p(\d\domain)} \le c \norm{u}_{W^{1,p}(\domain)}.
  \end{gather}
\end{Lemma}

\begin{Theorem*}{trace}{Trace theorem}
  Let $\domain$ be a bounded Lipschitz domain in $\R^d$. Then, every
  function $u\in W^{1,p}(\domain)$ has a well defined trace
  $\gamma u \in L^p(\d\domain)$ and there holds
  \begin{gather}
    \norm{\gamma u}_{L^p(\d\domain)} \le c \norm{u}_{W^{1,p}(\domain)},
  \end{gather}
  with the same constant as in the previous lemma. We simply write
  \begin{gather}
    u_{|\d\domain} = \gamma u.
  \end{gather}
\end{Theorem*}

\begin{remark}
  The trace theorem guarantees that the imposition of Dirichlet
  boundary conditions on Sobolev functions is a reasonable
  operation. In particular, it ensures that $H^1(\domain)$ and
  $H^1_0(\domain)$ are indeed different spaces. The same does not
  hold, if we complete $C^1(\domain)$ and $C^1_0(\domain)$ in
  $L^2(\domain)$.

  Remarkably, we set out defining $W^{1,p}(\domain)$ as a subset of
  $L^p(\domain)$, which consists of functions ``defined up to a set of
  measure zero''. Now it turns out, that functions in
  $W^{1,p}(\domain)$ can have well-defined values on certain sets of
  measure zero. From the point of view of subsets of $L^p(\domain)$,
  this is always to be understood by choosing representatives of the
  equivalence class. This is also the reason why we write
  ``$\hookrightarrow$'' instead of ``$\subset$''.
\end{remark}

\begin{Definition}{hoelder-spaces}
  A function $f \in C^0(\domain)$ is Hölder-continuous with exponent
  $\gamma \in (0,1]$, if there is a constant $C_f$ such that
  \begin{gather}
    \abs{f(\vx)-f(\vy)} \le C_f \abs{\vx-\vy}^\gamma
    \qquad\forall \vx,\vy\in\domain.
  \end{gather}
  In particular, for $\gamma = 1$, we obtain Lipschitz-continuity.

  We define the Hölder space $C^{k,\gamma}$ of $k$-times continuouly
  differentiable functions such that all derivatives of order $k$ are
  Hölder-continuous. The norm is
  \begin{gather}
    \norm{u}_{C^{k,\gamma}(\domain)} = \max_{\abs{\alpha}\le k} \sup_{\vx,\vy\in\domain}
    \frac{\abs{\d^\alpha u(\vx)-\d^\alpha u(\vy)}}{\abs{\vx-\vy}^\gamma}
  \end{gather}
\end{Definition}

\begin{Theorem}{hoelder-embedding}
  Let $\domain\subset\R^d$ be a bounded Lipschitz domain.
  If $s = k-\tfrac dp > j+\gamma$, then every function in $W^{k,p}$ has a
  representative in $C^{j,\gamma}$. We write
  \begin{gather}
    W^{k,p}(\domain) \hookrightarrow C^{j,\gamma}(\domain).
  \end{gather}
\end{Theorem}

\begin{Corollary}{sobolev-continuous}
  Elements of Sobolev spaces are continuous if the derivative
  order is sufficiently high. In particular,
  \begin{gather}
    \begin{aligned}
      H^1(\domain) & \hookrightarrow C(\domain) & d&= 1, \\
      H^2(\domain) & \hookrightarrow C(\domain) & d&= 2,3.
    \end{aligned}
  \end{gather}
\end{Corollary}

\begin{Lemma}{weak-well-posed}
  Let $a_{ij},c\in L^\infty(\domain), b_i\in C^1(\overline{\domain})$ such
  that there holds for a positive constant $\alpha$
  \begin{gather}
    \label{eq:hilbert:elliptic}
    \begin{split}
    \alpha\abs{\xi}^2 &\le \xi^T \mathbf A(\vx) \xi,
    \qquad \forall \xi \in \R^d,
    \\
    0 &\le c - \frac12 \nabla\cdot \vb.
    \end{split}
  \end{gather}
  Then, the associated bilinear form is coercive and bounded on
  $H^1_0(\domain)$, and thus the weak formulation has a unique
  solution.
\end{Lemma}

\begin{proof}
  In this proof we do not consider boundedness for the term containig
  $\mathbf A$ as it is exactly the same as in
  \slideref{Lemma}{semi-weak-well-posed}.

  Let us first consider boundedness for the $\vb$-term.
  For the standard dot product there holds
  $\vx\cdot \vy\leq \abs{\vx}\abs{\vy}$, which yields
  \begin{align*}
    \int_\domain \vb\cdot\nabla u v \dx \le \int_\domain \abs{\vb}\abs{\nabla u}v\dx.
  \end{align*}
  Now we can use Hölder's inequality which leads to
  \begin{align*}
    \abs{(\vb\cdot\nabla u,v)_{L^2}}&\le \Big| \int_\domain\abs{\vb}\abs{\nabla u}v\dx \Big| \\
    &\le \Big(\int_\domain \abs{\vb}^4\dx \Big)^{\frac14} \Big(\int_\domain \abs{\nabla u}^2\dx \Big)^{\frac12}
    \Big(\int_\domain \abs{v}^4 \dx\Big)^{\frac14} \\
    &= \norm{\vb}_{L^4} \norm{\nabla u}_{L^2} \norm{v}_{L^4}.
  \end{align*}
  Due to the Sobolev embedding theorem, holds
  \begin{align*}
    H^1 = W^{1,2} \hookrightarrow W^{0,4} = L^4
  \end{align*}
  which gives us $\norm{\vb}_{L^4},\norm{v}_{L^4}<\infty$.
  Additionally, this yields for $w\in H^1$ and $c\ge 0$ that
  \begin{align*}
    \norm{w}_{L^4} \le c \norm{w}_1.
  \end{align*}
  Hence, we obtain
  \begin{align*}
    (\vb\cdot\nabla u,v)_{L^2} \le c \norm{\vb}_{L^4} \abs{u}_1 \norm{v}_1
    \le c \norm{\vb}_{L^4} \abs{u}_1 \abs{v}_1.
  \end{align*}
  In the last step we used Friedrich's inequality for $v\in H^1_0$.
  
  Now, we prove boundedness for the $c$-term.
  Using the Bunyakovsky-Cauchy-Schwarz and Friedrich's inequality and
  $M:=\lambda(\domain)^2 \max_{x\in\domain}\abs{c(x)}>0$, there holds
  \begin{align*}
    (cu,v)_{L^2} = \int_\domain cuv\dx \le \abs{c} \norm{u}_{L^2} \norm{v}_{L^2} \le M \abs{u}_1 \abs{v}_1.
  \end{align*}
  Hence, our bilinear form is bounded.

  There is only the coercivity of our bilinear
  form left to prove.
  Let us first prove the identitiy
  \begin{gather*}
    (\vb\cdot\nabla u,u)_{L^2} = -\frac12 (\nabla\cdot \vb,u^2)_{L^2}.
  \end{gather*}
  Starting with the right hand side, integration by parts yields
  \begin{gather*}
    -\frac12 (\nabla\cdot \vb,u^2)_{L^2} = -\frac12 \int_\domain \nabla\cdot\vb~ u^2 \dx
    = \frac12 \int_\domain \vb\cdot\nabla(u^2)\dx - \frac12 \int_{\d\domain} \vb~ u^2\dx
  \end{gather*}
  where the boundary term vanishes as our solution space is $H^1_0(\domain)$.
  Then, there holds
  \begin{gather*}
    \begin{split}
      -\frac12 (\nabla\cdot \vb,u^2)_{L^2} &= \frac12 \int_\domain \vb\cdot\nabla(u^2)\dx
      = \frac12 \int_\domain \vb\cdot\nabla u~ 2u\dx
      \\
      &= \int_\domain \vb\cdot\nabla u u\dx = (\vb\cdot\nabla u,u)_{L^2}.
    \end{split}
  \end{gather*}
  As $c\in L^\infty(\domain)$ and $b_i\in C^1(\overline{\domain})$ the
  expression $C\coloneqq \min(c-\frac12 \nabla\cdot \vb)$ is
  well-defined and $C>0$ by assumption.
  Now consider the whole bilinear form
  \begin{align*}
    a(u,u) &= (\mathbf A\nabla u,\nabla u)_{L^2} - \frac12 (\nabla \cdot \vb~ u ,u)_{L^2} + (cu,u)_{L^2} \\
    &\ge \alpha \int_\domain \abs{\nabla u}^2 \dx + \int_\domain (c-\frac12 \nabla \cdot \vb)\abs{u}^2 \dx \\
    &\ge \alpha \int_\domain \abs{\nabla u}^2 \dx + C \int_\domain \abs{u}^2 \dx \\
    &\ge \underbrace{\min\{\alpha,C \}}_{\ge 0}\norm{u}_1^2 \ge \min\{\alpha,C \} \abs{u}_1^2.
  \end{align*}
  Thus, the bilinear form is coercive.
\end{proof}



%%% Local Variables: 
%%% mode: latex
%%% TeX-master: "main"
%%% End: 
