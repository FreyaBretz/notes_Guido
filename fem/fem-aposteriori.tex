\begin{Definition}{a-posteriori}
  Let $u\in V$ be the solution to a boundary value in weak form and
  $u_h\in V_h$ be its finite element approximation on the mesh
  $\mesh_h$. We call a quantity $\eta_h(u_h)$ \define{a posteriori
    error estimator},
  \begin{gather}
    \norm{u-u_h} \le c \eta_h(u_h).
  \end{gather}
  The estimator is \define{reliable}, if the constant $c$ is
  computable. It is \define{efficient}, if the converse estimate holds,
  namely
%  with addition of a quantity $\osc_h(f)$
  \begin{gather}
    \eta_h(u_h) \le c \norm{u-u_h}.
  \end{gather}
\end{Definition}

\subsection{Quasi-interpolation in $H^1$}

\begin{intro}
  Interpolation in Sobolev spaces in
  \slideref{Theorem}{fe-interpolation} relies on the nodal
  interpolation operator in
  \slideref{Definition}{nodal-interpolation}, which in turn requires
  point values of the interpolated function. Therefore, it is not
  defined on $H^1(\domain)$ in dimensions greater than one. Since we
  need such interpolation operators in the analysis of a posteriori
  error estimates, we provide them in this section.
  
  Most details in this section are from~\cite{Verfuerth13}. For the
  ease of presentation, we present the results for Dirichlet problem
  of the Laplacian, namely $V = H^1_0(\domain)$ and
  \begin{gather*}
    a(u,v) = \int_\domain \nabla u\cdot\nabla v\dvx.
  \end{gather*}
\end{intro}

\begin{Definition}{locally-quasi-uniform}
  A shape-regular family of meshes $\{\mesh_h\}$ is called
  \define{locally quasi-uniform}, if there is a constant $c$ such that
  for every pair of cells $\cell_1$ and $\cell_2$ sharing at least one
  vertex there holds
  \begin{gather}
    h_{\cell_1} \le c h_{\cell_2}.
  \end{gather}
\end{Definition}

\begin{Definition}{fem-neighborhood}
  For a vertex or higher-dimensional boundary facet $F$, we define the
  set of cells
  \begin{gather}
    \mesh_{F} = \bigl\{ \cell\in\mesh \big|F \subset\d\cell \bigr\}.
  \end{gather}
  Similarly, the set of cells sharing at least one vertex with $\cell$
  is called $\mesh_{\cell}$. Additionally, we define the subdomains
  \begin{gather}
    \overline\domain_{F} = \bigcup_{\cell\in \mesh_{F}}\overline\cell,
    \qquad
    \overline\domain_{\cell} = \bigcup_{\cell'\in \mesh_{\cell}}\overline\cell'.
  \end{gather}
\end{Definition}

\begin{Theorem*}{clement}{Clément quasi-interpolation}
  Let $\{\mesh_h\}$ be a locally quasi-uniform family of meshes with
  piecewise polynomial finite element spaces
  $V_h\subset H^1_0(\domain)$. Then, there exist bounded operators
  $\overline I_h\colon H^1_0(\domain) \to V_h$ such that for every
  function $u\in H^1_0(\domain)$, every mesh cell $\cell$, every face
  $F$, and for $m=0,1$ there holds
  \begin{align}
    \norm{u-\overline I_h u}_{m;T} &\le c h_\cell^{1-m} \snorm{u}_{1;\domain_\cell} \\
    \norm{u-\overline I_h u}_{0;F} &\le c h_\cell^{\nicefrac{1}{2}} \snorm{u}_{1;\domain_\cell}
  \end{align}
\end{Theorem*}

\begin{proof}
  We construct the quasi-interpolation operator on a mesh cell $\cell$
  into the lowest order space $\P_1$ or $\Q_1$ in two steps. First,
  for each vertex $\vertex_i$ of $\cell$, let
  \begin{gather}
    \overline u_i = \frac1{\abs{\domain_{\vertex_i}}}
    \int_{\domain_{\vertex_i}} u \dvx.
  \end{gather}
  By Poincaré inequality,
  \begin{gather}
    \label{eq:fem-aposteriori:1}
    \norm{u-\overline u_i}_{\domain_{\vertex_i}}
    \le c \diam(\domain_{\vertex_i}) \norm{\nabla u}_{\domain_{\vertex_i}}.
  \end{gather}
  For a vertex $\vertex_i$ on the boundary, we observe
  \begin{gather}
    \label{eq:fem-aposteriori:2}
    \begin{split}
      \norm{\overline u_i}_{\domain_{\vertex_i}}^2
      &= \frac1{\abs{\domain_{\vertex_i}}^2}
      \int_{\domain_{\vertex_i}}\left(\int u \dvy\right)^2\dvx\\
      & = \frac1{\abs{\domain_{\vertex_i}}}\left(\int 1 u \dvy\right)^2\\
      & \le \frac1{\abs{\domain_{\vertex_i}}} \int 1^2\dvy \int u^2\dvy \\
      & = \norm{u}^2_{{\domain_{\vertex_i}}}\\
      & \le c \diam({\domain_{\vertex_i}})^2 \norm{\nabla u}_{{\domain_{\vertex_i}}}^2,
    \end{split}
  \end{gather}
  by Friedrichs' inequality, since the subdomain
  ${\domain_{\vertex_i}}$ has at least one face on $\d\domain$.

  Now, we define the quasi-interpolation operator
  $\overline I_h\colon H^1_0(\domain)\to V_h$ cellwise on simplices by
  \begin{gather}
    \label{eq:fem-aposteriori:3}
    \overline I_h u_{|\cell}
    = \sum_{\vertex_i\in\d\cell\cap\domain} \lambda_i \overline u_i.
  \end{gather}
  Note, that zero boundary conditions are enforced by omitting
  vertices on the boundary.  On quadrilaterals, we use the bilinear
  shape functions associated with the vertices instead of the
  \putindex{barycentric coordinates} $\lambda_i$. Now, using
  $\sum\lambda_i=1$ and $0\le \lambda_i \le 1$, we estimate
  \begin{align*}
    \norm{u - \overline I_h u}_\cell
    & \le \sum_{\vertex_i\in\d\cell} \norm{\lambda_i(u-\overline u_i)}_{\cell}
      + \sum_{\vertex_i\in\d\cell\cap\d\domain}
      \norm{\lambda_u \overline u_i} \\
    & \le \sum_{\vertex_i\in\d\cell} \norm{u-\overline u_i}_{\cell}
      + \sum_{\vertex_i\in\d\cell\cap\d\domain}
      \norm{\overline u_i} \\
    & \le \diam({\domain_{\vertex_i}})\left(
      \sum_{\vertex_i\in\d\cell} \norm{\nabla u}_{\domain_{\vertex_i}}
      + \sum_{\vertex_i\in\d\cell\cap\d\domain} \norm{\nabla u}_{\domain_{\vertex_i}}
      \right),
  \end{align*}
  where we used~\eqref{eq:fem-aposteriori:1}
  and~\eqref{eq:fem-aposteriori:2} in the end. Observing that both
  sums extend over finitely many vertices and that by local
  quasi-uniformity we can bound the diameter of
  $\diam({\domain_{\vertex_i}})$ by that of $\cell$, namely
  $\diam({\domain_{\vertex_i}}) \le c h_\cell$, we obtain the estimate
  in $L^2(\cell)$. For the estimate in $H^1(\cell)$, we observe that
  the mean value calculation is a continuous operation on
  $H^1_0(\domain(\vertex_i))$, and that~\eqref{eq:fem-aposteriori:3} as a finite
  sum is continuous, therefore,
  \begin{gather}
    \snorm{u-\overline I_h u}_{1;\cell} \le \norm{\nabla u}_{\cell} +
    \norm{\nabla \overline I_h u}_{\cell}
    \le c \norm{\nabla u}_{\domain(\cell)}.
  \end{gather}
  Finally, we use the trace estimate for $u\in H^1(\cell)$
  \begin{gather}
    \norm{u}_F
    \le c \left(h^{-\nicefrac12} \norm{u}_\cell
      +h^{\nicefrac12} \norm{\nabla u}_\cell\right),
  \end{gather}
  to obtain the estimate on the edge.
\end{proof}

\begin{Theorem*}{scott-zhang}{Scott-Zhang quasi-interpolation}
  Let $\{\mesh_h\}$ be a locally quasi-uniform family of meshes with
  piecewise polynomial finite element spaces
  $V_h\subset H^1(\domain)$. Then, there exist bounded operators
  $\overline I_h\colon H^1(\domain) \to V_h$ such that for every
  function $u\in H^1(\domain)$, every mesh cell $\cell$, every face
  $F$, and for $m=0,1$ there holds
  \begin{align}
    \norm{u-\overline I_h u}_{m;T} &\le c h_\cell^{1-m} \snorm{u}_{1;\domain_\cell} \\
    \norm{u-\overline I_h u}_{0;F} &\le c h_\cell^{\nicefrac{1}{2}} \snorm{u}_{1;\domain_\cell},
  \end{align}
  and $\overline I_h u$ is a quasi-interpolation on the boundary $\d\domain$.
\end{Theorem*}

\begin{Theorem*}{schoeberl-interpolation}{Schöberl quasi-interpolation}
  Let $\{\mesh_h\}$ be a locally quasi-uniform family of meshes with
  piecewise polynomial finite element spaces
  $V_h\subset H^1(\domain)$. Then, there exist bounded
  \emph{projection} operators
  $\overline I_h\colon H^1(\domain) \to V_h$ such that for every
  function $u\in H^1(\domain)$, every mesh cell $\cell$, every face
  $F$, and for $m=0,1$ there holds
  \begin{align}
    \norm{u-\overline I_h u}_{m;T} &\le c h_\cell^{1-m} \snorm{u}_{1;\domain_\cell} \\
    \norm{u-\overline I_h u}_{0;F} &\le c h_\cell^{\nicefrac{1}{2}} \snorm{u}_{1;\domain_\cell}.
  \end{align}  
\end{Theorem*}

\begin{Definition}{residual}
  Let $a(u,v) = f(v)$ be the weak formulation of a BVP on the space
  $V$.  Then, we define the \define{residual}
\begin{gather}
  \begin{split}
    R\colon V &\to V^* \\
    w&\mapsto f(\cdot) - a(w,\cdot).
  \end{split}
\end{gather}
\end{Definition}
\begin{Lemma}{residual-hm1}
  For $u\in V=H^1_0(\domain)$ solution to Poisson's equation and any
  other function $w\in V$, there holds
  \begin{gather}
    \snorm{u-v}_1 = \norm{R v}_{-1}
    := \sup_{w\in V} \frac{\left<R v,w\right>}{\snorm{w}_1}.
  \end{gather}
\end{Lemma}

\begin{proof}
  We have
  \begin{gather}
    \left<Rv,w\right>
    = f(w) - a(v,w)
    = a(u-v,w).
  \end{gather}
  Since $a(.,.)$ is s.p.d., we can apply Cauchy-Schwarz to obtain the result.
\end{proof}

\begin{Definition}{mvl-jmp}
  Let $u$ be a piecewise continuous function on a mesh $\mesh$. On a
  face $F$ between two cells $\cell_1$ and $\cell_2$, we define the
  \define{mean value operator}
  \begin{gather}
    \mvl{u}(\vx) = \frac{u_1(\vx) + u_2(\vx)}2
    = \lim_{\epsilon\searrow 0}
    \frac{u(\vx-\epsilon \n_1) + u(\vx - \epsilon\n_2)}2,
  \end{gather}
  where $\n_i$ are the outer normal vectors of the two cells. The
  \define{jump operator} is
  \begin{gather}
    \mvl{u\n} = \frac{(u_1-u_2)\n_1}2 = \frac{(u_2-u_1)\n_2}2.
  \end{gather}
\end{Definition}
\begin{Definition}{residual-strong}
  Let $u\in V=H^1_0(\domain)$ be the solution to Poisson's equation
  and let $u_h\in V_h$ be a finite element function.  The strong form
  of the residual is
  \begin{gather}
    \left<Rv,w\right> = \sum_{\cell\in\mesh_h}\int_{\cell} r_\cell(u_h) w\dvx
    - \sum_{F\in \mathbb F_h^i} 2\mvl{\n\cdot\nabla u_h} w\ds,
  \end{gather}
  where
  \begin{gather}
    r_\cell(u_h) = f + \Delta u_h.
  \end{gather}
\end{Definition}

\begin{Lemma}{residual-upper-bound}
  Let $\mesh_h$ be a locally quasi-uniform mesh.
  There is a constant $c>0$ such that the error between the true
  solution $u\in V=H^1_0(\domain)$ and the finite element solution
  $u_h\in V_h \subset V$ is bounded by
  \begin{gather}
    \label{eq:fem-aposteriori:5}
    \snorm{u-u_h}_{1;\mesh_h} \le c \left(
      \sum_{\cell\in\mesh_h} h_\cell^2 \norm{r_\cell(u_h)}_\cell^2
      + \sum_{F\in \mathbb F_h^i} h_F
      \norm{\mvl{\n\cdot\nabla u_h}}_F^2
    \right)^{\nicefrac12}.
  \end{gather}
\end{Lemma}

\begin{proof}
  We begin with the equivalence of error in $V$ and residual in
  $V^*$. For the residual, there holds Galerkin orthogonality, such
  that we obtain
  \begin{gather}
    \label{eq:fem-aposteriori:4}
    \snorm{u-u_h}_1
    = \sup_{w\in V}\frac{\left<Ru_h,w\right>}{\snorm{w}_1}
    = \sup_{w\in V}\frac{\left<Ru_h,w- \overline I_h w\right>}{\snorm{w}_1}.
  \end{gather}
  Using the strong form and the quasi-interpolation $\overline I_h$,
  we get
  \begin{align*}
    \left<Ru_h,w\right>
    &= \sum_{\cell\in\mesh_h}\int_{\cell} r_\cell(u_h) (w-\overline I_h w)\dvx
    - \sum_{F\in \mathbb F_h^i} 2\mvl{\n\cdot\nabla u_h} (w-\overline I_h w)\ds
    \\
    & \le \sum_{\cell\in\mesh_h} \norm{r_\cell(u_h)}_\cell
      \norm{w-\overline I_h w}_\cell
      + \sum_{F\in \mathbb F_h^i} \norm{\mvl{\n\cdot\nabla u_h}}_\cell
      \norm{w-\overline I_h w}_\cell
      \\
    & \le c_1 \sum_{\cell\in\mesh_h} h_\cell
      \norm{r_\cell(u_h)}_\cell
      \norm{\nabla w}_{\domain_\cell}
      + c_2 \sum_{F\in \mathbb F_h^i} h_E^{\nicefrac12}
      \norm{\mvl{\n\cdot\nabla u_h}}_\cell
      \norm{\nabla w}_{\domain_\cell}
  \end{align*}
  Applying Hölder inequality, we obtain for the first term
  \begin{gather*}
    \sum_{\cell\in\mesh_h} h_\cell
    \norm{r_\cell(u_h)}_\cell
    \norm{\nabla w}_{\domain_\cell}
    \le
    \left(\sum_{\cell\in\mesh_h} h_\cell^2
    \norm{r_\cell(u_h)}_\cell^2\right)^{\nicefrac12}
  \left(\sum_{\cell\in\mesh_h} \norm{\nabla w}_{\domain_\cell}^2
  \right)^{\nicefrac12},
  \end{gather*}
  and similar for the second term. Both contain a term of the form
  \begin{gather*}
    \sum_{\cell\in\mesh_h} \norm{\nabla w}_{\domain_\cell}^2
    = \sum_{\cell\in\mesh_h} \sum_{\cell'\in\mesh_\cell}\norm{\nabla w}_{\cell'}^2
    \le n \norm{\nabla w}_{\domain}^2,
  \end{gather*}
  where $n$ is the maximal number of occurrences of a cell $\cell'$ in
  the double sum. This $n$ is bounded uniformly on shape regular
  family of meshes, since shape regularity prohibits degeneration of
  cells. Therefore, we conclude
  \begin{align*}
    \left<Ru_h,w\right>
    & \le c \left(\sum_{\cell\in\mesh_h} h_\cell^2
      \norm{r_\cell(u_h)}_\cell^2 +
      \sum_{F\in \mathbb F_h^i} h_E
      \norm{\mvl{\n\cdot\nabla u_h}}_\cell^2\right)^{\nicefrac12}
      \snorm{w}_{1;\domain}.
  \end{align*}
  Entering into~\eqref{eq:fem-aposteriori:4} yields the proposition.
\end{proof}

\begin{remark}
  The constant $c$ in the previous theorem depends on the constant in
  the quasi-interpolation estimate, which in turn was derived using
  Poincaré and trace inequalities. Traditionally, both are derived
  using indirect arguments, but a constructive proof with computable
  bounds is possible. For details, see~\cite[Chapter
  3]{Verfuerth13}. There still remains the question, whether these
  bounds are sufficiently sharp to be applicable in practice.
\end{remark}

\begin{Definition}{osc}
  For a function $f\in L^2(\domain)$, we define the cell-wise average
  \begin{gather}
    \overline f(\vx) = \frac1{\abs{\cell}}\int_\cell f \dvx
    \qquad\vx\in\cell, \quad\cell\in\mesh_h,
  \end{gather}
  and the \define{data oscillation}
  \begin{gather}
    \operatorname{osc}_\cell f = \norm{f-\overline f}_\cell.
  \end{gather}
\end{Definition}

\begin{Definition}{residual-estimator}
  The residual based error estimator for the finite element method for
  Poisson's equation on a mesh $\mesh$ is defined as
  \begin{gather}
    \eta_{h}(u_h) = \sum_{\cell\in\mesh} \eta_{\cell}(u_h),
  \end{gather}
  where
  \begin{gather}
    \eta_{\cell}(u_h)^2 = h_{\cell}^2 \norm{\overline f + \Delta u_h}_\cell^2
    + \frac12 \sum_{F\subset \d\cell} h_F\norm{\mvl{n\cdot\nabla u_h}}_E^2.
  \end{gather}
\end{Definition}

\begin{Theorem}{residual-estimate}
  There holds with positive constants $c_1$ and $c_2$
  \begin{align}
    \label{eq:fem-aposteriori:6}
    c_1 \snorm{u-u_h}_1^2
    &\le \eta_{h}(u_h)^2
      + \sum_{\cell\in\mesh_h} h^2_\cell\operatorname{osc}_\cell^2 f
    \\
    \label{eq:fem-aposteriori:7}
    c_2 \,\eta_\cell^2(u_h)
    &\le \snorm{u-u_h}_{1;\domain_{\cell}}^2
      + \sum_{\cell\in\mesh_\cell} h^2_\cell\operatorname{osc}_\cell^2 f
  \end{align}
\end{Theorem}

\begin{proof}
  Note that the right hand side of the
  estimate~\eqref{eq:fem-aposteriori:5} in
  \slideref{Lemma}{residual-upper-bound} differs from the estimator
  $\eta_h(u_h)$ only by the replacement of $f$ by $\overline
  f$. Therefore, we can start the proof of this lemma by
  changing~\eqref{eq:fem-aposteriori:4} to
  \begin{gather}
    \snorm{u-u_h}_1
    = \sup_{w\in V}\frac{\left<Ru_h,w\right>}{\snorm{w}_1}
    = \sup_{w\in V}\frac{\left<\overline Ru_h,w\right> + \form(f-\overline f, w)}{\snorm{w}_1},
  \end{gather}
  where we ad hoc defined $\overline R$ like $R$, but replacing
  $r_\cell = f+\nabla u_h$ by $\overline f + \nabla u_h$. Due to
  equality, we can still subtract the quasi-interpolant of $w$ and
  continue through the whole proof. What remains is the estimate
  \begin{gather}
    \label{eq:fem-aposteriori:8}
    \form(f-\overline f, w - \overline I_h w)
    \le c \sum_{\cell\in\mesh} h_\cell \norm{f-\overline f}_{\cell} \norm{\nabla w}_{\domain_\cell}
    \le c \left(\sum_{\cell\in\mesh} h_\cell \norm{f-\overline f}_{\cell}\right)^{\nicefrac12}
      \snorm{w}_{1;\domain},
    \end{gather}
    which is proven by Hölder inequality and the boundedness of the
    number of cells in $\mesh_\cell$. Thus, the upper
    bound~\eqref{eq:fem-aposteriori:6} is an immediate consequence of
    \slideref{Lemma}{residual-upper-bound}.
    
    Note: the proof for the lower bound is for linear elements
    only. It can be generalized to higher order by generalizing the
    bubble functions below, which is mostly technical.
    
    We observe that the lower bound is local, that is, the estimate on
    each cell $\cell$ is bounded from above by the error on the halo
    $\domain_\cell$. Therefore, we start by constructing test
    functions with support on a cell. On a simplex, we define the
    \define{bubble function} in terms of \putindex{barycentric coordinates}
    \begin{gather}
      \label{eq:bubble-simplex}
      B_\cell = \frac1{(d+1)^{d+1}}\lambda_0\lambda_1\dots\lambda_d,
    \end{gather}
    while on the reference hypercube $(0,1)^d$ we let
    \begin{gather}
      \label{eq:bubble-tensor}
      \reference B_\cell(\refvx) = b(\refx_1)b(\refx_2)\dots b(\refx_d)
      \qquad b(x) = 4 x(1-x).
    \end{gather}
    Both share the following properties: they vanish on $\d\cell$,
    they are positive inside $\cell$, and $\max B_\cell=1$. From
    positivity, we deduce the existence of a constant $c$ depending on
    shape regularity and the shape function space $\shapespace(\cell)$, such that
    \begin{gather}
      \int_\cell p^2 B_\cell \dvx \ge c \norm{p}_\cell^2
      \qquad\forall p\in \shapespace(\cell).
    \end{gather}
    Furthermore, $B_\cell$ is polynomial, such that the inverse estimate holds.

    Choose now the test function
    $w_\cell = (\overline f + \Delta u_h) B_\cell$. Since it vanishes
    on the boundary and outside of $\cell$, the strong and weak form
    of the residual reduce to
    \begin{gather}
      \int_\cell r_\cell(u_h) w_\cell \dvx = \int_\cell \nabla(u-u_h)\cdot\nabla w_\cell \dvx.
    \end{gather}
    Adding the difference of $f$ and $\overline f$ on both sides yields
    \begin{gather}
      \int_\cell (\overline f+\Delta u_h)^2 B_\cell \dvx
      =\int_\cell \nabla(u-u_h)\cdot\nabla w_\cell \dvx
      + \int_\cell (\overline f - f) w_\cell \dvx.
    \end{gather}
    On the left, we estimate
    \begin{gather}
      c \norm{\overline f + \Delta u_h}_\cell^2 \le \int_\cell (\overline f+\Delta u_h)^2 B_\cell \dvx.
    \end{gather}
    On the right, we estimate the residual term by
    \begin{align*}
      \int_\cell \nabla(u-u_h)\cdot\nabla w_\cell \dvx
      & \le \snorm{u-u_h}_{1;\cell} \,\snorm{w_\cell}_{1;\cell}
      \\
      & \le c \snorm{u-u_h}_{1;\cell} \, h_\cell^{-1} \norm{\overline f + \Delta u_h}_\cell.
    \end{align*}
    The data oscillation is estimated in a straightforward way by
    \begin{align*}
      \int_\cell (\overline f-f) w_\cell \dvx
      &\le \operatorname{osc}_\cell f \norm{\overline f+\Delta u_h}_\cell
    \end{align*}
    Combining these three estimates yields
    \begin{gather}
      \label{eq:fem-aposteriori:9}
      c \norm{\overline f+\Delta u_h}_\cell
      \le h_\cell^{-1} \snorm{u-u_h}_{1;\cell}
      + \operatorname{osc}_\cell f,
    \end{gather}
    which is the desired estimate for the cell term of the estimator
    $\eta_{\cell}(u_h)$ if we multiply by $h_\cell$.

    The estimate for the face term is similar, using a bubble function
    for the face. Since this functon is nonzero on the face, its
    support extends over two mesh cells. On simplicial cells, it is
    constructed like the cell bubble in
    equation~\eqref{eq:bubble-simplex}, but extending the product only
    over indices of vertices on the face. Therefore, it is a
    polynomial of degree $d-1$ on the face and also in the interior of
    each cell. For hypercubes, it is the product of the quadratic
    polynomials $b(x)$ of variables on the face times a linear
    function decaying linearly from 1 to zero. Again, we normalize
    such that the maximum equals 1. Choosing\footnote{For linear
      elements, the derivative on the face is constant and thus this
      definition is obvious. For higher order polynomials, we need a
      an extension from the face into the interior, which can be
      obtained from the barycentric coordinates or the tensor product
      structure.} $w_F = 2\mvl{\n\cdot\nabla u_h}_F B_F$, we obtain
    \begin{align*}
      c\norm{\mvl{\n\cdot\nabla u_h}}_F^2
      &\le \int_F \mvl{\n\cdot\nabla u_h}^2 B_F \ds
      \\&= \int_F \mvl{\n\cdot\nabla u_h} w_F \ds
      \\&= \sum_{\cell\in\mesh_F} \int_\cell r_\cell w_F \dvx - \left<R,w_F\right>
      \\&= \sum_{\cell\in\mesh_F} \int_\cell r_\cell w_F \dvx
      - \int_{\domain_F} \nabla (u-u_h)\cdot\nabla w_F \dvx
      \\&= \sum_{\cell\in\mesh_F} \int_\cell (\overline f + \Delta u_h) w_F \dvx
      - \int_{\domain_F} \nabla (u-u_h)\cdot\nabla w_F \dvx
      + \sum_{\cell\in\mesh_F} \int_\cell (\overline f - f)w_F \dvx.
    \end{align*}
    Again, we estimate the three terms on the right hand side
    separately. For the first, we estimate the norm of $w_F$ by its
    trace on the edge $F$. From the already proven estimate for the
    cell residual~\eqref{eq:fem-aposteriori:9}, we obtain
    \begin{align*}
      \sum_{\cell\in\mesh_F} \int_\cell (\overline f + \Delta u_h) w_F \dvx
      & \le \sum_{\cell\in\mesh_F} \norm{\overline f + \Delta u_h}_\cell
        \,\norm{w_F}_{\cell}
      \\&\le \sum_{\cell\in\mesh_F} \norm{\overline f + \Delta u_h}_\cell
      \,c h_F^{\nicefrac12} \norm{\mvl{\n\cdot\nabla u_h}}_F
      \\&\le 
      + c  \sum_{\cell\in\mesh_F} \left(h_F^{-\nicefrac12} \snorm{u-u_h}_{1;\cell}
      + h_F^{\nicefrac12}\operatorname{osc}_{\cell} f\right)\norm{\mvl{\n\cdot\nabla u_h}}_F
    \end{align*}
    Similarly, the inverse estimate for $w_F$ yields
    \begin{align*}
      \int_{\domain_F} \nabla (u-u_h)\cdot\nabla w_F \dvx
      &\le \snorm{u-u_h}_{1;\domain_F} \norm{\nabla w_F}_{\domain_F}
      \\&\le \snorm{u-u_h}_{1;\domain_F}
      \,c h_F^{-\nicefrac12}\norm{\mvl{\n\cdot\nabla u_h}}_F.
    \end{align*}
    Finally, the data oscillation terms becomes
    \begin{align*}
      \sum_{\cell\in\mesh_F} \int_\cell (\overline f - f)w_F \dvx
      &\le \sum_{\cell\in\mesh_F} \norm{\overline f - f}_\cell \norm{w_F}_\cell
      \\&\le \sum_{\cell\in\mesh_F} \norm{\overline f - f}_\cell
      \,c h_F^{\nicefrac12}\norm{\mvl{\n\cdot\nabla u_h}}_F.
    \end{align*}
    Summing and scaling with $h_F^{\nicefrac12}$ yields the result.
\end{proof}

%%% Local Variables: 
%%% mode: latex
%%% TeX-master: "main"
%%% End: 
