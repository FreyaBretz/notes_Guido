\begin{Definition}{rk}
  A \define{Runge-Kutta method} is a one-step method of the form
  \begin{subequations}
    \label{eq:implicit:1}
    \begin{xalignat}{2}
      \label{eq:implicit:1a}
      \rkg_i &= y_0 + h \sum_{j=1}^{\rks} \rka_{ij} k_j
      & i &= 1,\dots,\rks
      \\
      \label{eq:implicit:1b}
      k_i &= f(t_0+h \rkc_i, \rkg_i)
      & i &= 1,\dots,\rks
      \\
      \label{eq:implicit:1c}
      y_1 &= y_0 + h \sum_{i=1}^{\rks} \rkb_i k_i
    \end{xalignat}
  \end{subequations}
  The method is called
  \begin{description}
  \item[\textbf{ERK}] if $j \ge i\Rightarrow\rka_{ij} = 0$ (``explicit'')
    \index{Runge-Kutta method!explicit (ERK)|textbf}
  \item[\textbf{DIRK}] if $j>i\Rightarrow\rka_{ij} = 0$ (``diagonal implicit'')
    \index{Runge-Kutta method!diagonal implicit (DIRK)|textbf}
    \index{Diagonal implicit (DIRK)}
    \index{DIRK|see{Runge-Kutta method}}
  \item[\textbf{SDIRK}] if DIRK and $\forall i,j: \rka_{ii} =
    \rka_{jj}$ (``singly diagonal implicit'')
    \index{Runge-Kutta method!singly diagonal implicit (SDIRK)|textbf}
    \index{SDIRK|see{Runge-Kutta method}}
  \item[\textbf{IRK}] ``implicit'' in all other cases.
    \index{Runge-Kutta method!implicit (IRK)|textbf}
    \index{IRK|see{Runge-Kutta method}}
  \end{description}
\end{Definition}

%%% Local Variables: 
%%% mode: latex
%%% TeX-master: "../notes"
%%% End: 
