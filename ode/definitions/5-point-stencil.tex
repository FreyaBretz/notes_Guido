\begin{Definition}{5-point-stencil}
  The \define{5-point stencil} consists of the sum of a 3-point
  stencil in $x$- and a 3-point stencil in $y$-direction. Its
  graphical representation is
  \begin{center}
    \includegraphics[width=.25\textwidth]{fig/5-point-stencil.tikz}
  \end{center}
  For a generic row of the linear system, where the associated point is not
  neighboring the boundary, this leads to
  \begin{gather}
    \label{eq:5-point-stencil:1}
    \frac1{h^2}\bigl[4y_k - y_{k-1} - y_{k+1} - y_{k-(n-1)} -
    y_{k+(n-1)}\bigr] = f_k
  \end{gather}
  If the point $k$ is next to the boundary, the corresponding entry
  from the matrix must be omitted.
\end{Definition}

%%% Local Variables:
%%% mode: latex
%%% TeX-master: "../notes"
%%% End:
