\chapter{Boundary Value Problems}
\label{chapter:rwa}
\section{Introduction}
\begin{intro}
  This chapter deals with problems of a fundamentally different type
  than the problems we examined in chapter~\ref{cha:IVP} and which we
  solved with previous numerical methods, namely boundary value
  problems. Here, we have prescribed values at the beginning and at
  the end of an interval of interest.

  The representation we use here is based primarily
  on~\cite{DeuflhardBornemann08} and~\cite{Rannacher12}.
\end{intro}

\begin{Definition}{bvp}
  \index{BVP|see{boundary value problem}}
  A \define{boundary value problem} (BVP) is a differential equation
  problem of the form: Find $u:[a,b]\to \R^d$, such that
  \begin{subequations}
    \label{eq:rwa:1}
    \begin{xalignat}{2}
      \label{eq:rwa:2}
      u'(t) &= f\bigl(t,u(t)\bigr)
      & t &\in (a,b) \\
      \label{eq:rwa:3}
      r\bigl(u(a), u(b)\bigr) &= 0.
    \end{xalignat}
  \end{subequations}
\end{Definition}

%%% Local Variables: 
%%% mode: latex
%%% TeX-master: "../notes"
%%% End: 

\begin{Definition}{bvp-linear}
  A BVP~\eqref{eq:rwa:1} is called linear, if the right hand side 
  $f$ as well as the boundary conditions are linear in $u$. It has the form:
  find $u:[a,b]\to \R^d$, such that
  \begin{subequations}
    \label{eq:rwa:6}
    \begin{xalignat}{2}
      \label{eq:rwa:7}
      u'(t) &= A(t)u(t) + b(t) & \forall t &\in (a,b) \\
      \label{eq:rwa:8}
      \rwaa u(a) + \rwab u(b) &= g.
    \end{xalignat}
  \end{subequations}
\end{Definition}

%%% Local Variables: 
%%% mode: latex
%%% TeX-master: "../notes"
%%% End: 


\begin{remark}
  Since boundary values are imposed at two different
  points in time, the concept of local solutions from
  \slideref{Definition}{local-solution} is not applicable.
  Thus, tricks as going forward from interval to interval, which is
  for instance done with Euler's method in the proof of Péano's
  theorem, are here not applicable.  For this reason nothing can be
  concluded with the local properties of the right hand side $f$ at
  the points $a$ and $b$. In fact, it is just possible in a few
  special cases to conclude that a solution exists.
\end{remark}

\section{Derivatives of the solutions of IVP with respect to data}
\subsection{Derivatives with respect to the initial values}
\begin{intro}
  In order to understand BVP, we have to introduce the notion of the
  derivative of the solution to an IVP with respect to its initial
  values. We will thus denote in these cases $u=u(t;v)$, where $t$ is
  the usual ``time'' variable and $v$ the initial value, which now is
  a variable as well. Thus, $u(t;v)$ is the solution to the IVP
  \begin{gather}
    \label{eq:derivatives:1}
    \begin{split}
      u'(t;v) = \tfrac{\partial}{\partial t} u(t;v) &= f\bigl(t,u(t;v)\bigr) \\
      u(t_0;v) &= v.
    \end{split}
  \end{gather}
  The purpose of this section is the study of the derivative
  \begin{gather*}
    \tfrac{\partial}{\partial v} u(t;v),
  \end{gather*}
  which is fundamentally different from $\partial/\partial t
  u(t;v)$.
  It can be obtained by solving the variational equation of the
  original IVP, defined as follows.
\end{intro}

% \begin{remark}
%   In terms of linear \putindex{perturbation theory} this section deals
%   with the derivative of the solution $u(t)$ of the
%   IVP~\eqref{eq:awa}, but not with respect to the argument $t$, but
%   rather with respect to the initial values. That is why we will write
%   the solution $u = u(t;s)$ of the IVP
%   \begin{gather}
%     \label{eq:derivatives:1}
%     \begin{split}
%       u'(t;s) = \tfrac{\partial}{\partial t} u(t;s) &= f\bigl(t,u(t;s)\bigr) \\
%       u(t_0;s) &= s_0
%     \end{split}
%   \end{gather}
%   with two arguments, the actual time variable $t$ and the initial value $s$.
% \end{remark}

\begin{Definition}{variational-equation}
  The \define{variational equation} to the first order system of
  ODE
  \begin{gather*}
    u' = f,
  \end{gather*}
  of dimension $d$ is the linear matrix-valued system of ODE
  \begin{subequations}
  %  \label{eq:derivatives:1}
  \begin{gather}
    \label{eq:derivatives:2}
      \fundam' = \nabla_u f\bigl(t,u(t)\bigr) \fundam
  \end{gather}
  for $d\times d$ matrices $\fundam$. Here is $u$ a solution of the
  equation~\eqref{eq:awa:ode} and 
  \begin{gather*}
    \nabla_u f(t,u) =
    \begin{pmatrix}
      \tfrac{\partial f_1}{\partial u_1} & \cdots
      & \tfrac{\partial f_1}{\partial u_d} \\
      \vdots & & \vdots \\
      \tfrac{\partial f_d}{\partial u_1} & \cdots
      &  \tfrac{\partial f_d}{\partial u_d}
    \end{pmatrix}
  \end{gather*}
  is the matrix of the derivatives of $f$ with respect to the
  components of $u$. The \define{fundamental matrix}
  $\fundamental(t;t_0)$\index{Y@$\fundamental(t;t_0)$} is solution of
  the IVP to this equation with
  \begin{gather}
    \label{eq:derivatives:3}
    \fundam(t_0) = \identity.
  \end{gather}
  \end{subequations}
\end{Definition}


\begin{remark}
  The fundamental matrix $\fundam$ can also be read column
  by column.  Then each column is a vector-valued function
  $\phi^{(i)}(t)$ and solves the IVP
  \begin{gather*}
    \begin{split}
      \tfrac{d}{dt}\phi^{(i)}(t) &= \nabla_u
      f\bigl(t,u(t)\bigr)\phi^{(i)}(t),
      \\
      \phi^{(i)}(t_0) &= e_i.
    \end{split}
  \end{gather*}
\end{remark}

\begin{remark}
  The definition of the fundamental matrix here is consistent with the
  one in Definition~\ref{definition:fundamental-system} for linear
  equations. Namely, for $f(u) = Au$, we have $\nabla_u f(u) = A$.
\end{remark}

\begin{Lemma}{fundamental}
  For fundamental matrices there hold the relations
  \begin{align}
    \fundamental(t;s) &= \fundamental(s;t)^{-1}\\
    \label{eq:derivatives:4}
    \fundamental(t;r) &= \fundamental(t;s)\fundamental(s;r),
  \end{align}
  where $r,s,t$ are arbitrary real numbers, such that the solution $u$
  of the original IVP exists on the maximal interval spanned by these
  numbers.
\end{Lemma}

%%% Local Variables: 
%%% mode: latex
%%% TeX-master: "../notes"
%%% End: 


\begin{proof}
  In order to prove the first equation, denote by $V(\tau;s)$ the
  solution to the IVP
  \begin{gather*}
    V'(\tau;s) = \nabla_u f(\tau,u(\tau)) V(\tau;s),
    \qquad V(s;s) = \fundamental(s;t).
  \end{gather*}
  Because of uniqueness, we must have
  $V(\tau;s) = \fundamental(\tau;t)$ for any $\tau$ between $s$ and
  $t$, in particular for $r=t$, such that $V(t;s) = \identity$. On the
  other hand, by linearity, we have
  $V(\tau;s) = \fundamental(\tau;s)\fundamental(s;t)$, and thus the
  equation is proven by
  \begin{gather*}
    \identity = V(t;s) = \fundamental(t;s)\fundamental(s;t).
  \end{gather*}

  Now, assume without loss of generality that $s$ is between $r$ and
  $t$. Indeed, if for instance $t$ is between $r$ and $s$, multiply
  equation~\eqref{eq:derivatives:4} from the left by $U(s;t)$ and
  prove the equation for
  \begin{gather*}
    U(s;t)U(t;r) = U(s;t)U(t;s)U(s;r) = U(s;r).
  \end{gather*}
  Take the auxiliary function $V(\tau;s)$ as defined above. By
  uniqueness, it is equal to $U(\tau;t)$ for all $\tau$. But, on the
  other hand, we have by linearity $V(\tau;s) = U(\tau;s)U(s;r)$, in
  particular for $\tau=t$.

  The statement follows from the definition as a solution of an IVP
  and the fact that solutions of linear IVP are linear combinable.
\end{proof}

\begin{Theorem}{derivative}
  \index{conditioning!AWA} Let be $f(t,u)$ continuous in $t$ and
  continuously differentiable in $u$.  Then, the solution $u(t;v)$ of
  the IVP~\eqref{eq:derivatives:1} depends differentiably on the
  initial value $v$ and the derivative is given by
  \begin{gather}
    \label{eq:derivatives:5}
    \tfrac{\partial}{\partial v} u(t;v) = \fundamental(t;t_0),
  \end{gather}
  where $\fundamental(t;t_0)$ is the fundamental matrix with respect
  to the initial time $t_0$.
\end{Theorem}



\begin{proof}
  We write the IVP in its full dependence on $v$ as
  \begin{align*}
    \frac{\partial u(t;v)}{\partial t} &= f\bigl(t,u(t;v)\bigr) \\
    u(t_0;v) &= v.
  \end{align*}
  From the second equation, we immediately obtain
  \begin{gather*}
    \frac{\partial u(t_0;v)}{\partial v} = \identity.
  \end{gather*}
Assuming differentiability of $f$ with respect to $u$, the first
equation yields
\begin{gather*}
  \frac{\partial}{\partial v}\frac{\partial u(t;v)}{\partial t}
  = \frac{\partial f\bigl(t,u(t;v)\bigr)}{\partial v}
  = \nabla_u f(t,u(t;v)) \frac{\partial u(t;v)}{\partial v}.
\end{gather*}
Thus, $\nicefrac{\partial}{\partial v} u$ solves the
IVP~\eqref{eq:derivatives:1}. Since the solution exists and is unique,
this derivative is well defined and thus, the function is differentiable.
\end{proof}

\subsection{Derivatives with respect to the right hand side function}
\begin{intro}
  We close this section by studying the differential dependence of the
  solution $u(t)$ of an ODE at time $t$ on the function $f(t,u)$, that
  is, the derivative of a value with respect to a function. In order
  to keep things simple, we reduce this question to a regular
  derivative of a function with respect to a real variable by using
  the Gâteaux derivative. To this end, let $F(v)$ be a functional on
  some function space, that is, a real valued function on this
  space. Then, the Gâteaux derivative of $F$ at a point $v$ in
  direction $w$ with $v$ and $w$ in this function space is defined as
  \begin{gather}
    \label{eq:derivatives:6}
    \frac{\partial}{\partial w} F(v)
    = \lim_{\epsilon\to 0} \frac{F(v+\epsilon w) - F(v)}{\epsilon}.
  \end{gather}
  Here, we have used the notation for directional derivatives in
  $\R^n$, since this is indeed the character of the Gâteaux
  derivative.

  Back to differential equations, our task is now to compute the
  derivative of $u(t)$ with respect to changes in $f$, denoted as
  \begin{gather}
    \label{eq:derivatives:7}
    \frac{\partial}{\partial g} u(t)
    = \lim_{\epsilon\to 0} \frac{u_\epsilon-u}{\epsilon}
    = \left.\frac{d}{d\epsilon} u_\epsilon(t)\right|_{\epsilon=0},
  \end{gather}
  where $u$ and $u_\epsilon$ respectively solve the IVPs
  \begin{xalignat*}{2}
    u'&=f(t,u) & u(t_0)& = u_0 \\
    u_\epsilon'&=f(t,u_\epsilon)+\epsilon g(t,u_\epsilon)
    & u_\epsilon(t_0)& = u_0.
  \end{xalignat*}
  For this derivative, we have the following theorem.
\end{intro}

\begin{Theorem}{derivative-f}
  Let $f(t,u)$ and $g(t,u)$ be continuous in their first and
  continuously differentiable in their second argument. Let $u$ be the
  solution of the IVP $u'=f(u)$ with $u(t_0) = u_0$. Then, the Gâteaux
  derivative of $u$ in $f$ with respect to a perturbation $g$ exists
  and there holds
  \begin{gather}
    \label{eq:derivative-f:1}
    \frac{\partial}{\partial g} u(t)
    = \int_{t_0}^t \fundamental(t;s) g\bigl(s,u(s)\bigr)\, \ds.
  \end{gather}
\end{Theorem}
%%% Local Variables:
%%% mode: latex
%%% TeX-master: "../notes"
%%% End:


\begin{proof}
  We set out by devising a differential equation for the Gâteaux
  derivative $\mathcal U(t)$. The differential equation for $u$ yields
  \begin{multline*}
    \mathcal U(t) := \left.\frac{d}{d\epsilon} u_\epsilon(t)\right|_{\epsilon=0}
    = \left.\frac{d}{d\epsilon}
    \Bigl(f\bigl(t,u_\epsilon(t)\bigr)
    + \epsilon g\bigl(t,u_\epsilon(t)\bigr)\Bigr) \right|_{\epsilon=0}
  \\
    = \left.\nabla_u f(t,u_\epsilon) \mathcal U(t)
    + g\bigl(t,u_\epsilon(t)\bigr) \right|_{\epsilon=0}
  \end{multline*}
  Furthermore, we have
  \begin{gather*}
     \mathcal U(t_0) = \frac{d}{d\epsilon} u_\epsilon(t_0) = 0.
  \end{gather*}

  According to Lemma~\ref{lemma:linear-representation}, the solution
  of this initial value problem can be represented with the
  integrating factor $M(t)$ as
  \begin{gather*}
    \mathcal U(t) = M^{-1}(t) \int_{t_0}^t M(s) g\bigl(s,u(s)\bigr)
    \,d s.
  \end{gather*}
  Noticing that $M(\tau)^{-1} = U(\tau;t_0)$, we obtain
  \begin{gather*}
    u(t) = \int_{t_0}^t \fundamental(t;s) g\bigl(s,u(s)\bigr) \,d s
  \end{gather*}
\end{proof}


%%% Local Variables: 
%%% mode: latex
%%% TeX-master: "notes"
%%% End: 

%%%%%%%%%%%%%%%%%%%%%%%%%%%%%%%%%%%%%%%%%%%%%%%%%%%%%%%%%%%%%%%%%%%%%%
%%%%%%%%%%%%%%%%%%%%%%%%%%%%%%%%%%%%%%%%%%%%%%%%%%%%%%%%%%%%%%%%%%%%%%
\section{Theory of boundary value problems}
%%%%%%%%%%%%%%%%%%%%%%%%%%%%%%%%%%%%%%%%%%%%%%%%%%%%%%%%%%%%%%%%%%%%%%
%%%%%%%%%%%%%%%%%%%%%%%%%%%%%%%%%%%%%%%%%%%%%%%%%%%%%%%%%%%%%%%%%%%%%%

\begin{remark}
  The very general boundary condition (BC) \index{Boundary
    condition}~\eqref{eq:rwa:3} usually has more simple forms. Often
  it is a \textbf{linear} \putindex{linear boundary condition}
   which we can note in the following form
  \begin{gather}
    \label{eq:rwa:4}
    \rwaa u(a) + \rwab u(b) = g
  \end{gather}
  with $d\times d$ matrices $\rwaa$ and $\rwab$ as well as a vector $g\in
  \R^d$. Another very common case is the one of 
  \textbf{separated}
  \index{boundary condition!separated}
  \index{separated boundary condition}
  boundary conditions, which has the form
  \begin{xalignat}{2}
    \label{eq:rwa:5}
    r_a\bigl(u(a)\bigr) &= 0,
    &
    r_b\bigl(u(b)\bigr) &= 0,
    \\\intertext{or}
    \rwaa u(a) &= g_a,
    &
    \rwab u(b) &= g_b.
  \end{xalignat}
\end{remark}

\begin{example}
  Take the second order differential equation
  \begin{gather*}
    u''(t) = -2, \quad \forall t\in (0,1),
  \end{gather*}
  with boundary conditions
  \begin{gather*}
    u(0) = u(1) = 0.
  \end{gather*}
  We can deduce from the differential equation that the solution is a
  parabola open to the bottom, and we verify easily that
  \begin{gather*}
    u(t) = t(1-t)
  \end{gather*}
  is a solution.
\end{example}

\begin{example}
  Take the first order differential equation
  \begin{gather*}
    u'(t) = u(t), \quad \forall t\in (0,1),
  \end{gather*}
  with boundary values
  \begin{gather*}
    u(0) = u(1) = 1.
  \end{gather*}
  From the theory of the first chapter, we know that the initial value
  problem with only the initial value at zero has the unique solution
  $u(t) = e^t$. Thus, this BVP does not have a solution.

  If we changed the condition at the right end of the interval to
  $u(1) = e$, the problem is solvable, albeit we have to admit that
  this solution seems somewhat accidental.
\end{example}

%%%%%%%%%%%%%%%%%%%%%%%%%%%%%%%%%%%%%%%%%%%%%%%%%%%%%%%%%%%%%%%%%%%%%%
\begin{remark}
  As the examples show, a satisfying theory for the existence of
  solutions will be difficult to obtain for any boundary values. For
  instance, in the linear case it is obvious that neither $\rwaa$, nor
  $\rwab$ may have full rank, because that would imply the unique
  existence either of the IVP with $\rwaa u(a) = g_a$ or
  $\rwab u(b) = g_b$, and thus no freedom to match the condition at
  the other end.

  As a consequence we now turn our attention to a ``restricted''
  solution theory and wonder: assume a solution of the problem
  exists. Which further conditions are necessary to obtain
  well-posedness of the problem in terms of Hadamard
  (definition~\vref{Definition:hadamard}).

  The key is the following definition which grants us the possibility
  of the approximation of a solution, at least after a quantification
  of the neighborhood.
\end{remark}

%%%%%%%%%%%%%%%%%%%%%%%%%%%%%%%%%%%%%%%%%%%%%%%%%%%%%%%%%%%%%%%%%%%%%% 
\begin{Definition}{local-uniqueness}
  \defindex{solution!locally unique (BVP)} A solution $u(t)$ of the
  BVP~\eqref{eq:rwa:1} is called \textbf{locally unique}
  \defindex{locally unique solution (BVP)} or
  \textbf{isolated}\defindex{isolated solution}, if there is no second
  solution $v(t)$ of the BVP, which is arbitrary close to $u(t)$. In
  mathematical language: there exists an $\epsilon>0$, such that for
  any two solutions of the BVP there holds
  \begin{gather*}
    \max_{t\in [a,b]}|u(t)-v(t)| < \epsilon
    \qquad\Rightarrow\qquad
    u(t) = v(t) \quad \forall t\in [a,b].
  \end{gather*}
\end{Definition}

%%%%%%%%%%%%%%%%%%%%%%%%%%%%%%%%%%%%%%%%%%%%%%%%%%%%%%%%%%%%%%%%%%%%%%
\begin{Lemma}{bvp-derivative}
  Let be $f(t,u)$ continuous in $t$ and continuously differentiable in
  $u$.  Let additionally $r(x,y)$ be continuously differentiable and
  set
  \begin{gather}
    \label{eq:rwa:9}
    \rwaa = \left.\frac{\partial r(x,y)}{\partial
        x}\right|_{x=u(a),y=u(b)},
    \qquad
    \rwab = \left.\frac{\partial r(x,y)}{\partial
        y}\right|_{x=u(a),y=u(b)}.
  \end{gather}
  Let $u(t)$ be a continuously differentiable solution of the
  BVP~\eqref{eq:rwa:1}.  Then the derivative of the boundary condition
  $r(u(a),u(b))$ with respect to the function value $u(t)$ inside the
  interval $[a,b]$ is
  \begin{gather}
    \label{eq:rwa:12}
    E(t)
    := \frac{\partial r\bigl(u(a),u(b)\bigr)}{\partial u(t)}
    = \rwaa \fundamental(a;t) + \rwab \fundamental(b;t),
  \end{gather}
  where $\fundamental(t;t_0)$ is the fundamental matrix.
\end{Lemma}


\begin{remark}
  The definition of the matrices $\rwaa$ and $\rwab$ above is
  consistent with the usage of the matrix $\rwaa$ and $\rwab$ in
  equation~\eqref{eq:rwa:4}.
\end{remark}

\begin{proof}
  We consider the auxiliary function $w_t(\tau;v)$ as solution of the
  IVP with initial value $v$ in $t$:
  \begin{gather*}
    \frac{\partial}{\partial \tau} w_t(\tau;v)
    = f\bigl(\tau,w_t(\tau;v)\bigr), \qquad w_t(t;v) = v.
  \end{gather*}
  Choosing $v=u(t)$, we have by uniqueness $w_t(a;v) = u(a)$ and
  $u(b) = w_t(b;v)$. Our task of computing the derivative with respect
  to $u(t)$ has thus become computing the derivative with respect to
  $v$. The derivative of the boundary condition can therefore be
  written as
  \begin{align*}
    \frac{\partial r\bigl(u(a),u(b)\bigr)}{\partial u(t)}
    &=
    \frac{\partial r\bigl(w_t(a;v),w_t(b;v)\bigr)}{\partial v}
    \\
    &= \rwaa \frac{\partial w_t(a;v)}{\partial v}
    + \rwab \frac{\partial w_t(b;v)}{\partial v}
    \\
    &= \rwaa \fundamental(a;t) + \rwab \fundamental(b;t),
  \end{align*}
  where the last equality is due to Theorem~\ref{Theorem:derivative}.
\end{proof}

\begin{remark}
  In order to study local uniqueness of solutions and well-posedness
  of BVP, we have to change our view on boundary conditions and
  consider them as functions of solutions to the differential
  equation. Thus, we will consider the function
  \begin{gather*}
    \rho(v) := r\bigl(v(a),v(b)\bigr),
  \end{gather*}
  mapping solutions of the differential equation to their boundary
  values. $\rho$ is a continuous function, and, as the following
  theorem shows, even differentiable.
\end{remark}

\begin{Definition}{local-uniqueness}
  \defindex{solution!locally unique (BVP)} A solution $u(t)$ of the
  BVP~\eqref{eq:rwa:1} is called \textbf{locally unique}
  \defindex{locally unique solution (BVP)} or
  \textbf{isolated}\defindex{isolated solution}, if there is no second
  solution $v(t)$ of the BVP, which is arbitrary close to $u(t)$. In
  mathematical language: there exists an $\epsilon>0$, such that for
  any two solutions of the BVP there holds
  \begin{gather*}
    \max_{t\in [a,b]}|u(t)-v(t)| < \epsilon
    \qquad\Rightarrow\qquad
    u(t) = v(t) \quad \forall t\in [a,b].
  \end{gather*}
\end{Definition}


\begin{proof}
  First assume that $E(t)$ is regular for some $t$. Then we have for
  $\tau \neq t$:
  \begin{align*}
    E(\tau) &= \rwaa \fundamental(a;\tau) + \rwab \fundamental(b;\tau)
    \\
    &= \rwaa \fundamental(a;t)\fundamental(t;\tau) + \rwab
    \fundamental(b;t)\fundamental(t;\tau)
    = E(t) \fundamental(t;\tau),
  \end{align*}
  where the first factor is regular by assumption, the second one as the
  fundamental matrix of a linear ODE.

  Now, we consider the matrix $E(t)=E_u(t)$ as a function of $u(t)$, which is
  continuous by assumption. Therefore, if it is regular in $u(t)$, it
  is regular in a neighborhood of $u(t)$ of some positive diameter
  $\epsilon$. Let now $v$ be a second solution of the differential
  equation with $\abs{u(t)-v(t)}<\epsilon$. Then,
  \begin{gather*}
    \rho(u)-\rho(v) = E_\phi(t) \bigl(u(t)-v(t)\bigr),
  \end{gather*}
  where $\phi(t)$ is between $u(t)$ and $v(t)$ and thus $E_\phi(t)$ is
  regular. If both functions solve the BVP, then the left hand side is
  zero, and thus $v(t)=u(t)$.
\end{proof}

Now that we established the local uniqueness of the solution,
it yet remains to show the continuous dependency of data (stability).  
Here we are in particular interested, in analogy to the 
stability theorem, in the derivative
of the solution at time $t$ after perturbations of values on the boundary. 
For this purpose we have:

\begin{Theorem}{bvp-stability-1}
  \index{conditioning!BVP}
  Let the assumptions of Lemma~\ref{Lemma:bvp-derivative} hold and let
  $u(t)$ be the solution of the BVP~\eqref{eq:rwa:1} with
  \begin{gather*}
    u(a) = g_a\quad \text{and} \quad u(b) = g_b.
  \end{gather*}
  Then, there holds
  \begin{gather}
    \label{eq:rwa:20}
    \frac{\partial u(t)}{\partial g_a} = E^{-1}(t) \rwaa,
    \qquad
    \frac{\partial u(t)}{\partial g_b} = E^{-1}(t) \rwab.
  \end{gather}
  In particular, the conditioning with respect to changes of size
  $\epsilon$ in the boundary conditions $g_a$ and $g_b$ is
  \begin{gather}
    \delta u(t) \le \epsilon
    \max\bigl\{\norm{E^{-1}(t) \rwaa}, \norm{E^{-1}(t)\rwab}\bigr\}.
  \end{gather}
\end{Theorem}


\begin{proof}
  We demonstrate the proof for the derivative with respect to the left
  boundary value. The second equation can be proven the same way.
  With the chain rule we obtain
  \begin{gather*}
    \frac{\partial u(t)}{\partial g_a}
    = \frac{\partial u(t)}{\partial r(g_a,g_b)}
    \frac{\partial r(g_a,g_b)}{\partial g_a}
  \end{gather*}
  The second derivative is~\eqref{eq:rwa:9} $\rwaa$. For the first one
  we notice that $E(t)$ is the derivative of the inverse mapping of
  $\rho(u)$. By application of the implicit function theorem, we
  obtain the result.
\end{proof}

% \begin{example}
%   Consider the second order boundary value problem
%   \begin{gather*}
%     u'' = k^2 u,
%     \qquad u(a) = g_a, \quad u(b) = g_b.
%   \end{gather*}
%   We convert this into the system
%   \begin{gather*}
%     \frac{d}{dt}
%     \begin{pmatrix}
%       u_1 \\ u_2
%     \end{pmatrix}
%     =
%     \begin{pmatrix}
%       & k \\ k
%     \end{pmatrix}
%     \begin{pmatrix}
%       u_1 \\ u_2
%     \end{pmatrix}.
%   \end{gather*}
%   The fundamental matrix is
%   \begin{gather}
%     \label{eq:rwa:10}
%     \fundamental(\tau;t) =
%     \begin{pmatrix}
%       e^{k(\tau-t)} & e^{k(t-\tau)} \\
%       ke^{k(\tau-t)} & - ke^{k(t-\tau)}
%     \end{pmatrix}.
%   \end{gather}
% \end{example}

\begin{Theorem}{bvp-stability-2}
  Let $f(t,u)$ and $g(t,u)$ be continuous in $t$ and continuously
  differentiable in $u$. Then, the variation of the value $u(t)$ of
  the solution $u$ of the boundary value problem
  \begin{gather*}
    u'=f(t,u),\qquad r\bigl(u(a), u(b)\bigr)=0,
  \end{gather*}
  with respect to perturbations $f+g$ is
  \begin{gather}
    \label{eq:bvp-stability-2:1}
    \frac{\partial}{\partial g} u(t)
    = \int_{a}^b G(t,s) g\bigl(s,u(s)\bigr)\,d s,
  \end{gather}
  where
  \begin{gather}
    \label{eq:bvp-stability-2:2}
    G(t,s) =
    \begin{cases}
      -E(t)^{-1} \rwaa \fundamental(a;s) & a \le s \le t \\
      \phantom{-}E(t)^{-1} \rwab \fundamental(b;s) & t < s \le b
    \end{cases}.
  \end{gather}
\end{Theorem}

%%% Local Variables:
%%% mode: latex
%%% TeX-master: "../notes"
%%% End:


\begin{proof}
  We begin by estimating the influence of perturbations of the right
  hand side on the boundary values. Using the corresponding
  Theorem~\ref{Theorem:derivative-f} for IVP, where we swap the
  meaning of $t$ and the interval boundaries, we obtain
  \begin{align*}
    \frac{\partial u(a)}{\partial g}
    &= \int_t^a \fundamental(a;s) g\bigl(s,u(s)\bigr) \ds,
    \\
    \frac{\partial u(b)}{\partial g}
    &= \int_t^b \fundamental(b;s) g\bigl(s,u(s)\bigr) \ds,
  \end{align*}
  By the chain rule and Lemma~\ref{Lemma:bvp-derivative}
  and Theorem~\ref{Theorem:bvp-stability-1}, we get
  \begin{gather*}
    \frac{\partial r\bigl(u(a), u(b)\bigr)}{\partial g}
    = \frac{\partial r\bigl(u(a), u(b)\bigr)}{\partial u(t)}
    \frac{\partial u(t)}{\partial g}
    = E(t)\frac{\partial u(t)}{\partial g}.
  \end{gather*}
Assembling everything, we obtain
\begin{multline*}
  \frac{\partial u(t)}{\partial g}
  = E(t)^{-1} \frac{\partial r\bigl(u(a), u(b)\bigr)}{\partial g}
  = E(t)^{-1} \left(
    \rwaa \frac{\partial u(a)}{\partial g}
    + \rwab \frac{\partial u(b)}{\partial g}\right)
  \\
  = E(t)^{-1} \left(
    \rwaa \int_t^a \fundamental(a;s) g\bigl(s,u(s)\bigr) \ds
    + \rwab \int_t^b \fundamental(b;s) g\bigl(s,u(s)\bigr) \ds
    \right).
  \end{multline*}
\end{proof}

\begin{remark}
  Theorems~\ref{Theorem:local-uniqueness}, \ref{Theorem:bvp-stability-1}
  and~\ref{Theorem:bvp-stability-2} represent the verification of the
  second and third Hadamard conditions\index{Hadamard conditions}.
  Thus, even if the existence of a solution for the BVP is not always
  guaranteed, solutions can be approximated under certain conditions.
  
  The case for linear boundary value problems is much simpler and we
  have an existence an uniqueness result.
\end{remark}

\begin{Definition}{bvp-linear}
  A BVP~\eqref{eq:rwa:1} is called linear, if the right hand side 
  $f$ as well as the boundary conditions are linear in $u$. It has the form:
  find $u:[a,b]\to \R^d$, such that
  \begin{subequations}
    \label{eq:rwa:6}
    \begin{xalignat}{2}
      \label{eq:rwa:7}
      u'(t) &= A(t)u(t) + b(t) & \forall t &\in (a,b) \\
      \label{eq:rwa:8}
      \rwaa u(a) + \rwab u(b) &= g.
    \end{xalignat}
  \end{subequations}
\end{Definition}

%%% Local Variables: 
%%% mode: latex
%%% TeX-master: "../notes"
%%% End: 


\begin{proof}
  By Theorem~\ref{Lemma:bvp-derivative} and linearity of the BVP,
  we deduce that the mapping $\rho$ from $u(t)$ to the boundary
  condition is affine and can be written in the form
  \begin{gather*}
    r\bigl(u(a),u(b)\bigr) = E(t) u(t) + b(t),
  \end{gather*}
  where $b(t)$ is some vector in $\R^d$ or $\C^d$ independent of
  $u$. Since everything in this equation exept $u(t)$ is given, the
  unique solvability is equivalent to the invertibility of $E(t)$,
  which by Theorem~\ref{Theorem:local-uniqueness} is equivalent to
  regularity of $E(a)$ .
\end{proof}


%%%%%%%%%%%%%%%%%%%%%%%%%%%%%%%%%%%%%%%%%%%%%%%%%%%%%%%%%%%%%%%%%%%%%%
%%%%%%%%%%%%%%%%%%%%%%%%%%%%%%%%%%%%%%%%%%%%%%%%%%%%%%%%%%%%%%%%%%%%%%
\section{Shooting methods}
\subsection{Single shooting method}
%%%%%%%%%%%%%%%%%%%%%%%%%%%%%%%%%%%%%%%%%%%%%%%%%%%%%%%%%%%%%%%%%%%%%%
%%%%%%%%%%%%%%%%%%%%%%%%%%%%%%%%%%%%%%%%%%%%%%%%%%%%%%%%%%%%%%%%%%%%%%

\begin{example}
  We illustrate the shooting method on a simple, scalar example
  \begin{gather*}
    u'' = -g, \quad u(0) = 0, \quad u(1) = 0.
  \end{gather*}
	What we can solve is the IVP
  \begin{gather*}
    u'' = -g, \quad u(0) = 0, \quad u'(0) = s.
  \end{gather*}
	The latter has a unique solution u(t;s) for each initial value
  $s$. Now it is our task to find a value $s^*$, such that
  $u(1;s^*) = 0$. With other words, we search for a root of the function
  \begin{gather*}  
    F(s) = u(1;s).
  \end{gather*}
  This can be done with an arbitrary, convergent iteration method.
  For an example with the Bisection method.  Of course the Newton
  method would be a better choice but for that we need to calculate
  the derivatives of $F$.  This can be achieved with
  theorem~\ref{Theorem:derivative} by calculating the
  fundamental matrix $\fundam$.
\end{example}

\begin{Algorithm}{single-shooting}
  The single shooting method with standard \putindex{Newton method}
  consists of the following steps:
  \begin{enumerate}
  \item Start with an initial guess $v^{(0)}\in \R^d$.
  \item For $v^{(n)}$ given, solve the IVP and the associated variation equation
    \begin{xalignat*}{2}
      \tfrac\partial{\partial t}u^{(n)}(t) &=
      f\bigl(t,u^{(n)}(t)\bigr)
      & u^{(n)}(a) &= v^{(n)},\\
      \tfrac\partial{\partial t}\fundam^{(n)}(t;a)
      &= \nabla_uf\bigl(t,u^{(n)}(t)\bigr)\fundam^{(n)}(t;a)
      &\fundam^{(n)}(a;a) &= \identity.
    \end{xalignat*}
  \item Set
    \begin{gather}
      \label{eq:rwa:14}
      v^{(n+1)} = v^{(n)}
      - \bigl(\rwaa + \rwab \fundam^{(n)}(b;a)\bigr)^{-1}
      r\bigl(v^{(n)},u^{(n)}(b)\bigr)
    \end{gather}
  \item Stop the iteration if the value
    $r\bigl(s,u(t;u_0^{(n+1)})\bigr)$ is sufficiently small,\\
    otherwise repeat from 2.
  \end{enumerate}
\end{Algorithm}

%%% Local Variables:
%%% mode: latex
%%% TeX-master: "../notes"
%%% End:


\begin{remark}
  The task of the shooting method is solved normally with the help of
  Newton's method, which searches for a root (with respect to $v$) of
  the function
  \begin{gather}
    \label{eq:rwa:11}
    F(v) = r\bigl(v, u(b;v)\bigr)
  \end{gather}
	For Newton's method we require the partial derivatives
  \begin{gather*}
    \frac\partial{\partial v^{(i)}} F(v) =
    \partial_1 r\bigr(v,u(b;v)\bigr)
    + \partial_2r\bigr(v,u(b;v)\bigr) \frac\partial{\partial v^{(i)}} u(b;v),
  \end{gather*}
  % For linear and separate boundary conditions this leads to a significantly 	
  %       more readable form: 
  % \begin{gather*}
  %   \frac\partial{\partial u_0^{(i)}} F(s) = \rwaa s + \rwab
  %   \frac\partial{\partial u_0^{(i)}} u(b;u_0)
  % \end{gather*}
  which involves the \putindex{fundamental matrix} $\fundamental(b;a)$
  of derivatives of the solution at point $b$ with respect to the
  initial values (see Lemma~\ref{Lemma:bvp-derivative}).
\end{remark}

\begin{Algorithm}{single-shooting}
  The single shooting method with standard \putindex{Newton method}
  consists of the following steps:
  \begin{enumerate}
  \item Start with an initial guess $v^{(0)}\in \R^d$.
  \item For $v^{(n)}$ given, solve the IVP and the associated variation equation
    \begin{xalignat*}{2}
      \tfrac\partial{\partial t}u^{(n)}(t) &=
      f\bigl(t,u^{(n)}(t)\bigr)
      & u^{(n)}(a) &= v^{(n)},\\
      \tfrac\partial{\partial t}\fundam^{(n)}(t;a)
      &= \nabla_uf\bigl(t,u^{(n)}(t)\bigr)\fundam^{(n)}(t;a)
      &\fundam^{(n)}(a;a) &= \identity.
    \end{xalignat*}
  \item Set
    \begin{gather}
      \label{eq:rwa:14}
      v^{(n+1)} = v^{(n)}
      - \bigl(\rwaa + \rwab \fundam^{(n)}(b;a)\bigr)^{-1}
      r\bigl(v^{(n)},u^{(n)}(b)\bigr)
    \end{gather}
  \item Stop the iteration if the value
    $r\bigl(v,u(t;u_0^{(n+1)})\bigr)$ is sufficiently small,\\
    otherwise repeat from 2.
  \end{enumerate}
\end{Algorithm}

% \begin{corollary}
%   For linear a BVP~\eqref{eq:rwa:6} the initial value $v$, such that
%   $u(t;v)$ solves the BVP, is the solution of the linear system of
%   equations
%   \begin{gather}
%     \label{eq:rwa:13}
%     \bigl(\rwaa + \rwab\fundamental(b;a)\bigr) v = g - \rwab u(b;0).
%   \end{gather}
% \end{corollary}

% \begin{proof}
% 	For linear equations we know that the Newton's method converges in one step,
% 	for arbitrary initial values.
% 	We can use this to execute the upper algorithm with initial value
% 	 $u_0^{(0)} = 0$.
% 	Under application of $r(u(a),u(b)) = B_a u(a) + B_b u(b) -
%   g$ we obtain the upper formula by inserting in~\eqref{eq:rwa:14}.
% \end{proof}

\begin{remark}
  The IVP in this algorithm usually cannot be solved
  analytically. Thus, we have to use what we learned in the previous
  chapters to choose a time stepping scheme.

  Newton's method is known to converge locally, not
  globally. Therefore, a good initial value is needed for this
  algorithm to converge. This is the more true, since the solution of
  the IVP may grow much faster and may be less stable than that of the
  BVP (see homework), and is only approximated. There are two ways out
  of this problem: first, a Newton method should never be implemented
  without any globalization strategy, which modifies the update to
  increase the domain of convergence. The second part of the solution
  consists in choosing a method which is more robust than single
  shooting in the next section.
\end{remark}

\begin{remark}
  Step 2 of the single shooting algorithm requires solving the
  \putindex{variational equation} of our original ODE. If $f$
  describes a complex nonlinear process, the computation and
  implementation of its derivative may be a daunting and error prone
  task. This can be avoided by three different algorithmical
  tools. For each of them, we describe the main advantages and
  disadvantages:
  \begin{description}
  \item[Automatic differentiation] Software like for instance
    the module Sacado of the Trilinos package, has the standard rules
    of differentiation (Leibniz, quotient, chain rules, derivatives of
    polynomials and standard functions) built in. By implementing
    $f(u)$ in a conforming way, the software can automatically
    generate the code for its derivatives. Clearly, the advantage is
    that there is no approximation involved in those derivatives and
    thus equation and variational equation are always consistent. On
    the other hand, derivatives, which are not simplified
    analytically, can become fairly complex. Here, we rely on a good
    implementation of the automatic differentiation as well as on the
    optimizing capabilities of the compiler.
  \item[Internal numerical differentiation] While using a time
    stepping scheme for solving the IVP, for example a one-step
    method, in each time step, we compute
      \begin{gather*}
        y^k = \Phi(y^{k-1})
      \end{gather*}
      as well as approximations
      \begin{gather*}
        \frac{\partial}{\partial y_i} y^k
        \approx \frac{\Phi(y^{k-1}+\epsilon e_i) - \Phi(y^{k-1})}{\epsilon}
      \end{gather*}
with the same integration method $\Phi$ and the same step sizes $h_k$.       This approximation can also be replaced by more accurate
      differentiation formulas. Such an implementation requires $d$
      additional evaluations of $\Phi$ to compute the full gradient,
      which is feasible only for moderately sized $d$. Furthermore,
      the choice of $\epsilon$ is tricky, since the approximation is
      inaccurate for large values and unstable for small.
      Implementations of time stepping schemes which compute both $y$
      and its derivatives are available in the optimization community
      as well among reseach groups implementing filtering techniques
      in stochastic methods.
      
    \item[External numerical diferentiation] Here we forget about the
      variational equation altogether and approximate by difference
      quotients the derivative of $u(b)$ with respect to changes in
      the initial value directly:
      \begin{gather*}
        \frac{\partial}{\partial v_i} u(b)
        \approx \frac{u(b;v+\epsilon e_i) - u(b;v)}{\epsilon}.
      \end{gather*}
In practice, $u(b;v+\epsilon e_i)$ and $u(b;v)$ are computed numerically by any integration method and in general using different step sizes.       The robust choice of $\epsilon$ is more critical here and the
      search for a general algorithm to address this issue has
      failed. Therefore, this method is only used rarely nowadays
      (judgement from~\cite{DeuflhardBornemann08}).
  \end{description}

  A final remark on computing the Jacobian and its inverse: if an
  iterative method is used to solve the linear system in each Newton
  step, the complete matrix is not needed, but only the matrix applied
  to a direction vector. This is something that can be used to
  accelerate all methods described above by avoiding the computation
  of unneeded values.
\end{remark}

%%%%%%%%%%%%%%%%%%%%%%%%%%%%%%%%%%%%%%%%%%%%%%%%%%%%%%%%%%%%%%%%%%%%%%
%%%%%%%%%%%%%%%%%%%%%%%%%%%%%%%%%%%%%%%%%%%%%%%%%%%%%%%%%%%%%%%%%%%%%%
\subsection{Multiple shooting method}
\index{shooting method!multiple}

\begin{example}
 Take on the interval $[0,2]$ the (admittedly somewhat artificial) boundary value problem
 \begin{gather*}
   u' = u^2,\qquad u(2) = 1,
 \end{gather*}
 which has the bounded solution
 \begin{gather*}
   u(t) = \frac1{3-t}.
 \end{gather*}
 We might be tempted to start our shooting method with $v=1$ after
 realizing that $v=0$ leads nowhere. But then, we get
 \begin{gather*}
   u^{(0)}(t) = \frac1{1-t},
 \end{gather*}
 which only exists on the interval $[0,1)$. Clearly, the single
 shooting method is not suited to solve this otherwise harmless BVP.

 This observation leads to the idea of applying the shooting method on
 smaller subintervals and gluing those together.
\end{example}

\begin{Definition}{multiple-shooting}
  Choose a partitioning of the interval $[a,b]$ such that
  \begin{gather*}
    a=t_0<t_1<t_2<\dots<t_m=b.
  \end{gather*}
  On each subinterval $I_k = [t_{k-1}, t_k]$, $k=1,\ldots,m$ define
  the IVP
  \begin{gather*}
    u_k' = f(t, u_k), \qquad u_k\bigl(t_{k-1}\bigr) = v_k,
  \end{gather*}
  The \define{multiple shooting method} consists of finding vectors
  $v_1,\dots, v_m$, such that
  \begin{xalignat*}{2}
    v_{k+1}&= u_{k}\bigl(t_{k}\bigr)& k&=1,\dots,m-1,\\
    r(v_1, u_m(b)) &= 0.
  \end{xalignat*}
  The function
  \begin{gather*}
    u(t) = u_k(t) \qquad\text{for } t\in I_k
  \end{gather*}
  is continuous, solves the ODE, and obeys the boundary conditions.
\end{Definition}

%%% Local Variables:
%%% mode: latex
%%% TeX-master: "../notes"
%%% End:


\begin{remark}
  The numbering of intervals, vectors, and time partitioning has been
  chosen to be consistent with the previous definitions in this
  class. The governing entity is the subinterval $I_k = [t_{k-1},
  t_k]$ with solutions $u_k$ and initial value $v_k$. As a result, the
  initial values $v_k$ are imposed at $t_{k-1}$.

  Other authors have used the time subdivisions $t_k$ as governing
  entity, which leads to a shift of several of the indices. Whenever
  reading or writing about multiple shooting methods, connections
  between indices must be considered carefully, since every system
  will exhibits inconsistencies at some points.
\end{remark}

\begin{remark}
  The multiple shooting method is a typically nonlinear system of
  equations of dimension $d\cdot m$, where $d$ is the dimension of the
  ODE and $m$ the number of subintervals. Nevertheless, the
  formulation and typically the implementation hides a much larger
  number of unknowns involved in the discretization of the subdomain
  solves. We will keep ignoring this inner discretization of the
  intervals.
\end{remark}

\begin{remark}
  In order to keep the implementation and presentation simpler, we
  introduce an additional shooting vector $v_{m+1} = u_m(t_m)$. As a
  result, the boundary condition in the shooting method simplifies the
  last conditions to
  \begin{gather}
    \label{eq:rwa:15}
      r\bigl(v_1, v_{m+1}\bigr) = 0.
  \end{gather}
  The advantage becomes obvious, when we compute derivatives for the
  Newton method. With the new vector, we compute
  \begin{gather*}
    \frac{\partial r\bigl(v_1, v_{m+1}\bigr)}{\partial v_1}
    = \rwaa, \qquad
    \frac{\partial r\bigl(v_1, v_{m+1}\bigr)}{\partial v_{m+1}}
    = \rwab.
  \end{gather*}
  Without, we have to compute
  \begin{gather*}
    \frac{\partial r\bigl(v_1, u_m(b)\bigr)}{\partial v_{m}}
    = \rwab \fundamental(t_{m}; t_{m-1}).
  \end{gather*}
\end{remark}

\begin{Definition}{multiple-shooting-newton}
  A step of Newton's method for the multiple shooting system consists
  of the update
  \begin{gather}
    \label{eq:multiple-shooting-newton:1}
    v^{(n+1)} = v^{(n)} - \nabla F(v^{(n)})^{-1} F(v^{(n)}),
  \end{gather}
  where $v^{(n)} = [v^{(n)}_1,\dots,v^{(n)}_{m+1}]^T$,
  \begin{gather}
    \label{eq:multiple-shooting-newton:2}
    F(v) =
    \begin{bmatrix}
      F_1(v_1, v_2)\\
      \vdots\\
      F_{m}(v_{m}, v_{m+1})\\
      F_{m+1}(v_1,v_{m+1})
    \end{bmatrix}
    ,\quad
    \nabla F(v) =
    \begin{bmatrix}
      G_1 & -\identity \\
      &\ddots & \ddots \\
      &&G_{m} & -\identity \\
      \rwaa &&& \rwab
    \end{bmatrix},
  \end{gather}
  and,
  \begin{xalignat*}{2}
    F_k(v_k,v_{k+1})&= u_k(t_{k}) - v_{k+1} & k&=1,\dots,m, \\
    F_{m+1}(v_1, v_{m+1}) &= r\bigl(v_1, v_{m+1}\bigr),\\
    G_k &= \fundamental(t_k;t_{k-1}) & k&=1,\dots,m.
  \end{xalignat*}
\end{Definition}

\begin{remark}
  Whenever the shooting vectors $v = [v_1,\dots v_{m+1}]^T$ solve the
  multiple shooting equations $F(v) = 0$, the solution $u(t)$ of the
  multiple shooting method is also a solution of the original
  BVP. This is a consequence of the continuity enforced by these
  equations. Therefore, the existence of a solution to the original
  BVP implies existence of a solution to the multiple shooting
  problem.
\end{remark}

\begin{intro}
  We close this section by discussing two important extensions to the
  multiple shooting method. First, a system may have boundary values
  in more than just the two end points of the einterval of
  computation. In such a case, the additional points are included as
  shooting nodes, such that the boundary conditions can be applied to
  the shooting vectors $v_k$ instead of solutions of the initial value
  problems on subintervals.

  The second extension is to problems, where the equation itself has a
  parameter which we try to determine by the shooting method. It turns
  out that both extensions fit very well into concept of multiple
  shooting and change the underlying Newton method only slightly.
\end{intro}

\begin{Definition}{multiple-shooting-multiple}
  A multi-point boundary value problem has boundary conditions of the
  form
  \begin{gather}
    \label{eq:multiple-shooting-multiple:1}
    r\bigl(u(t_0), u(t_{k_1}), u(t_{k_2}),\dots, u(t_{k_\ell})\bigr) = 0,
  \end{gather}
  with $a=t_0$, $m=k_\ell$, and $b=t_m$
  A multiple shooting method for such a problem can be designed by
  including all values $t_{k_i}$ into the partitioning of the time
  interval. The corresponding shooting function and Jacobian are
  {\small
    \begin{gather}
      \label{eq:multiple-shooting-multiple:2}
      F(v) =
      \begin{bmatrix}
        F_1(v_1, v_2)\\
        \vdots\\
        F_{m}(v_{m}, v_{m+1})\\
        F_{m+1}(v_{1},\ldots,v_{m+1})
      \end{bmatrix}
      ,\quad
      \nabla F(v) =
      \begin{bmatrix}
        G_1 & -\identity \\
        &G_2 & -\identity \\
        &&\ddots & \ddots \\
        &&&G_{m} & -\identity \\
        \rwaa &\cdots&B_{k_i}&\cdots& \rwab
      \end{bmatrix}.
    \end{gather}
  }
  Here,
  \begin{xalignat*}{2}
    F_k(v_k,v_{k+1})&= u_k(t_{k}) - v_{k+1} & k&=1,\dots,m, \\
    F_{m+1}(v_1,\ldots,v_{m+1}) &= r\bigl(v_1, \ldots, v_{m+1}\bigr),\\
    G_k &= \fundamental(t_k;t_{k-1}) & k&=1,\dots,m.
  \end{xalignat*}
\end{Definition}

\begin{Definition}{bvp-parameter}
  Given a vector $p\in \R^q$, a parameter dependent boundary value
  problem depending on $p$ has the form
  \begin{gather}
    \label{eq:bvp-parameter:1}
    \begin{split}
      u' &= f(t, u; p),\\
      r\bigl(u(a), u(b); p\bigr) &= 0.
    \end{split}
  \end{gather}
  Here, $r(\ldots) \in \R^{d+q}$ where $d$ is the dimension of the ODE system.
\end{Definition}

\begin{remark}
  Every parameter dependent ODE can be transformed into a regular ODE
  by introducing the ${d+q}$-dimensional vector $v=(u,p)$ solving the
  ODE
  \begin{gather*}
    v' =
    \begin{pmatrix}
      f\bigl(t,u(t)\bigr)\\0
    \end{pmatrix}.
  \end{gather*}
  Solving it in this form is nevertheless inefficient, since it
  involves carrying a differentia lequation for $p$ through all
  integrators. Instead, we can modify the shooting method in a way,
  that we incorporate it directly.
\end{remark}

\begin{Definition}{multiple-shooting-parameter}
  The Jacobian of the Newton method for parameter dependent BVP is
  \begin{gather}
    \label{eq:multiple-shooting-parameter:1}
    \nabla F(v) =
    \begin{bmatrix}
      G_1 & -\identity &&& P_1\\
      &\ddots & \ddots \\
      &&G_{m} & -\identity & P_{m-1}\\
      \rwaa &&& \rwab & P_m
    \end{bmatrix}.
  \end{gather}
  Here,
  \begin{xalignat*}{2}
    P_k &= \frac{\partial u_k(t_k)}{\partial p} & k&=1,\dots,m,\\
    P_m &= \frac{\partial r(\dots)}{\partial p},\\
    G_k &= \fundamental(t_k;t_{k-1};p) & k&=1,\dots,m.
  \end{xalignat*}
\end{Definition}

%%%%%%%%%%%%%%%%%%%%%%%%%%%%%%%%%%%%%%%%%%%%%%%%%%%%%%%%%%%%%%%%%%%%%%
%%%%%%%%%%%%%%%%%%%%%%%%%%%%%%%%%%%%%%%%%%%%%%%%%%%%%%%%%%%%%%%%%%%%%%
%%% Local Variables: 
%%% mode: latex
%%% TeX-master: "notes"
%%% End: 
