\begin{Lemma}{taylor-u}
  The Taylor expansion of a single component of $u_1 = u(h)$ with
  respect to $h$ is
  \begin{gather}
    \label{eq:taylor-u:1}
    \begin{split}
      (u_1)_n &= (u_0)_n
      \\& + h f_n
      \\& + \frac{h^2}2 \sum_\lambda \partial_{\lambda} f_n f_\lambda
      \\& + \frac{h^3}6 \sum_{\lambda,\mu} \bigl[\partial_{\lambda\mu} f_n f_\lambda f_\mu
      + \partial_{\lambda}f_n \partial_{\mu}f_\lambda f_\mu\bigr]
      \\& + \frac{h^4}{24} \sum_{\lambda,\mu,\nu}
      \bigl[\partial_{\lambda\mu\nu} f_n f_\lambda f_\mu f_\nu
      + 3\partial_{\lambda\mu} f_n \partial_{m}f_\lambda f_\mu f_\nu
      % \\&+ \partial_{\lambda\mu} f_n \partial_{\nu}f_\mu f_\lambda f_\nu
      % \\&+ \partial_{\lambda\nu} f_n \partial_{\mu}f_\lambda f_\mu f_\nu
      \\&\left.\hphantom{+\frac{h^4}{24} \sum_{\lambda,\mu,\nu}}
      + \partial_{\lambda}f_n \partial_{\mu\nu} f_\lambda f_\mu f_\nu
      + \partial_{\lambda}f_n \partial_{\mu} f_\lambda \partial_{\nu} f_\mu f_\nu
      \right]
      \\&+\dots
    \end{split}
  \end{gather}
  where we have omitted the arguments $f = f(u(t_0))$ and all sums are
  taken from 1 to $d$.
\end{Lemma}

%%% Local Variables: 
%%% mode: latex
%%% TeX-master: "../notes"
%%% End: 
