\section{Derivatives of the solutions of IVP with respect to data}
\subsection{Derivatives with respect to the initial values}
\begin{intro}
  In order to understand BVP, we have to introduce the notion of the
  derivative of the solution to an IVP with respect to its initial
  values. We will thus denote in these cases $u=u(t;v)$, where $t$ is
  the usual ``time'' variable and $v$ the initial value, which now is
  a variable as well. Thus, $u(t;v)$ is the solution to the IVP
  \begin{gather}
    \label{eq:derivatives:1}
    \begin{split}
      u'(t;v) = \tfrac{\partial}{\partial t} u(t;v) &= f\bigl(t,u(t;v)\bigr) \\
      u(t_0;v) &= v.
    \end{split}
  \end{gather}
  The purpose of this section is the study of the derivative
  \begin{gather*}
    \tfrac{\partial}{\partial v} u(t;v),
  \end{gather*}
  which is fundamentally different from $\partial/\partial t
  u(t;v)$.
  It can be obtained by solving the variational equation of the
  original IVP, defined as follows.
\end{intro}

% \begin{remark}
%   In terms of linear \putindex{perturbation theory} this section deals
%   with the derivative of the solution $u(t)$ of the
%   IVP~\eqref{eq:awa}, but not with respect to the argument $t$, but
%   rather with respect to the initial values. That is why we will write
%   the solution $u = u(t;s)$ of the IVP
%   \begin{gather}
%     \label{eq:derivatives:1}
%     \begin{split}
%       u'(t;s) = \tfrac{\partial}{\partial t} u(t;s) &= f\bigl(t,u(t;s)\bigr) \\
%       u(t_0;s) &= s_0
%     \end{split}
%   \end{gather}
%   with two arguments, the actual time variable $t$ and the initial value $s$.
% \end{remark}

\begin{Definition}{variational-equation}
  The \define{variational equation} to the first order system of
  ODE
  \begin{gather*}
    u' = f,
  \end{gather*}
  of dimension $d$ is the linear matrix-valued system of ODE
  \begin{subequations}
  %  \label{eq:derivatives:1}
  \begin{gather}
    \label{eq:derivatives:2}
      \fundam' = \nabla_u f\bigl(t,u(t)\bigr) \fundam
  \end{gather}
  for $d\times d$ matrices $\fundam$. Here is $u$ a solution of the
  equation~\eqref{eq:awa:ode} and 
  \begin{gather*}
    \nabla_u f(t,u) =
    \begin{pmatrix}
      \tfrac{\partial f_1}{\partial u_1} & \cdots
      & \tfrac{\partial f_1}{\partial u_d} \\
      \vdots & & \vdots \\
      \tfrac{\partial f_d}{\partial u_1} & \cdots
      &  \tfrac{\partial f_d}{\partial u_d}
    \end{pmatrix}
  \end{gather*}
  is the matrix of the derivatives of $f$ with respect to the
  components of $u$. The \define{fundamental matrix}
  $\fundamental(t;t_0)$\index{Y@$\fundamental(t;t_0)$} is solution of
  the IVP to this equation with
  \begin{gather}
    \label{eq:derivatives:3}
    \fundam(t_0) = \identity.
  \end{gather}
  \end{subequations}
\end{Definition}


\begin{remark}
  The fundamental matrix $\fundam$ can also be read column
  by column.  Then each column is a vector-valued function
  $\phi^{(i)}(t)$ and solves the IVP
  \begin{gather*}
    \begin{split}
      \tfrac{d}{dt}\phi^{(i)}(t) &= \nabla_u
      f\bigl(t,u(t)\bigr)\phi^{(i)}(t),
      \\
      \phi^{(i)}(t_0) &= e_i.
    \end{split}
  \end{gather*}
\end{remark}

\begin{remark}
  The definition of the fundamental matrix here is consistent with the
  one in Definition~\ref{Definition:fundamental-system} for linear
  equations. Namely, for $f(u) = Au$, we have $\nabla_u f(u) = A$.
\end{remark}

\begin{Lemma}{fundamental}
  For fundamental matrices there hold the relations
  \begin{align}
    \fundamental(t;s) &= \fundamental(s;t)^{-1}\\
    \label{eq:derivatives:4}
    \fundamental(t;r) &= \fundamental(t;s)\fundamental(s;r),
  \end{align}
  where $r,s,t$ are arbitrary real numbers, such that the solution $u$
  of the original IVP exists on the maximal interval spanned by these
  numbers.
\end{Lemma}

%%% Local Variables: 
%%% mode: latex
%%% TeX-master: "../notes"
%%% End: 


\begin{proof}
  In order to prove the first equation, denote by $V(\tau;s)$ the
  solution to the IVP
  \begin{gather*}
    V'(\tau;s) = \nabla_u f(\tau,u(\tau)) V(\tau;s),
    \qquad V(s;s) = \fundamental(s;t).
  \end{gather*}
  Because of uniqueness, we must have
  $V(\tau;s) = \fundamental(\tau;t)$ for any $\tau$ between $s$ and
  $t$, in particular for $r=t$, such that $V(t;s) = \identity$. On the
  other hand, by linearity, we have
  $V(\tau;s) = \fundamental(\tau;s)\fundamental(s;t)$, and thus the
  equation is proven by
  \begin{gather*}
    \identity = V(t;s) = \fundamental(t;s)\fundamental(s;t).
  \end{gather*}

  Now, assume without loss of generality that $s$ is between $r$ and
  $t$. Indeed, if for instance $t$ is between $r$ and $s$, multiply
  equation~\eqref{eq:derivatives:4} from the left by $U(s;t)$ and
  prove the equation for
  \begin{gather*}
    U(s;t)U(t;r) = U(s;t)U(t;s)U(s;r) = U(s;r).
  \end{gather*}
  Take the auxiliary function $V(\tau;s)$ as defined above. By
  uniqueness, it is equal to $U(\tau;t)$ for all $\tau$. But, on the
  other hand, we have by linearity $V(\tau;s) = U(\tau;s)U(s;r)$, in
  particular for $\tau=t$.

  The statement follows from the definition as a solution of an IVP
  and the fact that solutions of linear IVP are linear combinable.
\end{proof}

\begin{Theorem}{derivative}
  \index{conditioning!AWA} Let be $f(t,u)$ continuous in $t$ and
  continuously differentiable in $u$.  Then, the solution $u(t;v)$ of
  the IVP~\eqref{eq:derivatives:1} depends differentiably on the
  initial value $v$ and the derivative is given by
  \begin{gather}
    \label{eq:derivatives:5}
    \tfrac{\partial}{\partial v} u(t;v) = \fundamental(t;t_0),
  \end{gather}
  where $\fundamental(t;t_0)$ is the fundamental matrix with respect
  to the initial time $t_0$.
\end{Theorem}



\begin{proof}
  We write the IVP in its full dependence on $v$ as
  \begin{align*}
    \frac{\partial u(t;v)}{\partial t} &= f\bigl(t,u(t;v)\bigr) \\
    u(t_0;v) &= v.
  \end{align*}
  From the second equation, we immediately obtain
  \begin{gather*}
    \frac{\partial u(t_0;v)}{\partial v} = \identity.
  \end{gather*}
Assuming differentiability of $f$ with respect to $u$, the first
equation yields
\begin{gather*}
  \frac{\partial}{\partial v}\frac{\partial u(t;v)}{\partial t}
  = \frac{\partial f\bigl(t,u(t;v)\bigr)}{\partial v}
  = \nabla_u f(t,u(t;v)) \frac{\partial u(t;v)}{\partial v}.
\end{gather*}
Thus, $\nicefrac{\partial}{\partial v} u$ solves the
IVP~\eqref{eq:derivatives:1}. Since the solution exists and is unique,
this derivative is well defined and thus, the function is differentiable.
\end{proof}

\subsection{Derivatives with respect to the right hand side function}
\begin{intro}
  We close this section by studying the differential dependence of the
  solution $u(t)$ of an ODE at time $t$ on the function $f(t,u)$, that
  is, the derivative of a value with respect to a function. In order
  to keep things simple, we reduce this question to a regular
  derivative of a function with respect to a real variable by using
  the Gâteaux derivative. To this end, let $F(v)$ be a functional on
  some function space, that is, a real valued function on this
  space. Then, the Gâteaux derivative of $F$ at a point $v$ in
  direction $w$ with $v$ and $w$ in this function space is defined as
  \begin{gather}
    \label{eq:derivatives:6}
    \frac{\partial}{\partial w} F(v)
    = \lim_{\epsilon\to 0} \frac{F(v+\epsilon w) - F(v)}{\epsilon}.
  \end{gather}
  Here, we have used the notation for directional derivatives in
  $\R^n$, since this is indeed the character of the Gâteaux
  derivative.

  Back to differential equations, our task is now to compute the
  derivative of $u(t)$ with respect to changes in $f$, denoted as
  \begin{gather}
    \label{eq:derivatives:7}
    \frac{\partial}{\partial g} u(t)
    = \lim_{\epsilon\to 0} \frac{u_\epsilon-u}{\epsilon}
    = \left.\frac{d}{d\epsilon} u_\epsilon(t)\right|_{\epsilon=0},
  \end{gather}
  where $u$ and $u_\epsilon$ respectively solve the IVPs
  \begin{xalignat*}{2}
    u'&=f(t,u) & u(t_0)& = u_0 \\
    u_\epsilon'&=f(t,u_\epsilon)+\epsilon g(t,u_\epsilon)
    & u_\epsilon(t_0)& = u_0.
  \end{xalignat*}
  For this derivative, we have the following theorem.
\end{intro}

\begin{Theorem}{derivative-f}
  Let $f(t,u)$ and $g(t,u)$ be continuous in their first and
  continuously differentiable in their second argument. Let $u$ be the
  solution of the IVP $u'=f(u)$ with $u(t_0) = u_0$. Then, the Gâteaux
  derivative of $u$ in $f$ with respect to a perturbation $g$ exists
  and there holds
  \begin{gather}
    \label{eq:derivative-f:1}
    \frac{\partial}{\partial g} u(t)
    = \int_{t_0}^t \fundamental(t;s) g\bigl(s,u(s)\bigr)\, \ds.
  \end{gather}
\end{Theorem}
%%% Local Variables:
%%% mode: latex
%%% TeX-master: "../notes"
%%% End:


\begin{proof}
  We set out by devising a differential equation for the Gâteaux
  derivative $\mathcal U(t)$. The differential equation for $u$ yields
  \begin{multline*}
    \mathcal U(t) := \left.\frac{d}{d\epsilon} u_\epsilon(t)\right|_{\epsilon=0}
    = \left.\frac{d}{d\epsilon}
    \Bigl(f\bigl(t,u_\epsilon(t)\bigr)
    + \epsilon g\bigl(t,u_\epsilon(t)\bigr)\Bigr) \right|_{\epsilon=0}
  \\
    = \left.\nabla_u f(t,u_\epsilon) \mathcal U(t)
    + g\bigl(t,u_\epsilon(t)\bigr) \right|_{\epsilon=0}
  \end{multline*}
  Furthermore, we have
  \begin{gather*}
     \mathcal U(t_0) = \frac{d}{d\epsilon} u_\epsilon(t_0) = 0.
  \end{gather*}

  According to Lemma~\ref{Lemma:linear-representation}, the solution
  of this initial value problem can be represented with the
  integrating factor $M(t)$ as
  \begin{gather*}
    \mathcal U(t) = M^{-1}(t) \int_{t_0}^t M(s) g\bigl(s,u(s)\bigr)
    \,d s.
  \end{gather*}
  Noticing that $M(\tau)^{-1} = U(\tau;t_0)$, we obtain
  \begin{gather*}
    u(t) = \int_{t_0}^t \fundamental(t;s) g\bigl(s,u(s)\bigr) \,d s
  \end{gather*}
\end{proof}


%%% Local Variables: 
%%% mode: latex
%%% TeX-master: "notes"
%%% End: 
