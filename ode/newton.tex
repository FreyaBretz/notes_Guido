\chapter{Newton and quasi-Newton methods}
\label{chapter:Newton}

\section{Basics of nonlinear iterations}
\begin{intro}
  The efficient solution of nonlinear problems is an important
  ingredient to implicit timestepping schemes as well as shooting
  methods. Without attempting completeness, we present some important
  facts about iterative methods for this problem. We introduce the two
  generic schemes, Newton and gradient methods, discuss their
  respective pros and cons and combine their features in order to
  obtain better methods.
\end{intro}

\begin{Definition}{nonlinear}
  We consider two formulations of nonlinear root finding problems
  \begin{xalignat}{2}
    \label{eq:nonlinear:1}
    f(x) &= 0, & f:\R^d &\to \R^d \\
    \intertext{and}
    \label{eq:nonlinear:2}
    x &= \operatorname{argmin} F(y)
    & F: \R^d &\to \R.
  \end{xalignat}
  These two problems are equivalent by either choosing for instance
  \begin{gather*}
    f(x) = \nabla F(x)
    \qquad\text{or}\qquad
    F(x) = \abs{f(x)}.
  \end{gather*}
\end{Definition}



%%% Local Variables:
%%% mode: latex
%%% TeX-master: "../notes"
%%% End:

\begin{Definition}{iteration-order}
  An \putindex{iteration}
  \begin{gather*}
    x^{(k+1)} = G\left(x^{(k)}\right)
  \end{gather*}
  is said to be \textbf{convergent of order $p$}, if there holds for
  $p\ge 1$:
  \begin{gather*}
    \norm{x^{(k+1)}-x^*} \le c \norm{x^{(k)}-x^*}^p,
  \end{gather*}
  and if for $p=1$ there holds $c<1$. For $p>1$, such a method
  converges only locally, namely if $\norm{x^{(0)}-x^*}$ is
  sufficiently small, for instance
  \begin{gather*}
    \norm{x^{(0)}-x^*}^{p-1} < \frac1c.
  \end{gather*}
\end{Definition}
%%% Local Variables:
%%% mode: latex
%%% TeX-master: "../notes"
%%% End:

\section{Das Newton-Verfahren}

\begin{Definition}{newton-verfahren}
  Das \define{Newton-Verfahren} ist ein Iterationsverfahren zum
  Auffinden einer Nullstelle einer Funktion $f\colon \R^n\to \R^n$. Zu
  einem Startwert $x^{(0)} \in \R^n$ berechnen sich die weiteren
  Iterierten durch
  \begin{gather}
    \label{eq:newton:1}
    x^{(k+1)} = x^{(k)} - \bigl(\nabla f(x^{(k)})\bigr)^{-1} f(x^{(k)}).
  \end{gather}
\end{Definition}

\begin{Algorithmus*}{newton}{Newton-Verfahren}
  \lstinputlisting[firstline=3,lastline=9]{code/newton.py} Die
  Parameter zu dieser Funktion sind der Startwert $x$, die Funktion
  $f(x)$, die Anwendung der inversen Ableitung
  \begin{gather}
    \operatorname{Dfinv}(x,r) = \bigl(\nabla f(x)\bigr)^{-1}r,
  \end{gather}
  sowie eine Toleranz als Abbruchkriterium.
\end{Algorithmus*}
\begin{Lemma}{newton-1}
  Sei $M\subset \R^n$ konvex. Sei $f\colon M\to \R^n$ stetig differenzierbar auf $M$ und die Ableitung genüge der Lipschitz-Abschätzung
  \begin{gather}
    \norm{\nabla f(x) - \nabla f(y)} \le \gamma \norm{x-y}
    \qquad \forall x,y\in M.
  \end{gather}
  mit einer Konstanten $\gamma$. Dann gilt für alle $x,y\in M$
  \begin{gather}
    \norm{f(x)-f(y) - \nabla f(y)(x-y)}
    \le \frac\gamma2 \norm{x-y}^2.
  \end{gather}
\end{Lemma}

\begin{proof}
  Wir folgen~\cite[Hilfssatz 5.3.1]{Stoer83}.
  Sei $\phi\colon[0,1] \to \R^n$ die Hilfsfunktion definiert durch
  \begin{gather}
    \phi(t) = f\bigl(y+t(x-y)\bigr),
  \end{gather}
  so dass
  \begin{gather}
    f(x)-f(y) - \nabla f(y)(x-y) = \phi(1) - \phi(0) - \phi'(0)
    = \int_0^1 (\phi'(t)-\phi'(0))\dt,
  \end{gather}
  denn nach der Kettenregel gilt
  \begin{gather}
    \phi'(t) = \nabla f\bigl(y+t(x-y)\bigr)(x-y).
  \end{gather}
  Den Integranden schätzen wir ab durch
  \begin{align}
    \norm{\phi'(t)-\phi'(0)}
    & = \norm{\Bigl(\nabla f\bigl(y-t(x-y)\bigr) - \nabla f(y)\Bigr)(x-y)}
    \\
    & \le \norm{\nabla f\bigl(y-t(x-y)\bigr) - \nabla f(y)}\norm{(x-y)}
    \\
    & \le \gamma\,t\norm{x-y}^2.
  \end{align}
  Einsetzen ins Integral ergibt
  \begin{gather}
    \norm{f(x)-f(y) - \nabla f(y)(x-y)}
    \le \frac\gamma2 \norm{x-y}^2.
  \end{gather}
\end{proof}

\begin{Satz}{newton-konvergenz}
  Sei $M\subset \R^n$ eine offene, konvexe Menge und
  $f\colon \overline{M} \to \R^n$ stetig differenzierbar in $M$ und
  stetig auf $\overline{M}$. Die \define{Jacobi-Matrix} $\nabla f(x)$
  sei auf ganz $M$ invertierbar und es gebe Konstanten $\beta$ und
  $\gamma$, so dass für $x,y\in M$ gilt
  \begin{gather}
    \label{eq:newton:2}
    \norm{\nabla f(x) - \nabla f(y)} \le \gamma \norm{x-y},
    \qquad \norm{\bigl(\nabla f(x)\bigr)^{-1}} \le \beta.
  \end{gather}
  Gibt es dann eine Konstante $\alpha$, so dass
  \begin{align}
    \label{eq:newton:3}
    \norm{\bigl(\nabla f(x^{(0)})\bigr)^{-1} f(x^{(0)})}
    &\le \alpha\\
    h := \frac{\alpha\beta\gamma}2 &<1\\
    \overline{B_r(x^{(0)})} &\subseteq M, \qquad\text{mit } r=\frac{\alpha}{1-h},
  \end{align}
  So ist die Folge $x^{(k)}$ des Newton-Verfahrens für alle
  $k=1,\dots$ wohldefiniert und liegt in $B_r(x^{(0)})$. Ferner
  konvergiert sie quadratisch gegen einen Wert
  $x^*\in\overline{B_r(x^{(0)})}$ und es gilt
  \begin{gather}
    \label{eq:newton:4}
    \norm{x^{(k)}-x^*} \le \alpha\frac{h^{2^k-1}}{1-h^{2^k}}.
  \end{gather}
\end{Satz}

\begin{proof}
  Wir folgen~\cite[Satz 5.3.2]{Stoer83}.
  Wir zeigen zunächst induktiv für alle $k=1,\dots$, dass das
  Folgenglied $x^{(k)}$ in $B_r(x^{(0)}) \subseteq M$ liegt. Damit
  existiert dann nach Voraussetzung
  $\bigl(\nabla f(x^{(k)})\bigr)^{-1}$ und $x^{(k+1)}$ ist
  wohldefiniert. Zur Verankerung bemerken wir, dass offensichtlich
  $x^{(0)}\in B_r(x^{(0)})$ und $x^{(1)}$ nach
  Voraussetzung~\eqref{eq:newton:3}. Nach der Verfahrensvorschrift
  können wir abschätzen:
  \begin{align}
    \norm{x^{(k+1)} - x^{(k)}}
    & = \norm{\bigl(\nabla f(x^{(k)})\bigr)^{-1} f(x^{(k)})}\\
    & \le \beta \norm{f(x^{(k)})}\\
    & = \beta \norm{f(x^{(k)}) - f(x^{(k-1)}) - \nabla f(x^{(k)})(x^{(k)}-x^{(k-1)})},
  \end{align}
  wobei wir die letzte Zeile aus der Multiplikation der
  Verfahrensvorschrift mit $\nabla f$ gewonnen haben. Hierauf wenden
  wir nun \slideref{Lemma}{newton-1} an und bekommen die quadratische
  Konvergenz, wenn der Abstand zweier Folgenglieder einmal klein genug
  ist:
  \begin{gather}
    \label{eq:newton:5}
    \norm{x^{(k+1)} - x^{(k)}}
    \le \frac{\beta\gamma}2 \norm{x^{(k)} - x^{(k-1)}}^2.
  \end{gather}
  Es bleibt zu zeigen, dass die Folge in $B_r(x^{(0)})$ bleibt. Dazu
  zeigen wir per Induktion, dass
  \begin{gather}
    \label{eq:newton:6}
    \norm{x^{(k+1)} - x^{(k)}} \le \alpha h^{2^k-1}.
  \end{gather}
  Für $k=0$ folgt $\norm{x^{(1)} - x^{(0)}} \le \alpha$ direkt
  aus~\eqref{eq:newton:3}. Für den Induktionsschritt benutzen wir
  unsere Konvergenzabschätzung~\eqref{eq:newton:5}:
  \begin{gather}
    \norm{x^{(k+1)} - x^{(k)}}
    \le \frac{\beta\gamma}2 \norm{x^{(k)} - x^{(k-1)}}^2
    \le \frac{\beta\gamma}2 (\alpha h^{2^{k-1}-1})^2
    = \frac{\alpha\beta\gamma}2 \alpha h^{2^k-2}
    = \alpha h^{2^k-1}.
  \end{gather}
  Nun können wir mit einer Teleskopsumme abschätzen
  \begin{align}
    \norm{x^{(k+1)} - x^{(0)}}
    &\le \sum_{j=0}^k \norm{x^{(j+1)} - x^{(j)}}\\
    & \le \alpha (1+h+h^3+h^7+\dots+h^{2^k-1})\\
    &<\frac\alpha{1-h} = r,
  \end{align}
  Aus~\eqref{eq:newton:6} folgt mit dieser Abschätzung, dass $x^{(k)}$
  Cauchy Folge ist und durch Grenzübergang die
  Abschätzung~\eqref{eq:newton:4}.
\end{proof}

\section{Abstiegsverfahren und Globalisierung}

\begin{intro}
  Die lokale Konvergenz ist beim Newtonverfahren ein großes Hindernis
  für die Anwendung. Wählt man den Startwert nicht im Einzugsbereich
  einer Nullstelle, so divergiert das Verfahren. Der Einzugsbereich,
  wie er sich aus dem Konvergenzsatz ergibt, ist dabei oft sehr klein
  und daher schwer zu finden.

  Ziel dieses Abschnitts ist daher, eine Modifikation des
  Newton-Verfahrens zu finden, die den Konvergenzbereich aufweitet,
  idealerweise sogar globale Konvergenz erzeugt. Solche Modifikationen
  findet man unter der Bezeichnung \define{Globalisierung}.
\end{intro}

\begin{Definition}{anstiegskegel}
  Sei $g\colon \R^n\to \R$ stetig differenzierbar. Dann definieren wir
  den Kegel positiven Anstiegs zum Parameter $\gamma$ als
  \begin{gather}
    S_\gamma(x) = \bigl\{ s\in\R^n \big|
    \norm{s} = 1 \;\wedge \;
    \nabla g(x)\cdot s \ge \gamma\norm{\nabla g(x)}
    \bigr\}.
  \end{gather}
  Die Richtung des steilsten Anstiegs im Punkt $x$ ist $\nabla g(x)$.
\end{Definition}

\begin{todo}
  Umgebung ``remark''.
\end{todo}
Da es sich um normierte Vektoren handelt, ist die Menge $S_\gamma(x)$
eigentlich kein Kegel, sondern eine Kugel in der Einheitssphäre mit
Zentrum im normierten Gradienten und Radius $\arccos \gamma$. Zusammen
mit den Skalierungsfaktoren, die unten eingeführt werden,
repräsentiert sie aber einen Kegel.


\begin{Lemma}{abstieg-reduktion}
  Sei $g\colon\R^n\to\R$ stetig differenzierbar und in einem Punkt
  $y\in\R^n$ gelte $\nabla g(y) \neq 0$. Dann gibt es eine Umgebung
  $U(y)$ und $\lambda>0$, so dass für alle $x\in U(y)$,
  $s\in S_\gamma(x)$ und $\mu\in [0,\lambda]$ gilt
  \begin{gather}
    g(x-\mu s) \le g(x) - \frac{\mu\gamma}4 \norm{\nabla g(y)}.
  \end{gather}
\end{Lemma}

\begin{proof}
  Wir definieren zunächst eine Umgebung um $y$ auf der sich die
  Gradienten nicht zu sehr unterscheiden:
  \begin{gather}
    \label{eq:newton:8}
    U_1(y) = \left\{
      x\in\R^n \middle| \;\norm{\nabla g(x)-\nabla g(y)} \le \frac\gamma4 \norm{\nabla g(y)}
      \right\}.
    \end{gather}
    Eine zweite Umgebung ist so gewählt, dass dort der Abstiegskegel
    in einem größeren Abstiegskegel im Punkt $y$ enthalten ist:
    \begin{gather}
      U_2(y) = \left\{
        x\in\R^n \middle| S_\gamma(x) \subseteq S_{\nicefrac\gamma2}(y)\right\}.
    \end{gather}

    Wähle nun $\lambda>0$, so dass
    \begin{gather}
      \overline{B_{2\lambda}(y)}\subseteq U_1(y) \cap U_2(y).
    \end{gather}
    und $U(y) = B_\lambda(y)$. Dann zeigen wir nun die Aussage für
    alle $x\in U(y)$, $s\in S_\gamma(x)$ und $\mu\in[0,\lambda]$. Nach
    dem Mittelwertsatz existiert $\theta\in(0,1)$ so dass
    \begin{gather}
      g(x)-g(x-\mu s) = \mu \nabla g(x-\theta\mu s)s.
    \end{gather}
    Wir formen weiter um:
    \begin{align}
      \nabla g(x-\theta\mu s)s
      &= \bigl(\nabla g(x-\theta\mu s) - \nabla g(y)\bigr)s + \nabla g(y)s\\
      &\ge -\frac{\gamma}4 \norm{\nabla g(y)}\norm{s} + \nabla g(y)s\\
      &\ge -\frac{\gamma}4 \norm{\nabla g(y)} + \frac\gamma2 \norm{\nabla g(y)}\\
      & \ge \frac\gamma4 \norm{\nabla g(y)}.
    \end{align}
\end{proof}

\begin{todo}
  Reihenfolge vertauschen  
\end{todo}

\begin{Definition}{abstiegsverfahren}
  Ein \define{Abstiegsverfahren} für eine stetig differenzierbare
  Funktion $g\colon \R^n\to\R$ ist eine Iterationsvorschrift aus den
  folgenden Schritten: gegeben $x^{(k)}$,
  \begin{enumerate}
  \item wähle $\gamma_k>\gamma>0$ und eine Abstiegsrichtung
    $s^{(k)} \in S_{\gamma_k}(x^{(k)})$.
  \item Wähle eine Schrittweite $\alpha_k>0$ und setze
    \begin{gather}
      x^{(k+1)} = x^{(k)} - \alpha_k s^{(k)},
    \end{gather}
    so dass die \define{Reduktionsbedingung}
    \begin{gather}
      \label{eq:newton:7}
      g\bigl(x^{(k+1)}\bigr)
      \le g\bigl(x^{(k)}\bigr) - \frac{\gamma_k\alpha_k}{4}\norm{\nabla g(x^{(k)})}
    \end{gather}
    gilt.
  \end{enumerate}
\end{Definition}

\begin{remark}
  Lemma \slideref{Lemma}{abstieg-reduktion} stellt sicher, dass es in
  jedem Schritt ein positives $\alpha_k$ gibt, das die Bedingung
  erfüllt.
\end{remark}

\begin{Beispiel*}{steepest-descent}{Verfahren des steilsten Abstiegs}
  Sei der Vektor $x^{(k)} \in\R^n$ gegeben, dann wähle
  $s^{(k)} = \nabla g(x^{(k)})$. Die Schrittweite $\alpha_k$ wird aus
  der eindimensionalen Minimierungsaufgabe (auch \define{line search}
  genannt)
  \begin{gather}
    \alpha_k = \operatorname*{argmin}_{\alpha>0}
    g\bigl(x^{(k)} - \alpha s^{(k)}\bigr)
  \end{gather}
  bestimmt. Danach setze
  \begin{gather}
    x^{(k+1)} = x^{(k)} - \alpha_k s^{(k)}.
  \end{gather}
\end{Beispiel*}

\begin{Satz}{abstieg-haeufung}
  Sei $g\colon \R^n\to \R$ und $x^{(0)} \in\R^n$ so gewählt, dass die Menge
  \begin{gather}
    K = \Bigl\{x\in\R^n \; \Big| \; g(x) \le g\bigl(x^{(0)}\bigr) \Bigr\}
  \end{gather}
  kompakt und $g$ stetig differenzierbar auf einer Umgebung von $K$
  ist. Dann besitzt die Folge $\{x^{(k)}\}$ des Abstiegsverfahrens
  mindestens einen Häufungspunkt in $K$. Gilt zusätzlich in der
  Umgebung eines Häufungspunkts $\gamma_k \ge \gamma>0$, so existiert
  $\alpha$, so dass $\alpha_k \ge \alpha>0$ gewählt werden kann. In
  diesem Fall ist der Häufungspunkt ein stationärer Punkt von $g$.
\end{Satz}

\begin{proof}
  Da die Folge monoton fällt, bleibt sie in $K$ und hat der
  Kompaktheit wegen mindestens einen Häufungspunkt $x^*$. Wir benennen
  nun ebenfalls mit $\{x^{(k)}\}$ ebenfalls eine Teilfolge, die gegen diesen
  Häufungspunkt konvergiert.

  Wir machen die Widerspruchsannahme, dass $x^*$ kein stationäre Punkt von $g$ ist, also
  \begin{gather}
    \nabla g(x^*) \neq 0.
  \end{gather}

  Wir bemerken, dass nach Voraussetzung $S_{\gamma_k}(x^*) \subseteq S_\gamma(x^*)$ gilt.
  Nun gibt es nach \slideref{Lemma}{abstieg-reduktion} eine Umgebung
  $U(x^*)$ und eine Zahl $\lambda>0$, so dass für alle $\mu\in[0,\lambda]$ gilt:
  \begin{gather}
    g(x-\mu s) \le g(x) - \mu \frac\gamma4\norm{\nabla g(x^*)}.
  \end{gather}
  Daraus folgt, dass zu jedem $\alpha_k \le \lambda$ die
  Reduktionsbedingung~\eqref{eq:newton:7} erfüllt ist und
  dementsprechend die Bedingung $\alpha_k\ge\alpha$ für
  $\alpha\le\lambda$ erfüllt werden kann.

  Sei nun $k_0$ gewählt, so dass $x^{(k)}\in U(x^*)$ für alle
  $k\ge k_0$. Dann gilt nach der Konstruktion von $U(x^*)$
  in~\eqref{eq:newton:8}, dass
  \begin{gather}
    \norm{\nabla g(x^{(k)})} \ge \norm{\nabla g(x^{*})} - \norm{\nabla g(x^{(k)}) - \nabla g(x^{*})}
    \ge \left(1-\frac\gamma4\right)  \norm{\nabla g(x^{*})}.
  \end{gather}
  Es gilt also
  \begin{gather}
    g(x^{k+1}) \le g(x^{k}) - \frac34 \alpha \norm{\nabla g(x^{*})}.
  \end{gather}
  Daraus folgt im Widerspruch zur Stetigkeit die Konvergenz
  $g(x^{(k)}) \to -\infty$, Es muss also $\nabla g(x^{*})=0$ gelten,
  $x^*$ ist also ein stationärer Punkt von $g$.
\end{proof}

\begin{remark}
  Der vorherige besteht aus zwei Teilen. Die Existenz eines
  Häufungspunktes wird unter einer der allgemeinen Bedingung
  getroffen, dass die Menge $K$ kompakt ist, also eine Abstiegsfolge
  nicht ins unendliche konvergieren kann. Diese Bedingung ist oft
  recht leicht nachzuprüfen. Insbesondere steht die Existenz mehrerer
  Häufungspunkte nicht im Widerspruch zu den Annahmen, so dass das
  Verfahren auch dann wenigstens einen davon findet.

  Die weiteren Bedingungen, die sicherstellen, dass es sich bei einem
  Häufungspunkt um einen stationären Punkt handelt, sind lokal in
  einer Umgebung eines solchen gestellt. An diesem Punkt sind die
  Folgen $\gamma_k$ und $\alpha_k$ nur sehr abstrakt fixiert. Wir
  zeigen nun, dass die Folge der $\alpha_k$ im Verfahren des steilsten
  Abstiegs die Bedingung erfüllt. Als zweite Anwendung von allgemeinen
  Abstiegsverfahren stellen wir dann das Newton-Verfahren mit
  Schrittweitensteuerung vor.
\end{remark}

\begin{Korollar}{abstieg-haeufung}
  Beim Verfahren des steilsten Abstiegs sind die Folgen $\gamma_k$ und
  $\alpha_k$ so gewählt, das \slideref{Satz}{abstieg-haeufung} gilt.
\end{Korollar}

\begin{proof}
  Da die Abstiegsrichtungen immer gleich dem (negativen) Gradienten
  sind, gilt $\gamma_k \equiv 1$. Für die Folge $\alpha_k$ zeigen wir
  nich die Beschränktheit durch $\alpha$. Stattdessen bemerken wir mit
  $\lambda$ aus \slideref{Lemma}{abstieg-reduktion}:
  \begin{align}
    g(x^{(k+1)})
    &= \min_{\alpha>0} g\left(x^{(k+1)}-\alpha s^{(k)}\right)\\
    &\le g(x^{(k)}-\lambda s^{(k)})\\
    &\le g(x^{(k)}) - \frac{\lambda}4 \norm{\nabla g(x^*)}.
  \end{align}
  Auch hier schließen wir, dass die Folge divergiert wenn die Punkte konvergieren.
\end{proof}

\begin{Lemma}{newton-abstieg}
  Sei $f\colon \R^n\to \R^n$ stetig differenzierbar und
  $g(x) = \norm{f(x)}_2^2$.  Dann sind die Suchrichtungen des Newton-Verfahrens
  \begin{gather}
    s^{(k)} = \frac{d^{(k)}}{\norm{d^{(k)}}_2},
    \qquad d^{(k)} = \bigl(\nabla f(x^{(k)})\bigr)^{-1}f(x^{(k)})
  \end{gather}
  Abstiegsrichtungen für $g(x)$ und es gilt
  \begin{gather}
    s^{(k)} \in S_\gamma\bigl(x^{(k)}\bigr),
    \qquad
    \gamma = \frac{1}{\cond_2(\nabla f(x^{(k)}))}
  \end{gather}
\end{Lemma}

\begin{proof}
  Es gilt (Nachrechnen!)
  \begin{gather}
    \nabla g(x) = 2 f(x)^T\nabla f(x).
  \end{gather}
  Daher ist
  \begin{align}
    \frac{\nabla g(x) s}{\norm{\nabla g(x)}_2}
    &= \frac{f(x)^T \nabla f(x) \bigl(\nabla f(c)\bigr)^{-1} f(x)}
      {\norm{f(x)^T\nabla f(x)}_2\norm{\bigl(\nabla f(c)\bigr)^{-1} f(x)}}_2\\
    &\ge \frac{\norm{f(x)}^2_2}{\norm{f(x)}_2\norm{\nabla f(x)}_2\norm{\bigl(\nabla f(c)\bigr)^{-1}}_2\norm{f(x)}_2}\\
    &= \frac1{\cond_2(\nabla f)}.
  \end{align}
\end{proof}

\begin{Korollar}{newton-abstieg}
  Das modifizierte Newton-Verfahren
  \begin{gather}
    x^{(k+1)} = x^{(k)} - \alpha_k \bigl(\nabla f(x^{(k)})\bigr)^{-1} f(x^{(k)})
  \end{gather}
  ist ein Abstiegsverfahren, wenn $\alpha_k$ so gewählt ist, dass die
  Reduktionsbedingung gilt.
\end{Korollar}

\begin{remark}
  Man kann nun zum Beispiel auch das Newton-Verfahren mit line search
  ausführen, um globale Konvergenzeigenschaften zu erzielen. Es gilt
  dann zunächst die Existenz von Häufungspunkten. In der Nähe eines
  solchen gilt aber natürlich, dass line search nicht schlechter
  konvergiert, als das normale Newton-Verfahren, woraus dann dort
  wieder die quadratische Konvergenz gefolgert werden kann.

  Wir betrachten stattdessen die folgende, einfachere Variante, die
  mit minimaler Modifikation ein global konvergierendes Verfahren
  ergibt.
\end{remark}

\begin{Definition}{newton-stepsize}
  Das Newton-Verfahren mit \define{Schrittweitensteuerung} berechnet iterativ
  $x^{(k+1)}\in \R^n$ aus $x^{(k)}\in \R^n$ in folgenden Schritten
  \begin{enumerate}
  \item Berechne $d^{(k)} = \bigl(\nabla f(x^{(k)})\bigr)^{-1}f(x^{(k)})$
  \item Berechne die kleinste ganze Zahl $j$, so dass
    \begin{multline}
      \norm*{f(x^{(k)}-2^{-j} d^{(k)})}_2^2
      \le \norm*{f(x^{(k)})}_2^2
      \\- 2^{-j} \frac{1}{4\cond_2(\nabla f(x^{(k)}))}
      \norm*{f^T\bigl(x^{(k)}\bigr)\nabla f\bigl(x^{(k)}\bigr)}_2
    \end{multline}
    \item Setze $x^{(k+1)}=x^{(k)}-2^{-j} d^{(k)}$
  \end{enumerate}
\end{Definition}

\begin{remark}
  Der Algorithmus benötigt viele zusätzliche Berechnungen, wie die von
  $\gamma_k$ oder $\nabla g$. Für die praktische Anwendung lässt er
  sich vereinfachen. Dazu beobachten wir zunächst, dass
  Bedingung~\eqref{eq:newton:7} dazu dient, eine hinreichende
  Kontraktion in der Nähe eines Häufungspunkts sicherzustellen. Die
  Existenz eines solchen kann bereits aus
  \begin{gather}
    g\bigl(x^{k+1}\bigr) < g\bigl(x^{k}\bigr)
  \end{gather}
  gefolgert werden. Umgekehrt wird, wenn die Funktion $f$ die
  Bedingungen des Konvergenzsatzes \slideref{Satz}{newton-konvergenz}
  erfüllt, in der Nähe eines Fixpunktes ohnehin $j=0$ gelten. Wir
  ersetzen daher die komplizierte Bedingung durch die wesentlich
  einfachere: sei $j$ die kleinste nichtnegative ganze Zahl, so dass
  \begin{gather}
    \norm*{f\bigl(x^{k} - 2^{-j} d^{(k)}\bigr)}_2^2 < \norm*{f\bigl(x^{k}\bigr)}_2^2.
  \end{gather}
  Es gibt in der Literatur weitere Heuristiken zur Wahl der
  Schrittweite im Newton-Verfahren, die man unter dem Stichwort \glqq
  Globalisierung\grqq{} findet. Hier wollen wir uns mit dieser
  besonders einfachen und gleichzeitig effektiven Variante begnügen.
\end{remark}

\begin{Algorithmus*}{newton-stepsize}{Newton-Verfahren mit Schrittweitensteuerung}
    \lstinputlisting[firstline=3,lastline=18]{code/newton-stepsize.py}
\end{Algorithmus*}

\begin{remark}
  Dieser Abschnitt gibt Hinweise darauf, wie das Newton-Verfahren
  modifiziert werden kann und trotzdem Konvergenz erhalten wird. Neben
  der Schrittweitensteuerung kommen hier insbesondere approximative
  Berechnungen der Ableitung in Frage. Diese werden dann oft als
  \define{Quasi-Newton-Verfahren} bezeichnet, konvergieren in der
  Regel nur von erster Ordnung, sind aber of viel effizienter als das
  Newton-Verfahren selbst.
\end{remark}

%%% Local Variables:
%%% mode: latex
%%% TeX-master: "main"
%%% End:

\begin{Theorem*}{newton-kantorovich}{Newton-Kantorovich}
  Let $f: \R^d \to \R^d$ be differentiable with
  \begin{xalignat}2
    \label{eq:newton-kantorovich:1}
    \norm{\nabla f(x) - \nabla f(y)}
    & \le L \norm{x-y} & x,y&\in \R^d,
    \\
    \label{eq:newton-kantorovich:2}
    \norm{\left(\nabla f\left(x^{(0)}\right)\right)^{-1}} &\le M.
  \end{xalignat}
  Then, if
  \begin{gather}
    \label{eq:newton-kantorovich:3}
    \beta_0 := LM \norm{f\left(x^{(0)}\right)} \le \frac12,
  \end{gather}
  the Newton method converges to a root of $f(x)$. If furthermore
  $\beta_0 < 1/2$, this convergence is quadratic.
\end{Theorem*}

%%% Local Variables:
%%% mode: latex
%%% TeX-master: "../notes"
%%% End:


\begin{remark}
  Instead of proving the Newton-Kantorovich theorem, we discuss its
  main assumptions and features. First, we note that it does not
  require that the initial value be close to a root, or even assumes
  the existence of a root. The theorem is actually an existence proof.
  
  The Lipschitz condition on $\nabla f$ can be seen as the deviation
  of $f$ from being linear. Indeed, if $f$ were linear, then $L=0$ and
  provided $M\neq 0$ the method converges in a single step for any
  initial value.

  The larger the constant $M$, the smaller wone of the eigenvalues of % wone?
  the Jacobian. Therefore, the function becomes flat in that
  direction and the root finding problem becomes unstable.
  
  If we have convergence due to $\beta_0 \le 1/2$ (the proof shows
  contraction) there holds $\beta_1 := LM \left\| f\left(x^{(1)}\right)
  \right\| < 1/2$ and we have quadratic convergence from the second step on.
\end{remark}

\begin{Definition}{gradient-method}
  The \define{gradient method} for finding minimizers of a nonlinear
  functional $F(x)$ reads:  given an initial value $x^{(0)}$, compute
  iterates $x^{(k)}$, $k=1,2,\ldots$ by the rule
  \begin{gather}
    \label{eq:gradient-method:1}
    \begin{split}
      d^{(k)} &= -\nabla F(x^{(k)}),
      \\
      \alpha_k &=
      \operatorname*{argmin}_{\gamma>0} F\left(x^{(k)} + \gamma d^{(k)}\right)
      \\
      x^{(k+1)} &= x^{(k)} + \alpha_k d^{(k)}.
    \end{split}
  \end{gather}
  The minimization process used to compute $\alpha_k$, also called
  \define{line search}, is one-dimensional
  and therefore simple. It may be replaced by a heuristic choice of
  $\alpha_k$.
\end{Definition}

%%% Local Variables:
%%% mode: latex
%%% TeX-master: "../notes"
%%% End:

\begin{Definition}{gradient-method}
  The \define{gradient method} for finding minimizers of a nonlinear
  functional $F(x)$ reads:  given an initial value $x^{(0)}$, compute
  iterates $x^{(k)}$, $k=1,2,\ldots$ by the rule
  \begin{gather}
    \label{eq:gradient-method:1}
    \begin{split}
      d^{(k)} &= -\nabla F(x^{(k)}),
      \\
      \alpha_k &=
      \operatorname*{argmin}_{\gamma>0} F\left(x^{(k)} + \gamma d^{(k)}\right)
      \\
      x^{(k+1)} &= x^{(k)} + \alpha_k d^{(k)}.
    \end{split}
  \end{gather}
  The minimization process used to compute $\alpha_k$, also called
  \define{line search}, is one-dimensional
  and therefore simple. It may be replaced by a heuristic choice of
  $\alpha_k$.
\end{Definition}

%%% Local Variables:
%%% mode: latex
%%% TeX-master: "../notes"
%%% End:


% \begin{proof}
%   First, we observe that in any point $x$ with $\nabla F(x) \neq 0$,
%   there exists $\epsilon > 0$ such that
%   \begin{gather*}
%     F(x) - F(x-\epsilon\nabla F(x))
%     = \epsilon \abs{\nabla F(x)}^2 > 0.
%   \end{gather*}
%   Thus, $F(x-\epsilon\nabla F(x)) < F(x)$. We conclude, that for such
%   $x$, the line search obtains a positive value of $\alpha$. Thus, the
%   sequence of the gradient iteration is monotonically decreasing and
%   stays within the set $K$.
% \end{proof}

\section{Globalization}

\begin{intro}
  The convergence of the Newton method is only local, and it is the
  faster, the closer to the solution we start. Thus, finding good
  initial guesses is an important task. A reasonable initial guess in
  a one-step method seems to be $y_0$, but on closer inspection, this
  is true only if the time step is small. Therefore, the convergence
  requirements of Newton's method would insert a new time step
  restriction, which we want to avoid in the context of implicit
  methods. Therefore, and for other cases like the shooting
  methods of chapter~\ref{chapter:rwa}, we present
  methods which extend the domain of convergence.

  As a rule, Newton'a method should never be implemented without some
  globalization strategy!
\end{intro}

\begin{Definition}{newton-line-search}
  The \define{Newton method with line search}\index{line search} for
  finding the root of the nonlinear equation $f(x) = 0$ reads: given
  an initial value $x^{(0)}$, compute iterates $x^{(k)}$,
  $k=1,2,\ldots$ by the rule
  \begin{gather}
    \label{eq:newton-line-search:1}
    \begin{split}
      J &= \nabla f\left(x^{(k)}\right),
      \\
       J d^{(k)} &= f(x^{(k)}),
      \\
      \alpha_k &= \operatorname*{argmin} f(x^{(k)} - \alpha d^{(k)})\\
      x^{(k+1)} &= x^{(k)} - \alpha_k d^{(k)}.
    \end{split}
  \end{gather}
\end{Definition}

%%% Local Variables:
%%% mode: latex
%%% TeX-master: "../notes"
%%% End:

\begin{Definition}{newton-step-size}
  The \define{Newton method with step size control} for finding the
  root of the nonlinear equation $f(x) = 0$ reads: given an initial
  value $x^{(0)}$, compute iterates $x^{(k)}$, $k=1,2,\ldots$ by the
  rule
  \begin{gather}
    \label{eq:newton-step-size:1}
    \begin{split}
      J &= \nabla f\left(x^{(k)}\right),
      \\
      J d^{(k)} &= f(x^{(k)}),
      \\
      x^{(k+1)} &= x^{(k)} - 2^{-j} d^{(k)}.
    \end{split}
  \end{gather}
  Here, $j$ is the smallest integer number, such that
  \begin{gather}
    \label{eq:newton-step-size:2}
    f(x^{(k)} - 2^{-j} d^{(k)}) < f(x^{(k)}),
  \end{gather}
  for practical purposes.
\end{Definition}

%%% Local Variables:
%%% mode: latex
%%% TeX-master: "../notes"
%%% End:


\begin{remark}
  The step size control algorithm can be implemented with very low
  overhead. In fact, in each Newton step we only have to compute the
  norm of the residual, which is typically needed for the stopping
  criterion anyway. Additional work is only needed if the residual
  grows. But this is the case, when the original method was likely to
  fail.

  The convergence proof does not guarantee that the values of $j$
  remain bounded. Practically, this is irrelevant, since typically the
  step size control only triggers within the first few steps, then the
  quadratic convergence of the Newton method starts.
\end{remark}

\begin{Definition}{descent-methods}
  For a given vector $v \in \R^d$ and $\gamma > 0$, we define the
  spherical disc
    \begin{gather}
    \label{eq:descent-methods:1}
    \mathcal S_\gamma(v) = \Bigl\{s\in \R^d \Big|
    \abs{s} = 1 \;\wedge \;
    v\cdot s \ge \gamma \abs{v} \Bigr\}.
  \end{gather}
  A \define{descent method} is an iterative method for finding
  minimizers of the functional $F(x)$ computes iterate $x^{(k+1)}$
  from iterate $x^{(k)}$ by the following steps:
  \begin{enumerate}
  \item Choose a search direction:
    \begin{gather*}
      s \in \mathcal S_\gamma\bigl(\nabla F(x^{(k)})\bigr),
    \end{gather*}
    and a positive parameter $\mu$.
  \item Update:
    \begin{gather*}
      x^{(k+1)} = x^{(k)} -\mu s.
    \end{gather*}
  \end{enumerate}
\end{Definition}
%%% Local Variables:
%%% mode: latex
%%% TeX-master: "../notes"
%%% End:


\begin{remark}
  Obviously, the gradient method is a descent method, where the direction $s$
  is chosen parallel to $\nabla F(x^{(k)})$ and $\mu$ is chosen in an
  optimal way. It is also called the method of \define{steepest
    descent}.
\end{remark}

\begin{Lemma}{newton-descent}
  The Newton method applied to the function $f(x)$ is a descent method
  applied to the functional $F(x) = \abs{f(x)}^2$. The same holds for
  the Newton method with line search or step size control.
\end{Lemma}

\begin{proof}
  By the product rule, there holds
  \begin{gather*}
    \nabla F(x) = 2 f^T(x) \nabla f(x).
  \end{gather*}
  The search direction of the Newton method is
  \begin{gather*}
    s = -\frac{d^{(k)}}{\abs{d^{(k)}}}
    = \frac{\bigl(\nabla f(x^{(k)})\bigr)^{-1}f(x^{(k)})}{\abs{\dots}}
  \end{gather*}
  Thus, omitting the arguments $x^{(k)}$, we obtain
  \begin{gather*}
    \frac{\nabla F\, s}{\norm{\nabla F}}
    = \frac{f^T \nabla f(x)\bigl(\nabla f\bigr)^{-1} f}%
    {\norm{\bigl(\nabla f\bigr)^{-1} f}\,\norm{f^T \nabla f(x)}}
    \ge \frac{\abs{f}^2}{\norm{\bigl(\nabla f\bigr)^{-1}}\;
      \norm{f}^2 \norm{\nabla f(x)} }
    = \frac1{\operatorname{cond}_2 (\nabla f(x))},
  \end{gather*}
  where we used the operator norm $\norm{.}$ of matrices with respect
  tot the Euclidean norm of $\R^d$. $\operatorname{cond}_2(A)$ is the
  spectral condition of $A$, namely
  \begin{gather*}
    \operatorname{cond}_2(A) = \norm{A} \, \norm{A^{-1}}.
  \end{gather*}
  With ~\ref{eq:descent-methods:1} we conclude that $s\in \mathcal S_\gamma(\nabla F)$ for any
  $\gamma$ with
  \begin{gather*}
    \gamma \le \frac1{\operatorname{cond}_2 (\nabla f(x))}.
  \end{gather*}
  The different variants of the Newton method are only distinguished by
  a different choice of the scaling parameter $\mu$.
\end{proof}

\begin{Lemma}{downhill}
  Let $F: \R^d \to \R$ be continuously differentiable. For a given
  point $x$, assume $\nabla F = \nabla F(x) \neq 0$.  Then, there is a
  constant $\lambda > 0$ such that for any
  $s\in \mathcal S_\gamma(\nabla F(x))$ and any
  $0 \le \mu \le \lambda$ there holds
  \begin{gather}
    \label{eq:downhill:1}
    F(x-\mu s) \le F(x) - \frac{\gamma\mu}2 \abs{\nabla F(x)}.
  \end{gather}
  In particular, a positive scaling factor $\mu$ for the descent method can
  always be found.
\end{Lemma}

% Stoer/Bulirsch I, 4. Auflage, p. 238
\begin{proof}
  First, define
  \begin{gather*}
    U_1(x) = \bigl\{ y\in \R^d \big\vert \abs{\nabla F(y)-\nabla F(x)}
    \le \tfrac{\gamma}{2} \abs{\nabla F(x)}\bigr\}.
  \end{gather*}
  Since $\nabla F$ is continuous and $\nabla F(x) \neq 0$, this set is
  a nonempty neighborhood of $x$.
% Similarly, the set
%  \begin{gather*}
%    U_2(x) =
%    \bigl\{ y\in \R^d \big\vert \mathcal S_\gamma(\nabla F(y)) \subseteq
%    \mathcal S_{\nicefrac{\gamma}{2}}(\nabla F(x))\bigr\}
%  \end{gather*}
%  is a nonempty neighborhood of $x$.
Choose now $\lambda$ such that
  \begin{gather*}
    B_{\lambda}(x) \subseteq U_1(x),% \cap U_2(x).
  \end{gather*}
  Hence, for any $\mu\in(0,\lambda)$ and
  $s\in \mathcal S_\gamma(\nabla F(x))$, there holds by the mean value
  theorem with $0<\theta<1$
  \begin{multline*}
    F(x)-F(x-\mu s) = \mu \nabla F(x-\theta\mu s)s \\
    = \mu \Bigl(\bigl(\nabla F(x-\theta\mu s) - \nabla F(x)\bigr)
    +\nabla F(x)\Bigr).
  \end{multline*}
  Using the definitions of $U_1(x)$ and $U_2(x)$, we obtain
  \begin{align*}
    F(x)-F(x-\mu s)
    &\ge -\tfrac{\gamma\mu}{2}\abs{\nabla F(x)}
      + \mu DF(x)s\\
    &\ge -\tfrac{\gamma\mu}{2}\abs{\nabla F(x)}
      + \mu\gamma\abs{\nabla F(x)}\\
    &= \tfrac{\gamma\mu}{2}\abs{\nabla F(x)}.
  \end{align*}
\end{proof}

\section{Practical considerations}

\begin{intro}
  Quadratic convergence is an asymptotic statement, which for any
  practical purpose can be replaced by ``fast'' convergence. Most of
  the effort spent in a single Newton step consists of setting up the
  Jacobian $J$ and solving the linear system in the second line
  of~\eqref{eq:newton-def:1}. Therefore, we will consider techniques
  here, which avoid some of this work. We will have to consider two
  cases
  \begin{enumerate}
  \item Small systems with $d\lesssim 1000$. For such systems, a
    direct method like $LU$- or $QR$-decomposition is advisable in
    order to solve the linear system. To this end, we compute the
    whole Jacobian and compute its decomposition, an effort of order
    $d^3$ operations. Comparing to $d^2$ operations for applying the
    inverse and order $d$ for all other tasks, this must be avoided as
    much as possible.
  \item Large systems, where the Jacobian is typically sparse (most of
    its entries are zero). For such a system, the effort of order
    $d^2$ for a full matrix vector multiplication is already not
    affordable. Therefore, the linear problem is solved by an
    iterative method and we will not have to compute the Jacobian at
    all.
  \end{enumerate}
\end{intro}

\begin{remark}
  In order to save numerical effort constructing and inverting
  Jacobians, the following strategies have been successful.
  \begin{itemize}
  \item Fix a threshold $0<\eta<1$ which will be used as a bound for
    error reduction. In each Newton step, first compute the update
    vector $\widehat d$ using the Jacobian $\widehat J$ of the
    previous step. This yields the modified method
    \begin{gather}
      \label{eq:newton:1}
      \begin{alignedat}2
        &&J_k &= J_{k-1} \\
        &&\widehat x &= x^{(k)} - J_{k}^{-1}f(x^{(k)})\\
        \text{If }\abs{f(\widehat x)} &\le \eta
          \abs{f(x^{(k)})} \quad& x^{(k+1)} &= \widehat x\\
          \text{Else } J_k &= \bigl(\nabla f(x^{(k)}\bigr)^{-1}
          \quad & x^{(k+1)} &=x^{(k)}- J_{k}^{-1}f(x^{(k)}). 
      \end{alignedat}
    \end{gather}
    Thus, an old Jacobian and its inverse are used until convergence
    rates deteriorate. This method is again a quasi-Newton method
    which will not converge quadratically. However, we can obtain linear
    convergence at any rate $\eta$.
  \item If Newton's method is used within a time stepping scheme, the
    Jacobian of the last Newton step in the previous time step is
    often a good approximation for the Jacobian of the first Newton
    step in the new time step. This holds in particular for small time
    steps and constant extrapolation. Therefore, the previous method
    should also be extended over the bounds of time steps.
  \item An improvement of the method above can be achieved by so
    called rank-1 updates. Given $x^{(k)}$ and $x^{(k-1)}$, compute
    \begin{gather}
      \label{eq:newton:2}
      \begin{split}
        p &= x^{(k)} - x^{(k-1)}\\
        q &= f(x^{(k)}) - f(x^{(k-1)})\\
        J_k &= J_{k-1} + \frac1{\abs{p}^2}
        \left(q - J_{k-1} p\right)p^T
      \end{split}
    \end{gather}
    The fact that the rank of $J_k - J_{k-1}$ is at most one can be
    used to obtain a decomposition of $J_k$ in a cheap way from one
    for $J_{k-1}$.
  \end{itemize}
\end{remark}

\begin{remark}
  For problems leading to large, sparse Jacobians, typically space
  discretizations of partial differential equations, computing
  inverses of $LU$-decompositions is infeasible. These matrices
  typically only feature a few nonzero elements per row, while the
  inverse and the $LU$-decomposition is fully populated, thus
  increasing the amount of memory from $d$ to $d^2$.

  Linear systems like this are often solved by iterative methods,
  leading for instance to so called Newton-Krylov methods. Iterative
  methods approximate the solution of a linear system
  \begin{gather*}
    J d = f
  \end{gather*}
  only using multiplications of a vector with the matrix $J$. On
  the other hand, for any vector $v\in \R^d$, the term $Jv$ denotes
  the directional derivative of $f$ in direction $J$. Thus, it can be
  approximated easily by
  \begin{gather*}
    J v \approx \frac{f\left(x^{(k)}+\epsilon v\right) -
      f\left(x^{(k)}\right)}{\epsilon}.
  \end{gather*}
  The term $f\left(x^{(k)}\right)$ must be calculated anyway as it is
  the current Newton residual. Thus, each step of the iterative linear
  solver requires one evaluation of the nonlinear function, and no
  derivatives are computed.

  The efficiency of such a method depends on the number of linear
  iteration steps which is determined by two factors: the gain in
  accuracy and the contraction speed. It turns out that typically
  gaining two digits in accuracy is sufficient to ensure fast
  convergence of the Newton iteration. The contraction number is a
  more difficult issue and typically requires preconditioning, which
  is problem-dependent and as such must be discussed when needed.
\end{remark}

%%% Local Variables:
%%% mode: latex
%%% TeX-master: "notes"
%%% End:
