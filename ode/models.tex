\section{Modeling with ordinary differential equations}

\begin{example}[Exponential growth]
  Bacteria are living on a substrate with ample nutrients. Each
  bacteria splits into two after a certain time $\Delta t$. The time
  span for splitting is fixed and independent of the individuum. Then,
  given the amount $u_0$ of bacteria at time $t_0$, the amount at
  $t_1 = t_0+\Delta t$ is $u_1 = 2 u_0$. Generalizing, we obtain
  \begin{gather*}
    u_n = u(t_n) = 2^n u_0, \qquad t_n = t_0 + n\Delta t.
  \end{gather*}

  After a short time, the number of bacteria will be huge, such that
  counting is not a good idea anymore. Also, the cell division does
  not run on a very sharp clock, such that after some time, divisions
  will not only take place at the discrete times $t_0+n\Delta t$, but
  at any time between these as well. Therefore, we apply the continuum
  hypothesis, that is, $u$ is not a discrete quantity anymore, but a
  continuous one that can take any real value. In order to accommodate
  for the continuum in time, we make a change of variables:
  \begin{gather*}
    u(t) = 2^{\frac{t-t_0}{\Delta t}} u_0.
  \end{gather*}

  Here, we have already written down the solution of the problem,
  which is hard to generalize. The original description of the problem
  involved the change of $u$ from one point in time to the next. In
  the continuum description, this becomes the derivative, which we can
  now compute from our last formula:
  \begin{gather*}
    \tfrac{d}{dt} u(t) = \frac{\ln 2}{\Delta t} 2^{\frac{t-t_0}{\Delta t}} u_0
    = \frac{\ln 2}{\Delta t} u(t).
  \end{gather*}
  
  We see that the derivative of $u$ at a certain time depends on $u$
  itself at the same time and a constant factor, which we call the
  growth rate $\alpha$. Thus, we have arrived at our first
  differential equation
  \begin{gather}
    \label{eq:models:1}
    u'(t) = \alpha u(t).
  \end{gather}
  What we have seen as well is, that we had to start with some
  bacteria to get the process going. Indeed, any function of the form
  \begin{gather*}
    u(t) = c e^{\alpha t}
  \end{gather*}
  is a solution to equation~\eqref{eq:models:1}. It is the initial
  value $u_0$, which anchors the solution and makes it unique.
\end{example}

\begin{example}[Predator-prey systems]
  We add a second species to our bacteria example. Let's say, we
  replace the bacteria by sardines living in a nutrient rich sea, and
  we add tuna eating sardines. The amount of sardines eaten depends on
  the likelyhood that a sardine and a tuna are in the same place, and
  on the hunting efficiency of the tuna. Thus,
  equation~\eqref{eq:models:1} is augmented by a negative change in
  population depending on the product of sardines $u$ and tuna $v$:
  \begin{gather*}
    u' = \alpha u - \beta u v.
  \end{gather*}

  In addition, we need an equation for the amount of tuna. In this
  simple model, we will make two assumptions: first, tuna die of
  natural causes at a death rate of $\gamma$. Second, tuna procreate
  if there is enough food (sardines), and the procreation rate is
  proportional to the amount of food. Thus, we obtain
  \begin{gather*}
    v' = \delta u v - \gamma v.
  \end{gather*}

  Again, we will need initial populations at some point in time to
  compute ahead from there.
\end{example}

\begin{example}[Graviational two-body systems]
  According to Newton's law of universal gravitation, two bodies of
  masses $m_1$ and $m_2$ attract each other with a force
  \begin{gather*}
    \vec F_1 = G \frac{m_1m_2}{r^3} \vec r_1,
  \end{gather*}
  where $\vec F_1$ is the force vector acting on $m_1$ and $\vec r_1$
  is the vector pointing from $m_1$ to $m_2$ and $r = \lvert\vec r_1\rvert = \lvert\vec r_2\rvert$.

  Newton's second law of motion on the other hand relates forces and
  acceleration:
  \begin{gather*}
    \vec F = m \vec x'',
  \end{gather*}
  where $\vec x$ is the position of a body in space.

  Combining these, we obtain equations for the positions of the two bodies:
  \begin{gather*}
    \vec x''_i = G \frac{m_{3-i}}{r^3} (\vec x_i - \vec x_{3-i}), \qquad i=1,2.
  \end{gather*}
  This is a system of 6 independent variables. Nevertheless, it can be
  reduced to three by using that the center of mass moves
  inertially. Then, the distance vector is the only variable to be
  computed for:
  \begin{gather*}
    \vec r'' = - G \frac{m}{r^3} \vec r.
  \end{gather*}
  Intuitively, that we need an initial position and an initial
  velocity for the two bodies. Later on, we will see that this can
  actually be justified mathematically.
\end{example}

\begin{example}[Celestial mechanics]
  Now we extend the two-body system to a many-body system. Again, we
  subtract the center of mass, such that we obtain $n$ sets of 3
  equations for an $n+1$-body system. Since forces simply add up, this
  system becomes
  \begin{gather}
    \label{eq:celestial}
    \vec x_i = -G \sum_{j\neq i} \frac{m_j}{r_{ij}^3} \vec r_{ij}.
  \end{gather}
  Here, $\vec r_{ij} = \vec r_j - \vec r_i$ and $r_{ij} = \lvert \vec r_{ij}\rvert$.
  Initial data for the solar system can be obtained from
  \begin{center}
    \texttt{https://ssd.jpl.nasa.gov/?horizons}
  \end{center}
\end{example}

%%% Local Variables: 
%%% mode: latex
%%% TeX-master: "notes"
%%% End: 
