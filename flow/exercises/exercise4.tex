% --------------------------------------------------------------
% This is all preamble stuff that you don't have to worry about.
% Head down to where it says "Start here"
% --------------------------------------------------------------
 
\documentclass[12pt]{article}
 
\usepackage[margin=1in]{geometry} 
\usepackage{amsmath,amsthm,amssymb}
\usepackage{graphicx}
 
\newcommand{\N}{\mathbb{N}}
\newcommand{\Z}{\mathbb{Z}}
\newcommand{\bh}{\boldsymbol{h}}
\newcommand{\bx}{\boldsymbol{x}}
\newcommand{\bu}{\boldsymbol{u}}
\newcommand{\bv}{\boldsymbol{v}}
\newcommand{\bomega}{\boldsymbol{\omega}}
 
\newenvironment{theorem}[2][Theorem]{\begin{trivlist}
\item[\hskip \labelsep {\bfseries #1}\hskip \labelsep {\bfseries #2.}]}{\end{trivlist}}
\newenvironment{lemma}[2][Lemma]{\begin{trivlist}
\item[\hskip \labelsep {\bfseries #1}\hskip \labelsep {\bfseries #2.}]}{\end{trivlist}}
\newenvironment{exercise}[2][Exercise]{\begin{trivlist}
\item[\hskip \labelsep {\bfseries #1}\hskip \labelsep {\bfseries #2.}]}{\end{trivlist}}
\newenvironment{problem}[2][Problem]{\begin{trivlist}
\item[\hskip \labelsep {\bfseries #1}\hskip \labelsep {\bfseries #2.}]}{\end{trivlist}}
\newenvironment{question}[2][Question]{\begin{trivlist}
\item[\hskip \labelsep {\bfseries #1}\hskip \labelsep {\bfseries #2.}]}{\end{trivlist}}
\newenvironment{corollary}[2][Corollary]{\begin{trivlist}
\item[\hskip \labelsep {\bfseries #1}\hskip \labelsep {\bfseries #2.}]}{\end{trivlist}}

\setlength\parindent{0pt}
 
\begin{document}
 
\title{Finite Element Methods for Flow Simulations\\ - Exercise Sheet 4 -}
\author{Daniel Arndt}
\date{}
 
\maketitle

\begin{exercise}{7}
Consider a stationary, incompressible ($\nabla \cdot \bv = 0$) and inviscid flow field $\bv$, i.e. $\nu = 0$. 
Moreover, suppose that the external source term $\boldsymbol{f}$ is a potential, 
i.e. there exists a function $\phi$ such that $\boldsymbol{f} = -\nabla \phi$, 
and define the expression $B = \frac12 \bv^2 + \phi + \rho p$.
\begin{enumerate}
 \item Show the Bernoulli theorem: $B$ is constant along streamlines, i.e. $(\bv \cdot \nabla)B = 0$.
 \item What changes if the flow field is additionally irrotational, i.e. $\nabla \times \bv = \boldsymbol{0}$?
\end{enumerate}
Hint: Start from the stationary Euler model $(\bv \cdot \nabla)\bv + \rho \nabla p = \boldsymbol{f}$ and show the identity
\begin{align*}
(\bv \cdot \nabla)\bv = \nabla( |\bv|^2) - \bv \times (\nabla \times \bv).
\end{align*}

\end{exercise}

\begin{exercise}{8}
Consider the Stokes problem in $\Omega \subset \mathbb{R}^d§$ with mixed boundary conditions on $\partial \Omega = \Gamma_D \cup \Gamma_N$ with
$\Gamma_D \cap \Gamma_N = \emptyset$ and $\operatorname{meas}_{d-1} \Gamma_D > 0$:
\begin{align*}
&\hfill &                  -\nu \Delta \bu + \nabla p &= \boldsymbol{f} &&\text{ in } \Omega,&\hfill\\
&\hfill &                            \nabla \cdot \bu &= 0              &&\text{ in } \Omega,&\hfill\\
&\hfill &                                         \bu &= \boldsymbol{0} &&\text{ on } \Gamma_D,&\hfill\\
&\hfill & (\nu \nabla \bu - p I) \cdot \boldsymbol{n} &= 0              &&\text{ on } \Gamma_N.&\hfill
\end{align*}
\begin{enumerate}
\item Derive a weak formulation of the above problem in appropriate function spaces.
\item Explain why the variational formulation admits one and only one (generalized) solution. Why is
the pressure already uniquely defined?
\end{enumerate}
\end{exercise}
\end{document}